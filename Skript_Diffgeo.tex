\documentclass[fleqn, 12pt, letterpaper]{article}

\usepackage[utf8]{inputenc}
\usepackage[ngerman]{babel}
\usepackage[a4paper, total={6in, 8in}]{geometry}
\usepackage{graphicx}
\usepackage{wrapfig}
\usepackage{amsmath, amssymb, amsthm}
\usepackage{textgreek}
\usepackage{textcomp, mathcomp}
\usepackage{verbatim}
\usepackage{pdfpages}
\usepackage{siunitx}
\usepackage{booktabs}
\usepackage[shortlabels]{enumitem}
\usepackage{hyperref}
\usepackage{fancyhdr}
\usepackage{xparse}
\usepackage{marginnote}
\usepackage{tabularx}
\usepackage{xcolor}
\usepackage{float}
\usepackage{isotope}
\usepackage{mhchem}
\usepackage[export]{adjustbox}
\usepackage{bbm}
\usepackage{mathrsfs}
\usepackage{tikz}

% Eigene Makros
\newcommand{\steigung}[2]{\frac{\Delta #1}{\Delta #2}}
\newcommand{\inv}[1]{#1^{-1}}
\newcommand{\ten}[1]{\cdot 10^{#1}}
\newcommand{\deldel}[2]{\frac{\partial #1}{\partial #2}}
\newcommand{\diff}[1]{\text{d}#1}
\newcommand{\w}{\omega}
\newcommand{\diffdiff}[2]{\frac{\diff{#1}}{\diff{#2}}}
\newcommand{\relerr}[2]{\sqrt{\left(\frac{\Delta #1}{#1}\right)^2+\left(\frac{\Delta #2}{#2}\right)^2}}
\newcommand{\txt}[1]{\text{#1}}
\newcommand\norm[1]{\left\lVert#1\right\rVert}

\setlength{\parindent}{0pt}
\textheight 23cm

% Kopfteil ohne Versuchsnr. und Titel
\newcommand{\instinfo}[3]{
	\begin{large}
		\textbf{#2}\hspace{0.2cm}
		\textbf{#1}\hfill
		Zusammenfassung von \textsc{#3}
		\vspace{.1cm}
		\hrule
	\end{large}
}

\newcommand{\papheader}[3]{	
	\instinfo{}{#2}{#1}
	\vspace{0.5em}	
	\begin{center}
		\textbf{\large{Grundlagen der Geometrie und Topologie}}\\
		{\large{Sommersemester 2025}} \\
		\large{Dozent: #3}
		\vspace{1em}
		\hrule
	\end{center}
}

%%%%%%%%%%%%%%%%%%%%%%%%%%%%%%%%%%%%%%%%%%%%%%%%%%%%%%%%%%%%%%%
\begin{document}

\vspace*{-2.5cm}

% Kopf mit Name, Datum und Dozent
\papheader{Yuting Shi und Tianming Zhao} % Name
{14.04.2025}           % Datum
{Kevin Wiegand}        % Dozent

\section{Differenzierbare Mannigfaltigkeiten}

\underline{Organisatorisches:} Übungsaufgabe Dienstags, Termin: Mo, Di 11-13 Uhr.\\

\underline{Ziele der Vorlesung:}
\begin{itemize}
	\item Topologische und differenzierbare Mannigfaltigkeit
	\item Vektorbündel
	\item Lie- Gruppen
	\item Triangulierung und 
	\item Überlagerung \& Fundamentalgruppe
\end{itemize}
\subsection{Wiederholung Topologischer Grundlage und Mannigfaltigkeit}
\textbf{Definition 1.1:} Sei $M$ eine Menge, ein Mengensystem $O=O_M \subset \mathcal{P} (M)$ heißt \underbar{Topologie} auf M, falls
\begin{itemize}
	\item $\varnothing, M\in O$
	\item $U,V\in O\;\implies \;U\cap V\in O\qquad \text{(endlicher Schnitt)}$
	\item $U_i\in O\;\forall i\in I,\;I\;\txt{beliebig}\;\implies \bigcup_{i\in I} U_i\in O \qquad \txt{(beliebige Vereinigung)}$
\end{itemize}
Das Paar $(M,O)$ heißt \underbar{topologischer Raum}. $U\subseteq M$ heißt \underbar{offen}, falls $U\in O$; $A\subseteq M$ heißt \underbar{abgeschlossen}, falls $M\setminus A$ offen.\\

\underline{Beispiel:}\\

1) Standardtopologie auf $\mathbb{R} ^n$: $U\subseteq \mathbb{R}^n \;\text{offen} \;\Leftrightarrow \;\forall x\in U \;\exists r_x>0 \;\text{mit} B_{r_x}(x):=\{y\in \mathbb{R}^n||x-y|<r_x\} \subseteq U$\\
Allgemein: $(M,d)$ metrischer Raum $U\subseteq M \; \text{offen} \;:\Leftrightarrow \; \forall x\in U \exists r_x>0 \; \text{mit}\;B_{r_x}(x):=\{y\in \mathbb{R}^n|d(x,y)<r_x\}\subseteq U$\\

2) Sei M beliebige Menge\begin{itemize}
	\item diskrete Topologie: $O=\mathcal{P}(M)$, also die Topologie, bei der jede Teilmenge offen ist. Diese Topologie kommt von einer Metrik (diskrete Metrike)
	\[
  d(x, y) = \begin{cases}
  0 & \text{wenn } x = y \\
  1 & \text{wenn } x \ne y
  \end{cases}
  \]
	\item indiskrete Topologie: $O=\{\varnothing ,M\}$, diese Topologie kommt im Allgemeinen NICHT von einer Metrik, wenn M mehr als ein Element hat. Denn sonst wäre (M,d) metrischer Raum, der hausdorff ist. Muss es ausreichend viele offene Mengen geben, um Punkte unterscheiden zu können.
\end{itemize}

3) Teilraumtopologie: $(\mu, O_{\mu})$ topologischer Raum, $X\subseteq M$. Wir definieren eine Topologie $O_X$ auf $X$ durch 
\begin{equation*}
	U\in O_X :\Leftrightarrow \exists V\in O_{\mu} \;\text{mit}\; U=X\cap V $, $O_X
\end{equation*}
heißt \underbar{Teilraumtopologie} oder \underbar{induzierte Topologie} auf X (vgl. Übungsaufgabe).\\

4) $M=\mathbb{Z} $, $U\subseteq \mathbb{Z}$ offen $:\Leftrightarrow \mathbb{Z}\setminus U \;\txt{endlich oder}\;U=\varnothing \qquad $\textcolor{blue}{(Ko-endliche Topologie)} \\

5) Die \underbar{Quotiententopologie:} M topologischer Raum, $\sim$ eine Äquivalenzrelation auf M. Dann wird $M/_{\sim}:=$ Menge der Äquivalenzklassen $=\{[x]|x\in M\}$ wird zu einem topologischen Raum durch: 
\begin{equation*}
	U\subseteq M/_{\sim} \; \txt{offen}\; \Leftrightarrow  \pi^{-1}(U)\subseteq M \;\txt{offen, wobei}\; \pi:M\rightarrow M/_{\sim}, \;x\mapsto [x]
\end{equation*}
6) $M=\mathbb{R}\times \{+,-\}$, $(x, \xi )\sim (y, \eta)\Leftrightarrow \xi = \eta, x=y$ oder $\xi\neq \eta, x=y\neq 0$

offene Menge von $M/_{\sim}$ sind\begin{itemize}
	\item offene Intervalle I mit $0\notin I$
	\item $(a, 0)\cup \{0,-\} \cup (0, b)\; \; a<0<b$
	\item $(a, 0)\cup \{0,+\} \cup (0, b)\; \; a<0<b$
	\item Vereinigung davon
\end{itemize}

\begin{figure}[H]
    \centering
    \includegraphics[width=10cm]{Image Diffgeo/1.jpg}
	\caption{Darstellung Quotientenmenge}
 \end{figure}
$U\subseteq M$ offen, falls $U\cap \mathbb{R}\times\{+\}$ offen und $U\cap \mathbb{R}\times\{-\}$ offen.\\

\textbf{Definition 1.2:} Ein Mengensystem $\mathcal{B} \subset \mathcal{P} (M)$ heißt \underbar{Basis der Topologie} $O_M$, falls die offene Menge aus $O_M$ genau die Vereinigung der Mengen aus $\mathcal{B}$ sind. Insbesondere $\mathcal{B}\subseteq O_M$. Eine (topologische) Basis auf $X$ ist eine Menge $\mathcal{B}$ mit folgenden Eigenschaften:
\begin{itemize}
  \item Jeder Punkt $x\in X$ ist in mindestens einer Menge von $\mathcal{B}$ enthalten. (Mit anderen Worten, es gilt $\cup \mathcal{B}=X$.)
  \item Ist ein Punkt $x\in X$ in zwei Mengen $B_1$,$B_2\in \mathcal{B}$ enthalten, dann gibt es $B_3\in \mathcal{B}$ mit $x\in B_3$ und $B_3\subset B_1\cap B_2$
\end{itemize}

\underline{\textbf{Bemerkung:}} $\mathcal{B}\subseteq O_M$ ist eine Basis der Topologie $\Leftrightarrow$ Für alle $U\in O_M$ und alle $p\in U$ ein $B\in \mathcal{B}$ mit $p\in B\subseteq U$ existiert.\\

\underline{Beispiel:}\\

1) $\mathcal{B}=\{B_r(q)\subseteq \mathbb{R}^n|r\in \mathbb{Q}_{>0}, q\in\mathbb{Q}^n\}$ ist eine (abzählbare) Basis der Standardtopologie auf $\mathbb{R}^n$.\\
Allgemein: (M,d) metrischer Raum und $D\subseteq M$ abzählbare dichte Teilmenge, $\mathcal{B}=\{B_r(q)\subseteq M|r\in \mathbb{Q}_{>0}, q\in D\}$\\

2) $\mathcal{B}=\{\{x\}\subseteq M|x\in M\}$ ist eine Basis der diskreten Topologie\\
$\mathcal{B}=\{M\}$ ist eine Basis der indiskreten Topologie (Konvention: $\bigcup_{i\in \varnothing }B_i=\varnothing $)\\

\textbf{Definition 1.3}\begin{itemize}
	\item Eine Abbildung $f:(M, O_M)\rightarrow (N, O_N)$ heißt \underbar{stetig}, falls $f^{-1}(V)\in O_M$ für alle $V\in O_N$ (Urbild offener Mengen wieder offen).
	\item Eine bijektive Abbildung $f:M\rightarrow N$ heißt \underbar{Homöomorphismus}, falls $f$ und $f^{-1}$ stetig sind. In diesem Fall nennt man die Räume $M$ und $N$ \underbar{homöomorph}.
	\item Ein topologischer Raum $M$ heißt \underbar{lokal homöomorph} zu $\mathbb{R}^n$ (häufig: \underbar{lokal euklidisch}), falls für alle $p\in M$ eine offene Umgebung $U\subseteq M$ mit $p\in U$, eine offene Menge $V\subseteq \mathbb{R}^n$ und ein Homöomorphismus $\phi:p\in U\rightarrow V$ existieren.
\end{itemize}

\underline{Beispiel.}
\begin{itemize}
	\item $(M, \mathcal{P}(M))$ ist lokal homöomorph zu $\mathbb{R}^0=\{0\}$. (nehme $U:=\{p\}\in \mathcal{P}(M)$ offen, $V=\{0\}$ für alle p)
	\item Reelle Achse mit zwei Nullen ist lokal homöomorph zu $\mathbb{R}^1$
\end{itemize}

\begin{figure}[H]
    \centering
    \includegraphics[width=11cm]{Image Diffgeo/2.jpg}
	\caption{Lokale Homöomorphie zu $\mathbb{R}^1$}
 \end{figure}

\textbf{Definition 1.4}\begin{itemize}
	\item Ein topologischer Raum $M$ heißt \underbar{zusammenhängend}, falls kein $\varnothing \neq U \subsetneq M$ existiert, dass zugleich offen und abgeschlossen ist. (M lässt sich nicht in nicht-leere, disjunkte, offene Mengen zerlegen.)
	\item $X\subseteq M$ heißt \underbar{zusammenhängend}, falls X mit der Teilraumtopologie zusammenhängend ist.
	\item Ein topologischer Raum $M$ heißt \underbar{weg-zusammenhängend}, falls sich je zwei Punkte in M durch einen stetigen Weg in M verbinden lassen, d.h. 
	\begin{equation*}
		\forall p, q\in M \exists  \gamma:[0,1]\rightarrow M \;\txt{stetig mit}\; \gamma(0)=p, \gamma(1)=q 
	\end{equation*}
\end{itemize}

\textbf{Lemma 1.5}\begin{itemize}
	\item $M\;\txt{wegzusammenhängend}\;\implies M$ zusammenhängend
	\item $M \; \txt{zusammenhängend und lokal euklidisch}\; \implies \;M \txt{ weg-zusammenhängend}$
\end{itemize}
\begin{proof}
	Sei $p\in M$, betrachte $W_p=\{q\in M\;|\;\exists \gamma:[0,1]\rightarrow M \;\txt{stetig mit}\;\gamma(0)=p, \gamma(1)=q)\}$ und zeige $W_p$ ist offen und abgeschlossen:\\

	(a) offen: Sei $q\in W_p$, dann existiert ein Weg $\gamma$ zwischen p und q.\\
	M lokal euklidisch $\implies \exists$ offene Umgebung U von q, wobei U homöomorph zu einer offenen Umgebung V in $\mathbb{R}^n$ ist. Da $\mathbb{R}^n$ weg-zusammenhängend ist, ist V selbst auch wegzusammenhängend sowie U. So für alle $q'$ in U existiert ein Weg $\delta$ zwischen q und $q'$. Verbinde $\delta$ und $\gamma$ hat man einen Weg $\gamma \circ \delta$ zwischen p und $q'$ konstruiert. $\implies q'\in W_p \implies U \subset W_p$ offen, aus Definition von Offenheit folgt dann $W_p$ offen.\\
	
	(b) abgeschlossen: Sei $(q_n)_{n\in \mathbb{N}}\subset W_p, \;q_n\overset{n\rightarrow \infty}{\longrightarrow}q\in \overline{W_p}$, zu zeigen $q\in W_p$. Da M lokal euklidisch, existiert eine weg-zusammenhängende Umgebung U von q. Wegen Konvergenz gibt es ein $N_0\in\mathbb{N}, q_n\in U\; \forall n\geqslant N_0$. Es gibt also einen Weg $\gamma_n$ zwischen $q_n$ und q, hänge $\gamma_n$ an dem Weg $\delta_n$ zwischen p und $q_n$ ergibt sich der Weg $\gamma_n\circ\delta_n$ zwischen q und p. $\implies q\in W_p$
\end{proof}
\begin{figure}[H]
    \centering
    \includegraphics[width=13cm]{Image Diffgeo/3.jpg}
	\caption{Links: Offenheit; Rechts: Abgeschlossenheit}
 \end{figure}

\textbf{Definition 1.6}\begin{itemize}
	\item Ein topologischer Raum heißt (Überdeckungs-)kompakt, falls jede offene Überdeckung eine endliche Teilüberdeckung besitzt. (i.A. nicht dasselbe wie Folgen-kompakt, für metrische Räume schon)
	\item Ein topologischer Raum heißt \underbar{hausdorff} (auch $T_{\alpha}$), falls für alle $p, q\in M, p\neq q$ offene Mengen $U, V\subset M$ existiert mit \begin{itemize}
		\item $p\in U,\; q\in V,\;U\cap V =\varnothing$
	\end{itemize}
\end{itemize}
\begin{figure}[H]
    \centering
    \includegraphics[width=7cm]{Image Diffgeo/4.jpg}
	\caption{Hausdorff'scher Raum}
 \end{figure}
\textbf{Bemerkung:} Lokal euklidisch impliziert NICHT hausdorff'sch, siehe Beispiel reelle Achse mit zwei Nullen (Problem bei 0+ und 0-).\\

\textbf{Definition 1.7}\\
Ein $n$-dimensionale \underline{topologische Mannigfaltigkeit} ist ein topologischer Raum $M$ mit:
\begin{itemize}
    \item[(i)] $M$ ist lokal homöomorph zu $\mathbb{R}^n$ \hfill \textcolor{orange}{(selbes $n$ für alle Punkte)}
    \item[(ii)] $M$ ist hausdorff (Trennungseigenschaft)
    \item[(iii)] $M$ besitzt eine abzählbare Basis der Topologie \hfill \textcolor{blue}{(zweitabzählbar)}
\end{itemize}

Die lokalen Homöomorphismen $\varphi: U \subset M \to V \subset \mathbb{R}^n$ heißen \underline{Karten} 
\textcolor{blue}{(lokale Koordinatensysteme / lokale Koordinaten)}
\begin{figure}[H]
    \centering
    \includegraphics[width=10cm]{Image Diffgeo/5.jpg}
	\caption{Torus $T^2\subseteq \mathbb{R}^3$ als 2-dim topologische Mannigfaltigkeit in $\mathbb{R}^3$}
 \end{figure}
Die zweitabzählbarkeit garantiert, dass die Mannigfaltigkeit nicht zu groß wird.\\

\textbf{Bemerkung}\\
Für $M \neq \emptyset$ ist die Dimension von $M$ eindeutig bestimmt.\\
D.\,h. sind $p \in M$, $U, V$ offene Mengen mit $p \in U$, $p \in V$ und $\varphi: U \to \mathbb{R}^n, \quad \psi: V \to \mathbb{R}^m$ Homöomorphismen, dann gilt: $n = m$.

\begin{proof}
	Benutzt den \underline{Satz von Brouwer}: Sei $U \subset \mathbb{R}^n$ offen und $f: U \to \mathbb{R}^n$ stetig und injektiv. Dann ist die Menge $f(U) \subset \mathbb{R}^n$ offen, d.\,h. $f$ ist ein Homöomorphismus aufs Bild.\\

    Angenommen $n \neq m$ (o.\,E. $n > m$).\\
    Wir betrachten die stetige, injektive Abschneidung
    \[
    z: \mathbb{R}^n \to \mathbb{R}^m, \quad (x_1, \dots, x_m) \mapsto (x_1, \dots, x_n,0, \dots, 0)
    \]
    Aber: $z(\mathbb{R}^n)$ enthält keine offene, nicht-leere Menge. ($\forall x, \forall \epsilon>0, B_{\epsilon}(x)\nsubseteq z(\mathbb{R}^m)$, denn es gibt $y\in B_{\epsilon}(x), y_i\neq 0$ für ein i nach n)\\

	\( p \in U\cap V\subseteq M\) ist offen:
    \[
    \underbrace{\phi(U\cap V)}_{\in \mathbb{R}^n} \xrightarrow{\phi^{-1} \;\txt{Homöom}} \underbrace{U\cap V}_{\in M} \xrightarrow{\psi \;\txt{Homöom}} \underbrace{\psi(U\cap V)}_{\in \mathbb{R}^m}     \xrightarrow{z \;\txt{injektiv, stetig}} \mathbb{R}^n
    \]

    Die Verkettung ist also stetig und injekitv, aber das Bild $(z\circ \psi\circ\phi^{-1})(\phi(U\cap V))\subseteq \mathbb{R}^n$ ist nicht offen, Widerspruch zum Satz v. Browser!\\
    \[
    \Rightarrow m = n.
    \]
\end{proof}

\textbf{Bemerkung}
\begin{itemize}
    \item Sei $M$ eine Mannigfaltigkeit, aus dem Metrisierungssatz von Nagata-Smirnov folgt, dass es eine Metrik auf $M$ gibt, welche die Topologie auf $M$ erzeugt.
    
    \item O.E. können wir annehmen, dass das Bild einer Karte ganz $\mathbb{R}^n$ ist. \\
    Sei $\phi : U \rightarrow V \subseteq \mathbb{R}^n$ eine Karte und $p \in U$. Dann gibt es Ball $B_\varepsilon(\phi(p)) \subseteq V$.
    \[
        \Rightarrow \text{Die Einschränkung } \widetilde{\phi} : \phi^{-1}(B_\varepsilon(\phi(p))) \rightarrow B_\varepsilon(\phi(p)) \text{ ist ein Homöomorphismus.}
    \]
    Durch Verketten von $\widetilde{\phi}$ mit einem Homöomorphismus, der $B_\varepsilon(\phi(p))$ mit $\mathbb{R}^n$ identifiziert, folgt die Behauptung.
\end{itemize}
\begin{figure}[H]
    \centering
    \includegraphics[width=13cm]{Image Diffgeo/6.jpg}
	\caption{Bild von Karte ganz auf $\mathbb{R}^n$}
 \end{figure}

\textbf{Beispiele:}
\begin{itemize}
    \item[(i)] $M = \mathbb{R}^n$, der $n$-dimensionale euklidische Raum, Karte: $\text{id} : \mathbb{R}^n \to \mathbb{R}^n$ \\
    \textbf{\textit{Allgemein:}} Jede offene Teilmenge von $\mathbb{R}^n$ ist eine $n$-dimensionale Mannigfaltigkeit.

    \item[(ii)] $S^n = \left\{ x \in \mathbb{R}^{n+1} \ \middle| \ x_1^2 + x_2^2 + \ldots + x_{n+1}^2 = 1 \right\}$, die $n$-dimensionale Sphäre, mit der Teilraumtopologie des $\mathbb{R}^{n+1}$ \\
    Hausdorff (Teilmenge von hausdorff'schem Raum), zweitabzählbar (Teilraum Basistopologie $\{B_r(q)|q\in \mathbb{Q}^n, r\in Q_{>0}\}$)

    \textbf{Karten:} (stereographische Projektion)
    \begin{align*}
        g_1 : S^n \setminus \{0, \ldots, 0, 1\} &\to \mathbb{R}^n \qquad \txt{(Nordpol)} \\
        x = (x_1, \ldots, x_{n+1}) &\mapsto \frac{1}{1 - x_{n+1}} (x_1, \ldots, x_n)
    \end{align*}

    \begin{align*}
        g_2 : S^n \setminus \{0, \ldots, 0, -1\} &\to \mathbb{R}^n \qquad \txt{(Südpol)}\\
        (x_1, \ldots, x_{n+1}) &\mapsto \frac{1}{1 + x_{n+1}} (x_1, \ldots, x_n)
    \end{align*}
\end{itemize}
\begin{figure}[H]
    \centering
    \includegraphics[width=9cm]{Image Diffgeo/7.jpg}
	\caption{stereographische Projektion n=1}
 \end{figure}
 \begin{figure}[H]
    \centering
    \includegraphics[width=13cm]{Image Diffgeo/8.jpg}
	\caption{stereographische Projektion n=2}
 \end{figure}

 \begin{itemize}
    \item[(iii)] $\mathbb{RP}^n$, der $n$-dimensionale reell projektive Raum.

    \textit{„Der Raum aller 1-dimensionalen Unterräume von } $\mathbb{R}^{n+1}$ \textit{“} \\
    \textit{„Alle Geraden durch Null in} $\mathbb{R}^{n+1}$\textit{“}

    \medskip

    Genauer: $\mathbb{RP}^n = \mathbb{R}^{n+1}  \setminus \{0\} / \sim$, wobei
    \[
    x \sim y \iff x = \lambda y \quad \text{für ein } \lambda \in \mathbb{R} \setminus \{0\}
    \]
    $\mathbb{RP}^n$ trägt die Quotiententopologie, d.h. sei
    \[
    \pi : \mathbb{R}^{n+1} \setminus \{0\} \longrightarrow \mathbb{RP}^n,\quad x \mapsto [x] \;\txt{(alle Vektoren in einer Richtung)}
    \]
    die Quotientenabbildung, dann gilt:
    \[
    U \subseteq \mathbb{RP}^n \text{ offen} \iff \pi^{-1}(U) \subseteq \mathbb{R}^{n+1} \text{ offen}
    \]
    Wir schreiben die Äquivalenzklasse von $x$ als
    \[
    \pi((x_0, \ldots, x_n)) = [(x_0, \ldots, x_n)] = [x_0 : \ldots : x_n] = [\lambda x_0 : \ldots : \lambda x_n], \quad \forall \lambda \neq 0
    \]

    \medskip

    $\mathbb{RP}^n$ ist kompakt, hausdorffsch, zusammenhängend und zweitabzählbar. \\
    (n-dim reelle, kompakte, glatte Mannigfaltigkeit, nicht orientierbar für $n\geqslant 2$)
\end{itemize}
\textbf{Karten:}
\begin{align*}
V_i &= \left\{ (x_0, \ldots, x_n) \mid x_i \neq 0 \right\} \subset \mathbb{R}^{n+1} \\
U_i &= \pi(V_i) = \left\{ [x_0 : \ldots : x_n] \mid x_i \neq 0 \right\}=\{[\frac{x_0}{x_i}:\dots:1:\dots:\frac{x_n}{x_i}]
\end{align*}
\[
\varphi_i : U_i \to \mathbb{R}^n, \quad [x] \mapsto \left( \frac{x_0}{x_i}, \ldots, \frac{x_{i-1}}{x_i}, \frac{x_{i+1}}{x_i}, \ldots, \frac{x_n}{x_i} \right)
\]
$\varphi_i$ ist wohldefiniert, denn:
\begin{align*}
[x] = [y] &\iff x \sim y \\
&\iff \exists \lambda \neq 0 : x = \lambda y \\
&\iff x_j = \lambda y_j \quad \forall j \quad\lambda \neq0\\\
&\iff \frac{x_j}{x_i} = \frac{\lambda y_j}{\lambda y_i} = \frac{y_j}{y_i} \quad \forall j \neq i \\
&\Rightarrow \varphi_i([x]) = \varphi_i([y])
\end{align*}
Die Karte misst einfach die Steigung der Geraden durch den Ursprung.
\begin{itemize}
	\item[(iv)] Die Kegel \(M = \left\{ (x, y, z) \in \mathbb{R}^3 \mid z^2 = x^2 + y^2 \right\}\) ist keine topologische Mannigfaltigkeit (nicht lokal euklidisch bei $(0,0,0)$).
\end{itemize}

\medskip

\textbf{Satz 1.8} Eine 0-dimensionale topologische Mannigfaltigkeit ist eine abzählbare Menge von Punkten mit der diskreten Topologie. Insbesondere sind 0-dimensionale kompakte Mannigfaltigkeiten endlich.
\begin{proof}
	Sei $M$ eine 0-dimensionale Mannigfaltigkeit und $p \in M$.\\
    Dann existiert eine offene Menge $U_p \subseteq M$ mit $p \in U_p$, die homöomorph zu $\mathbb{R}^0 = \{0\}$ ist.
    \[
    \Rightarrow U = \{p\} \Rightarrow \text{Die Punkte in } M \text    { sind offene Mengen.}
    \]
    \begin{itemize}[label={--}]
      \item zweitabzählbar $\Rightarrow$ $M$ abzählbar
      \item kompakt $\Rightarrow$ $M$ endlich (endliche Teilüberdeckung)
    \end{itemize}
\end{proof}

\textbf{Bemerkung:} Eine zusammenhängende 1-dimensionale topologische Mannigfaltigkeit ist homöomorph zu $S^1$ oder $\mathbb{R}$.

\medskip

\textbf{Satz 1.9:} Sei $M$ eine $m$-dimensionale und $N$ eine $n$-dimensionale Mannigfaltigkeit. Dann ist $M \times N$ eine topologische Mannigfaltigkeit der Dimension $m+n$. \textit{(s. Übungen)}

\medskip

\textbf{Beispiel:} Sei $M = S^1$, dann ist $T^2$ der 2-dimensionale Torus. Allgemein: $T^n = \underbrace{S^1 \times \ldots \times S^1}_{n\text{-mal}}$ ist der $n$-dimensionale Torus.

\textbf{Bemerkung:} $S^1 \subset \mathbb{R}^2$, $T^n \subset \mathbb{R}^{2n}$, aber $T^2 \subset \mathbb{R}^3$. Das das kleinste $m$ mit $T^n \subset \mathbb{R}^m$ ist n+1. Um einen n-dim Torus zu überdecken wird mindestens 2 Karten benötigt(Stichwort: group action, Lens Space). Im Allgemeinen benötigt man $n+1$ Karten, um eine topologische Mannigfaltigkeit abzudecken, wobei $n$ die Dimension der Mannigfaltigkeit ist. Diese Invariante wird als \textit{Lusternik-Schnirelmann-Kategorie} bezeichnet und steht in tiefer Verbindung zur Morsetheorie.\\

Für $S^1$ braucht man mindestens 2 Karten (stereographische Projektion + Nordpol).
Produktkarten können $T^n$ mit $2^n$ Karten überdecken ($T^n=S^1\times\dots\times S^1$). Wie viele braucht man mindestens?
\[
T^n = \mathbb{R}^n / \mathbb{Z}^n \quad \text{mit Quotiententopologie}
\quad x \sim y \Leftrightarrow x - y \in \mathbb{Z}^n
\]

%%%%%%%%%%%%%%%%%%%%%%%%%%%%%%%%%%%%%%%%%%%%%%%%%%%%%%%%%%%%%%%%%%%%%%%% Vorlesung 2 %%%%%%%%%%%%%%%%%%%%%%%%%%%%%%%%%%%%%%%%%%%
\subsection{Differenzierbare Mannigfaltigkeiten}

In der Analysis I,II war Differenzierbarkeit eine gute Eigenschaft von Funktionen, aber es funktioniert nur auf $\mathbb{R}^n$. Nun wollen wir \textbf{differenzierbare Funktionen auf Mannigfaltigkeiten definieren}.(Abb. zwischen Mannigfaltigkeiten definieren). Dafür benötigen wir differenzierbare Strukturen.

Sei $M$ eine topologische Mannigfaltigkeit und $f: M \to \mathbb{R}$ eine stetige Abbildung, $p \in M$, sowie $(U, \phi)$, $(V, \psi)$ Karten um $p$. Dann betrachten wir die Komposition
\[
f\circ \phi^{-1} = (f\circ\psi^{-1})\circ(\psi\circ\phi^{-1}) : \mathbb{R}^n \to \mathbb{R}
\]
(auf $\phi(U \cap V)$). Können über Differenzierbarkeit von $f\circ\phi^{-1}$ bzw. $(f\circ\psi^{-1})\circ(\psi\circ\phi^{-1})$ nachdenken.
 Diese sollte unabhängig von der gewählten Karte sein.\\

\textbf{Definition 1.20 (Verträglichkeit von Karten)}
Zwei Karten $(U, \phi)$ und $(V, \psi)$ heißen \underline{\emph{verträglich}}, falls die Abblidung
\[
\phi \circ \psi^{-1}: \underbrace{\psi(U \cap V)}_{\in \mathbb{R}^n} \to \underbrace{\phi(U \cap V)}_{\in \mathbb{R}^n}
\]
ein \emph{Diffeomorphismus} von offenen Mengen in $\mathbb{R}^n$ ist. Die Abbildung $\phi \circ \psi^{-1}$ heißt \underbar{Kartenwechsel}.

\begin{figure}[H]
  \centering
  \includegraphics[width=13cm]{Image Diffgeo/14.jpg}
\caption{Verträglichkeit zwei Karten}
\end{figure}

Eine Menge von Karten $\mathcal{A} = \{(U_i, \phi_i)\}_{i\in I}$ heißt \emph{Atlas}, wenn die Karten \begin{itemize}
  \item M überdecken
  \item je zwei Karten verträglich sind
\end{itemize}

Ein Atlas heißt \emph{maximal}, wenn jede Karte, die mit allen Karten in $\mathcal{A}$ verträglich ist, bereits in $\mathcal{A}$ enthalten ist.\\

\textbf{Lemma 1.21}
Sei $A$ ein Atlas für $M$, und sei $A_{\max}$ die Menge aller Karten von M, die mit allen Karten aus $A$ verträglich sind. Dann gilt:
\begin{enumerate}
    \item $A \subseteq A_{\max}$
    \item $A_{\max}$ ist ein Atlas
    \item $A_{\max}$ ist ein maximaler Atlas
    \item Jeder Atlas ist in genau einem maximalen Atlas enthalten (Widerspruchsbeweis)
\end{enumerate}
\begin{proof}
  Angenommen, es gibt zwei maximale Atlanten $\mathcal{M}_1$ und $\mathcal{M}_2$, die $\mathcal{A}$ enthalten. Da $\mathcal{M}_1$ maximal ist, muss er alle mit $\mathcal{A}$ kompatiblen Karten enthalten, also $\mathcal{M}_2 \subseteq \mathcal{M}_1$. Analog folgt $\mathcal{M}_1 \subseteq \mathcal{M}_2$. Somit gilt $\mathcal{M}_1 = \mathcal{M}_2$. Ein maximaler Atlas ist also durch die Forderung, \glqq{}alle kompatiblen Karten zu enthalten\grqq{}, eindeutig festgelegt.
\end{proof}

\subsubsection*{Definition 1.22 (Differenzierbare Mannigfaltigkeit)}
Eine \emph{$n$-dimensionale differenzierbare Mannigfaltigkeit} $C^\infty$- oder glatte Mannigfaltigkeit $(M,A)$ ist eine n-dimensionale topologische Mannigfaltigkeit $M$ zusammen mit einem maximalen Atlas (differenzierbare Struktur) $A$.\\

\textbf{Bemerkungen:}
\begin{itemize}
    \item Ein maximaler Atlas definiert eine \underbar{differenzierbare Struktur}.
    \item Es gibt topologische Mannigfaltigkeiten, die keine differenzierbare Struktur zulassen (z.B. Alexander’s horned sphere (Einbettung der 2-Sphäre $S^2$ in $\mathbb{R}^3$) \url{https://en.wikipedia.org/wiki/Alexander_horned_sphere}).
    \item Auf einer topologischen Mannigfaltigkeit kann es unterschiedliche differenzierbare Strukturen geben.
    \item Mit Mannigfaltigkeit meinen wir im Folgeneden differenzierbare Mannigfaltigkeit.
\end{itemize}

\textbf{Beispiel:}
\begin{enumerate}
  \item Jede offene Teilmenge $U \subseteq \mathbb{R}^n$ ist ein n-dim (differenzierbare) Mannigfaltigkeit mit dem Atlas $A = \{(U, \text{id})\}$
  \item Die Sphäre $S^n$ ist eine n-dim (differenzierbare) Mannigfaltigkeit.

  $\{(U_i, g_i)\}$ (i = 1,2) die Karten aus der letzten Vorlesung (stereographische Projektionen).\\

  Zu zeigen: Der Kartenwechsel $g_1\circ g_2^{-1}$ ist differenzierbar.

  Für $x \in g_2(U_1\cap U_2)= g_2(S^n\setminus \{\pm e_{n+1}\})= \mathbb{R}^{n}\setminus \{0\} \subset  \mathbb{R}^{n}$ gilt:
  \[
    g_1 \circ g_2^{-1}(x) = g_1(\frac{2x}{1+\| x\Vert ^2},\frac{1-\norm{x}^2 }{1+\| x\Vert ^2}) =\dots=\frac{x}{\norm{x}^2}\;\; \text{(Spiegelung an $S^{n-1}$)}
  \]
  Damit sind die Karten $(U_i, g_i)$ (i = 1,2) verträglich.\\
  Da sei $S^n$ überdecken erhalten wir den Atlas $A = \{(U_i, g_i), i = 1,2\}$
  \item Der reell-projektive Raum $\mathbb{RP}^n$ ist eine n-dim. (differenzierbare) Mannigfaltigkeit.

  Mit den Karten $(U_i, \psi_i)$ aus der letzten Vorlesung gilt:
  \[
    \psi_j \circ \psi_i^{-1} : \psi_i(U_i \cap U_j) \to \psi_j(U_i \cap U_j)
  \]
  \begin{align*}
    (\psi_j \circ \psi_i^{-1})(y_0, \dots, y_{n-1}) = \psi_j \left( [y_0 : \dots : y_{i-1} : 1 : y_{i+1} : \dots : y_{n-1}] \right)\\
    = \left( \frac{y_0}{y_j}, \dots, \frac{y_{i-1}}{y_j}, 1, \frac{y_{i+1}}{y_j}, \dots,\frac{y_{j-1}}{y_j}, \frac{y_{j+1}}{y_j}, \dots \frac{y_{n-1}}{y_j} \right)
  \end{align*}
  Dies ist ein Diffeomorphismus.
\end{enumerate}
\textbf{Bemerkung} $\mathbb{RP}^n$ ist eine n-dim., kompakte, (weg-)zusammenhängende Mannigfaltigkeit.

Allgemeiner: Sei $V$ ein $(n+1)$-dimensionaler Vektorraum über einem (Schief-) körper $K$, dann definieren wir den projektiven Raum
\[
  KP^n := \text{Menge aller 1-dim. Unterräume von } V
\]
M.a.W: $KP^n$ ist der Quotient von $V\setminus \{0\}$\[
  KP^n = V \setminus \{0\} / \sim \quad \text{wobei } v \sim w \Leftrightarrow v,w \text{ linear abhängig}
\]
\[
  K = \mathbb{C} \Rightarrow \text{komplex projektiver Raum}\quad K = \mathbb{H} \Rightarrow \text{quaternionischprojektiver Raum}
\]

\subsection{Differenzierbare Abbildungen}

\textbf{Definition 1.30:} Seien $M, N$ zwei differenzierbare Mannigfaltigkeiten. Eine stetige Abbildung $f: M \to N$ heißt \emph{differenzierbar}, wenn es um jeden Punkt $p \in M$ Karten $(U,\phi)$ und um $f(p)\in N\; (V,\psi)$ gibt, so dass auf einer Umgebung $W \subseteq \phi(U \cap f^{-1}(V))$ von $\phi(p)$ die Abbildung
\[
  \psi\circ f\circ\phi^{-1}:W \subset\underbrace{\phi(f^{-1}(V)\cap U)}_{\in \mathbb{R}^m} \to \underbrace{\psi(V)}_{\in \mathbb{R}^n}
\]
differenzierbar ist.\\
\begin{figure}[H]
  \centering
  \includegraphics[width=14cm]{Image Diffgeo/9.jpg}
%\caption{stereographische Projektion n=2}
\end{figure}

Ein \emph{Diffeomorphismus} ist ein Homöomorphismus $f$, für den $f$ und $f^{-1}$ differenzierbar sind. In diesem Fall nennen wir die Mannigfaltigkeiten $M$ und $N$ \emph{diffeomorph} (schreibe $M \cong N$).\\

\textbf{Bemerkung}
\begin{itemize}
  \item[i)] Die Definition ist unabhängig von der Wahl der Karten. D.h.: Sei $f$ differenzierbar bzgl. der Karten $(U,\phi)$ von M und $(V,\psi)$ von N; $p\in U$,$f(p)\in V$,
  und seien $(U', \phi')$, $(V', \psi')$ verträgliche Karten von M bzw. N; $p\in U'$,$f(p)\in V'$, so gilt nach Einschränkung:
  \begin{align*}
    \psi' \circ f \circ (\phi')^{-1} = (\psi' \circ \underbrace{\psi^{-1}) \circ (\psi}_{id_V} \circ f \circ \underbrace{\phi^{-1}) \circ (\phi}_{id_U} \circ (\phi')^{-1})\\=\underbrace{(\psi'\circ\psi^{-1})}_{\txt{diff'bar}}\circ\underbrace{(\psi\circ f\circ\phi^{-1})}_{\txt{diff'bar nach Annahme}}\circ\underbrace{(\phi\circ\phi')}_{\txt{diff'bar}}
  \end{align*}
  \begin{figure}[H]
    \centering
    \includegraphics[width=9cm]{Image Diffgeo/10.jpg}
  %\caption{stereographische Projektion n=2}
  \end{figure}


  \item[ii)] Den unendlich-dimensionalen Vektorraum der differenzierbaren Funktionen auf $M$ mit Werten in $\mathbb{R}$ bezeichnen wir mit $C^\infty(M)=C^\infty(M, \mathbb{R})$.

  \item[iii)] Analog zu Def.\ 1.30 kann man Differenzierbarkeit in einem Punkt definieren.\\
  Für Differenzierbarkeit auf ganz $M$ genügt es, die Existenz von Karten $(U, \phi)$ und $(V, \psi)$ mit $f(U) \subseteq V$ zu fordern, für die $\psi \circ f \circ \phi^{-1}$ differenzierbar auf $\phi(U)$ ist.
  \textcolor{blue}{(Die $(U, \phi)$ müssen $M$ überdecken.)}

  \item[iv)] Analog zu Def.\ 1.15 sagen wir $f$ ist $k$-fach differenzierbar, falls $\psi \circ f \circ \phi^{-1}$ $k$-fach differenzierbar ist.\\
  Im Weiteren meinen wir mit \emph{differenzierbar} immer „beliebig oft differenzierbar“.\\
  Alternative Sprechweise: „glatt“ oder $C^\infty$.

\end{itemize}
\textbf{Beispiele:}
\begin{enumerate}
  \item Die antipodale Abbildung 
  \[
    f : S^n \to S^n,\quad y \mapsto -y
  \]
  ist differenzierbar.

  \item Sei $M = \mathbb{R} = N$ mit den Atlanten
  \[
    \mathcal{A}_1 = \{ \phi = \text{id} : \mathbb{R} \to \mathbb{R} \},\quad
    \mathcal{A}_2 = \{ \gamma : \mathbb{R} \to \mathbb{R},\; \gamma(t) = t^3 \}
  \]

  $\phi$ und $\gamma$ sind keine verträglichen Karten, denn
  \[
    \phi \circ \gamma^{-1}(t) = \sqrt[3]{t}
  \]
  ist nicht differenzierbar.
  \[
    \Rightarrow\ \text{id} : (\mathbb{R}, \mathcal{A}_1^{\max}) \to (\mathbb{R}, \mathcal{A}_2^{\max})\ \text{ist differenzierbar} \;(\psi \;\circ id \;\circ\phi^{-1}\;\;t\mapsto t^3)
  \]
  aber kein Diffeomorphismus. Denn
  \[
    \text{id} : (\mathbb{R}, \mathcal{A}_2^{\max}) \to (\mathbb{R}, \mathcal{A}_1^{\max})
  \]
  ist nicht differenzierbar $(\phi \;\circ id \;\circ\psi^{-1}\;\;t\mapsto \sqrt[3]{t})$.

  Allerdings ist die Abbildung $f(t) = t^3$ ein Diffeomorphismus ($\psi\circ f\circ\phi^{-1}\;\;t\mapsto t$)
  \[
    f : (\mathbb{R}, \mathcal{A}_1^{\max}) \to (\mathbb{R}, \mathcal{A}_2^{\max}).
  \]
 
   \item $\mathbb{RP}^1 \cong S^1,\quad \mathbb{CP}^1 \cong S^2,\quad \mathbb{HP}^1 \cong S^4\qquad$ $\cong$ diffeomorph\\
  Später: Für $n \geqslant 2$ ist $\mathbb{RP}^n$ nicht diffeomorph zu $S^n$. \quad (Fundamentalgruppe)
  
  \item Die Abbildungen
  \[
    S^n \to \mathbb{RP}^n,\quad S^{2n+1} \to \mathbb{CP}^n,\quad S^{4n+3} \to \mathbb{HP}^n
  \]
  \[
    \pi(z_0, \dots, z_n) = [z_0 : \dots : z_n]
  \]
  sind differenzierbar und heißen \emph{Hopf-Faserungen}.
  
  \[
    \pi^{-1}(p) \cong 
    \begin{cases}
      \mathbb{Z}_2 & \text{für } \mathbb{RP}^n \quad \text{(zwei Punkte)} \\
      S^1 & \text{für } \mathbb{CP}^n \\
      S^3 & \text{für } \mathbb{HP}^n \quad \text{(auch für $n = 1$, $S^3 \to S^2$)}
    \end{cases}
  \quad \text{\color{blue}sog. Fasern.}
  \]
\end{enumerate}
\begin{figure}[H]
  \centering
  \includegraphics[width=15cm]{Image Diffgeo/11.jpg}
\caption{$\mathbb{RP}^1$ diffeomorph zu $S^1$}
\end{figure}
\begin{figure}[H]
  \centering
  \includegraphics[width=15cm]{Image Diffgeo/12.jpg}
\caption{Innerhalb $\mathbb{RP}^2$ ist Mobius-Band enthalten}
\end{figure}


\textbf{Bemerkung}
\begin{enumerate}
  \item Für $n \neq 4$ ist jede differenzierbare Struktur auf $\mathbb{R}^n$ diffeomorph zur Standardstruktur $\mathcal{A}_{\text{max}}$ zu $\mathcal{A} = \{(\mathbb{R}^n, \text{id})\}$

  \item Auf $\mathbb{R}^4$ gibt es überabzählbar viele differenzierbare Strukturen, die paarweise nicht diffeomorph sind. („Exotische Strukturen!“)

  \item Jede topologische Mannigfaltigkeit in Dimension 1, 2 und 3 besitzt genau eine differenzierbare Struktur. (eindeutige differenzierbare Struktur)

  \item Es gibt topologische Mannigfaltigkeiten in Dimension 4, die keine differenzierbare Struktur zulassen.

  \item Für $n \geqslant  7$ gibt es Mannigfaltigkeiten, die homöomorph zu $S^n$ aber nicht diffeomorph sind.\\
  In jeder Dimension $n$, $n > 7$, gibt es höchstens endlich viele exotische Sphären.

  Die Existenz einer exotischen $S^4$ ist unklar.
\end{enumerate}


\subsection{Der Satz vom regulären Wert}

Nachrechnen, dass $M$ durch eine Menge von Karten $A$ zu einer Mannigfaltigkeit wird, ist aufwendig. Deshalb wollen wir einen anderen Weg finden.

Sei $f : M \to N$ eine differenzierbare Abbildung.\\

\textbf{Definition 1.40:} Seien $(U, \phi)$ bzw.\ $(V, \psi)$ Karten um $p$ bzw.\ $f(p)$.\\
Dann ist der \underbar{Rang} von $f$ in $p$ definiert als der Rang der Jacobi-Matrix von $\, \psi \circ f \circ \phi^{-1}$, d.h.
\[
\operatorname{rg}_p(f) := \operatorname{Rang} \left( J_p(\psi \circ f \circ \phi^{-1}) \right) = \txt{Rang}(\deldel{(\psi \circ f \circ \phi^{-1})}{x_j})
= \operatorname{Rang} J_{\psi(f(p))} \left( \psi \circ f \circ \phi^{-1} \right)
\]
wobei $J_p(f)$ die Jacobi-Matrix einer Abbildung $f$ im Punkt $p$ bezeichnet.\\


\textbf{Bemerkung:} Der Rang der Jacobi-Matrix $J_{\psi(f(p))}(\psi \circ f \circ \phi^{-1})$ hängt nicht von den gewählten Karten $(U,\phi)$ und $(V,\psi)$ ab, d.h. der Rang $\operatorname{rg}_p(f)$ ist wohldefiniert.
\begin{proof}
  Seien $(U, \phi)$ und $(\tilde{U}, \tilde{\phi})$ Karten um $p$, $(V, \psi)$ und $(\tilde{V}, \tilde{\psi})$ Karten um $f(p)$. Dann gilt:
  \[
  \tilde{\psi} \circ f \circ \tilde{\phi}^{-1} = (\tilde{\psi}  \circ\psi^{-1}) \circ (\psi\circ f\circ\phi^{-1}) \circ   (\phi\circ\tilde{\phi}^{-1})
  \]
  Die Kartenwechsel $\phi^{-1}\circ\tilde{\phi}$ und $\tilde{\psi}\circ\psi^{-1}$ sind Diffeos und ihre Jacobi-Matrizen Isomorphismen, die den Rang der linearen Abbildung nicht ändern.\\
  \textbf{\textit{Kettenregel}:}
 \[
 \operatorname{rg} \left[ J \left( \tilde{\psi} \circ f \circ   \tilde{\phi}^{-1} \right) \right]
 = \operatorname{rg} \left[ J(\tilde{\psi}\circ\psi^{-1}) \circ J(\psi\circ f\circ\phi^{-1}) \circ J(\phi\circ\tilde{\phi}^{-1}) \right] = \operatorname{rg} J(\psi\circ f\circ\phi^{-1}) \quad \Box
 \]
\end{proof}


\textbf{Bemerkung:} Die Abbildung $p \mapsto \operatorname{rg}_p(f)$ ist unterhalbstetig, d.h. hat $f$ in $p$ den Rang $r$, dann gilt $\operatorname{rg}_q(f) \geq r$ für alle $q$ in einer kleinen Umgebung von $p$.\\

\textbf{Rangsatz}
Sei $f : M \to N$ eine differenzierbare Abbildung (nicht zwingend Diffemorphismus), die in einer Umgebung von $p \in M$ (es gibt eine Menge $V$ mit $p\in V$ und ein $U\subseteq M$ offen mit $p\in U\subseteq V$) konstanten Rang $r$ hat.\\
Dann ist $f$ bzgl. \textit{\underbar{geeigneter lokaler Koordinaten um $p$}} von der Form
\[
\mathbb{R}^r \times \mathbb{R}^s \longrightarrow \mathbb{R}^r \times \mathbb{R}^t, \quad (x,y) \mapsto (x, 0, \dots, 0)
\]
wobei $\dim M = r + s$ und $\dim N = r + t$.

D.h.: Es gibt Karten $(U,\phi)$ um $p$ und $(V,\psi)$ um $f(p)$, so dass
\[
(\psi \circ f \circ \phi^{-1})(x, y) = (x,0,\dots,0)
\]
$f$ Diffeomorphismus (voller Rang) $\Rightarrow J_p(f)$ invertierbar.

\subsubsection*{Umkehrsatz}
Sei $f : M \to N$ eine differenzierbare Abbildung zwischen zwei Mannigfaltigkeiten der Dimension $n$. Sei $p \in M$ ein Punkt mit $\operatorname{rg}_p(f) = n$. Dann existiert eine Umgebung von $p$, auf der $f$ ein Diffeomorphismus ist.\\

\underline{Beweisskizze}:\\
Ist der Rang von $f$ in $p$ gleich $n$, dann ist die Jacobi-Matrix von $\psi \circ f \circ \phi^{-1}$ (in $p$) invertierbar und damit ist $\psi \circ f \circ \phi^{-1}$ ein lokaler Diffeomorphismus um $\phi(p)$.

Bei Def.\ ist $f$ also ein lokaler Diffeomorphismus zwischen $M$ und $N$.

\vspace{0.5em}
\textcolor{blue}{\small(Spezialfall des Rangsatzes)}\\

\textbf{Definition 1.41:} Sei $f : M \to N$ eine differenzierbare Abbildung. Dann heißt ein Punkt $p \in M$ \emph{regulär}, falls
\[
\operatorname{rg}_p(f) = \dim N
\]
d.h. falls $J(\psi\circ f\circ\phi^{-1})$ surjektiv ist.\\
Andernfalls heißt ein Punkt \emph{kritisch} oder \emph{singulär}.

\subsubsection*{Bemerkung:}
\begin{itemize}
  \item[i)] Ist $p$ regulär, dann folgt $\dim M \geq \dim N$
  \item[ii)] Ist $\dim M < \dim N$, dann sind alle Punkte in $M$ singulär.
\end{itemize}

\subsubsection*{Satz vom regulären Punkt}
Sei $f : M \to N$ eine differenzierbare Abbildung und $p \in M$ regulär.\\
Dann existieren Karten $(U, \phi)$ um $p$ und $(V, \psi)$ um $f(p)$ mit $f(U) \subseteq V$ und
\[
(\psi \circ f \circ \phi^{-1})(x_1, \dots, x_{r+s}) = (x_1, \dots, x_r)
\]
wobei $\dim M = r + s$ und $\dim N = r$, d.h. \underbar{in lokalen Koordinaten stimmt $f$ mit} \underbar{der Projektion $\mathbb{R}^{r+s} \to \mathbb{R}^r$ überein.}\\


\textbf{Definition 1.16:} Sei $f : M \to N$ eine differenzierbare Abbildung. Ein Punkt $q \in N$ heißt \emph{regulärer Wert} von $f$, falls jedes $p \in f^{-1}(q)$ ein regulärer Punkt von f ist.

\subsubsection*{Bemerkung:}
\begin{itemize}
  \item[i)] Jeder Punkt, der nicht im Bild von $f : M \to N$ liegt, ist ein regulärer Wert.

  \item[ii)] \textbf{Satz von Sard:} Die Menge der regulären Werte hat Lebesgue-Maß 0 in $N$, ist dicht in $N$ und speziell kann eine differenzierbare Abbildung $f:M\to N $ für $\txt{dim}(M)<\txt{dim}(N)$ nicht surjektiv sein.\\
  (Die Menge der kritischen Punkte ist eine Lebesgue-Nullmenge: Sei f diff'bar $K:=\{f(x)|f'(x)=0\}\;\;\mathcal{L} (K)=0$)

  \item[iii)] Es gibt keine surjektive differenzierbare Abbildung $f : \mathbb{R} \to \mathbb{R}^2$ (hätte nur singuläre Werte laut Bem (ii), insbe. liegen reguläre Werte nicht dicht in $\mathbb{R}^2$),\\
  jedoch stetige surjektive Abbildungen existieren (Fass-kurven stetig aber nicht bijektiv\url{https://de.wikipedia.org/wiki/Peano-Kurve}).
\end{itemize}


\textbf{Definition 1.17:} Eine Teilmenge $M_0 \subseteq M$, einer $n$-dimensionalen Mannigfaltigkeit $M$ nennt man \emph{$k$-dimensionale Untermannigfaltigkeit}, wenn es um jeden Punkt von $M_0$ eine Karte $(U, \phi)$ von $M$ gibt mit
\[
\phi(U \cap M_0) = \mathbb{R}^k\cap\phi(U)
\]

wobei $\mathbb{R}^k = \{(x_1, \dots, x_k, 0, \dots, 0)\} \subseteq \mathbb{R}^n$. Die Differenz $\dim M - \dim M_0$ nennt die \emph{Kodimension} von $M_0$ in $M$.

(Vgl.\ Analysis II mit $M = \mathbb{R}^n$.)
\begin{figure}[H]
  \centering
  \includegraphics[width=13cm]{Image Diffgeo/13.jpg}
\caption{Untermannigfaltigkeit $M_0$ von $M:=S^2$}
\end{figure}

\subsubsection*{Bemerkung:}
\begin{itemize}
  \item[i)] Jede Untermannigfaltigkeit ist wieder eine Mannigfaltigkeit (mit der Teilraumtopologie von $M$).
  
  \begin{itemize}
    \item Untermannigfaltigkeiten der Kodimension 0 sind genau die offenen Teilmengen in $M$
    \item Untermannigfaltigkeiten der Dimension 0 sind genau die diskreten Teilmengen in $M$, also diskrete Punkte die keine Häufung haben. (endlich, falls $M$ kompakt)
  \end{itemize}

  \item[ii)] Eine Teilmenge $M_0 \subseteq \mathbb{R}^n$ ist genau dann eine Untermannigfaltigkeit der Dimension $k$ von $\mathbb{R}^n$, wenn $M_0$ lokal diffeomorph zu einer offenen Teilen von $\mathbb{R}^k$ ist, z.\,B. reguläre Fläche in $\mathbb{R}^3$ (Def. in Analysis II)

  \item[iii)] Einschränkungen differenzierbarer Abbildungen auf Untermannigfaltigkeiten sind wieder differenzierbar.
\end{itemize}

\subsubsection*{Theorem (Einbettungs-Satz von Whitney):}
Jede $n$-dimensionale Mannigfaltigkeit ist diffeomorph zu einer Untermannigfaltigkeit von $\mathbb{R}^{2n}$.

Das Resultat kann nicht verbessert werden. Denn man kann zeigen: $\mathbb{RP}^2$ ist keine Untermannigfaltigkeit von $\mathbb{R}^3$ und Möbius-Band ist auch keine Untermannigfaltigkeit von $\mathbb{R}^3$.

\vspace{1em}
\subsubsection*{Satz vom regulären Wert:}
Ist $q \in N$ ein regulärer Wert einer differenzierbaren Abbildung $f : M \to N$,\\
dann ist $f^{-1}(\{q\})$ eine Untermannigfaltigkeit von $M$ und es gilt:
\[
\dim f^{-1}(\{q\}) = \dim M - \dim N
\]
%%%%%%%%%%%%%%%%%%%%%%%%%%%%%%%%%%%%%%%%%%%%%%%%%%%%%%%%%%%%%%%%%%%%%%%%%%%%%%%%%%% Vorlesung3 %%%%%%%%%%%%%%%%%%%%%%%%%%%%%%%
\begin{itemize}
  \item[*$_1$:] Jedes $p \in f^{-1}(\{q\})$ ist ein regulärer Punkt, d.h. $\forall p \in f^{-1}(\{q\})$ ist $J_{\varphi(p)}(\psi \circ f \circ \varphi^{-1})$ surjektiv.
  
  \item[*$_2$:] Für alle Karten $(U, \varphi)$ von $M$ und $(V, \psi)$ von $N$ ist $\psi \circ f \circ \varphi^{-1}: \mathbb{R}^m \to \mathbb{R}^n$ differenzierbar (dort wo definiert).
  
  \item[*$_3$:] Für alle $p \in f^{-1}(\{q\})$ gibt es eine Karte $(U, \phi)$ von $M$ um $p$ mit
  \[
  \phi \left(U \cap f^{-1}(\{q\}) \right) = \mathbb{R}^k \times \{0\}\subseteq \mathbb{R}^m \quad k = \dim M - \dim N \quad \txt{(Definition von Untermannigfaltigkeit)}
  \]
\end{itemize}
Der Beweis verwendet den \underline{Satz vom regulären Punkt}:

Sei $f: M \to N$ differenzierbar, $p \in M$ regulär. Dann existieren Karten $(U, \phi)$ um $p$ und $(V, \psi)$ um $f(p)$ mit $f(U) \subseteq V$ und
\[
(\psi \circ f \circ \phi^{-1})(x_1,\dots, x_{n+k}) = (x_1,\dots,x_n)
\]
wobei $\dim M = m\quad \dim N = n$.

\vspace{0.5cm}
\hrule
\vspace{0.5cm}
\begin{proof}
  (Beweis vom „Satz vom regulären Wert")\\
  Sei $\dim M = m$, $\dim N = n$ und $k := m - n \geq 0$, sonst gibt es keinen regulären Punkt. Sei $p_0 \in f^{-1}(\{q\})$ ein regulärer Punkt.

  \textit{Satz vom regulären Punkt:} Wir können Karten $(U, \varphi)$ um $p_0$ und $(V, \psi)$ um $q = f(p_0)$ wählen mit
  \[
  (\psi \circ f \circ \varphi^{-1})(x_1,\dots,x_m) = (x_1,\dots,x_n) \quad \textcolor{blue}{\text{(lokale Darstellung von }f\text{)}}
  \]
  Nach Umbenennen der Koordinaten: (damit $\times \{0\}$ rauskommt)
  \[
  (\psi \circ f \circ \varphi^{-1})(x_1, \dots, x_m) = (x_{k+1} \dots, x_m)\qquad \textcolor{blue}{(m = k + (m - k) = k + n)}
  \]
  O.B.d.A.: \quad \(\psi(q) = \psi(f(p_0)) = 0\). Dann gilt: 
  \[
  p \in U \cap f^{-1}(\{q\}) 
  \iff 
  p \in U\; \text{und} \; f(p)=q 
  \]
  \[
  \iff p \in U \;\txt{und}\;\psi\circ f(p)=(\psi\circ f\circ\varphi^{-1})\circ\varphi(p)=0\in \mathbb{R}^n
  \]
  \[
  \iff \varphi(p)_i=0\quad \forall \;i=k+1,\dots,m
  \]

  laut \(\psi\circ f \circ \varphi^{-1} (x_1, \dots, x_m) = (x_{k+1}, \dots, x_m)\)\\
  Zusammenfassend: 
  \[
  p \in U \cap f^{-1}(\{q\}) \iff \varphi(p) \in \mathbb{R}^k\times \{0\}
  \]
  Also die Untermannigfaltigkeit-Gleichung:
  \[
  \varphi(U \cap f^{-1}(\{q\})) \subseteq \mathbb{R}^k\times \{0\}
  \]
  Die Untermannigfaltigkeit \(f^{-1}(\{q\})\) hat die Dimension $k=m-n=\text{dim}(M)-\txt{dim}(N)$
\end{proof}

\textbf{\underline{Anwendung / Beispiel:}}
\begin{enumerate}
  \item[i)] Die $n$-dimensionale Sphäre
  \[
  \mathcal{M} = \mathbb{R}^{n+1}, \quad \mathcal{N} = \mathbb{R}, \quad f : \mathbb{R}^{n+1} \to \mathbb{R}, \quad x \mapsto \norm{x}^2
  \]
  Jacobi-Matrix von \( f \) im Punkt \( x = (x_1, \dots, x_{n+1}) \): \quad \( J(f) = (2x_1, \dots, 2x_{n+1}) \)

  Für alle \( x \neq 0 \) hat \( J(f) \) vollen Rang \( (=1) \), d.h. 1 ist ein regulärer Wert von \( f \)
  \[
  \Rightarrow f^{-1}(\{1\}) = S^n \text{ ist eine Untermannigfaltigkeit von } \mathbb{R}^{n+1}
  \]
  \item[ii)] Die orthogonale Gruppe \(O(n) = \{ A \in M(n,\mathbb{R}) \mid A^TA=\mathbb{I} \}\)
  \[
  \mathcal{M} = M(n,\mathbb{R}) = \mathbb{R}^{n^2}, \quad \mathcal{N} = \mathrm{Sym}(n,\mathbb{R}) = \{ A \in M(n,\mathbb{R}) \mid A^T=A\} = \mathbb{R}^{\frac{1}{2}n(n+1)}
  \]
\end{enumerate}
\textbf{Lemma 1.18:} Die Einheitsmatrix \(\mathbb{I}\) ist ein regulärer Wert der Abbildung
\[
f : M(n,\mathbb{R}) \to \mathrm{Sym}(n,\mathbb{R}), \quad A \mapsto A^T A
\]

Daher ist die orthogonale Gruppe \( O(n) = f^{-1}(\mathbb{I}) \) eine Untermannigfaltigkeit der Dimension $\frac{1}{2}n(n-1)$
\begin{proof}
  Laut Satz vom regulären Wert folgt die zweite Behauptung aus der ersten.

  Sei \( p \in f^{-1}(\mathbb{I}) = O(n) \). Zu zeigen: \( J(f)_p \) ist surjektiv.

  Berechne die Jacobi-Matrix mittels der Richtungsableitung, \( v \in \mathbb{R}^{n^2} \):
  \[
  J(f)_p v = \left. \frac{d}{dt} \right|_{t=0} f(p + tv) = \left. \frac{d}{dt} \right|_{t=0} \left( p+tv \right)^T \left( p+tv \right) = \left. \frac{d}{dt} \right|_{t=0} (p^Tp+tp^Tv+tv^Tp+t^2v^Tv)
  \]
  \[
  = p^Tv+v^Tp
  \]
  Für \( B \in \mathrm{Sym}(n, \mathbb{R}) \) und \( p \in O(n) = f^{-1}(\{\mathbb{I}\}) \), sei \( v = \frac{1}{2}pB\)
  \[
  \Rightarrow p^Tv+v^Tp=\frac{1}{2}p^T(pB)+\frac{1}{2}(pB)^Tp=\frac{1}{2}B+\frac{1}{2}B^T=B \qquad (p^Tp=\mathbb{I})
  \]
  D.h. die Jacobi-Matrix der oben definierten Abbildung \( f \) ist surjektiv in allen Punkten von \( f^{-1}(\{\mathbb{I}\}) \).

  \[
  \Rightarrow \mathbb{I} \;\text{ist ein regulärer Wert von } f \Rightarrow \text{ Beh.}
  \]
\end{proof}
\textbf{Bemerkung:}
\begin{enumerate}
  \item \( \mathrm{SO}(n) \), die speziell orthogonale Gruppe, ist eine offene Teilmenge von \( O(n) \), daher eine Untermannigfaltigkeit der Kodimension 0.
  \item \( O(n) \) hat genau zwei Zusammenhangskomponenten (kann in zwei zusammenhängende disjunkte Teilmenge zerlegen $O(n)=SO(n)\bigoplus SO(n)^{\bot}$)
  \item Die Matrixgruppen \( \mathrm{GL}(n), \mathrm{U}(n), \mathrm{SU}(n), \mathrm{Sp}(n) \) sind Mannigfaltigkeiten $O(n)\times O(n)\to O(n) \quad (A,B)\mapsto AB$.\\
  Die Gruppenoperation (Multiplikation) ist jeweils eine differenzierbare Abbildung.
\end{enumerate}
\textcolor{red}{\textbf{Später: Solche Mannigfaltigkeiten heißen \underline{Lie-Gruppen}}}

\textcolor{black}{
\begin{align*}
\mathrm{GL}(n) &= \{ \text{invertierbare Matrizen} \} \\
\mathrm{U}(n) &= \{ A \in \mathcal{M}(n, \mathbb{C}) \mid (A^*)^T = A^{-1} \} \\
\mathrm{SU}(n) &= \{ A \in \mathcal{M}(n, \mathbb{C}) \mid \det(A) = 1 \} \\
\mathrm{Sp}(n) &= \left\{ A \in \mathcal{M}(2n, \mathbb{R}) \,\middle|\, A^T 
\begin{pmatrix}
0 & \mathbb{1}_n \\
-\mathbb{1}_n & 0
\end{pmatrix}
A =
\begin{pmatrix}
0 & \mathbb{1}_n \\
-\mathbb{1}_n & 0
\end{pmatrix}
\right\}
\end{align*}
}
\subsection{Immersion, Submersion, Einbettung}

\textcolor{black}{\textbf{Definition 1.19:}} Sei \( f: M^m \to N^n \) eine differenzierbare Abbildung. 

\begin{itemize}
  \item Die Abbildung \( f \) heißt \underline{Submersion}, wenn $m\geqslant n$
  \[
  \txt{Rang}J_p(f) = \txt{dim} (N) \quad \forall p \in M 
  \qquad \textcolor{blue}{\text{(Rang der Jacobi-Matrix)}}
  \]

  \item Die Abbildung \( f \) heißt \underline{Immersion}, wenn $m\leqslant n$
  \[
  \txt{Rang}J_p(f) = \txt{dim}(M) \quad \forall p \in M
  \]

  \item Die Abbildung \( f \) heißt \underline{Einbettung}, wenn \(f(M) \subset N\) eine Untermannigfaltigkeit und $f:M\to f(M)$ ein Diffeomorphismus auf seine Bildmenge ist.\\
(Intuition: Eine Einbettung ist eine „saubere“ Einlagerung von $M$ in $N$, ohne Selbstüberschneidung oder komische Topologie. )
\end{itemize}
\subsubsection*{Lemma 1.20}
Die Abbildung \( f: M^m \to N^n \) ist genau dann eine \textit{Submersion}, wenn die Jacobi-Matrix \(J(\psi \circ f \circ \varphi^{-1})\)
in allen Punkten \( p \in M \) bezüglich Karten \( (U,\varphi) \) um \( p \) und \( (V,\psi) \) um \( f(p) \) surjektiv ist.

Weiter ist \( f \) eine Submersion genau dann, wenn \( f \) in lokalen Koordinaten von der Form
\[
(x_1, \dots, x_m) \mapsto (x_1,\dots,x_n)
\]
ist.

\subsubsection*{Lemma 1.21}
Die Abbildung \( f: M^m \to N^n \) ist genau dann eine \textit{Immersion}, wenn die Jacobi-Matrix \(J(\psi \circ f \circ \varphi^{-1})\) 
in allen Punkten \( p \in M \) bezüglich Karten \( (U,\varphi) \) um \( p \) und \( (V,\psi) \) um \( f(p) \) injektiv ist.

Weiter ist \( f \) eine Immersion genau dann, wenn \( f \) in lokalen Koordinaten von der Form
\[
(x_1, \dots, x_m) \mapsto (x_1,\dots,x_m,0,\dots,0)
\]
ist.
\subsubsection*{Bemerkung}
\begin{itemize}
\item[i)] Immersionen sind im Allgemeinen nicht injektiv, und injektive Immersionen sind im Allgemeinen keine Einbettungen.\\
\textbf{Beispiel:}
\begin{itemize}
  \item \( f(t) = (t^2-1,t^3-t ) \) ist eine Immersion, da \( f'(t) \neq 0 \,\,   \forall t \), aber nicht injektiv, da \( f(1) = f(-1) \). 

  \item \( g(t) = (\cos(t),\sin(2t)) \), \quad \( t \in \left( -\frac{\pi}{2},   \frac{3\pi}{2} \right) \), ist injektive Immersion, aber keine Einbettung. (Am Rand $-\pi/2$ und $3\pi/2$ ist g gegen 0, und $g(\pi/2)=0$: Umkehrfunktion $g^{-1}$ nicht stetig!)\\
  \textit{Skizze:} (Acht-förmige Kurve, zeigt Selbstüberschneidung)

  Es gilt:
  \[
  g\left( \frac{3\pi}{2} - \frac{1}{n} \right) \to (0 , 0) = g(\frac{\pi}{2}), \quad   \text{aber} \quad \frac{3\pi}{2} - \frac{1}{n} \not\to 0
  \]
  \[
  \Rightarrow \quad g^{-1}: \operatorname{Bild}(g) \to \left(-\frac{\pi}  {2}, \frac{3\pi}{2}\right) \quad \text{ist nicht stetig.}
  \]
  \begin{figure}[H]
    \centering
    \includegraphics[width=13cm]{Image Diffgeo/15.jpg}
  \caption{Achtförmige Schleife (rechts) und nicht-injektive Immersion (links)}
  \end{figure}
\end{itemize}
\end{itemize}
\begin{itemize}
\item[ii)] Einbettungen sind Immersion, die Homöomorphismen auf ihr Bild sind. (D.h. die Differenzierbarkeit der Immersion folgt aus den anderen Eigenschaften.)\\
(Beispiel: Der Einheitskreis $S^1\subset \mathbb{R}^2$ mit Einbettung $f(\theta)=(\cos(\theta), \sin(\theta))$)

\item[iii)] Sei \( M \) kompakt und \( f: M \to N \) eine injektive Immersion, dann ist \( f \) eine Einbettung. 
\textit{(M kompakt, \( f(M) \) hausdorff, \( f: M \to f(M) \) bijektiv \( \Rightarrow \) \( f: M \to f(M) \) Homöomorphismus)}\\
(Problem für achtförmige Schleife: Definitionsbereich nicht kompakt)

\item[iv)] Eine Abbildung \( f \) ist genau dann ein \textbf{lokaler Diffeomorphismus}, wenn f Immersion und Submersion ist (\( f \) und \( f^{-1} \) lokal differenzierbar sind).\\
\textit{(Ana II: Satz von der lokalen Umkehrabbildung:}  \( \psi \circ f \circ \varphi^{-1} \) hat lokale Inverse \( \Rightarrow \varphi\circ (\varphi^{-1} \circ f^{-1} \circ \psi)\circ \psi \) ist lokale Inverse zu \( f \)).

\item[v)] Diffeomorphismen sind bijektive lokale Diffeomorphismen.

\item[vi)] Submersionen sind im Allgemeinen nicht surjektiv.\\
\textit{Beispiel:} \( \emptyset \neq U \subsetneq \mathbb{R}^n \) offen, \( U \to \mathbb{R}^n \), Inklusion ist Submersion, aber nicht surjektiv.
\end{itemize}
\begin{itemize}
\item[vii)] Submersionen sind offene Abbildungen, d.h. ist \( f: M \to N \) eine Submersion und \( U \subseteq M \) offen, dann ist \( f(U)\subseteq N \) offen.

\item[viii)] Sei \( f: M \to N \) eine Submersion, \( M \) kompakt und \( N \) zusammenhängend,\\
\hspace*{1.5em} dann ist \( f \) surjektiv.\\
\textit{(Tipp: \( f(M) \) ist offen (aus vii)), abgeschlossen auch nicht-leer.)}\\
$M$ kompakt $\implies$ $f(M)\subset N$ kompakt $\overset{N zsh. hausdorff}{\implies}$ $f(M)$ abgeschlossen
\end{itemize}
\subsection{Immersierte Untermannigfaltigkeiten}

In Definition 1.7 haben wir Untermannigfaltigkeiten eingeführt, genauer gesagt haben wir eingebettete Untermannigfaltigkeiten definiert. Man kann auch immersierte Untermannigfaltigkeiten betrachten.\\

\underline{\textbf{Über eingebettete Untermannigfaltigkeiten}}:\\

\( N \subseteq \mathbb{R}^m \) eine \( n \)-dimensionale eingebettete Untermannigfaltigkeit, wenn es zu jedem Punkt \( p \in N \) eine Karte \( (U, \varphi) \) von \( M \) um \( p \) existiert, so dass \( \varphi(U \cap N) = \mathbb{R}^n \cap \varphi(U) \) gilt.

In diesem Fall ist \( N \) mit der \textit{\underline{Teilraum-Topologie}} eine topologische Mannigfaltigkeit und es existiert eine \underline{eindeutige } bestimmte differenzierbare Struktur auf \( N \), so dass die Inklusionsabbildung \( \tau : N \hookrightarrow M \) eine Einbettung ist.\\

\textit{Erinnerung:} Einbettung \( \Longleftrightarrow \) injektive Immersion, die Homöomorphismus auf Bild ist.

\textit{Zusammenfassend:} Bilder von Einbettungen sind \textbf{eingebettete} Untermannigfaltigkeiten (und umgekehrt).\\

Eine Teilmenge \( N \subseteq M \) heißt \underline{immersierte Untermannigfaltigkeit}, falls \( N \) eine differenzierbare Mannigfaltigkeit ist (die differenzierbare Struktur darf unabhängig von der auf \( M \) sein!) und die Inklusionsabbildung \( \tau : N \hookrightarrow M \) eine \textbf{Immersion} ist.\\

Beispiel: Die 8-Kurve ist eine immersierte aber \underline{keine} eingebettete Mannigfaltigkeit (da $f(t)=(\cos(t),\sin(2t))$ Immersion aber keine Einbettung).

\textbf{Konstruktion:} \( N \) eine Untermannigfaltigkeit und \( F : N \hookrightarrow M \) eine injektive Immersion. Dann existiert auf \( F(N) \subseteq M \) eine eindeutig bestimmte Topologie und differenzierbare Struktur, für die \( F : N \to F(N) \) ein Diffeomorphismus ist (i.A. nicht die  Teilraumtopologie),\\

(Neue Topologie): \( U \subseteq F(N)   \) offen \( \Leftrightarrow F^{-1}(U) \) offen in \( N \).\\

Differenzierbare Struktur: Sei \( (V, \psi) \) eine Karte von \( N \), definiere eine Karte von \( F(N) \) durch
\[ (F(V), \psi \circ F^{-1}) \]
Durch diese Wahlen wird die Inklusion \( \tau : F(N) \hookrightarrow M \) eine injektive Immersion. Dann
\[\tau = F \circ F^{-1} \]
ist die Komposition eines Diffeomorphismus und einer Immersion.\\

\textbf{{Bemerkung.:}}
\begin{itemize}
\item Immersierte Untermannigfaltigkeiten sind genau die Bilder injektiver Immersion.
\item Immersierte Untermannigfaltigkeit tragen i.A. nicht die Teilraumtopologie.
\end{itemize}

Man kann ein Beispiel für folgende Situation konstruieren:

\( N \subseteq M \) immersierte Untermannigfaltigkeit, \( f : P \to M \) differenzierbare Abbildung, \( f(P) \subseteq N \). Aber:
\begin{itemize}
\item \( f : P \to N \) nicht stetig
\item Ist \( f : P \to N \) stetig, so ist \( f \) auch differenzierbar.
\end{itemize}

\textbf{Bemerkung:}
\begin{enumerate}
\item[i)] Ist \( n \) eine 2er-Potenz, so lässt sich \( \mathbb{RP}^n \) nicht nach \( \mathbb{R}^{2n-1} \) einbetten.\\
(\( \mathbb{RP}^2 \) ist nicht diffeomorph zu einer Untermannigfaltigkeit von \( \mathbb{R}^3 \))

\item[ii)] Ist \( n \) keine 2er-Potenz, so existiert für jede \( n \)-dim Untermannigfaltigkeit eine Einbettung nach \( \mathbb{R}^{2n-1} \). (Haefliger–Hirsch–Wall)

\item[iii)] 
\end{enumerate}

\begin{figure}[H]
  \centering
  \includegraphics[width=13cm]{Image Diffgeo/16.png}
\caption{Torus und Kleinsche Flasche (Stichwort: Linsenraum, $\mathbb{R}^n\setminus \mathbb{Z}^n$, Homotopie, Riemannsche Gitter, Orientierbarkeit)}
\end{figure}

\begin{figure}[H]
  \centering
  \includegraphics[width=11cm]{Image Diffgeo/17.png}
\caption{Kleinsche Flasche}
\end{figure}

\section{Der Tangentialraum}

\subsubsection*{\underline{Erinnerung:}} Sei \( f: \mathbb{R}^n \rightarrow \mathbb{R}^m \) eine differenzierbare Abbildung. Dann ist die Ableitung, die „beste“ lineare Approximation. Genauer gilt:
\[
f(x + v) = f(x) + Df_x(v) + \varphi(v)
\]
mit
\[
\lim_{v \to 0} \frac{\varphi(v)}{\|v\|} = 0.
\]

\medskip

Wobei:
\[Df_x(v) = \underbrace{J_f(x)\cdot v }_{\txt{Jacobi-Matrix angewandt auf v}} = \underbrace{v(f)_x\equiv \frac{d}{dt}\mid_{t=0}f(x+tv)}_{\text{Richtungsableitung von } f \text{ in Richtung } v \qquad \text{ jeweils im Punkt } x} 
\]
Partielle Ableitungen von \( f \) sind die Bilder der Basisvektoren \( e_i \) der kanonischen Basis im \( \mathbb{R}^n \) unter der linearen Abbildung \( Df_x(e_i)\).\\

\textit{\underline{Ziel:}} Definiere die Ableitung zwischen Mannigfaltigkeiten.\\

Die Jacobi-Matrix \( J_{\varphi(p)} (\psi \circ f \circ \varphi^{-1}) \) hängt von der Wahl der Karten ab, liefert also nur qualitative Aussagen, keine quantitativen.

\medskip

\textbf{\underline{Brauchen}:} \begin{itemize}
  \item Abstrakten Vektorraum $T_pM$ , den Tangentialraum von \( M \) in \( p \)\hfill \textit{(\textcolor{blue}{abhängig von \( M \) und \( p \)})}
  \item Lineare Abbildung $Df: T_pM \to T_{f(p)}N$ ,das Differential von \( f \) in \( p \) \hfill \textit{(\textcolor{blue}{abhängig von \( f \) und \( p \)})}
\end{itemize}

\medskip

Für Untermannigfaltigkeiten von \( \mathbb{R}^n \) gibt es eine anschauliche Definition des Tangentialraums mittels der Kurven durch \( p \):
\[
T_p M := \left\{ \dot{\gamma}(0) \,\middle|\, \gamma : (-\epsilon,\epsilon)\to M\subseteq \mathbb{R}^n \;\textbf{differenzierbar}, \gamma(0) = p \right\}.
\]
Für \textbf{abstrakte Mannigfaltigkeiten} haben wir keine kanonische Einbettung, daher müssen wir den Tangentialraum \( T_p M \) 
durch innere Eigenschaften von \( M \) bzw. mit Hilfe von Karten definieren.

\medskip

\underline{Beispiel.} Der Tangentialraum in \( p \in S^n \subset \mathbb{R}^{n+1} \) ist gegeben als das orthogonale Komplement von \( p \), d.h.
\[
T_p S^n = p^\perp = \left\{ v \in \mathbb{R}^{n+1} \,\middle|\, \langle v, p \rangle = 0 \right\}.
\]
\begin{figure}[H]
  \centering
  \includegraphics[width=13cm]{Image Diffgeo/18.jpg}
\caption{Tangentialraum auf Mannigfaltigkeit}
\end{figure}
%%%%%%%%%%%%%%%%%%%%%%%%%%%%%%%%%%%%%%%%%%%%%%%%%%%%%%%%%%%%%%%%%%%%%%%%%%%%%% Vorlesung 4 %%%%%%%%%%%%%%%%%%%%%%%%%%%%%%%%%%%

\textbf{Erinnerung:} $M \subseteq \mathbb{R}^n$ Untermannigfaltigkeit, \quad 
$T_pM = \{ \dot{\gamma}(0) \mid \gamma : (-\epsilon, \epsilon) \to M \text{ differenzierbar}, \gamma(0) = p \}$
\subsection{Die geometrische Definition des Tangentialraums}

\textbf{Idee:} Benutze Äquivalenzklassen von Kurven.

\medskip

Die \underline{Äquivalenzrelation}: $\alpha, \beta : (-\epsilon, \epsilon) \to M$ zwei (differenzierbare) Kurven in $M$, mit
\[
\alpha(0)=p=\beta(0)
\]
dann sind $\alpha$, $\beta$ äquivalent, $\alpha \sim \beta$, falls für eine (und damit in allen) Karte $(U, \varphi)$ um $p$ gilt:
\[
\frac{d}{dt}|_{t=0}\;\varphi \circ \alpha(t)= \frac{d}{dt}|_{t=0}\;\varphi\circ\beta(t)
\]

\begin{figure}[H]
  \centering
  \includegraphics[width=13cm]{Image Diffgeo/4.01.jpg}
\caption{Zwei äquivalente Kurven}
\end{figure}

Die Definition ist unabhängig von der Wahl der Karte. Denn sind $\varphi, \tilde{\varphi}$ Karten um $p$, so gilt:
\[
\left. \frac{d}{dt} \right|_{t=0} (\tilde{\varphi} \circ \alpha)(t) = 
\left. \frac{d}{dt} \right|_{t=0} (\tilde{\varphi} \circ \underbrace{\varphi^{-1} \circ \varphi}_{\mathrm{id}} \circ \alpha)(t)
\overset{\text{Kettenregel}}{=} 
D(\tilde{\varphi} \circ \varphi^{-1}) \cdot \left. \frac{d}{dt} \right|_{t=0} (\varphi \circ \alpha)(t)
\]
\[
= D(\tilde{\varphi} \circ \varphi^{-1}) \cdot \left. \frac{d}{dt} \right|_{t=0} (\varphi \circ \beta)(t)
= \dots = \left. \frac{d}{dt} \right|_{t=0} (\tilde{\varphi} \circ \beta)(t)
\]


Für $\epsilon$ klein genug liegen $\alpha(-\epsilon, \epsilon)$ und $\beta(-\epsilon, \epsilon)$ in $U\cap \tilde{U}$

Das führt zu der Definition:


\subsubsection*{Definition 2.1}
\underline{Tangentialvektoren} an $M$ im Punkt $p \in M$ sind Äquivalenzklassen bezüglich $\sim$ von Kurven in $M$ durch $p$. Der \underline{Tangentialraum} von $M$ in $p$ ist die Menge all dieser Äquivalenzklassen:
\[
T_p M = \{\gamma :I_{\gamma}\rightarrow M \;\txt{differenzierbar},\;\gamma(0)=p\}/\sim
\]
für hinreichend kleines Intervall mit $0 \in I_\gamma$\\
Schreibweise:
\[
[\gamma] = \dot{\gamma}(0) = \left. \frac{d}{dt} \right|_{t=0} \gamma(t)
\]

\subsubsection*{Lemma 2.2} Sei \( M \) eine \( n \)-dimensionale Mannigfaltigkeit, \( p \in M \). Dann ist der Tangentialraum \( T_pM \) ein \( n \)-dimensionaler reeller Vektorraum.
\begin{proof}
    Sei \((U, \varphi)\) eine Karte um \( p \), dann definieren wir
    \[
    D_p\varphi: T_pM \to \mathbb{R}^n \qquad [\gamma] \mapsto \left. \frac{d}{dt} \right|_{t=0} \varphi\circ\gamma(t)
    \]
    Die Abbildung ist linear, wohldefiniert (unabhängig von der gewählten Repräsentanten) nach Definition der Äquivalenzrelation \(\sim\).\\
    
    \underline{Behauptung:} \(D_p\varphi\) ist bijektiv.
    \begin{itemize}
        \item Injektivität: Direkt aus der Definition von \(\sim\).
        \item Surjektivität: Sei \( v \in \mathbb{R}^n \). Gesucht: Kurve \(\gamma\) in \(M\) mit \(\gamma(0) = p\) und \\
        Definiere 
        \[
        \gamma(t) = \varphi^{-1}(\varphi(p)+tv)
        \]
        und wähle \(\varepsilon > 0\) klein genug, um $\varphi(p)+tv$ \(\in \varphi(U)\). für \(|t| < \varepsilon\) zu garantieren, Dann folgt:
        \[
        \gamma(0) = p \qquad D_p\varphi([\gamma]) = \left. \frac{d}{dt} \right|_{t=0} \varphi(\varphi^{-1}(\varphi(p)+tv)) =  \left. \frac{d}{dt} \right|_{t=0} \varphi(p)+tv = v
        \]
    \end{itemize}
    \underbar{Vektorraumstruktur auf \(T_pM\)}: Wir wollen erreichen, dass \( D_{\varphi_p} \) eine lineare Abbildung wird. Für \( v, w \in T_pM \), \( \lambda \in \mathbb{R} \) definieren wir:
    \[
    v + w := (D_p\varphi)^{-1}\circ (D_p\varphi(v)+D_p\varphi(w)) \qquad \lambda \cdot v := (D_p\varphi)^{-1}\circ(\lambda D_p\varphi(v))
    \]
    \underbar{Bleibt zu zeigen}: Die Vektorraumstruktur hängt nicht von der Wahl der Karte ab.
    \[
    (D_p{\tilde{\varphi}})^{-1} \left( D_p\tilde{\varphi}(v) +     D_p\tilde{\varphi}(w) \right) \overset{?}{=} (D_p\varphi)^{-1} \left( D_p{\varphi}    (v) + D_p{\varphi}(w) \right)
    \]
    Bedenke:
\[
(D_p \widetilde{\varphi})(v) \overset{v = [\gamma]}{=} 
\left. \frac{d}{dt} \right|_{t=0} \widetilde{\varphi}(\gamma(t)) = 
\left. \frac{d}{dt} \right|_{t=0} \left( (\widetilde{\varphi} \circ \varphi^{-1}) \circ (\varphi \circ \gamma) \right)(t)
\]
\[
=\dots= D_{\varphi(p)} (\widetilde{\varphi} \circ \varphi^{-1}) D_p\varphi(v)
\]
und:
\[
D_{\varphi(p)}(\widetilde{\varphi} \circ \varphi^{-1}) = D_p\widetilde{\varphi} \circ (D_p\varphi)^{-1}.
\]

\end{proof}
\textbf{Bemerkung:}
\begin{itemize}
    \item[i)] Die Abbildung \( D_p\varphi \) ist das Differential der Kartenabbildung \(\varphi\) im Punkt \( p \).
    
    \item[ii)] Die kanonische differenzierbare Struktur auf \(\mathbb{R}^n\) ist definiert durch die Karte \((\mathbb{R}^n, \varphi = \mathrm{id})\). Dabei stimmt hier \( D_p{\varphi} \) überein mit der kanonischen Identifikation
    \[
    \Phi: T_p\mathbb{R}^n\rightarrow \mathbb{R}^n \quad [\gamma]\mapsto \dot{\gamma}(0)
    \]
    die nach Definition ein linearer Isomorphismus ist. Die Kurve ist gegeben durch \( v \mapsto [\alpha] \), wobei \( \alpha(t)=p+tv \) eine Kurve mit:
    \[
    \alpha(0) = p , \quad \dot{\alpha}(0) = v .
    \]
\end{itemize}

\subsubsection*{Definition 2.3}
Sei \( f: M \to N \) eine differenzierbare Abbildung. Dann ist das \emph{\underbar{Differential}} von \( f \) in \( p \) definiert durch
\[
Df_p: T_p M \to T_{f(p)} N \qquad [\gamma]\mapsto [f \circ \gamma]
\]

Alternative Schreibweisen: \( Tf_p \), \( df_p \), \(Df_p(\dot{\gamma}(0)) = (f \circ \gamma)'(0).\)

\vspace{0.5cm}

\underline{Anschaung}: Wenn \( \gamma \) eine Kurve in \( M \) mit \( \gamma(0) = p \) ist, dann ist \( f \circ \gamma \) eine Kurve in \( N \) mit \( (f \circ \gamma)(0) = f(p) \).

\begin{figure}[H]
  \centering
  \includegraphics[width=13cm]{Image Diffgeo/4.02.jpg}
\caption{Differential von f in p}
\end{figure}

Das Differential von \( f \) ordnet dem Tangentialvektor gegeben durch die Kurve \( \gamma \) den Tangentialvektor gegeben durch die Kurve \( f \circ \gamma \) zu.\\

\textbf{Lemma 2.4}\\
Das Differential einer differenzierbaren Funktion ist
\begin{itemize}
    \item wohldefiniert (unabhängig von der Wahl der Repräsentanten)
    \item linear
    \item erfüllt Kettenregel
\end{itemize}
\begin{proof}
  \underline{1. Wohldefiniert}\\
        Seien \( \alpha \) und \( \tilde{\alpha} \) zwei äquivalente Kurven auf \( M \) durch \( p \), d.h. es gilt 
        \[(\varphi \circ \alpha)'(0) = (\varphi \circ \tilde{\alpha})'(0)\]
        für alle Karten \( \varphi \) um \( p \). Sei \((U, \varphi)\), \((V, \psi)\) eine Karte um \( p \) bzw. \( f(p) \), dann folgt
        \[
        D{\psi_{f(p)}}( [f \circ \alpha]) \overset{\txt{Def.}}{=} \left. \frac{d}{dt} \right|_{t=0} (\psi \circ f \circ \alpha)(t) = \left. \frac{d}{dt} \right|_{t=0} (\psi \circ f \circ  \varphi^{-1} \circ \varphi \circ \alpha)(t)
        \]
        \[ \overset{\txt{Kettnregel}\;dt}{=} D(\psi \circ f \circ \varphi^{-1})(\varphi \circ \alpha)'(0) \overset{\txt{Voraus}}{=} D(\psi \circ f \circ \varphi^{-1})(\varphi \circ \tilde{\alpha})'(0) =\dots=  D{\psi_{f(p)}}( [f \circ \tilde{\alpha}])\]
        
        \( D{\psi_{f(p)}} \) bijektiv \(\rightarrow\) $[f \circ \alpha]=[f \circ \tilde{\alpha}]$
        d.h. \( Df \) ist unabhängig von der Wahl des Repräsentanten des Tangentialvektors.\\

        \underline{2. Kettenregel}: Sei $M\overset{f}{\rightarrow}N\overset{g}{\rightarrow}L$
     \[
(Dg \circ Df)([\alpha]) = Dg(Df([\alpha])) = Dg([f \circ \alpha]) = [g \circ f \circ \alpha] = D(g \circ f)([\alpha]).
\]
        \underline{3. Linearität}\\
        Sei \(\varphi : U \subseteq M \to \mathbb{R}^m\) eine Karte um \( p \). Das Differential von \(\varphi\) in \(p\) ist eine Abbildung
\[
D\varphi_p: T_p M=T_pU \to T_{\varphi(p)} \mathbb{R}^m
\]
Verknüpft mit der kanonischen Identifikation \(\Phi: \mathbb{R}^m\to \mathbb{R}^n\) von \(T_{\varphi(p)} \mathbb{R}^m\) mit \(\mathbb{R}^m\), so erhält man genau die oben definierte Abbildung \(D\varphi_p\). Diese ist nach Definition linear.
\[
\Rightarrow \text{Das Differential der Kartenabbildung } \varphi \text{ in } p \text{ ist linear.}
\]

Im Folgenden unterscheiden wir nicht mehr zwischen \( D\varphi_p: T_p\mathcal{M} \to T_{\varphi(p)} \mathbb{R}^n \) und \( \Phi \circ D\varphi_p: T_p\mathcal{M} \to \mathbb{R}^n \). Schreiben wir nun \( f = \psi^{-1}(\psi \circ f \circ \varphi^{-1}) \circ \varphi\), so folgt aus der Kettenregel \( Df =\underbrace{(D\psi)^{-1}D(\psi \circ f \circ \varphi^{-1})D\varphi}_{\txt{alle Ableitungen linear}\implies Df\;\txt{linear}} \)

\begin{figure}[H]
  \centering
  \includegraphics[width=11cm]{Image Diffgeo/4.03.jpg}
%\caption{Tangentialraum auf Mannigfaltigkeit}
\end{figure}
\end{proof}

\textbf{Bemerkung:}
\begin{itemize}
    \item[(i)] Für \(\mathcal{M} = \mathbb{R}^m\), \(\mathcal{N} = \mathbb{R}^n\) stimmt das Differential aus Def. 2.3 mit der Jacobi-Matrix aus Analysis II überein.
    
    \item[(ii)] \((D\operatorname{id}_{\mathcal{M}})_p = \operatorname{id}_{T_pM} \qquad \qquad [\gamma]\rightarrow[id\circ\gamma]=[\gamma]\)
    
    \item[(iii)] Ist \( f \) ein Diffeomorphismus, dann ist \( Df_p \) für alle \( p \in \mathcal{M} \) ein Isomorphismus. \([\gamma]\rightarrow[f \circ \gamma]\rightarrow[f^{-1} \circ f \circ \gamma]\)
    
    \item[(iv)] Ist \( Df_p \) ein Isomorphismus für alle \( p \in \mathcal{M} \), dann ist \( f \) ein lokaler Diffeomorphismus um $p$.
    
    \item[(v)] Da die Jacobi-Matrix von \(\psi \circ f \circ \varphi^{-1}\) eine andere Beschreibung des Differentials \( D(\psi \circ f \circ \varphi^{-1}) \) ist, folgt:
    \[
    \txt{Rg}_p(f)=\txt{Rg}_p(Df_p) \quad \txt{im Sinne LA für abstrakte VR}
    \]
    
    \item[(vi)] Sei \(\iota: \mathcal{M} \hookrightarrow \mathbb{R}^N\) eine Untermannigfaltigkeit mit Inklusionsabbildung \(\iota\). Dann ist
    \[
    \Phi\circ D_{\iota} : T_p \mathcal{M} \to T_{\iota(p)} \mathbb{R}^N \to \mathbb{R}^N
    \]
    ein Isomorphismus aufs Bild, also ein Monomorphismus.

    Es gilt
    \[
    D\iota([\gamma]) = [\iota \circ \gamma], \quad \txt{also }(\Phi\circ d\iota)([\gamma]) = \dot{\gamma}(0)
    \]
    \[
    T_p\mathcal{M} \equiv  \text{Vektorraum aller Tangentialvektoren } \dot{\gamma}(0) \text{ für Kurven mit } \gamma(0) = p. \;\txt{(vgl. Analysis)}
    \]
\end{itemize}
\subsection{Die Physiker-Definition von Tangentialvektoren}

\textbf{Idee:} 
Tangentialvektoren werden mittels Karten als Vektoren in \(\mathbb{R}^m\) definiert, die bei Kartenwechseln bestimmte Transformationsverhalten aufweisen.

\vspace{0.5cm}

\textbf{Bezeichnungen:}
\[
K_p\mathcal{M} = \text{Menge aller Karten um } p \in \mathcal{M}
\]
\[
V_p(\mathcal{M}) := \text{Menge aller Abbildungen } v: K_p\mathcal{M} \to \mathbb{R}^n \text{ mit }
\]
\[
v(V, \psi) = J_{\varphi(p)}(\psi \circ \varphi^{-1}) v(U, \varphi) \quad \forall (U, \varphi), (V, \psi) \in K_p\mathcal{M}.
\]

Die Menge \(V_p(\mathcal{M})\) ist isomorph zu \(T_p\mathcal{M}\), der Isomorphismus ist gegeben durch:
\[
T_p\mathcal{M} \longrightarrow V_p(\mathcal{M}) \qquad \xi \mapsto \left( (U, \varphi) \mapsto D\varphi_p(\xi \right)).
\]
Tatsächlich gilt \[((U,\varphi)\mapsto D\varphi_p(\xi))\in V_pM \]denn:
\[
D_{\psi_p}(\xi) = D(\psi \circ \varphi^{-1} \circ \varphi)_p(\xi)
= D(\psi \circ \varphi^{-1})\left( D_{\varphi_p}(\xi) \right).
\]
und
\(D(\psi \circ \varphi^{-1})_{\varphi(p)}\)
ist genau die Jacobi-Matrix \(J_{\varphi(p)}(\psi\circ \varphi^{-1})\).

\subsection{Die algebraische Definition des Tangentialraums}
\textbf{Idee:}
\begin{itemize}
    \item Identifizieren \( T_p\mathcal{M} \) mit Derivationen auf Funktionenkeimen um \( p \).
    \item Verallgemeinerung von Richtungsableitungen: Ist \( U \subseteq \mathbb{R}^n \) offen, $v\in T_pU=\mathbb{R}^n$ und \( f: U \longrightarrow \mathbb{R} \) eine glatte Funktion auf \( U \), dann ist die Richtungsableitung von \( f \) in \( p \) in Richtung \( v \):
    \[
     \left.\frac{d}{dt}\right|_{t=0} f(p+tv)=J(f)_p(v)
    \]
\end{itemize}

\textbf{Bemerkung:}
\begin{itemize}
    \item Wir müssen \( f \) nur lokal bei \( p \) kennen.
    \item Wenn wir \( J(f)_p(v) \) für genügend viele \( f \) kennen, können wir \( v \) bestimmen.
    
    Zum Beispiel: Ist
    \[
    f_i: U \to \mathbb{R} \qquad x=(x_1,...,x_n)\mapsto x_i ,
   \quad \txt{dann}\; J(f_i)_p(v)=v_i
    \]
  
\end{itemize}
\textbf{Schlussfolgerung:} Ableitungen von Funktionen enthalten Informationen über Tangentialvektoren.\\

Seien \( p \in \mathcal{M} \) und \( f, h: U \to \mathbb{R} \) glatte Funktionen, definiert auf einer Umgebung \( U \) von \( p \).
Wir sagen, \( f \) ist äquivalent zu \( h \) (geschrieben \( f \sim h \)), falls es eine Umgebung \( V \subseteq U \) von \( p \) gibt, sodass
\[
f|_V = h|_V.
\]

\textbf{Erinnerung:} 
Eine Menge \( V \subseteq \mathcal{M} \) heißt \emph{Umgebung} von \( p \), falls eine offene Menge \(\widetilde{V}\) mit
\[
p \in \widetilde{V} \subseteq V
\]
existiert. 
Man kann also immer von offenen Umgebungen ausgehen.

\subsubsection*{Definition 2.5:}
Die Äquivalenzklassen bezüglich $\sim$ differenzierbarer Funktionen, definiert auf Umgebungen von \( p \in \mathcal{M} \), heißen \textbf{Funktionenskeime in \( p \)}. 
Man schreibt \(C^{\infty}_p(M)\) für den Raum der Funktionskeime in $p$.\\

\textbf{Bemerkung:} 
\[
C_p^\infty(\mathcal{M}) \text{ ist eine reelle Algebra (Vektorraum mit Multiplikation) mit}
\]
\[
[f] + [g] = [f+g], \quad [f] \cdot [g] = [f \cdot g].
\]

\textbf{Lemma 2.6:} 
Sei \(M\) eine \(n\)-dimensionale Mannigfaltigkeit. Dann ist
\[
\mathcal{C}_p^\infty \cong \mathcal{C}_{0} ^\infty
\]
\begin{proof}
    Sei \((U, \varphi)\) eine Karte um \(p\) mit \(\varphi(p) = 0\). Dann ist die Abbildung
\[
[f] \mapsto [f \circ \varphi^{-1}]
\]
der gesuchte Isomorphismus.
\end{proof}
\subsubsection*{Definition 2.7:}
Eine \textbf{Derivation} auf \( C_p^\infty(\mathcal{M}) \) ist eine lineare Abbildung
\[
v: C_p^\infty(\mathcal{M}) \to \mathbb{R}
\]
mit der Eigenschaft
\[
v([f][g]) = v([f])g(p) + f(p)v([g]) \quad \text{(Produktregel)}.
\]

Die Menge der Derivationen auf \( C_p^\infty(\mathcal{M}) \) wird mit \( \mathcal{D}_p(\mathcal{M}) \) bezeichnet.

\vspace{0.5cm}

\textbf{Bemerkung:}
\begin{itemize}
    \item[(i)] \( \mathcal{D}_p(\mathcal{M}) \) ist ein reeller Vektorraum.
    \item[(ii)] \( \mathcal{D}_p(\mathcal{M}) \cong \mathcal{D}_0(\mathbb{R}^n) \).
    
    Wir definieren den Isomorphismus durch: Sei \( v \in \mathcal{D}_0(\mathbb{R}^n) \), dann ist
    \[
    [f] \mapsto v([f \circ \varphi^{-1}])
    \]
    eine Derivation in \( \mathcal{D}_p(\mathcal{M}) \).
\end{itemize}

\textbf{Beispiel:}\\

Jeder Vektor \( v = (v_1, \dotsc, v_n) \in \mathbb{R}^n \) definiert durch die Richtungsableitung eine Derivation auf \( C_{x_0}^\infty(\mathbb{R}^n) \):
\[
v([f]) = \left. \frac{d}{dt} \right|_{t=0} f(x_0 + t v) = Df_{x_0}(v) = \langle v, \nabla f \rangle = \sum_{j=1}^n v_j \left. \frac{\partial f}{\partial x_j} \right|_{x_0}.
\]
für eine Funktion \( f \), definiert in einer Umgebung von \( x_0 \).

Insbesondere entsprechen die Basisvektoren \( e_i \) der kanonischen Basis den partiellen Ableitungen \(\frac{\partial}{\partial x_i}, \quad i = 1, \dotsc, n.\)

Im Wesentlichen sind alle Derivationen von dieser Form, d.h.:
\[
\mathbb{R}^n \cong \mathcal{D}_{x_0}(\mathbb{R}^n) \cong \mathcal{D}_p(M) 
\]
\textit{(Das ist die Aussage des nächsten Satzes.)}\\

\textbf{Satz 2.8:} 

Jede Derivation \( \delta \) auf \( C_0^\infty(\mathbb{R}^n) \) schreibt sich, angewendet auf \([f] \in C_0^\infty(\mathbb{R}^n)\), als
\[
\delta([f]) = \sum_{i=1}^n \delta([x_i]) \frac{\partial f}{\partial x_i}|_0
\]


\vspace{0.5cm}

Insbesondere hat der Raum der Derivationen auf \( C_0^\infty(\mathbb{R}^n) \) die Dimension \( n \), mit einer Basis aus den Derivationen 
\[
\frac{\partial}{\partial x_i}, \quad i=1, \dotsc, n.
\]
Der Beweis benötigt folgendes Lemma:

\vspace{0.5cm}

\textbf{Lemma 2.9:}

Sei \( f: U \to \mathbb{R} \) eine differenzierbare Funktion auf einer konvexen Umgebung \( U \) von \( 0\in \mathbb{R}^n \).
Dann kann \( f \) geschrieben werden als
\[
f(x) = f(0) + \sum_{i=1}^n x_i h_i(x)
\]
für gewisse differenzierbare Funktionen \( h_i \), mit
\[
h_i(0) = \int_{0}^{1}  \frac{\partial f}{\partial x_i}(0)\,dt  = \frac{\partial f}{\partial x_i}(0).
\]
\begin{proof}
    (Beweis von Lemma 2.9)\\
    Hauptsatz der Differential- und Integralrechnung zusammen mit der Kettenregel:
\[
f(x) - f(0) = \int_0^1 \frac{d}{dt} f(tx) \, dt 
\overset{\text{Kettenregel}}{=} \int_0^1 \sum_{j=1}^n x_j \frac{\partial f}{\partial x_j}(tx) \, dt\]
\[= \sum_{j=1}^n x_j \left( \int_0^1 \frac{\partial f}{\partial x_j}(tx) \, dt \right)
=: \sum_{j=1}^n x_j h_j(x),\]

\vspace{0.5cm}

Setze:
\[
h_i(x) = \int_0^1 \frac{\partial f}{\partial x_i}(tx) \, dt,
\]
dann folgt:
\[
f(x) = f(0) + \sum_{i=1}^n x_i h_i(x).
\]
\underbar{Beobachtung}: 
\( h_i(0)= \int_0^1\frac{\partial f}{\partial x_i}(0) dt= \frac{\partial f}{\partial x_i}(0) \). \( h \) ist glatt da Differential und Integral vertauscht werden dürfen, denn $\deldel{f}{x_j}(tx)$ ist $C^{\infty}$ in $x$.
\end{proof}

\underline{\textit{Beweis von Satz 2.8:}}

Wir wenden die Derivation \( S \) auf \( f \in C_0^\infty(\mathbb{R}^n) \) an:

\[
\delta([f]) 
 \overset{Lem. 2.9}= \delta\left( [f(0) + \sum_{j=1}^n x_j h_j ]\right)
\overset{\delta\;\txt{linear}}= \delta([f(0)]) + \sum_{j=1}^n \delta\left( [x_j h_j] \right)
\]
\[
\overset{\txt{Produktregel}}{=} \delta([f(0)]) + \sum_{j=1}^n \left( \delta([x_j]) h_j(0) + \underbrace{x_j(0)}_{=0}\delta([h_j]) \right)
\]
\[
= \delta([f(0)]) + \sum_{j=1}^n \delta([x_j]) \left. \frac{\partial f}{\partial x_j} \right|_{0}.
\]

Da \( x_i(0) = 0 \) und \( h_i(0) \) konstant ist.\\

\underline{bleibt zu zeigen:} \(\delta([f(0)]) = 0\)

Allgemein gilt für alle konstanten Funktionen \(c\): \(\delta([c]) = 0.\)
Durch die Linearität von \(\delta\) reicht es, c=1 zu betrachten.
\[\delta([1 \cdot 1]) = \delta([1]) + \delta([1]) \quad \rightarrow \quad \delta([1]) = 0\]

\hfill \(\square\)

\underline{Zur Basis:}

Die partiellen Ableitungen nach \(x_j\) im Punkt \(0\) sind die Derivationen
\[
[f] \mapsto \left. \frac{\partial f}{\partial x_j} \right|_{x=0}.
\]

Wir schreiben diese Ableitungen als
\(\frac{\partial}{\partial x_j}.\)

In der obigen Rechnung haben wir gezeigt, dass diese den Raum der Derivationen aufspannen.  
Weiterhin sind sie linear unabhängig.

Somit:
\[
\left\{ \frac{\partial}{\partial x_j} \right\}_{j=1,\dotsc,n}
\]
ist eine Basis von \( \mathcal{D}_0(\mathbb{R}^n) \).

\begin{figure}[H]
  \centering
  \includegraphics[width=13cm]{Image Diffgeo/4.04.png}
\caption{Gekrümmte Richtungskurven nach Kartenabbildung}
\end{figure}

\subsubsection*{Kurze Zusammenfassung:}
\begin{itemize}
  \item Geometrische Definition von Tangentialraum und Differential 
  \[T_pM:=\{\gamma:I_\gamma \to M\;|\;\gamma \;\txt{ist differenzierbar}, \; \gamma(0)=p\}/\sim\]
  \[\gamma \sim \tilde{\gamma}\;:\Leftrightarrow\; \txt{Für eine/jede Karte }(U, \varphi)\;\txt{um}\;p\;\txt{gilt }\frac{d}{dt}|_{t=0}\;\varphi \circ \gamma(t)= \frac{d}{dt}|_{t=0}\;\varphi \circ \tilde{\gamma}(t)\]
  f ist differenzierbar $\implies$ Differential von f in p $(Df)_p: T_pM \to T_{f(p)}N\quad [\gamma]\mapsto[f \circ \gamma]$\\
  Nach Einführung der kanonischen Identifikation $\Phi:T_{\varphi(p)}\mathbb{R}^m\to \mathbb{R}^m$ lässt sich der Tangentialraum auffassen als Menge der Kurvenableitungen 
  \[\{\dot{\gamma}(0)\;|\;\gamma(0)=p\}=\{v\;|\;<v,p>=0\}\]

  \item Algebraische Definition des Tangentialraums\\
   Funktionskeimen (um p) $\mathcal{C}_p^\infty(M):=\{f:M\to\mathbb{R}\;\txt{glatt}\}/\sim$
   \[f \sim \tilde{f}\;:\Leftrightarrow \;\exists V\subseteq M \;\txt{Umgebung von }p \;\txt{mit } f|_V=\tilde{f}|_V\]
   Menge der Derivationen\\
   \[\mathcal{D}_p(M)=\{v:\mathcal{C}_p^\infty(M)\to \mathbb{R}\;|\;v \;\txt{linear}\;v([f g])=v([f])g(p)+f(p)v([g])\}\]
   \[\cong \mathcal{D}_0(\mathbb{R}^{\txt{dim}(M)})\quad (\txt{druch Wahl einer Karte})\qquad \cong \mathbb{R}^n \quad (\txt{Richtungsableitung})\]
\end{itemize}
%%%%%%%%%%%%%%%%%%%%%%%%%%%%%%%%%%%%%%%%%%%%%%%%%%%%%%%%%%%%%%%%%%%%%%%%%%%%%%%%%%%%%%%% Vorlesung 5 %%%%%%%%%%%%%%%%%%%%%%%%%

\textbf{Definition 2.16} \\
Sei \(\xi \in T_p M\), \(f \in C^\infty(M)\). Dann ist die \emph{\underline{Lie-Ableitung}} von \(f\) in Richtung \(\xi\) definiert als
\[
\mathcal{L}_\xi(f) = \xi(f)=\frac{d}{dt}|_{t=0}f(\alpha(t))\quad \xi = [\alpha].
\]

Auf Funktionsräumen definieren wir die Lie-Ableitung durch
\[
\mathcal{L}_\xi([f]) = \mathcal{L}_{\xi}(f).
\]

\vspace{1em}

\textbf{Bemerkung:}
\begin{enumerate}[i)]
    \item Identifizieren wir \(\mathbb{R}\) mit \(T_{f(p)} \mathbb{R}\), so schreiben wir die Lie-Ableitung auch als
    \[
    \mathcal{L}_\xi(f) = Df_p(\xi).
    \]
    
    \item Die Lie-Ableitung ist eine Derivation:
    \begin{enumerate}
        \item Linearität:
    \[
    \mathcal{L}_{\xi}([\lambda f+g]) = \left. \frac{d}{dt} \right|_{t=0} (\lambda f+g )(\alpha(t))  = \lambda \mathcal{L}_{\xi}([f]) + \mathcal{L}_{\xi}([g])
    \]
        
        \item Für Produkte:
\[
    \mathcal{L}_{\xi}([f] \cdot [g]) = \left. \frac{d}{dt} \right|_{t=0} (fg)(\alpha(t) = \left. \frac{d}{dt} \right|_{t=0} f(\alpha(t))g(\alpha(t))
    \]
\[
    \overset{\txt{Produktregel dt}}{=}[\left. \frac{d}{dt} \right|_{t=0}f(\alpha(t))] \cdot g(\alpha(0)) + f(\alpha(0)) \cdot [\left. \frac{d}{dt} \right|_{t=0} g(\alpha(t))]
    \]
    \[
    = \mathcal{L}_{\xi}([f]) \cdot g(p) + f(p) \cdot \mathcal{L}_{\xi}([g])
    \]
   \end{enumerate}
\end{enumerate}

\textbf{Satz 2.11:} \\
Die Abbildung \(\xi \mapsto \mathcal{L}_\xi\) ist ein linearer Isomorphismus zwischen \(T_p M\) und dem Raum der Derivationen $\mathcal{D}_p(M)$ auf \(C_p^\infty(M)\), d.\,h.
\[
T_p M \cong \mathcal{D}_p(M).
\]
Daraus stimmen die geometrische sowie die algebraische Definitionen überein!
\vspace{1em}

\begin{proof}
    Sei \((U, \varphi)\) eine Karte um \(p \in M\) mit \(\varphi(p) = 0 \in \mathbb{R}^m\).  
Sei \(\xi \in T_p M\) ein Tangentialvektor mit
\[
D\varphi_p(\xi) = v=(v_1,\dots,v_m) \in \mathbb{R}^m.
\]
Die Zuordnung \(\xi \mapsto \mathcal{L}_\xi\) ist linear, denn \(Df\) ist linear (Lemma 2.4).

Es gilt:
\[
\mathcal{L}_{\xi}(f) = Df_{p}(\xi) = D(f \circ \varphi^{-1} \circ \varphi)_{p}(\xi) =  D(f \circ \varphi^{-1})_{0} \cdot \underbrace{(D\varphi)_{p}(\xi)}_{=v}   
\]
\[
\underset{\txt{vgl. Bsp 2.8 oben}}{\overset{\txt{Def. Derivation}}{=}} v(f \circ \varphi^{-1})(0) \quad (v \;\txt{definiert Derivation durch Richtungsableitung)}
\]

\[
\overset{2.8}{=}  \sum_{j=1}^{n}v([x_j])\deldel{(f \circ \varphi^{-1})}{x_j}(0)   \quad (*)
\]
\underline{Injektivität:} \\
Sei \(\mathcal{L}_\xi(f) = 0\) für alle \(f \in C_p^\infty(M)\), insb. für \(f = \overbrace{Pr_k}^{\txt{Projetion k-te Komponente}} \circ \;\varphi=:\varphi_k\)
\[
0 = \mathcal{L}_{\xi}(\varphi_k) \overset{(*)}{=} \sum_{j=1}^m v_j\deldel{(Pr_k \circ \varphi \circ \varphi^{-1})}{x_j}(0) = \sum_{j=1}^m v_j\delta_{jk} =v_k \quad \Rightarrow \quad \xi = 0.
\]
\underline{Surjektivität:} \\
\[
\dim T_p M = m \overset{2.8}{=} \dim \mathcal{D}_0(\mathbb{R}^m) \overset{2.6}{=} \dim \mathcal{D}_p(M)
\]
Aus Lineare Algebra ist die Abbildung dann surjektiv.
\end{proof}

\textbf{Bezeichnung:} \\
\(f \in C^\infty(M)\), \((U, \varphi)\) Karte um \(p \in M\). Schreiben
\[
\frac{\partial f}{\partial x_j}(p) = \deldel{}{x_j}|_p(f) = \deldel{(f \circ \varphi^{-1})}{x_j}(\varphi(p)). \quad \txt{Partielle Ableitung abhängig von der Wahl der Karte}
\]
Genau genommen muss \(f\) auf der Kartenumgebung \(U\) definiert sein, d.\,h. die partielle Ableitung hängt nur von dem Funktionskeim \([f]\) ab.

\vspace{1em}

\textbf{Bemerkungen:}
\begin{enumerate}[i)]
    \item Die Derivationen \(\frac{\partial}{\partial x_j}\), \(j=1,\dots,m\) bilden eine Basis in \(\mathcal{D}_p(M) \cong T_p M\).
    \[
    \frac{\partial}{\partial x_j}|_p \longleftarrow (D\varphi)_p^{-1} e_j
    \]
Denn nach (*) können wir jeden Tangentialvektor \(\xi \in T_p M\) bzw. die zugehörige Derivation \(\mathcal{L}_\xi\) schreiben als
    \[
    \mathcal{L}_\xi = \sum_{j=1}^{m}v_j\deldel{}{x_j}|_p, \quad v_j \in \mathbb{R}.
    \]
    
    Dabei sind die Koeffizienten \(v_j\) eindeutig bestimmt durch
    \[
    v_k = \mathcal{L}_{\xi}(\varphi_k) = (D\varphi_k)(\xi),
    \]
    denn
    \[
    \frac{\partial}{\partial x_j} (\varphi_k) = \frac{\partial (Pr_k \circ \varphi \circ \varphi^{-1})}{\partial x_j} = \delta_{kj}.
    \]
    Aus (*) wissen wir
\[
v = (v_1, \dotsc, v_n) = D\varphi_p(\xi). \quad \text{Insbesondere:}
\]
\[
\left. \frac{\partial}{\partial x_j} \right|_p = (D\varphi_p)^{-1}(e_j).
\]
  \item Die Derivation \(\left. \frac{\partial}{\partial x_j} \right|_p\) entspricht dem Tangentialvektor \(\xi \in T_p M\) mit $D\varphi_p(\xi)=e_j$, d.h. der Äquivalenzklasse von Kurven \(\gamma\) mit \(\alpha_j(t) = \varphi^{-1}(\varphi(p)+te_j).\)
  \begin{figure}[H]
    \centering
    \includegraphics[width=13cm]{Image Diffgeo/5.01.jpg}
	\caption{Derivationen identifizieren mit Basisvektoren}
 \end{figure}
    
    \item Seien \((U, \varphi)\) und \((V, \psi)\) zwei Karten um \(p\). Dann gilt
    \[
    \left. \frac{\partial}{\partial x_i} \right|_p = \sum_j \left. \frac{\partial \psi_j}{\partial x_i}(p) \right. \left. \frac{\partial}{\partial y_j} \right|_p,
    \]
    wobei
    \[
    \frac{\partial}{\partial x_i} = (D\varphi_p)^{-1}(e_i), \quad \frac{\partial}{\partial y_j} = (D\psi_p)^{-1}(e_j), \quad     1 \leq i, j \leq n.
    \]
    \(\text{\small{Häufig wird diese Formel für Karten \((U, x)\), \((V, y)\) aufgestellt und liest sich als}}\)
\[
\frac{\partial}{\partial x_i} = \sum_j \frac{\partial y_j}{\partial x_i} \frac{\partial}{\partial y_j}.
\]
\begin{proof}
    \[
\frac{\partial}{\partial x_i} = (D\varphi)^{-1}(e_i)
= (D\psi)^{-1}(D\psi)(D\varphi)^{-1}(e_i) = (D\psi^{-1})(D(\psi \circ \varphi^{-1})(e_i))
\]
\[
= (D\psi^{-1})(J(\psi \circ \varphi^{-1})(e_i)) 
= (D\psi^{-1})\left( \sum_{j=1}^m \frac{\partial (\psi \circ \varphi^{-1})_j}{\partial x_i} e_j \right)
\]
\[
= \sum_{j=1}^m \frac{\partial (\psi \circ \varphi^{-1})_j}{\partial x_i} (D\psi^{-1})(e_j)
\]
\[
= \sum_{j=1}^m \underbrace{\frac{\partial (\psi \circ \varphi^{-1})_j}{\partial x_i}}_{\deldel{\psi_j}{x_i}} \frac{\partial}{\partial y_j}
\quad \text{(vgl. Bezeichnung)}
\]
\end{proof}
\end{enumerate}



\begin{enumerate}[iv)]
    \item \(f: M \to N\) differenzierbare Abbildung, \((U, \varphi)\) Karte um \(p\), \((V, \psi)\) Karte um \(f(p)\). Dann gilt:
\[
Df_p\left( \left. \frac{\partial}{\partial x_i} \right|_p \right)
= \sum_{j=1}^n \frac{\partial f_j}{\partial x_i}(p)
\left. \frac{\partial}{\partial y_j} \right|_{f(p)}
\]
    wobei:
    \[
    f_j = \psi_j \circ f = Pr_j \circ \psi \circ f.
    \]
    \[
    \left. \frac{\partial}{\partial x_i} \right|_p = (D\varphi_p)^{-1}(e_i), \quad \left. \frac{\partial}{\partial y_j} \right|_{f(p)} = (D\psi_{f(p)})^{-1}(e_j).
    \]
    Insbesondere gilt für Funktionen \(f: M \to \mathbb{R}\)
\[
Df_p\left( \left. \frac{\partial}{\partial x_i} \right|_p \right)
= \frac{\partial f}{\partial x_i}(p)
= \frac{\partial (f \circ \varphi^{-1})}{\partial x_i} \Big|_{\varphi(p)}.
\quad \text{(siehe Bezeichnung nach 2.11)}
\]

\end{enumerate}

\begin{enumerate}[v)]
    \item Sei \(\varphi_j : U\subset M \to \mathbb{R}\) die \(j\)-te Koordinatenfunktion. Dann bilden die linearen Abbildungen
    \[
    (D\varphi_j)_p : T_p M \to \mathbb{R}, \quad j=1,\dotsc,m,
    \]
    eine Basis des Dualraums \((T_p M)^*\quad \text{(Linearformen)}.\)
    Dies ist die zu
    \(\left\{ \left. \frac{\partial}{\partial x_j} \right|_p \right\}_{j=1,\dotsc,m}\) duale Basis, denn nach iv) gilt für \(f = \varphi_j\):
    \[
    (D\varphi_j)_p\left( \left. \frac{\partial}{\partial x_i} \right|_p \right) = \deldel{\varphi_j}{x_i} = \deldel{(\varphi_j \circ \varphi^{-1})}{x_i} = \deldel{Pr_j}{x_i} = \delta_{ij}.
    \]
\end{enumerate}   

\begin{enumerate}[vi)]
        \item Sei \(\Phi: M \to N\) eine differenzierbare Abbildung, \(f \in C^\infty(N)\) und \(X \in T_p M\), dann gilt:
    \[
    \underbrace{D\Phi_p(X)}_{\text{Tangentialvektor an}\; f(p) \in N }(f) = \mathcal{L}_X\underbrace{(f\circ\Phi)}_{M\longrightarrow \mathbb{R}} = X(f\circ\Phi).\quad \txt{(Kettenregel)}
    \]
      \begin{figure}[H]
    \centering
    \includegraphics[width=13cm]{Image Diffgeo/5.02.jpg}
	%\caption{Derivationen identifizieren mit Basisvektoren}
 \end{figure}
\end{enumerate}


\begin{enumerate}[label=\roman*), start=7]
    \item Sei \(X = [\gamma]\in T_p M\), \(\gamma: I_\gamma \subseteq\mathbb{R} \to M\) und \(\gamma(0) = p\). Dann gilt:
    \[
    X = D\gamma\left( \left. \frac{d}{dt} \right|_{t=0} \right).
    \]
    Wobei
    \[
    \left. \frac{d}{dt} \right|_{t=0} = (D_{\mathrm{id}})_0^{-1}(e_1) \in T_0 \mathbb{R}.
    \]

    \vspace{1em}
    
    Tatsächlich:
    \[
    X(f) = \mathcal{L}_X(f) = \left. \frac{d}{dt} \right|_{t=0}(f\circ\gamma) = D\gamma(\left. \frac{d}{dt} \right|_{t=0})(f).
    \]

    \vspace{1em}
    
    \underline{Alternative:} \\
    Per Definition:
    \[
    (D\varphi)_p(X) = \left. \frac{d}{dt} \right|_{t=0} (\varphi \circ \gamma)(t)
    \quad \text{und}
    \]
    \[
    (D\varphi)_p\left( D\gamma\left( \left. \frac{d}{dt} \right|_{t=0} \right) \right) = D( \varphi\circ\gamma)_0(\left. \frac{d}{dt} \right|_{t=0} ) = \left. \frac{d}{dt} \right|_{t=0} (\varphi\circ\gamma)(t).
    \]
    \[D\varphi_p\;\txt{Isomorphismus}\implies X = D\gamma(\left. \frac{d}{dt} \right|_{t=0})\]

    \vspace{1em}

    Schließlich:
    \[
    D\varphi_p(\xi) = (\mathcal{L}_\xi(x_1), \dotsc, \mathcal{L}_\xi(x_n)).
    \]
\end{enumerate}

\subsection{Das Tangentialbündel (Spezialfall Vektorbündel)}

\textbf{Idee:} \(M\) Mannigfaltigkeit, für alle \(p \in M\) haben wir den Tangentialraum \(T_p M\).  
Wollen alle Tangentialräume zu einem topologischen Raum zusammenfassen, sodass wir zwischen verschiedenen Tangentialräumen „umherlaufen“ können.


      \begin{figure}[H]
    \centering
    \includegraphics[width=13cm]{Image Diffgeo/5.03.jpg}
	\caption{Links: $S^1$ mit eigenen seiner Tangentialräume; Rechts: fasst die Tangentialräume zum Tangentialbündel, einem besonderen Vektorbündel zusammen}
 \end{figure}

\textbf{Definition 2.12:} \\
Das \underline{Tangentialbündel} \(TM\) einer Mannigfaltigkeit \(M\) ist definiert als die disjunkte Vereinigung aller Tangentialräume:
\[
TM := \bigcup_{p\in M}T_pM = \bigcup_{p\in M}\{p\}\times T_pM \quad \text{(als Menge)}.
\]

\underline{Nicht klar:} \\
Was ist eine gute topologische Struktur auf \(TM\)?

\vspace{0.5em}

\textbf{Wunschliste:}
\begin{itemize}
    \item Soll Informationen über die Topologie von \(M\) enthalten.
    \item Soll Informationen über die Topologie der Tangentialräume enthalten.
    \item  In einer Umgebung \(U\) von \(p\) ist \[
\bigcup_{p\in U}\{p\}\times T_p M \quad \cong \quad U \times \mathbb{R}^m.
\]
\end{itemize}
   
\vspace{0.5em}

\textbf{Satz 2.13:} \\
Sei \(M\) eine \(m\)-dimensionale differenzierbare Mannigfaltigkeit. Dann trägt \(TM\) die Struktur einer \(2m\)-dimensionalen differenzierbaren Mannigfaltigkeit.  
Die Fußpunktabbildung \(\pi: TM \to M\) ist surjektiv und differenzierbar.

\begin{proof}
    Wir definieren die (kanonische) Fußpunktabbildung \(\pi: TM \to M\) durch
\[
\pi: TM \to M, \quad \pi(\xi) = p, \quad \text{falls} \quad \xi \in T_pM.
\]

Sei \((U, \varphi)\) eine Karte von \(M\), dann definieren wir Karte $(\pi^{-1}(U), \Phi)$ auf $TM$:
\[
\Phi: \pi^{-1}(U) \to \varphi(U) \times \mathbb{R}^n \subseteq \mathbb{R}^{2n}
\]
durch
\[
\xi \mapsto (\varphi(\pi(\xi)), D\varphi_{\pi(\xi)}(\xi)).
\]

\vspace{1em}

\textbf{Klar:} Diese Abbildung ist bijektiv.

\vspace{0.5em}

Hier kommt die Topologie von \(TM\) ins Spiel, da genau Definition ist trickreich. Aber sie wird so gewählt, dass alle Abbildungen \(\Phi\) Homöomorphismen sind.\\

Wir betrachten die \textbf{Kartenwechsel:}

Seien \((U, \varphi)\), \((V, \psi)\) Karten um \(p \in M\) mit assoziierten Karten \((\pi^{-1}(U), \Phi)\), \((\pi^{-1}(V), \Psi)\).

Uns interessiert
\[
\Psi \circ \Phi^{-1}: \varphi(U \cap V) \times \mathbb{R}^m \longrightarrow \psi(U \cap V) \times \mathbb{R}^m
.
\]

Sei \((q, v) \in \varphi(U \cap V) \times \mathbb{R}^m\) und \(\xi = \Phi^{-1}((q, v))\), dann gilt:
\[
\pi(\xi)= \varphi^{-1}(q)  \quad (\leftrightarrow \varphi(\pi(\xi))=q) \quad v=D\varphi_{\pi(\xi)}(\xi)
\]
\[
\Rightarrow \quad \Psi \circ \Phi^{-1}(q, v) = \Psi(\xi) = \left( \psi \circ \pi(\xi), D\psi_{\pi(\xi)}(\xi) \right)
= \left( \psi\circ\varphi^{-1}(q), D\psi_{\pi(\xi)} (D\varphi_{\pi(\xi)} )^{-1} v\right).
\]

\vspace{1em}

Wir sehen: Der Kartenwechsel ist differenzierbar.
\[
\Rightarrow \quad \text{Diese Karten definieren eine differenzierbare Struktur auf } TM.
\]

\vspace{1em}

\hfill \(\text{\small{(s. lokal euklidisch mit differenzierbarem Kartenwechsel)}}\)

\textbf{Hausdorff:} \\
Seien \(\xi = (p,a)\) und \(s = (q,b)\) zwei verschiedene Punkte in \(TM\), mit \(a \in T_p M\), \(b \in T_q M\).

\vspace{1em}

\underline{1. Fall:} \(p \neq q\).  
Dann existieren offene Umgebungen \(U,V\) von \(p,q\) mit $U\cap V=\emptyset$. (wegen Hausdorff-Eigenschaft von $M$)

Dann sind $\pi^{-1}(U)$ und $\pi^{-1}(V)$ disjunkte offene Umgebungen von \((p,a)\) und \((q,b)\).

\vspace{1em}

\underline{2. Fall:} \(p=q\), \(a \neq b \in T_p M\).

Sei \((U,\varphi)\) eine Karte um \(p\) mit der zugehörigen Karte
\[
\Phi: \pi^{-1}(U) \to \varphi(U) \times \mathbb{R}^n \quad \text{für}\; TM
\]
und
\[
\Phi(a) = (\varphi(p),v), \quad \Phi(b) = (\varphi(p),w), \quad v \neq w.
\]

\vspace{1em}

Sind \(V,W\) offene Mengen in \(\mathbb{R}^n\) mit \(v \in V\), \(w \in W\) und \(V \cap W = \emptyset\), dann sind
\[
\Phi^{-1}(\varphi(u)\times V) \quad \text{und} \quad \Phi^{-1}(\varphi(U)\times W)
\]
disjunkte offene Umgebungen von \((p,a)\) und \((p,b)\).
  \begin{figure}[H]
    \centering
    \includegraphics[width=14cm]{Image Diffgeo/5.04.jpg}
	\caption{$TM$ ist hausdorff'sch}
 \end{figure}

\textbf{Zweitabzählbar:} \\
Heuristisch, da wir keine genaue Definition der Topologie gegeben haben.
\(M\) besitzt abzählbare Basis der Topologie (o.E. besteht diese aus Kartengebieten), für alle Mengen \(U\) aus dieser Basis ist \(\pi^{-1}(U)\) homöomorph zu \(\varphi(U) \times \mathbb{R}^n \subset \mathbb{R}^{2n}\) und dort finden wir eine abzählbare Basis \(\leadsto\) Diagonalargument.

\vspace{1em}

\textbf{Differenzierbarkeit von \(\pi\):} \\
\(\pi: TM \to M\) ist differenzierbar, da
\[
\varphi\circ\pi\circ\Phi^{-1}:\varphi(U) \times \mathbb{R}^n \longrightarrow \varphi(U) \quad (x_1, \dotsc, x_{2n}) \mapsto (x_1, \dotsc, x_{n}).
\]
Stetigkeit kommt wieder aus der Definition von Topologie.
\end{proof}



\subsection{Vektorbündel (Familie von Vektorräumen)}

\textbf{Idee:} \\
\(B\) Mannigfaltigkeit, über jedem \(b \in B\) soll ein Vektorraum \(V_b\) angeheftet werden, sodass $	\bigcup_{b\in B}\{b\}\times V_b$ eine Mannigfaltigkeit ist,  
\(V_b \cong V_{b'}\) als Vektorräume, Kartenwechsel sollen Isomorphismen in den Vektorräumen sein.
  \begin{figure}[H]
    \centering
    \includegraphics[width=14cm]{Image Diffgeo/5.05.jpg}
	%\caption{$TM$ ist hausdorff'sch}
 \end{figure}
\[
\tilde{\varphi} \circ \varphi^{-1} = \mathrm{id} \times A
\quad \text{mit einem linearen Isomorphismus \(A\)}.
\]
Vektorbündel oder manchmal auch Vektorraumbündel sind Familien von Vektorräumen, die durch die Punkte eines topologischen Raumes parametrisiert sind. Anschaulich besteht ein Vektorbündel aus je einem Vektorraum für jeden Punkt des Basisraumes. \\

\textbf{Definition 2.14:} \\
Seien \(E\) und \(B\) differenzierbare Mannigfaltigkeiten, \(\pi: E \to B\) eine surjektive Submersion, sodass jede Faser
\[
E_p :=\pi^{-1}(\{p\})
\]
die Struktur eines \(k\)-dimensionalen Vektorraums besitzt.  

Eine \emph{Trivialisierung} von \(\pi\) über einer offenen Menge \(U\) von \(B\) ist ein Diffeomorphismus
\[
\chi :E_U\longrightarrow U\times \mathbb{R}^k \subset \mathbb{R}^{m+k} \quad E_U := \pi^{-1}(U), \; E|_U \;\txt{enthält sowie Standpunkt als auch Faser}
\]
so dass für alle \(p \in U\):
\begin{enumerate}[i)]
    \item \(\chi(E_p) = \{p\}\times\mathbb{R}^k\)
    \item \(\chi|_{E_p}: E_p \to\{p\}\times\mathbb{R}^k \cong \mathbb{R}^k\) ist ein linearer Isomorphismus
\end{enumerate}

Für alle \(p \in U\) existiert also ein linearer Isomorphismus \(\xi(p):E_p\longrightarrow\mathbb{R}^k\) so dass:
\[
\chi(e) = (\pi(e), \xi(\pi(e))e)
\]
\begin{figure}[H]
  \centering
  \includegraphics[width=9cm]{Image Diffgeo/5.06.png}
\caption{Illustration des Vektorbündels \((E, B, \pi)\): 
\(E = \mathbb{R}^2\) der Totalraum und \(B = \mathbb{R}\) der Basisraum. Die Abbildung \(\pi \colon E \to B\) projiziert jeden Punkt \(p = (x, y) \in E\) auf \(x \in B\). Die Faser über \(x \in B\) ist definiert als
\(E_x = \{ p \in E \mid \pi(p) = x \}.\)
Der Totalraum \(E\) ist die Vereinigung aller Fasern:
\(
E = \bigsqcup_{x \in B} E_x.
\)
}
\end{figure}


\textbf{Bemerkung:}
\begin{enumerate}[i)]
    \item Die erste Bedingung lässt sich auch als \(\Pr_1 \circ \chi = \pi\) schreiben, wobei \(\Pr_1 : U\times\mathbb{R}^k \to U\) die Projektion auf den ersten Faktor ist.
    
    \item Sind \(\chi_1, \chi_2\) zwei Trivialisierungen von \(\pi\) über \(U_1, U_2 \subseteq B\) offen.
    
    Bedingung i) \(\oplus\) ii): Es gibt glatte Abbildungen \(A_{12}: U_1 \cap U_2 \to \underbrace{GL_k(\mathbb{R})}_{\txt{invertierbar, linear}}\) mit
    \[
    \chi_2 \circ \chi_1^{-1}: (U_1 \cap U_2) \times \mathbb{R}^k \to (U_1 \cap U_2) \times \mathbb{R}^k,
    \]
    \[
    (p,v) \mapsto (p, A_{12}(p)v).
    \]
\end{enumerate}

\(A_{12}\) heißt \underline{Übergangsabbildung} zwischen den Trivialisierungen.\\

b) Ein reelles \emph{\underline{Vektorbündel}} \((E, B, \pi)\) von Rang \(k\) über einer Mannigfaltigkeit \(B^m\) der Dimension \(m\) ist eine surjektive Submersion \(\pi:E\longrightarrow B\) zwischen einer Mannigfaltigkeit \(E\) (der Dimension $k+m$), dem sog. \underline{Totalraum} des Bündels, und der Mannigfaltigkeit \(B\), der sog. \underline{Basis} des Bündels, mit den folgenden Eigenschaften:
\begin{enumerate}[i)]
    \item jede Faser \(E_p := \pi^{-1}(\{p\})\) besitzt die Struktur eines k-dim Vektorraums, dessen Nullelement mit \(0_p\) bezeichnet wird,
    
    \item es gibt eine offene Überdeckung \(\{U_i\}_{i \in I}\) von \(B\) und eine Familie von lokalen (bijektiven, differenzierbaren) Trivialisierungen \(\{\chi_i:E_{U_i}\longrightarrow U_i\times\mathbb{R}^k\}_{i\in I}\) von \(\pi\) über den Elementen der Überdeckung, die auf jeder Faser einen $\mathbb{R}$-Vektorraum-Isomorphismus induziert, und mit $\mathrm{Pr}_1 \circ \chi_i = \pi$ \\
    
    Zu jedem Paar $(i, j) \in I \times I$ mit $U_{ij} := U_i \cap U_j \neq \emptyset$ gebe es
eine differenzierbare Abbildung
\[
g_{ij} : U_{ij} \to \mathrm{GL}_k(\mathbb{R}),
\]
so dass gilt:
\[
\chi_i \circ \chi_j^{-1}(x, v^\top) = (x, g_{ij}(x) \cdot v^\top).
\]
Dann gibt es auf $E$ eine (eindeutig bestimmte) differenzierbare Struktur, so dass
$E$ ein Vektorbündel vom Rang $k$ über $X$ mit Bündelprojektion $\pi$ und lokalen
Trivialisierungen $\varphi_i$ ist.\\

Jede lokale Trivialisierung eines Vektorbündels liefert zugleich eine Karte der Gesamtmannigfaltigkeit $E$, da sie lokal einen Diffeomorphismus in den Standardraum $\mathbb{R}^{m+k}$ gibt. Man identifiziert $U_i \times \mathbb{R}^k \subset \mathbb{R}^{m+k}$ über eine Karte der Basis $X$:
\[
\varphi_i: U_i \to \mathbb{R}^m
\]
Zusammengenommen ergibt das eine Karte $\chi_i \circ (\varphi_i \times \mathrm{id})$ der Gesamtmannigfaltigkeit $E$ in der Form:
\[
\pi^{-1}(U_i) \xrightarrow{\chi_i} U_i \times \mathbb{R}^k 
\xrightarrow{\varphi_i \times \mathrm{id}} \mathbb{R}^{m+k}.
\]


\end{enumerate}

\vspace{1em}

c) Wenn \((E,B,\pi)\) eine Trivialisierung \(\chi: E \to B \times\mathbb{R}^k\) (über ganz \(B\)) besitzt, sagen wir, dass \(\pi\) ein \emph{triviales Vektorbündel} ist. Das bedeutet, dass es eine globale stetige Auswahl von Basen (eine globale Trivialisierung) gibt.

\begin{figure}[H]
  \centering
  \includegraphics[width=6cm]{Image Diffgeo/5.07.png}
\caption{Möbiusband als reelles Linienbündel (also ein $\mathbb{R}^1$-Bündel) über dem Kreis. Lokal ist es homöomorph zu \(U \times \mathbb{R}\), wobei \(U\) eine offene Teilmenge von \(S^1\) ist. Allerdings ist das Möbiusband nicht homöomorph zu \(S^1 \times \mathbb{R}\), was einem Zylinder entspricht.
}
\end{figure}

\textbf{Bemerkung:} \\
Häufig nennt man \(E\) das Bündel (anstatt \((E, B, \pi)\) oder \(\pi: E \to B\)) und geht implizit von einer Abbildung \(\pi: E \to B\) aus.\\

\textbf{Beispiele:}
\begin{enumerate}[i)]
    \item Das triviale Bündel über \(B\): \(E = B \times \mathbb{R}^k\), \quad \(\pi: B \times \mathbb{R}^k \to B,\quad (x,v) \mapsto x.\)

    Ein Vektorbündel heißt trivial, wenn es isomorph zu einem trivialen Bündel ist. (Per Def ist jedes Vektorbündel lokal trivial.)

    \vspace{1em}

    \item Das \emph{Tangentialbündel} \(\pi: TM \to M\) einer \(m\)-dimensionalen Mannigfaltigkeit \(M\) ist ein Vektorbündel vom Rang m.

    \vspace{1em}

    \item Analog zum Tangentialbündel ist das \emph{Kotangentialbündel} \(T^*M\) als Vereinigung der Dualräume von Tangentialräumen
    \[
    T_p^* M = (T_p M)^*
    \]
    auch ein Vektorbündel und wir können das \emph{Endomorphismenbündel} \(\mathrm{End}(TM)\) definieren.

    \vspace{1em}

    \item Die Vektorbündel $TS^2$ und das Möbiusband sind nicht trivial.\\
    Für das triviale Bündel wählen wir eine globale Trivialisierung, sodass der Übergang
\[
\chi_1 \circ \chi_2^{-1}(x,v) = (x,v) \quad \text{für alle } x \in U_1 \cap U_2.
\]
Wäre das Möbiusband trivial, gäbe es glatte Funktionen 
\[
h_\pm : E|_{U_\pm} \to U_{\pm}\times \mathbb{R}\quad \text{mit} \quad \mu_{+-}(x) = h_+(x) \circ h_-(x)^{-1}.
\]
Dies würde bedeuten, dass ein glatter Weg von einem orientierungserhaltenden zu einem orientierungsumkehrenden Diffeomorphismus existiert. Ein solcher Weg kann jedoch nicht existieren, da orientierungserhaltende (bzw. -umkehrende) Diffeomorphismen überall positive (bzw. negative) Ableitungen haben, während die Ableitung entlang eines glatten Weges stetig sein müsste.


      \begin{figure}[H]
    \centering
    \includegraphics[width=9cm]{Image Diffgeo/5.99.png}
	\caption{Möbiusband: nicht triviales Bündel über $S^1$}
 \end{figure}

    \(TS^1\), \(TS^3\) und \(TS^7\) sind trivial.

    \item Der Raum der Differentialformen ist als Bündel der äußeren Algebra auch ein Vektorbündel.
\end{enumerate}

\textbf{Def 2.14 1/2:} \\
Seien \(\pi_1: E_1 \to B_1\) und \(\pi_2: E_2 \to B_2\) Vektorbündel.  
Eine glatte Abbildung \(F:E_1\longrightarrow E_2\) heißt \emph{Bündelhomomorphismus}, wenn:
\begin{enumerate}[i)]
    \item eine glatte Abbildung \(\overline{F}:B_1\longrightarrow B_2\) existiert, sodass \(\pi_2\circ F = \overline{F}\circ \pi_1\) \quad (\(F\) bildet Fasern auf Fasern ab). Als Diagramm:
 \[
\begin{array}{ccc}
E_1 & \xrightarrow{F} & E_2 \\
\downarrow \pi_1 & & \pi_2 \downarrow \\
B_1 & \xrightarrow{\overline{F}} & B_2
\end{array}
\]

    \item die Einschränkung \(F|_{(E_1)_p}:\) $(E_1)_p$ \(\longrightarrow\)  $(E_2)_{\overline{F}(p)}$  linear ist.\\
    Sprechweise: \(F\) hebt die Abbildung \(\overline{F}\) hoch/an.
\end{enumerate}

\(F\) heißt \emph{Bündelisomorphismus}, wenn ein Bündelhomomorphismus \(G: E_2 \to E_1\) existiert mit:
\[
F \circ G = \mathrm{id_{E_2}}, \quad G \circ F = \mathrm{id_{E_1}}.
\]
\textbf{Definition 2.15:} \\
Eine differenzierbare Abbildung \(s: B \to E\) heißt \emph{\underline{Schnitt}} des Vektorbündels $\pi:E\to B$, falls
\[
\pi\circ s = id_B,
\]
d.h. falls \(s(x) \in E_x\) für alle \(x \in B\).\\
Den unendlich-dimensionalen Vektorraum der Schnitte in \(E\) bezeichnet man mit \(\Gamma(E)\).\\

\textbf{Beispiele.}
\begin{itemize}
    \item Trivialbündel: Sei \( E = B \times \mathbb{R}^k \), dann ist jeder Schnitt eine Abbildung
    \[
    s: B \to B \times \mathbb{R}^k, \quad s(x) = (x, v(x)),
    \]
    wobei \( v: B \to \mathbb{R}^k \) eine glatte Funktion ist. Die Schnitte entsprechen also genau den glatten \( \mathbb{R}^q \)-wertigen Funktionen auf \( X \).

    \item Tangentialbündel \( TM \): Ein Schnitt des Tangentialbündels
    \[
    s: M \to TM
    \]
    mit \( \pi \circ s = \mathrm{id}_M \) ist ein \textbf{Vektorfeld} auf \( M \), da \( s(x) \in T_xM \) für alle \( x \in X \) ist.
\end{itemize}
Ein Schnitt ist wie ein „Feld von Vektoren“, das sich kontinuierlich (oder glatt) über die ganze Mannigfaltigkeit $X$ erstreckt, wobei jeder Vektor aus dem passenden Vektorraum (der Faser) über dem Punkt kommt.\\

\textbf{Definiton 2.16:} \\
Eine \emph{Nullstelle} eines Schnittes \(s: B \to E\) ist ein Punkt \(x \in B\) mit
\[
s(x)=0 \in E_x.
\]
Jedes Vektorbündel hat den \emph{Nullschnitt} \(s \equiv 0\), für den jedes Punkt \(x \in B\) eine Nullstelle ist.

%%%%%%%%%%%%%%%%%%%%%%%%%%%%%%%%%%%%%%%%%%%%%%%%%%%%%%%%%%%%%%%%%%%%%%%%%%%%%%%%%%%%% Vorlesung 6 %%%%%%%%%%%%%%%%%%%%%%%%5%%%

Der Spezialfall \( E = TM \) verdient einen eigenen \textit{Namen}.
\section{Vektorfeld}
\bigskip
\textbf{Definition 3.1:} Ein (differenzierbares) \underline{Vektorfeld} ist ein Schnitt des Tangentialbündels, d.h. eine differenzierbare Abbildung 
\[
X \colon M \to TM \quad \txt{mit } \pi \circ X =\mathrm{id}_M.
\]
Sei $ X_p = \mathrm{Pr}_2 \circ X(p) \in T_pM \;\text{für alle } p \in M.$\\

\textbf{\underline{Beispiel:}} \quad
\( U \subseteq \mathbb{R}^n \) offen. \quad
Identifizieren \( TU \cong U \times \mathbb{R}^n \)

\bigskip

\noindent
Vektorfelder \quad \( \Longleftrightarrow \) \quad differenzierbare Abbildung \( X \colon U \to U \times \mathbb{R}^n \) mit \( X(p) = (p,X_p) \)

\smallskip

\noindent
\phantom{Vektorfelder} \quad \( \Longleftrightarrow \) \quad differenzierbare Abbildung \( X \colon U \to \mathbb{R}^n \)\\
\begin{figure}[H]
  \centering
  \includegraphics[width=10cm]{Image Diffgeo/6.01.png}
\caption{Vektorfeld auf einer Mannigfaltigkeit M}
\end{figure}

\textbf{Bemerkung:} Sei \( M \subseteq \mathbb{R}^n \) eine Untermannigfaltigkeit. 
Wir identifizieren \( TM \) mit einer Teilmenge von \( \mathbb{R}^{2n} \).

\medskip

Für \( p \in M \colon T_p M \longleftrightarrow \) Untervektorraum von \( \mathbb{R}^n \)

\medskip

Sei \( \Psi \colon TM \to \mathbb{R}^n \times \mathbb{R}^n \) definiert durch
\[
\Psi(v) = (p, v) \quad \text{wobei } v \in T_p M \subseteq TM
\]
Dann ist \( \Psi \) eine Einbettung.

\emph{\underline{Beschreibung des Vektorfelds$:^{(*)}$}}Differenzierbare Abbildung \( X \colon M \to \Psi(TM) \subseteq \mathbb{R}^{2n} \) mit 
\(X(p) = (p,X_p) \)\\
(bzw. \( X \colon M \to \mathbb{R}^n \) mit Bild in einem von \( p \) abhängigen Untervektorraum)

\vspace{1em}

\noindent
\textbf{Beispiel:} \( M = S^n \), dann schreiben wir den Tangentialraum \( T_p S^n \) von \( S^n \) an \( p \) als
\[
T_p S^n = \left\{ v \in \mathbb{R}^{n+1} \,\middle|\, \langle v, p \rangle = 0 \right\}
\]
\(\Rightarrow\) Ein Vektorfeld auf \( S^n \) ist eine glatte Abbildung \( X \colon S^n \to \mathbb{R}^{n+1} \) mit
\[
\langle X(p), p \rangle = 0. \quad (\txt{Hier } X(p)=X_p\in T_pM)
\]
\colorbox{blue!20}{Frage:}
\( M \) Mannigfaltigkeit, gibt es ein Vektorfeld \( X \) ohne Nullstellen? \\
(Also \( X(p) \neq 0 \) für alle \( p \in M \))

\medskip

Auf der Sphäre $S^{n}$ existiert genau dann ein nichtverschwindendes Vektorfeld, wenn $n$ ungerade ist.
Für \( M = S^{2n+1} \colon X(x_0, \ldots, x_{2n+1}) = (-x_1, x_0, \ldots, -x_{2n+1}, x_{2n}) \)
ist ein konkretes Beispiel für ein nichtverschwindendes Vektorfeld, das tangential zur Sphäre $S^{2n+1}$ liegt. 
\vspace{1em}

\noindent
\subsubsection*{Satz vom Igel}
Jedes Vektorfeld auf \( S^{2n} \) hat mindestens eine Nullstelle. \\
(\emph{Offensichtlich kann dieser Satz nur für differenzierbare / stetige Vektorfelder richtig sein.})

\vspace{1em}

\noindent
\colorbox{blue!20}{{Weiterführende Frage:}}
Wie viele linear unabhängige solcher Vektorfelder gibt es?

\vspace{1em}

\noindent
\textbf{Definition:}
Zwei oder mehr Vektorfelder heißen \emph{linear unabhängig}, falls sie punktweise linear unabhängig sind. (\emph{Stichwort: Rahmen})\\

\textbf{Bemerkung:}
\begin{itemize}
  \item[(i)] Sei \( n+1 = (2r+1) \cdot 2^{c+4d} \) mit \( 0 \leq c \leq 3 \), \( d, r \in \mathbb{N}_0 \). \\
  Dann gibt es auf \( S^n \) genau \( 2^c + 8d - 1 \) linear unabhängige Vektorfelder (insebsondere haben diese keine Nullstellen).

  \item[(ii)] \( S^n \) hat \( n \) linear unabhängige Vektorfelder genau dann, wenn \( n = 1, 3, 7 \)
  \[
  \left( S^1 \subseteq \mathbb{R}^2 = \mathbb{C}, \quad S^3 \subseteq \mathbb{R}^4 = \mathbb{H}, \quad S^7 \subseteq \mathbb{R}^8 = \mathbb{O} \right)
  \]

  \item[(iii)] Gibt es auf einer \( m \)-dimensionalen glatten Mannigfaltigkeit \( M \) die maximal mögliche Anzahl von \( m \) linear unabhängigen Vektorfeldern, so sagen wir, dass \( M \) \emph{\underline{parallelisierbar}} bzw. das \emph{\underline{Tangentialbündel trivial}} ist. ($T_M\cong M\times \mathbb{R}^m)$

  \item[(iv)] Den unendlich-dimensionalen (z.B. auf einem Funktionenraum) (falls \( \dim M > 0 \)) reellen Vektorraum der Vektorfelder auf \( M \) bezeichnen wir mit \( \mathcal{X}(M) = \Gamma(TM) \). Der Vektorraum \( \Gamma(TM) \) ist ein Modul über dem Ring \( C^\infty(M) \), da \( f\cdot X \) (punktweises Produkt) für jede Funktion \( f \) und jedes Vektorfeld \( X \) wieder ein Vektorfeld ist.
\end{itemize}

\subsection{Vektorfelder als Derivationen}

\begin{itemize}
  \item Schon gesehen: Tangentialvektoren \( \longleftrightarrow \) Derivationen auf Funktionenkeimen
  \item Wollen sehen: Vektorfelder \( \longleftrightarrow \) Derivationen auf \( C^\infty(M) \) \textcolor{blue}{(Raum der glatten Funktionen auf \( M \))}
\end{itemize}

\textbf{Definition 3.2:}
Eine \underline{Derivation} \( \delta \) auf der Algebra \( C^\infty(M) \) der differenzierbaren Funktionen auf \( M \) ist eine lineare Abbildung
\[
\delta \colon C^\infty(M) \longrightarrow C^\infty(M),
\]
die die Produktregel
\[
\delta(fg) = \delta(f)g + f\delta(g)
\]
erfüllt. Der Vektorraum der Derivationen auf \( M \) heißt \( \mathcal{D}(M) \).\\

\textbf{Beispiel:}
Jedes Vektorfeld \( X \in \Gamma(TM) \) definiert durch die Lie-Abbildung
\[
\mathcal{L}_X \colon C^\infty(M) \to C^\infty(M)
\]
eine Derivation auf \( M \). Dabei definieren wir punktweise:
\[
\mathcal{L}_X(f)(p) = \mathcal{L}_{X_p}(f) = Df_p(X_p).
\]

Die Funktion \( \mathcal{L}_X(f) \) ist in \( C^\infty(M) \), da \( f \in C^\infty(M) \)
(partielle Ableitung einer glatten Funktion ist glatt).

\vspace{1em}

\noindent
\textbf{Satz 3.3:}
Die Abbildung
\[
\mathcal{L} \colon \Gamma(TM) \longrightarrow \mathcal{D}(M), \quad X \mapsto \mathcal{L}_X
\]
ist ein Isomorphismus unendlich-dimensionaler Vektorräume.
\begin{proof}
\quad
\begin{itemize}
  \item \( \mathcal{L} \) ist eine lineare Abbildung \quad \checkmark
  \item \( \mathcal{L}_X \) ist eine Derivation \quad \checkmark
\end{itemize}

\medskip

\underline{Injektivität:} 
Zu zeigen: Für \( X \in \Gamma(TM) \) mit \( X \neq 0 \) ist \( \mathcal{L}_X \neq 0 \)

\medskip

Sei \( X \neq 0 \) ein Vektorfeld auf \( M \), d.h. es gibt ein \( p \in M \) mit \( X_p = X(p) \neq 0 \)

\medskip

Haben schon gesehen: Die Abbildung
\[
T_p M \longrightarrow \mathcal{D}_p(M) \quad X_p \mapsto \mathcal{L}_{X_p}
\]
ist injektiv. (Satz 2.11, ist zwar linearer Isomorphismus)
\[
\Rightarrow \text{Es gibt eine Funktion } f, \text{ definiert auf einer Umgebung } U \text{ von } p \text{ mit }
\]
\[\mathcal{L}_{X_p}(f) = Df_p(X_p) \neq 0\]

\textbf{Behauptung:} Wir können \( f \) zu einem auf ganz \( M \) definierten, differenzierbaren Funktion fortsetzen. Wähle dazu eine Funktion \( \chi \in C^\infty(M) \) mit $\mathrm{supp}(\chi)\subset U$ und $\chi|_V=1$ auf einer Umgebung $V\subset U$ von p:
\textcolor{blue}{
\begin{itemize}
  \item \(\text{supp}(g) = \{ x \in M \mid g(x) \neq 0 \}\)
  \item Nicht klar, dass es eine solche Funktion \( \chi \) gibt.
  \item \( \chi \) heißt \emph{Abschneide Funktion}.
\end{itemize}
}



Dann ist \( \chi \cdot f \in C^\infty(M) \), definiert durch
\[
(\chi f)(p) = 
\begin{cases}
(\chi f)(p) & \text{für } p \in U \\
0 & \text{für } p \in M \setminus U
\end{cases}
\]
die gesuchte glatte Fortsetzung von \( f \).

\underline{Beobachtung:} \footnote{\(\mathcal{L}_X(\chi f) = \mathcal{L}_X(\chi) \cdot f + \chi \cdot \mathcal{L}_X(f)\)}
\[
\mathcal{L}_X(\chi f) = D(\chi f)_p(X_p) \overset{\chi|_V=1 }{=} Df_p(X_p) \neq 0
\]
\underline{Alternative:}
\[
\mathcal{L}_X(\chi f)(p) = \mathcal{L}_{X_p}(\chi f) \overset{\txt{produktregel}}{=} \underbrace{\mathcal{L}_{X_p}(\chi)}_{=0 \;\txt{konst. Fkt}}f(p)+\underbrace{\chi(p)}_{=1} \mathcal{L}_{X_p}(f) = \mathcal{L}_{X_p}(f) \neq 0.
\]

\underline{Surjektivität:} \\
Sei \( \delta \) eine Derivation auf \( C^\infty(M) \). Gesucht: \( X \in \Gamma(TM) \) mit \( \mathcal{L}_X = \delta \).

\medskip

Klar gesehen: Durch Abschneide Funktion lässt sich jede lokal definierte Funktion zu einer global definierten Funktion fortsetzen. (Daher sprechen wir von \emph{Funktionenkeimen}.)

\medskip

\( \delta \) definiert durch \( \delta_p \colon f \mapsto \delta(f)(p) \) eine Derivation auf den Funktionenkeimen \( C^\infty_p(M) \). (Def. 2.7 Derivation durch algebraische Definition von Tangentialraum) \\
(Dazu wählen wir einen Repräsentanten von \( f \) und setzen diesen zu einer globalen Funktion fort.)
\[
\overset{2.8}{\Rightarrow} \delta_p = \sum_{i=1}^m \delta_p(\varphi_j) \frac{\partial}{\partial x_j}|_p
\]
Definiere das Vektorfeld auf \( M \) durch: \( X_p \) ist der Tangentialvektor, der zu \( \delta_p \) gehört. (Durch Isomorphismus $T_pM\cong \mathcal{D}_pM$ Satz 2.11) Für alle p definiert man ein $\delta_p$, aus dem ein $X_p$ stammt. Damit ist $X$ bestimmt so auch dessen Lie-Ableitung $\mathcal{L}_X$. 
\[
\Rightarrow \mathcal{L}_X = \delta
\]
\end{proof}
In lokalen Koordinaten \( (U, \varphi) \) schreibt sich \( X \) mittels Basis $\{\deldel{}{x_j}\}$ von $\mathcal{D}_pM$ als
\[
X|_U = \sum_{j=1}^m \mathcal{L}_X(\varphi_j)\deldel{}{x_j}= \sum_{j=1}^m \delta(\varphi_j)\deldel{}{x_j}, \quad \deldel{}{x_j}\;\txt{als Tangentialvektor}
\]
\[
\Rightarrow X \text{ ist ein glattes Vektorfeld.}
\]

Zur Existenz einer \emph{Abschneidefunktion}:

\medskip

\textbf{Lemma 3.4:}
Sei \( R > 0 \), dann existiert ein glattes Feld 
\[
\chi = \chi_R \colon \mathbb{R}^n \to [0,1]
\]
mit
\[
\chi \big|_{\overline{B_{1/3 R}(0)}} \equiv 1 \quad \text{und} \quad \chi \big|_{\mathbb{R}^n \setminus \overline{B_{2/3 R}(0)}} \equiv 0
\]
\textbf{Insbesondere:} Dieses Feld ist
\begin{itemize}
  \item glatt auf \( \mathbb{R}^n \)
  \item identisch \( 1 \) für \( |x| \leq \frac{1}{3} R \)
  \item identisch \( 0 \) für \( |x| \geq \frac{2}{3} R \)
\end{itemize}

\textbf{Bemerkung:}
\begin{itemize}
  \item[(i)] Es reicht, das Lemma für \( n = 1 \) zu zeigen, danach können wir $\chi(\norm{x}^2)$ für \( n > 1 \) wählen.
  
  \item[(ii)] Um die Abschneidefunktion auf \( M^m \) um den Punkt \( p \) zu erhalten, wählen wir eine Karte \( (U, \varphi) \) mit \( p \in U \) und $\varphi(p)=0$ \\
  Dann gibt es ein \( R > 0 \) mit $B_R(0)\subset \varphi(U)$. Die gesuchte Abschneidefunktion ist:
\[
\quad
\begin{cases}
\chi \circ \varphi & \txt{auf } U \\
0 & \txt{sonst.} \quad \qquad \qquad \qquad 
\end{cases}
\]
(Zurückziehen von $\chi$ mit $\varphi$.)
\end{itemize}

\begin{proof}
    Betrachte die Funktionen
\[
f \colon \mathbb{R} \longrightarrow [0,1], \quad
x \mapsto
\begin{cases}
e^{-\frac{1}{x^2}} & x > 0 \\
0 & x \leq 0
\end{cases}
\]
\[
g \colon \mathbb{R} \longrightarrow [0,1], \quad x \mapsto \frac{f(x)}{f(x)+f(1-x)}
\]

\medskip

Es gelten: 
\begin{itemize}
  \item \( f, g \in C^\infty(\mathbb{R}) \)
  \item \( g(x) = 0 \quad \text{für } x \leq 0 \)
  \item \( g(x) = 1 \quad \text{für } x > 1 \)
\end{itemize}

\begin{figure}[H]
  \centering
  \includegraphics[width=14cm]{Image Diffgeo/6.02.jpg}
\caption{Links: $\chi$; Mittel: f; Rechts: g}
\end{figure}
Die gesuchte Funktion ist
\[
\chi \colon \mathbb{R} \longrightarrow [0,1], \quad
x \mapsto g\left( x+2 \right) \cdot g\left( x-2 \right)
\]

\medskip

Es gilt \( \chi \in C^\infty(\mathbb{R}) \):
\begin{itemize}
  \item \( \chi(x) = 1 \quad \text{für } x \in [-1,1] \)
  \item \( \chi(x) = 0 \quad \text{für } x \in \mathbb{R} \setminus (-2,2) \)
\end{itemize}

\medskip

Das ist die Funktion für \( R = 3 \), die allgemein erhalten wir durch Strecken/Stauchen.
\end{proof}


\textbf{Bemerkung:}
Sei \( X \) ein Vektorfeld auf \( M \) und \( (U, \varphi) \) eine Karte von \( M \). Dann können wir \( X \) auf \( U \) als
\[
X|_U = \sum a_i \frac{\partial}{\partial x_i}
\]
schreiben, mit glatten Funktionen \( a_i \in C^\infty(U) \). Diese sind durch
\(a_i = X(\varphi_i) \) eindeutig bestimmt, wobei $\varphi_i$ die i-te Koordinatenfunktion bezeichnet, $\varphi_i(q) = (\varphi(q))_i = (Pr_i \circ \varphi)(q)$

Die Derivationen \( \frac{\partial}{\partial x_j} \) sind glatte Vektorfelder auf \( U \), die in jedem Punkt eine Basis bilden 
(sog. \emph{Basisvektorfelder}).

\subsection{Die Lie-Algebra der Vektorfelder}

Was passiert, wenn wir zwei Derivationen hintereinander ausführen? Ist sie wieder einen Derivation?
\[
\delta_1 \delta_2 (f \cdot g) 
= \delta_1 \left( \delta_2(f) \cdot g + f \cdot \delta_2(g) \right)
\]
\[
= \delta_1(\delta_2(f)) \cdot g + \delta_2(f) \cdot \delta_1(g)
+ \delta_1(f) \cdot \delta_2(g) + f \cdot \delta_1(\delta_2(g))
\]
\[
\neq \left( \delta_1 \delta_2(f) \cdot g + f \cdot \delta_1 \delta_2(g) \right)
\quad \text{\textcolor{red}{(im Allg.)}}
\]
(Stichwort: Paralleltransport)
\medskip

\textbf{Lemma 3.5:} 
Seien \( \delta_1, \delta_2 \) zwei Derivationen auf \( C^\infty(M) \). Dann ist der \emph{\underline{Kommutator}}
\[
[\delta_1, \delta_2] := \delta_1 \circ \delta_2 - \delta_2 \circ \delta_1
\]
wieder eine Derivation.\\

Der Isomorphismus \( \mathcal{L} \colon \Gamma(TM) \to \mathcal{D}(M) \) motiviert die folgende Definition:

\medskip

\subsubsection*{Definition 3.6:}
Seien \( X, Y \) Vektorfelder auf \( M \). Dann ist \([X, Y]\) das eindeutig bestimmte Vektorfeld mit
\[
\mathcal{L}_{[X,Y]} =\mathcal{L}_X \circ \mathcal{L}_Y-\mathcal{L}_Y \circ \mathcal{L}_X= [\mathcal{L}_X, \mathcal{L}_Y]
\]
Das Vektorfeld \([X,Y]\) heißt die \emph{\underline{Lie-Klammer}} bzw. der \emph{\underline{Kommutator}} der Vektorfelder \( X, Y \).\\

\medskip

\textbf{Definition 3.7:}
Die \emph{\underline{Lie-Ableitung}} von \( Y \) nach \( X \) ist \( \mathcal{L}_X Y := [X, Y] \).\\

\medskip

\textbf{Bemerkung:}
Das Vektorfeld \([X,Y]\) ist als Derivation bestimmt durch die Gleichung
\[
[X, Y](f) = X(Y(f)) - Y(X(f)) \quad \forall f \in C^\infty(M)
\]

\textbf{Lemma 3.8:} \\

Zwei Vektorfelder \( X, Y \in \Gamma(TM) \) seien in einer Karte \( (U, \varphi) \) gegeben als
\[
X|_U = \sum a_i \frac{\partial}{\partial x_i}, \quad
Y|_U = \sum b_i \frac{\partial}{\partial x_i}
\]
wobei $a_i=X(\varphi_i)$, $b_i=Y(\varphi_i)$. Dann gilt
\[
[X, Y]|_U = \sum_{i,j=1}^m \left( a_j\deldel{b_i}{x_j}-b_j\deldel{a_i}{x_j} \right) \frac{\partial}{\partial x_i}
\]
\begin{proof}
    Schreibe \( [X, Y] \) auf \( (U, \varphi) \) als 
\[
[X, Y] = \sum_{i=1}^{m} c_i \frac{\partial}{\partial x_i}
\]

Die Koeffizienten \( c_i \) sind bestimmt durch
\[
c_i = [X, Y](\varphi_i)
= X\left( Y(\varphi_i) \right)
- Y\left( X(\varphi_i) \right)
\]
\[
= X(b_i) - Y(a_i)
\]
\[
= \sum_j a_j \frac{\partial b_i}{\partial x_j}
- \sum_j b_j \frac{\partial a_i}{\partial x_j}
\]
\end{proof}

\textbf{Satz 3.9:} \\
Seien \( X, Y, Z \) Vektorfelder auf \( M \), \( a, b \) reelle Zahlen und \( f \) eine differenzierbare Funktion auf \( M \), dann gilt:

\begin{itemize}
  \item[(i)] \( [X, Y] = -[Y, X] \) \hfill (Schiefsymmetrie bzw. Antikommutativität)
  \item[(ii)] \( [aX + bY, Z] = a[X, Z] + b[Y, Z] \) \hfill (Bilinearität über \( \mathbb{R} \))
  \item[(iii)] \( [X, [Y, Z]] + [Y, [Z, X]] + [Z, [X, Y]] = 0 \) \hfill (Jacobi-Identität)
  \item[(iv)] \( [fX, Y] = f[X, Y] - Y(f)X \)
  
  \( \mathcal{L}_X(fY) = [X, fY] = f[X, Y] + X(f)Y \).
\end{itemize}

\medskip

\textbf{Definition 3.10:}
Eine reelle \underline{\textbf{Lie-Algebra}} $(V, [\cdot,\cdot])$ ist ein reeller Vektorraum \( V \) mit einer Abbildung:
\[
[ \cdot , \cdot ] \colon V \times V \longrightarrow V,
\]
welche die Eigenschaften (i)--(iii) erfüllt. Die \emph{\underline{Lie-Klammer}} \( [ \cdot , \cdot ] \) ist also eine bilineare schiefsymmetrische Abbildung, welche die Jacobi-Identität erfüllt.\\

\textbf{Beispiele:}
\begin{itemize}
  \item[(i)] Der Raum \( V = \Gamma(TM) \) der Vektorfelder auf einer Mannigfaltigkeit \( M \) zusammen mit dem Kommutator von Vektorfeldern ist eine unendlich-dimensionale reelle Lie-Algebra.

  \item[(ii)] Jeder Vektorraum \( V \) wird mit \( [\cdot,\cdot] = 0 \) zu einer Lie-Algebra. Diese Lie-Algebren heißen \emph{abelsch}.

  \item[(iii)] Der Vektorraum \( V = M(n, \mathbb{R}) \) der quadratischen reellen Matrizen mit der Lie-Klammer
  \[
  [A, B] = AB - BA
  \]
  ist eine \( n^2 \)-dimensionale reelle Lie-Algebra.

  \item[(iv)] Der Vektorraum
  \[
  \mathfrak{so}(n) = \{ A \in M(n, \mathbb{R}) \mid A + A^T = 0 \}
  \]
  der schiefsymmetrischen Matrizen ist mit dem Kommutator
  \[
  [A, B] = AB - BA
  \]
  eine \( \frac{1}{2} n(n-1) \)-dimensionale reelle Lie-Algebra.
\end{itemize}
\begin{itemize}
  \item[(v)] \( V = \mathbb{R}^3 \) mit dem Vektorprodukt \( \times \) als Lie-Klammer
  \[
  [v, w] = v \times w, \quad \forall v, w \in \mathbb{R}^3
  \]
  ist eine 3-dimensionale Lie-Algebra. Diese Lie-Algebra ist isomorph zu \( \mathfrak{so}(3) \).
\end{itemize}

\medskip

\textbf{Definition 3.11:}
Seien \( (V, [\cdot,\cdot]_V) \) und \( (W, [\cdot,\cdot]_W) \) Lie-Algebren. Eine Abbildung \( f \colon V \to W \) heißt \emph{\underline{Lie-Algebren-Homomorphismus}}, falls \( f \) linear ist und
\[
f([v_1, v_2]_V) = [f(v_1), f(v_2)]_W \quad \forall v_1, v_2 \in V
\]

\medskip

Ein bijektiver Lie-Algebren-Homomorphismus heißt \emph{\underline{Lie-Algebren-Isomorphismus}}. \\
Zwei Lie-Algebren heißen \emph{isomorph}, falls es einen Lie-Algebren-Isomorphismus zwischen ihnen gibt.

%%%%%%%%%%%%%%%%%%%%%%%%%%%%%%%%%%%%%%%%%%%%%%%%%%%%%%%%%%%%%%%%%%%%%%%%%%%%%%%%%% Vorlesung 7 %%%%%%%%%%%%%%%%%%%%%%%%%%%%%%%

\subsection{Das Bild von Vektorfeldern unter Diffeomorphismen}

\(\mathcal{M}, \mathcal{N}\) Mannigfaltigkeiten  
\(\phi: \mathcal{M} \to \mathcal{N}\) differenzierbare Abbildung  
\(X \in \Gamma(T\mathcal{M})\) Vektorfeld auf \(\mathcal{M}\)

Gibt es ein Vektorfeld \(Y\) auf \(\mathcal{N}\) mit
\[
T_{\phi(p)} \mathcal{N} \ni Y_{\phi(p)} = \left( D\phi_p (X_p) \right) \in T_{\phi(p)} \mathcal{N} \, ?
\]
  \begin{figure}[H]
    \centering
    \includegraphics[width=12cm]{Image Diffgeo/7.01.jpg}
	%\caption{Derivationen identifizieren mit Basisvektoren}
 \end{figure}
Falls \(\phi\) nicht injektiv ist: Mehrere Punkte können auf \(q \in \mathcal{N}\) landen, \(\phi(p) = q = \phi(p')\) mit $p\neq p'$
\(\Rightarrow\) Es gibt keinen Grund warum
\[
D\phi_p(X_p) = D\phi_{p'}(X_{p'})
\]

Falls \(\phi\) nicht surjektiv ist: Was sollte \(Y\) außerhalb des Bildes von \(X\) sein?  
Stetige Fortsetzungen sind nicht immer möglich.

 \begin{figure}[H]
    \centering
    \includegraphics[width=12cm]{Image Diffgeo/7.02.png}
	%\caption{Derivationen identifizieren mit Basisvektoren}
 \end{figure}

\textbf{Definition 3.12:}  
Sei \(\phi: \mathcal{M} \to \mathcal{N}\) ein Diffeomorphismus, dann ist der \emph{\underline{Push-Forward}} (oder das Bild) des Vektorfeldes \(X \in \Gamma(T\mathcal{M})\) unter \(\phi\) ein Vektorfeld auf \(\mathcal{N}\), das wir mit \(\phi_* X\) auf \(\mathcal{N}\) bezeichnen und durch
\[
(\phi_* X)_q := \, (D\phi)_{\phi^{-1}(q)}(X_{\phi^{-1}(q)}) \qquad \forall q \in \mathcal{N}
\]
bzw.
\[
(\phi_* X)_{\phi(p)} := \, (D\phi)_p(X_p) \qquad \forall p \in \mathcal{M}
\]
definieren.

\vspace{0.5cm}

\textbf{Satz 3.13:}  
Sei \(\phi: \mathcal{M} \to \mathcal{N}\) ein Diffeomorphismus und \(X \in \Gamma(T\mathcal{M})\) ein glattes Vektorfeld.  
Dann ist \(\phi_* X\) ein glattes Vektorfeld auf \(\mathcal{N}\) und es gilt:
\begin{itemize}
  \item[(i)] \(\mathcal{L}_{\phi_* X}(f) = \, \mathcal{L}_X(f \circ \phi) \circ \phi^{-1}\)
  \item[(ii)] \(\phi_* [X, Y] = \, [\phi_{*}X,\phi_{*}Y]\)
\end{itemize}
für alle \(f \in C^\infty(\mathcal{N})\), \(Y \in \Gamma(T\mathcal{M})\).\\

\textbf{Sanity Check:}(ergeben die Formeln Sinn)

\(f: \mathcal{N} \to \mathbb{R}\), dann ist auch \(\mathcal{L}_{\phi_* X}(f)\) eine Funktion \(\mathcal{N} \to \mathbb{R}\)

\(f \circ \phi: \mathcal{M} \to \mathbb{R}\), dann ist \(\mathcal{L}_X(f \circ \phi)\) eine Funktion \(\mathcal{M} \to \mathbb{R} \Rightarrow \mathcal{L}_{X}(f \circ \phi) \circ \phi^{-1}: \mathcal{N} \to \mathbb{R}\)

\vspace{0.3cm}

\(X, Y\) Vektorfelder auf \(\mathcal{M}\), dann ist \([X, Y]\) Vektorfeld auf \(\mathcal{M} \Rightarrow\) $\phi_*[X,Y]$ Vektorfeld auf \(\mathcal{N}\)\\

\(\phi_* X, \phi_* Y\) Vektorfelder auf \(\mathcal{N}\), dann ist $[\phi_*X, \phi_*Y]$ Vektorfeld auf \(\mathcal{N}\)

\vspace{0.5cm}
\begin{proof}
    \textbf{Zu (0)}: Der Diffeomorphismus \(\phi: \mathcal{M} \to \mathcal{N}\) induziert einen Diffeomorphismus \(D\phi: T\mathcal{M} \to T\mathcal{N}\) durch
\[
D\phi: T\mathcal{M} \to T\mathcal{N}, \quad (p, v) \mapsto (\phi(p), D\phi_p(v))
\]

Als Abbildung ist \(\phi_* X: \mathcal{N} \to T\mathcal{N}\) so definiert, dass das Diagramm kommutiert:
 \begin{figure}[H]
    \centering
    \includegraphics[width=6cm]{Image Diffgeo/7.03.jpg}
	%\caption{Derivationen identifizieren mit Basisvektoren}
 \end{figure}
\[
 \Rightarrow \quad \phi_* X \text{ glattes Vektorfeld auf } \mathcal{N}.
\]
(Aus Definition entspricht push-forward genau dem linken Verlauf, der aus glatten Abbildungen besteht.)\\

\textbf{Zu i)} Wir rechnen nach:
\[
(\mathcal{L}_{\phi_* X} f)(\phi(p)) \overset{(\phi_* X)_{\phi(p)} = D\phi_p(X_p)}{=} \mathcal{L}_{D\phi_p(X_p)} f\overset{\txt{Def. Lie-Ableitung}}{=} Df_p(D\phi_p(X_p))
\]
\[\overset{\txt{Kettenregel}}{=} D(f \circ \phi )_p(X_p)\overset{\txt{Def. Lie-Ableitung}}{=}\mathcal{L}_{X_p}(f \circ \phi) = \mathcal{L}_{X}(f \circ \phi)(p).\]
\[
\Rightarrow (\mathcal{L}_{\phi_* X} f) \circ \phi = \mathcal{L}_X(f \circ \phi)
\quad \# 
\]

\textbf{Zu ii)} Wir benutzen die Gleichung aus i):
\[
\mathcal{L}_{[\phi_* X, \phi_* Y]} f 
= \mathcal{L}_{\phi_* X} (\mathcal{L}_{\phi_* Y} f)
- \mathcal{L}_{\phi_* Y} (\mathcal{L}_{\phi_* X} f)
\qquad \textcolor{orange}{[\text{Def 3.6}]}
\]
\[
= \mathcal{L}_{\phi_* X} (\mathcal{L}_Y (f \circ \phi) \circ \phi^{-1})
- \mathcal{L}_{\phi_* Y} (\mathcal{L}_X (f \circ \phi) \circ \phi^{-1}) \quad \textcolor{orange}{[i]}
\]
\[
= \mathcal{L}_X (\mathcal{L}_Y(f \circ \phi) \circ \phi^{-1} \circ \phi) \circ \phi^{-1}
- \mathcal{L}_Y (\mathcal{L}_X(f \circ \phi)) \circ \phi^{-1}
\qquad \textcolor{orange}{[i]}
\]
\[
= \mathcal{L}_{[X,Y]}(f \circ \phi) \circ \phi^{-1}
\qquad \textcolor{orange}{[\text{Def 3.6}]}
\]
\[
= \mathcal{L}_{\phi_*[X,Y]} f \quad \textcolor{orange}{[i]}
\]


\end{proof}

\subsection{\(\phi\)-verknüpfte Vektorfelder}

\textbf{Definition 3.14:}  
Sei \(\phi: \mathcal{M} \to \mathcal{N}\) eine differenzierbare Abbildung. Zwei gegebene Vektorfelder \(X \in \Gamma(T\mathcal{M})\), \(Y \in \Gamma(T\mathcal{N})\)
heißen \underline{\(\phi\)-verknüpft ($\phi$-verwandt)}, falls
\[
Y_{\phi(p)} = D\phi_p(X_p)
\]
für alle \(p \in \mathcal{M}\) erfüllt ist.

\vspace{0.3cm}

\textbf{Beispiel:}  
\(\phi: \mathcal{M} \to \mathcal{N}\) Diffeomorphimus und \(X\) Vektorfeld auf \(\mathcal{M}\).  
Dann sind die Vektorfelder \(X\) und \(\phi_* X\)  
\(\phi\)-verknüpft.\\

\textbf{Bemerkung:}  
Zwei Vektorfelder \(X \in \Gamma(T\mathcal{M}),\ Y \in \Gamma(T\mathcal{N})\) sind genau dann \(\phi\)-verknüpft, wenn
\[
\mathcal{L}_Y(f) \circ \phi = \mathcal{L}_X(f \circ \phi) \quad (*)
\]
für alle Funktionen (bzw. Funktionenskeime) \(f\) auf \(\mathcal{N}\) gilt.
\begin{proof}
    Für \(p \in \mathcal{M}\) ist
\[
\mathcal{L}_Y(f) \circ \phi(p) = \mathcal{L}_{Y_{\phi(p)}}(f)
\]
und
\[
\mathcal{L}_X(f \circ \phi)(p) = \mathcal{L}_{X_p}(f \circ \phi) = \mathcal{L}_{D\phi_p(X_p)}(f)
\]

(Kettenregel, Bem. vi) am Ende von Abschnitt 2.3.)\\

Die Gleichheit in \((*)\) gilt, wenn \(Y_{\phi(p)}\) und \(D\phi_p(X_p)\) dieselbe Derivation definieren, also gleich sind.
\end{proof}


\textbf{Lemma 3.15:}  
Sei \(\phi: \mathcal{M} \to \mathcal{N}\) eine differenzierbare Abbildung und für \(i = 1, 2\) seien \(X_i \in \Gamma(T\mathcal{M})\) bzw. \(Y_i \in \Gamma(T\mathcal{N})\) zwei Paare von $\phi$-verknüpften Vektorfeldern.  
Dann sind die Kommutatoren \([X_1, X_2]\) und \([Y_1, Y_2]\) ebenfalls $\phi$-verknüpft, d.\,h.
\[
D\phi\left([X_1, X_2]_p\right) = \left[ Y_1, Y_2 \right]_{f(p)}.
\]
\begin{proof}
    Übung.
\end{proof}

\subsection{Vektorfelder und Differentialgleichungen}

\underline{Aufgabe / Fragestellung:}  
\begin{itemize}
  \item Sei \(\mathcal{M}\) eine Mannigfaltigkeit. Finde einen Diffeomorphismus \(\phi: \mathcal{M} \to \mathcal{M}\) mit $\phi\neq \mathrm{id}$
  \item Sei \(\mathcal{M}\) weg-zusammenhängend und \(p \neq q \in \mathcal{M}\). Finde einen Diffeomorphimus \(\phi: \mathcal{M} \to \mathcal{M}\) mit \(\phi(p) = q\)
\end{itemize}

\vspace{0.3cm}

Wir wollen beide Fragen mit Hilfe von Vektorfeldern beantworten.\\

\textbf{Definition 3.16:}  
Eine differenzierbare Kurve \(\gamma: (-\varepsilon, \varepsilon) \to \mathcal{M}\) heißt \emph{\underline{Integralkurve}} des Vektorfeldes \(X \in \Gamma(T\mathcal{M})\) durch \(p\), falls
\begin{align*}
\gamma(0) &= p \tag{1} \\
(D\gamma)_t(\deldel{}{t})=\dot{\gamma}(t) &= X_{\gamma(t)}\in T_{\gamma(t)}\mathcal{M} \quad \forall\, t \in (-\varepsilon, \varepsilon) \tag{2}
\end{align*}

\vspace{0.5em}

\underline{Fragen:} Gegeben ein Vektorfeld \(X\) auf \(\mathcal{M}\).
\begin{itemize}
  \item Gibt es für jedes \(p \in \mathcal{M}\) ein \(\varepsilon = \varepsilon(p) > 0\) und eine Integralkurve \(\gamma: (-\varepsilon, \varepsilon) \to \mathcal{M}\) durch \(p\)?
  \item Was ist das größtmögliche \(\varepsilon\)?
  \item Wie ändert sich die Integralkurve, wenn wir \(p\) verschieben?
  \item Können zwei verschiedene Integralkurven durch denselben Punkt verlaufen?
\end{itemize}
  \begin{figure}[H]
    \centering
    \includegraphics[width=8cm]{Image Diffgeo/7.04.jpg}
	\caption{Integralkurve auf einem Vektorfeld}
 \end{figure}
\textbf{Zunächst lokal:}

Sei \((U, \varphi)\) eine Karte um \(p \in \mathcal{M}\).  
Wir schreiben \(X\) auf \(U\) als
\[
X|_U = \sum_{i=1}^m a_i \frac{\partial}{\partial x_i}
\]
mit glatten Funktionen \(a_i: U \to \mathbb{R}\). (Bemerkung Section 3.2 oben)\\

\textbf{Erinnerung:}
$\frac{\partial}{\partial x_1}, \dots, \frac{\partial}{\partial x_m}$
ist die Basis des Tangentialraums, induziert von der Karte $\varphi$. Gegeben durch:
$\left[ \varphi^{-1} \left( \varphi(p) + t \cdot e_i \right) \right]$


\vspace{0.3cm}

Sei \(\gamma: (-\varepsilon, \varepsilon) \to \mathcal{M}\) eine Kurve durch \(p \in \mathcal{M}\).  
Für \(\varepsilon\) klein genug liegt das Bild der Kurve \(\gamma\) in \(U\) (da \(\gamma\) stetig ist).  
Auch den Tangentialvektor \(\dot{\gamma}(t)\) können wir auf \(U\) durch \(\frac{\partial}{\partial x_1}, \dotsc, \frac{\partial}{\partial x_m}\) darstellen:

\[
\dot{\gamma}(t) = \sum_{i=1}^m b_i(t) \left. \frac{\partial}{\partial x_i} \right|_{\gamma(t)}
\]

für glatte \(b_i: (-\varepsilon, \varepsilon) \to \mathbb{R}\).

\vspace{0.3cm}

Wir wollen die Funktion $b_i$ mit Hilfe von \(\gamma\) und \(\varphi\) ausdrücken.

In der Karte \(\varphi\) gilt:
\[
(\varphi \circ \gamma)(t) = (\gamma_1(t), \dotsc, \gamma_m(t)) : (-\varepsilon, \varepsilon) \to \mathbb{R}^m,
\]
d.\,h. \(\gamma_i = \underbrace{pr_i \circ \varphi}_{=:\varphi_i} \circ \gamma\), wobei \(pr_i\) die \(i\)-te Koordinate auswählt.

{\underline{Erinnerung:}} Zwei Vektorfelder sind genau dann gleich, wenn sie die gleichen Derivationen definieren.

\[b_j(t_0) = \sum_{i=1}^m b_i(t_0) \underbrace{\left. \frac{\partial}{\partial x_i} \right|_{\gamma(t_0)} (\varphi_j)}_{\delta_{ij}}
= \dot{\gamma}(t_0)(\varphi_j) = \left. \frac{d}{dt} \right|_{t=t_0} \varphi_j(\gamma(t))
= \dot{\gamma}_j(t_0)
\]
\[
\Rightarrow \quad \dot{\gamma}(t) = \sum_{i=1}^m \dot{\gamma}_i(t) \left. \frac{\partial}{\partial x_i} \right|_{\gamma(t)}
\]

Dadurch wird Gleichung (2): 
\[
\dot{\gamma}(t) = X_{\gamma(t)} \quad \text{zu}
\]
\[
\dot{\gamma}_i(t) = a_i(\gamma(t)) = (a_i \circ \varphi^{-1})(\varphi(\gamma(t)))
= (a_i \circ \varphi^{-1})(\gamma_1(t), \ldots, \gamma_m(t))
\quad \text{für } 1 \leq i \leq m.
\]

Das ist ein System von \(m\) gekoppelten gewöhnlichen Differentialgleichungen erster Ordnung:
\[
\dot{\gamma}_1(t) = (a_1\circ\varphi^{-1})(\gamma_1(t),...,\gamma_m(t)) \\ 
\]
\[
\vdots \\
\]
\[
\dot{\gamma}_m(t) = (a_m\circ\varphi^{-1})(\gamma_1(t),...,\gamma_m(t))
\]
für die Funktionen
\[
\gamma_i: (-\varepsilon, \varepsilon) \to \mathbb{R}, \quad i = 1, \dotsc, m,
\]
mit den Anfangsbedingung
\[
\gamma_i(0) = \varphi_i(p).
\]

\textbf{Bemerkung:}  
Jedes System von \(m\) gekoppelten Differentialgleichungen erster Ordnung entspricht der Differentialgleichung für Integralkurven eines Vektorfeldes auf (einer Teilmenge von) \(\mathbb{R}^m\).\\


Aus dem Satz von Picard–Lindelöf folgt:

\textbf{Satz 3.17 (lokale Existenz und Eindeutigkeit):}  
Sei \(\mathcal{M}\) eine Mannigfaltigkeit und \(X \in \Gamma(T\mathcal{M})\) ein Vektorfeld.  
Für alle Punkte \(p \in \mathcal{M}\) existiert ein offenes Intervall \(I_p \subset \mathbb{R}\) um \(0\) und eine eindeutig bestimmte Kurve
\[
\gamma_p: I_p \to \mathcal{M}
\]
mit
\[
\gamma_p(0) = p, \quad \dot{\gamma}_p(t) = X_{\gamma(t)} \quad \forall t \in I_p,
\]
d.h. die Integralkurven von \(X\) existieren lokal und sind (lokal) eindeutig bestimmt.

\vspace{0.5em}

\textbf{Bemerkung:}  
Im Allgemeinen kann man nicht \(I_p = \mathbb{R}\) wählen, d.h. Integralkurven müssen nicht für alle Zeiten definiert sein.


\textbf{Beispiele:}

\begin{itemize}
  \item[(i)] \(\mathcal{M} = \mathbb{R}^2,\quad X(p) = (1,0)\)  
  
  \textit{Integralkurven:} \(\gamma_p(t) = p+te_1\)
  
\includegraphics[width=6cm]{Image Diffgeo/7.95.png}

  \item[(ii)] \(\mathcal{M} = \mathbb{R}^2,\quad X(p) = p\) , das radial Vektorfeld. 
  
  \textit{Integralkurven:} \(\dot\gamma_p(t) = e^tp=X(e^tp)=X(\gamma_p(t))\)

 \includegraphics[width=6cm]{Image Diffgeo/7.96.png}


  \item[(iii)] \(\mathcal{M} = \mathbb{R}^2,\quad X(p) = X(p_1, p_2) = (-p_2,p_1) =ip\)
  
  \textit{Integralkurven:} \(\gamma_p(t) = e^{it}p = \left( \begin{array}{c} p_1\cos(t)-p_2\sin(t) \\ p_1\sin(t)+p_2\cos(t) \end{array} \right)\)

  
  \includegraphics[width=6cm]{Image Diffgeo/7.97.png}

\end{itemize}


\textit{Test für (iii):}  
\[
\gamma_p(0) = \left( p_1\cdot1-p_2\cdot0, p_1\cdot0+p_2\cdot1 \right) = (p_1, p_2)
\]
\[
X(\gamma_p(t)) = \left( -p_1\sin(t)-p_2\cos(t), p_1\cos(t)-p_2\sin(t) \right)
\]
\[
\dot{\gamma}_p(t) = \left( -p_1\sin(t)-p_2\cos(t), p_1\cos(t)-p_2\sin(t) \right) = X(\gamma_p(t))
\]

\begin{itemize}
  \item[(iv)] \(\mathcal{M} = \mathbb{R}^2 \setminus \{0\}, \quad X(p) = (1,0)\)

  \textit{Integralkurven:} \(\gamma_p(t) = p + (t, 0)\), aber für \(p = (-1,0)\) existiert die Integralkurve nur für \(t \in (-\infty,1)\)

  \item[(v)] \(\mathcal{M} = \mathbb{R}^2, \quad X(p) = (p_y^2, p_x^2)\)

  \textit{Die Integralkurve durch \(p_0 = (1,1)\) ist:} \(\gamma_{p_0}(t) = \left( \frac{1}{1-t}, \frac{1}{1-t} \right)\)

  \textit{Existiert nur für } \(t \in (-\infty, 1)\)
\end{itemize}


\textbf{Korollar 3.18 (Globale Eindeutigkeit):}  
Durch jeden Punkt \(p\) der Mannigfaltigkeit \(\mathcal{M}\) verläuft genau eine Integralkurve von \(X\), d.h. insbesondere, dass sich Integralkurven nur schneiden, wenn sie identisch sind.

\includegraphics[width=6cm]{Image Diffgeo/7.98.png}

Aus dem Satz über die differenzierbare Abhängigkeit von Lösungen folgt:\\

\textbf{Satz 3.19 (Existenz eines lokalen Flusses):}  
Für alle \(p \in \mathcal{M}\) existiert eine offene Umgebung \(U\) von \(p\) und ein nicht-leeres offenes Intervall \(I\) um \(0\), so dass für alle \(q \in U\) die Kurve \(\gamma_q\) auf \(I\) definiert ist.  
Die Abbildung
\[
I \times U \longrightarrow \mathcal{M}, \quad (t, q) \longmapsto \gamma_q(t)
\]
ist differenzierbar.

\textit{Insbesondere können wir für alle \(q\) in einer kleinen offenen Menge \(U\) die Intervalle \(I_q\) gleich einem festen Intervall \(I\) wählen.}

\includegraphics[width=6cm]{Image Diffgeo/7.99.png}

\textbf{Definition 3.20:}  
Die Abbildung \((t, q) \mapsto \gamma_q(t)\) heißt der \underline{\emph{lokale Fluss}} des Vektorfeldes \(X\).  
Da Integralkurven von \(X\) heißen \underline{\emph{Flusslinien}} von \(X\). Der Fluss definiert für alle hinreichend kleinen Parameter \(t \in I\) eine lokale Abbildung
\[
\varphi_t: U \subset \mathcal{M} \to \mathcal{M}, \quad q \mapsto \varphi_t(q)=\gamma_q(t),
\]
den sogenannten \underline{\emph{Zeit-\(t\)-Fluss}}.

\vspace{1em}

\textbf{Satz 3.21:}  
Die Abbildungen \(\varphi_t\) sind lokale Diffeomorphismen von \(U\) auf \(\varphi_t(U)\) und es gilt
\[
\varphi_{t+s} = \varphi_t \circ \varphi_s
\]
für alle Parameter \(t,s \in I\), für die \(t+s \in I\).  
Insbesondere kommutieren die lokalen Diffeomorphismen \(\varphi_t\) und \(\varphi_s\).  
Man sagt: \(\{\varphi_t\}\) ist eine \emph{1-parametrige Gruppe (lokaler) Diffeomorphismen}.

Außerdem gilt:
\[
(\varphi_t)_* X|_U = X|_{\varphi_t(U)} \quad \forall t \in I.
\]

\begin{proof}
    Die Abbildungen
\[
t \mapsto \gamma_q(s + t) \quad \text{und} \quad t \mapsto \gamma_{\varphi_s(q)}(t)
\]
sind Integralkurven von \(X\) mit Anfangspunkt
\[
\gamma_q(s) = \varphi_s(q)=\gamma_{\varphi_s(q)}(0).
\]
\[
\varphi_{t+s}(q) = \gamma_q(t + s) \underbrace=_{\txt{Eindeutigkeit von Flusskurven}} \gamma_{\varphi_s(q)}(t) = \varphi_t(\varphi_s(q))=\varphi_t\circ\varphi_s(q).
\]

Insbesondere folgt aus dieser Gleichung:
\[
\mathrm{id} = \varphi_0 = \varphi_{t + (-t)} = \varphi_t \circ \varphi_{-t}, \quad \text{also} \quad \varphi_t^{-1} = \varphi_{-t}.
\]

Also ist \(\varphi_t\) invertierbar und die Umkehrabbildung ist wieder differenzierbar, d.h. \(\varphi_t\) ist ein (lokaler) Diffeomorphismus.


\textbf{Bleibt zu zeigen:}  
\[
(\varphi_t)_* X|_U = X|_{\varphi_t(U)} \quad \text{(d.h. } ((\varphi_t)_* X)_p = X_{\varphi_t(p)})
\]

\textbf{Beobachtung:}  
\[
(\psi_*X)_p = (D\psi)_{\psi^{-1}(p)}(X_{\psi^{-1}(p)}) \text{ ist äquivalent zu } (\psi_*X)_{p}  = (D\psi)_p(X_p) \quad (p\leftrightarrow \psi(p))
\]

\textbf{Wollen zeigen:} ($\psi \leftrightarrow \varphi_t$)
\[
((\varphi_t)_* X)_{p} = (D \varphi_t)_p (X_p) \overset{?}{=} X_{\varphi_t(p)}
\]

\[
(D\varphi_t)_p (X_p)
= \left. \frac{d}{ds} \right|_{s=0} \varphi_t(\gamma_p(s))
= \left. \frac{d}{ds} \right|_{s=0} \varphi_t(\varphi_s(p))
= \left. \frac{d}{ds} \right|_{s=0} \varphi_{t+s}(p)
\]
\[= \left. \frac{d}{d\tau} \right|_{\tau=t} \varphi_\tau(p)= X_{\varphi_t(p)}\]
\end{proof}

%%%%%%%%%%%%%%%%%%%%%%%%%%%%%%%%%%%%%%%%%%%%%%%%%%%%%%%%%%%%%%%%%%%%%%%%%%%%%%%%%% Vorlesung 8 %%%%%%%%%%%%%%%%%%%%%%%%%%%%%%%
\newpage

\underline{In der letzten Vorlesung:}
Sei $M$ eine Mannigfaltigkeit, $X$ ein Vektorfeld auf $M$ und $p \in M$. Dann gibt es ein $\varepsilon > 0$ und eine Kurve
\(
\gamma_p: (-\varepsilon, \varepsilon) \rightarrow M
\)
mit 
\[
\gamma_p(0) = p, \quad \dot{\gamma}_p(t) = X_{\gamma(t)}  \quad\quad \txt{Existenz} \;(*)
\]
Diese Kurve ist bis auf die Wahl des Intervalls eindeutig, d.h. sind 
\[
\gamma_p: (a, b) \rightarrow M \quad \text{und} \quad \tilde{\gamma}_p: (\tilde{a}, \tilde{b}) \rightarrow M
\]
zwei Kurven, die (*) erfüllen, dann stimmen $\gamma_p$ und $\tilde{\gamma}_p$ auf dem Schnitt der Intervalle $(a, b) \cap (\tilde{a}, \tilde{b})$ überein. (Eindeutigkeit)\\
Auf dem maximalen Definitionsintervall ist die Kurve eindeutig. Dadurch können wir den Fluss definieren.\\

\includegraphics[width=14cm]{Image Diffgeo/8.01.png}

\underline{Gegeben:} $M$ Mannigfaltigkeit, $p \in M$ und $X$ Vektorfeld.

Dann gibt es eine offene Umgebung $U$ von $p$ und ein nicht-leeres offenes Intervall $I$, s.d. $\gamma_q$ 
für alle $q \in U$ und alle $t \in I$ definiert ist.

Die Abbildung 
\[
\varphi: I \times U \longrightarrow M, \quad (t, q) \longmapsto \gamma_q(t)
\]
heißt der \underline{(lokale) Fluss} von $X$ und $\gamma_q(t)$ ist die Lösung von
\[
\begin{cases} 
\gamma_q(0) = q \\ 
\dot{\gamma}_q(t) = X_{\gamma_q(t)}
\end{cases}
\]

Wird $q \in U$ festgehalten, bekommen wir die Flusskurve von $X$ durch $q$:
\[
\gamma_q: I \longrightarrow M
\]
Wird $t \in I$ festgehalten, bekommen wir den Zeit-$t$-Fluss:
\[
\varphi_t: U \longrightarrow M, \quad q \longmapsto \varphi_t(q) = \gamma_q(t)
\]

\textbf{Beispiel:}

Sei $\mathcal{M} = \mathbb{R}$ und $X_x = x^2 e_1 = x^2 \frac{\partial}{\partial x} \quad (\text{für alle } x \in \mathbb{R})$. Die Integralkurve von $X$ durch $x_0$ ist gegeben durch
\[
\gamma_{x_0}(t) = \frac{x_0}{1 - t x_0}.
\]

\includegraphics[width=10cm]{Image Diffgeo/8.02.png}


\underline{Test:} 
\[
\gamma(0) = x_0, \quad \dot{\gamma}(t) \overset{\txt{identify $T_p\mathbb{R}$ mit $\mathbb{R}$}}{=} 
\frac{-x_0}{(1 - t x_0)^2} \cdot (-x_0) = \frac{x_0^2}{(1 - t x_0)^2} = \gamma(t)^2 = X_{\gamma(t)}
\]

Die Integralkurve von $X$ durch den Startpunkt $x$ ist maximal für die folgende Intervalle $I_x$ um 0 definiert:
\[
I_x = 
\begin{cases} 
(-\infty, \frac{1}{x}) & \text{, falls } x > 0 \\ 
(\frac{1}{x}, \infty) & \text{, falls } x < 0 \\ 
(-\infty, \infty) & \text{, falls } x = 0 
\end{cases}
\]

\includegraphics[width=10cm]{Image Diffgeo/8.03.png}

Auf $U = (-a, a)$ mit $a > 0$, ist der lokale Fluss $\varphi_t$ für alle $t \in I= \left(-\infty, \frac{1}{a}\right) \cap \left(-\frac{1}{a}, \infty \right) = \left(-\frac{1}{a}, \frac{1}{a}\right)$ definiert. Das Intervall wird klein, wenn $a$ groß wird.

\includegraphics[width=10cm]{Image Diffgeo/8.04.png}
\subsubsection*{Definition 3.22}

Sei $M$ ein topologischer Raum und $f: M \rightarrow \mathbb{R}^n$ eine Abbildung, dann heißt die Menge
\[
\operatorname{supp}(f) := \overline{\left\{ x \in \mathcal{M} \mid f(x) \neq 0 \right\}}
\]
der \underline{Träger} von $f$.

\vspace{1em}

\textbf{Bemerkung:} Wir sagen, dass ein Vektorfeld $X \in \Gamma(TM)$ einen \underline{kompakten Träger} hat, falls
\[
\operatorname{supp}(X) =\overline{ \{ p \in M \mid X_p \neq 0 \}}
\]
kompakt ist.\\

\textbf{Lemma 3.23:}

Sei $X$ ein Vektorfeld und 
\[
\varphi : \mathbb{R} \times M \longrightarrow M, \quad (t, p) \longmapsto \varphi_t(p)
\]
eine differenzierbare Abbildung mit
\[
\varphi_0 = \varphi(0, \cdot) = \operatorname{id}_M
\]
\[
\varphi_t \circ \varphi_s =\varphi_{t+s} \quad \forall \, s, t \in \mathbb{R}
\]
\[
\frac{d}{dt} \varphi_t(p) = X_p \quad \forall \, p \in M
\]
\[
(\varphi_t)_\ast X = X \quad \forall \, t \in \mathbb{R}
\]

Dann ist $\{\varphi_t\}$ der (globale) Fluss von $X$. (Existiert für alle $t \in \mathbb{R}$)

\begin{proof}
    Wir müssen zeigen, dass 
\(\gamma_p(t) = \varphi_t(p) = \varphi(t, p) \quad \text{eine Integralkurve von } X \text{ ist.}\)
\[
\dot{\gamma}_p(s) \overset{\txt{Def. von }\gamma}{=} \frac{d}{dt} \big|_{t = s}\varphi_t(p) \overset{\txt{2. Eigenschaft}}{=} \frac{d}{dt} \big|_{t = s} \varphi_s\circ\varphi_{t-s}(p)\overset{\tau:=t-s}{=} \frac{d}{d\tau} \big|_{\tau = 0}\varphi_s\circ\varphi_\tau(p)
\]

\[
\underset{\txt{3. Eigenschaft}}{\overset{\txt{Kettenregel }\varphi_s \;\txt{Abb. } M\rightarrow M}{=}}D\varphi_s\circ(X_p) \underset{+ \txt{Def. Push-forward}}{\overset{\txt{4. Eigenschaft}}{=}} X_{\varphi_s(p)}=X_{\gamma_p(s)}
\]

\[
\gamma_p(0) = \varphi_0(p) = p 
\quad \text{(1. Eigenschaft)}
\]
\end{proof}

\textbf{Satz 3.24:}

Ist $X$ ein Vektorfeld mit kompaktem Träger, so ist der Fluss $\varphi_t$ für alle $t \in \mathbb{R}$ definiert:
\[
\varphi_t: \mathbb{R} \times M \longrightarrow M, \quad (t, p) \longmapsto \varphi_t(p)
\]
und für jedes $t \in \mathbb{R}$ ist $\varphi_t: M \longrightarrow M$ ein globaler Diffeomorphismus.\\

\vspace{1em}

\textbf{Bemerkung:} 
\begin{itemize}
    \item Die Voraussetzung des Satzes ist für kompakte Mannigfaltigkeiten erfüllt.
    \item Ein Vektorfeld $X$, dessen lokaler Fluss für alle $t \in \mathbb{R}$ definiert ist, heißt \underline{vollständig}.
\end{itemize}

\vspace{1em}
\begin{proof}
\quad

Laut Satz 3.19 existiert für alle $p \in M$ eine offene Umgebung $V$ von $p$ und ein Intervall $I = (-\varepsilon, \varepsilon)$, so dass 
 der Fluss $\varphi_t(q)$
\(\text{für alle } q \in V, \, t \in I\) definiert ist

Da der Träger $\operatorname{supp}(X)$ von $X$ kompakt ist, reichen endlich viele der V's um diesen zu überdecken

\vspace{0.5em}

Wir bezeichnen diese offene Menge mit $V_1,...,V_r$.
\[
\implies X_q = 0 \quad \text{für alle } q \in M \setminus \bigcup_{i=1}^r V_i
\]

Durch die Punkte $q \in M$ mit $X_q = 0$ existiert immer die konstante Integralkurve $\gamma(t)=q$, $\forall t \in \mathbb{R}$. 

Seien $(-\varepsilon_i, \varepsilon_i)$ die zugehörigen Definitionsbereiche der Flüsse auf $V_i$ und $\epsilon:=\mathrm{min}\{\epsilon_i\}$

Dann ist $\varepsilon > 0$.

Der Fluss $\varphi_t(q)$ ist definiert für alle $q \in M$ und $t \in (-\varepsilon, \varepsilon)$. Der Fluss ist eine Abbildung
\[
(-\varepsilon, \varepsilon) \times M \longrightarrow M, \quad (t, p) \longmapsto \varphi_t(p).
\]
Für $t \in \mathbb{R}$ schreibe $|t| = k\frac{\epsilon}{2}+r $ mit eindeutig bestimmten $k \in \mathbb{N}_0$ und $r \in [0, \frac{\varepsilon}{2})$.\\

Wir definieren
\[
\hat{\varphi}_t := 
\begin{cases} 
\left(\varphi_{\frac{\varepsilon}{2}}\right)^k \circ \varphi_r
 & \text{für } t \geq 0 \\[1em]
\left(\varphi_{-\frac{\varepsilon}{2}}\right)^k \circ \varphi_{-r}
 & \text{für } t < 0 
\end{cases}
\]

\includegraphics[width=13cm]{Image Diffgeo/8.05.png}

Durch diese Definition bekommen wir eine Abbildung
\[
\hat{\varphi} : \mathbb{R} \times M \longrightarrow M, \quad (t, p) \longmapsto \hat{\varphi}(t, p) := \hat{\varphi}_t(p)
\]

\vspace{1em}

Für $\lvert t \rvert \leq \frac{\varepsilon}{2}$ gilt $k = 0$, d.h. $\lvert t \rvert = r$ und $\hat{\varphi}_t$ stimmt mit dem ursprünglichen $\varphi_t$ überein.

Für $\frac{\varepsilon}{2} \leq \lvert t \rvert < \varepsilon$ gilt $k = 1$ und 
\[
\hat{\varphi}_t = 
\begin{cases} 
  \varphi_{\frac{\varepsilon}{2}} \circ \varphi_r = \varphi_t
  & \text{für } t > 0 \\[1em]
  \varphi_{-\frac{\varepsilon}{2}} \circ \varphi_{-r} = \varphi_t
  & \text{für } t < 0 
\end{cases}
\]

\vspace{1em}

Die Gruppengesetze dürfen wir anwenden, da 
\(
\frac{\varepsilon}{2} + r,\; \frac{-\varepsilon}{2} - r \in (-\varepsilon, \varepsilon).
\)

\vspace{1em}

Insgesamt folgt, dass $\hat{\varphi}$ eine Fortsetzung des auf $(-\varepsilon, \varepsilon)$ definierten $\varphi_t$ zu einer Abbildung auf ganz $\mathbb{R}$ ist.

Wir überprüfen die Bedingungen aus Lemma 3.23, um zu zeigen, dass $\hat{\varphi}_t$ der globale Fluss von $X$ ist.
\paragraph{1.} $\hat{\varphi}_0(\cdot) = \varphi_0(\cdot) = \operatorname{id}_M$ \hspace{1em} \textcolor{green}{\checkmark}

\paragraph{2.} $\hat{\varphi}_t \circ \hat{\varphi}_{t'} = \hat{\varphi}_{t + t'} \quad \forall \, t, t' \in \mathbb{R}$

Seien $t, t' \in \mathbb{R}$, zum Beispiel $t, t' > 0$. Schreiben wir dann:
\[
t = k \cdot \frac{\varepsilon}{2} + r, \quad t' = k' \cdot \frac{\varepsilon}{2} + r'
\]
\[
\implies t + t' = 
\begin{cases} 
    (k + k') \frac{\varepsilon}{2} + (r + r') & \text{, falls } r + r' \in [0, \frac{\varepsilon}{2}) \\[1em]
    (k + k' + 1) \frac{\varepsilon}{2} + (r + r' - \frac{\varepsilon}{2}) & \text{, falls } r + r' \in [\frac{\varepsilon}{2}, \varepsilon)
\end{cases}
\]
Es folgt
\[
\hat{\varphi}_t \circ \hat{\varphi}_{t'} = \left(\varphi_{\frac{\varepsilon}{2}}\right)^k \circ \varphi_r \circ \left(\varphi_{\frac{\varepsilon}{2}}\right)^{k'} \circ \varphi_{r'}
\] 

\hspace{2em} \textcolor{orange}{schrittweises Anwenden der Gruppengesetze (Verkettungen), da $r + \frac{\varepsilon}{2} \in (-\varepsilon, \varepsilon)$. Vgl. oben.}
\[
= \left(\varphi_{\frac{\varepsilon}{2}}\right)^k \circ \left(\varphi_{\frac{\varepsilon}{2}}\right)^{k'} \circ \varphi_r \circ \varphi_{r'}= \left(\varphi_{\frac{\varepsilon}{2}}\right)^{k + k'} \circ \varphi_{r + r'}
\]
\hspace{2em} \textcolor{orange}{evtl. nach dem Ausklammern eines weiteren $\varphi_{\frac{\varepsilon}{2}}$ aus $\varphi_{r+r'}$.}
\[= \hat{\varphi}_{t + t'}
\]
\paragraph{3.} $(\hat{\varphi}_t)_\ast X = X$

Sei $t \in \mathbb{R}$, z.B. $t = k \frac{\varepsilon}{2} + r > 0$. Da
\((\varphi_s)_\ast X = X \;\; \forall \, s \in (-\varepsilon, \varepsilon)\), (Satz 3.21)
folgt
\[
(\hat{\varphi}_t)_\ast X = \left((\varphi_{\frac{\varepsilon}{2}})^k \circ \varphi_r \right)_\ast X = \left((\varphi_{\frac{\varepsilon}{2}})_\ast\right)^k \circ \underbrace{(\varphi_r)_\ast X}_{= X} = \left((\varphi_{\frac{\varepsilon}{2}})_\ast\right)^{k-1} \underbrace{(\varphi_{\frac{\varepsilon}{2}})_\ast X}_{=X}=\dots=X
\]
\end{proof}


\subsection{Lie-Ableitung von Vektorfeldern}

Sei $X$ ein glattes Vektorfeld auf $M$ mit dem lokalen Fluss $\varphi_t: M \rightarrow M$.
Dann ist $X$ (nach Definition) tangential zu den Flusslinien 
\(\gamma_p(t) = \varphi_t(p),\; \text{daraus folgt}\)
\[
X_p(f)=\mathcal{L}_X (f)_p = (Df)_p(X_p) = \frac{d}{dt} \big|_{t=0} f(\gamma_p(t)) = \frac{d}{dt} \big|_{t=0} f(\varphi_t(p))
\]
\[
\overset{\varphi_0=\mathrm{id}}{=} \lim_{t \to 0} \frac{1}{t} \left(f(\varphi_t(p)) - f(p)\right) = \lim_{t \to 0} \frac{1}{t} \left((\varphi_t^* f)(p) - f(p)\right)
\]
\textcolor{orange}{(Lie-Ableitung der Funktion $f \in C^\infty(M)$ in Richtung $X$.)}

\vspace{1em}

wobei $\varphi_t^\ast f:= f\circ\varphi $ das Zurückziehen (\underline{Pull-Back}) der Funktion $f \in C^\infty(M)$ durch $\varphi$ ist.
\[
\Phi: M \overset{\Phi\;\txt{glatt}}{\rightarrow} N \rightarrow \mathbb{R}
\]
\[
\Gamma(TM) \overset{\txt{Push-forward}}{\rightarrow} \Gamma(TN) \quad X\rightarrow\Phi_*X
\]
\[
\Gamma(TM) \overset{\txt{Pull-Back}}{\leftarrow} \Gamma(TN) \quad \Phi^*Y=(\Phi^{-1})_*Y \leftarrow Y
\]

Durch (lokale) Flüsse können die Lie-Ableitungen anderer Objekte beschrieben werden.

\vspace{1em}

Das Zurückziehen eines Vektorfeldes mittels der lokalen Diffeomorphismen $\varphi_t$ ist definiert als ($X$ und $Y$ sind zwei Vektorfelder auf $M$)
\[
(\varphi_t^\ast Y)_p = ((\varphi_{-t})_* Y)_p = \underbrace{(D \varphi_{-t})_{\varphi_t(p)}}_{
T_{\varphi_t(p)} M \longrightarrow T_p M}\, \underbrace{Y_{\varphi_t(p)}}_{T_{\varphi_t(p)} M }
\quad \in {T_pM}
\]

\includegraphics[width=12cm]{Image Diffgeo/8.06.png}

\vspace{1em}
\textbf{Satz 3.25:}

Seien $X, Y$ glatte Vektorfelder auf $M$. Dann gilt in $p \in M$:
\[
\mathcal{L}_X Y_p = \lim_{t \to 0} \frac{1}{t} \left[(\varphi_t^\ast Y)_p - Y_p\right] 
\overset{\textcolor{red}{(*)}}{=} \lim_{t \to 0} \frac{1}{t} \left[Y_p - (\varphi_{t})_\ast Y|_p \right] 
\overset{\textcolor{red}{(**)}}{=} [X, Y]_p 
\]
\begin{proof}
\quad

\underline{Zu \textcolor{red}{$(*)$}}:
Per Definition
\[
(\varphi_t^\ast Y)_p - Y_p =  ((\varphi_{-t})_* Y)_p - Y_p = (\varphi_{-t})_*  (Y - (\varphi_t)_* Y)_p
\]
\[
\implies \lim_{t \to 0} \frac{1}{t} \left[(\varphi_t^\ast Y)_p - Y_p\right] \overset{\txt{Def. Pull-back}}{=} \lim_{t \to 0} (\varphi_{-t})_* \frac{1}{t} \left(Y_p - ((\varphi_t)_* Y)_p \right) \overset{(\varphi_t^\ast Y)_p|_{t=0}}{=} \underbrace {(\varphi_0)_*}_{id} \lim_{t \to 0} \frac{1}{t} \left(Y_p - (\varphi_t)_* Y_p \right) 
\]


\underline{Zu \textcolor{red}{$(**)$}}: Sei $f: M \rightarrow \mathbb{R}$ eine glatte Funktion. Wir definieren:
\[
F(t, p) := f(\varphi_t(p))-f(p)
\]
Aus Lemma 2.9 folgt: Es gibt eine glatte Funktion $g$ mit $F(t, p) = tg(t,p) $, insbesondere
\[
g(0, p) = \lim_{t \to 0} g(t, p) = \lim_{t \to 0} \frac{1}{t} \left(f(\varphi_t(p)) - f(p)\right) = \frac{d}{dt} \bigg|_{t=0} f(\varphi_t(p)) = X_p(f) = (\mathcal{L}_X f)_p
\]

Wir können nachrechnen
\[
(\varphi_t^\ast Y)_p(f) \overset{\txt{Def. }\varphi_t^\ast Y}{=} (D \varphi_{-t})_{\varphi_t(p)} \, Y_{\varphi_t(p)}(f)  \overset{\txt{Kettenregel}}{=} Y_{\varphi_t(p)}(f \circ \varphi_{-t})  \overset{\txt{Def. von } F}{=}Y_{\varphi_t(p)}(F( - t, \cdot) + f)
\]
\[
\underset{\txt{Def. von }g}{\overset{\txt{Linearität der Ableitung}}{=}}  -t\,Y_{\varphi_t(p)}(g(-t, \cdot) )+Y_{\varphi_t(p)}(f)   =(Y(f))_{\varphi_t(p)} - t \, Y_{\varphi_t(p)}(g(-t, \cdot))
\]


Für den Grenzwert folgt also:
\[
\Rightarrow \lim_{t \to 0} \frac{1}{t} \left((\varphi_t^* Y)_p(f) - Y_p(f)\right) = \lim_{t \to 0} \frac{1}{t} \left(Y(f)_{\varphi_t(p)} - Y(f)_p \right) - \lim_{t \to 0} Y_{\varphi_t(p)}(g(-t, \circ))
\]
\[= \frac{d}{dt}\big|_{t=0} Y(f)(\varphi_t(p)) - Y_p(X(f)) = X_p(Y(f)) - Y_p(X(f)) = [X, Y]_p(f)
\]
\end{proof}
%%%%%%%%%%%%%%%%%%%%%%%%%%%%%%%%%%%%%%%%%%%%%%%%%%%%%%%%%%%%%%%%%%%%%%%%%%%%%%%%%%%%%%%% Vorlesung 9 %%%%%%%%%%%%%%%%%%%%%%%%%

\subsection{Ergänzende Informationen zu Vektorfeldern}
\textbf{Satz 3.26:}
Sei $X$ ein Vektorfeld auf $M$ mit $X_p \neq 0$ für ein $p \in M$. Dann existiert eine Karte $(U, \varphi)$ um $p$ mit 
\[
X = \frac{\partial}{\partial x_1} \quad \text{auf } U. \quad (\text{m.a.W. } \;\; \varphi_\ast X = e_1)
\]
  \begin{figure}[H]
    \centering
    \includegraphics[width=12cm]{Image Diffgeo/9.01.png}
	%\caption{Derivationen identifizieren mit Basisvektoren}
 \end{figure}
\begin{proof}
    Da es sich um eine lokale Aussage handelt, können wir o.E. annehmen, dass $M$ eine offene Teilmenge von $\mathbb{R}^m$ ist. \\
Wir nehmen an, dass die Aussage für offene Mengen von $\mathbb{R}^m$ korrekt ist. Sei 
\(
\tilde{\varphi}: \tilde{U} \rightarrow \tilde{V}
\)
eine Karte um $p$. Dann ist 
\(
\tilde{V}\subset \mathbb{R}^m \;\txt{offen und }\tilde{\varphi}_*X\; \text{ein Vektorfeld auf } \tilde{V} \text{ mit } 
\)
\[(\tilde{\varphi}_*X)_{\varphi(p)}\neq 0\]

Nach Annahme gibt es eine Karte $(V', \psi)$ um $\tilde{\varphi}(p)$ mit 
\[
\psi_*(\tilde{\varphi}_*X)=\deldel{}{x_1}\; \text{auf } \psi(V').
\]

Nach Einschränken $\tilde{V} = V'$ Die gesuchte Karte um $p$ ist 
\[
\varphi = \psi \circ \tilde{\varphi}, \quad (\varphi_\ast X) = \psi_*\tilde{\varphi_*}X = \psi_*(\tilde{\varphi_*}X) = \deldel{}{x_1}
\]
\underline{Zum Beweis für $\mathbb{R}^n$:} Sei $X$ ein Vektorfeld mit $X_p \neq 0$.

Durch Anwenden einer linearen Transformation und einer Verschiebung können wir 
\[
p = 0 \quad \text{und} \quad X_0 = e_1 = \frac{\partial}{\partial x_1} \big|_0
\]
annehmen.

\underline{Idee:}
  \begin{figure}[H]
    \centering
    \includegraphics[width=9cm]{Image Diffgeo/9.02.jpg}
	%\caption{Derivationen identifizieren mit Basisvektoren}
 \end{figure}

\noindent
Durch den Punkt $(0, a_2, \dots, a_n)$ verläuft genau eine Integralkurve von $X$.

\vspace{1em}

\noindent
Wenn wir $(0, a_2, \dots, a_n)$ vorgeben, ist jeder Punkt auf der Kurve eindeutig durch die Zeit festgelegt, die $(0, a_2, \dots, a_n)$ braucht, um bis zu diesem Punkt zu gelangen.\\

Sei $\varphi_t$ der lokale Fluss von $X$, wir definieren eine Abbildung $\psi$ auf einer hinreichend kleinen Umgebung von $0$ durch
\[
\psi(a_1, \dots, a_n) = \varphi_{a_1}(0,a_2,\dots,a_n)
\]

\underline{Wollen zeigen:} Auf einer hinreichend kleinen Umgebung von $0$ ist $\psi$ ein Diffeomorphismus.

\underline{Hilfsmittel:} Inverser Funktionssatz, es reicht zu zeigen, dass $D\psi|_0$ invertierbar ist.

\vspace{1em}

In $a = (a_1, \dots, a_n)$ gilt:
\[
D \psi \left(\frac{\partial}{\partial x_1} \big|_a \right)(f) = \frac{\partial}{\partial x_1} \big|_a (f \circ \psi) = \lim_{t \to 0} \frac{1}{t} \left[f \left(\psi(a_1 + t, a_2, \dots, a_n)\right) - f(\psi(a_1, \dots, a_n)) \right]
\]
\[
= \lim_{t \to 0} \frac{1}{t} [f (\underbrace{\varphi_{a_1+t}(0, a_2, \dots, a_n)}_{=\varphi_t(\varphi_{a_1}(0,a_2,\dots,a_n))}) - f(\psi(a)) ] = \lim_{t \to 0} \frac{1}{t} \left[f \left(\varphi_t(\psi(a))\right) - f(\psi(a)) \right]
\]
\[\overset{\txt{Lie-Ableitung von Fkt. }p=\psi(a)}{=} X_{\psi(a)}(f) \qquad \forall f \in C^\infty(\mathbb{R}^n)
\]

Analog für $a = 0$ und $i \geq 2$: Das Differential von $\psi$ auf den Basisvektoren $\frac{\partial}{\partial x_i} \big|_0$ ist
\[
D \psi \left(\frac{\partial}{\partial x_i} \big|_0 \right)(f) = \frac{\partial}{\partial x_i} \big|_0 (f \circ \psi) = \lim_{t \to 0} \frac{1}{t} \left[f(\psi(0, \dots, t, \dots, 0)) - f(0)\right]
\]
\[
= \lim_{t \to 0} \frac{1}{t} \left[f(0, \dots, t, \dots, 0) - f(0)\right] = \frac{\partial}{\partial x_i} \big|_0 (f)
\]
Da $X_0 = \frac{\partial}{\partial x_1} \big|_0$, folgt $D\psi_0 = \operatorname{id}$.
\[
\implies \psi \text{ ist ein lokaler Diffeomorphismus und } \varphi = \psi^{-1} \text{ ist die gesuchte Karte um 0.}
\]
In dieser Karte gilt $X = \frac{\partial}{\partial x_1}$, denn
\[
\psi : \left(\mathbb{R}^n, \frac{\partial}{\partial x_1}, \dots, \frac{\partial}{\partial x_n} \right) \longrightarrow \left(\mathbb{R}^n, \frac{\partial}{\partial y_1}, \dots, \frac{\partial}{\partial y_n} \right)
\]
\[
X_{\psi(a)}(f) = D\psi \left(\frac{\partial}{\partial x_1} \big|_{\psi^{-1}(\psi(a))} \right)(f) = \frac{\partial (f \circ \psi)}{\partial x_1} \big|_{\psi^{-1}(\psi(a))} = \frac{\partial f}{\partial y_1} \big|_{\psi(a)}
\]
Denn nach Definition der partiellen Ableitung in einer Karte $(U, \varphi)$ gilt andererseits
\[\frac{\partial}{\partial y_1} \big|_{\psi(a)} (f) 
= \frac{\partial}{\partial x_1} (f \circ \psi) \big|_{\psi^{-1}(\psi(a))}
\]
\end{proof}

\textbf{Lemma 3.27} 

Sei $\psi: M \rightarrow M$ ein Diffeomorphismus und $X$ ein Vektorfeld auf $M$ mit dem lokalen Fluss $\varphi_t$. Dann hat das Vektorfeld $\psi_\ast X$ den Fluss 
\(
\psi \circ \varphi_t \circ \psi^{-1}
\)

\begin{proof}
Sei $f$ ein Funktionskeim um $q \in M$. Dann berechnen wir:
\[
(\psi_* X)_{q}(f) \overset{\txt{Def. Push-forward}}{=} (D \psi)_{\psi^{-1}(q)} \big(X_{\psi^{-1}(q)}\big)(f) = X_{\psi^{-1}(q)}(f \circ \psi)
\]
\[\overset{\varphi_t\txt{ Fluss}}{=} \frac{d}{dt} \big|_{t=0} (f \circ \psi)(\varphi_t(\psi^{-1}(q))) = \frac{d}{dt} \big|_{t=0} f \big(\psi \circ \varphi_t \circ \psi^{-1}(q)\big)\]
\[
\implies \text{Das Vektorfeld } \psi_\ast X \text{ in dem Punkt } q \text{ ist tangential an die Kurve } \psi \circ \varphi_{t} \circ \psi^{-1}(q).
\]
\[
\text{Diese Kurven sind genau die Flusslinien von } \psi_\ast X.
\]
\end{proof}

\textbf{Korollar 3.28}

Sei $\psi: M \rightarrow M$ ein Diffeomorphismus und $X$ ein Vektorfeld auf $M$ mit dem lokalen Fluss $\varphi_t$. Dann gilt 
\(
\psi_\ast X = X \; \text{genau dann, wenn} \; \varphi_t \circ \psi = \psi \circ \varphi_t.
\)
\begin{proof}
    \[
\psi_\ast X = X \;\; \Leftrightarrow \;\;\txt{Der lokale Fluss von}\;\psi_\ast X \txt{ ist gleich den lokalen Fluss von X}
\]
\[
\overset{3.27}{\Leftrightarrow } \;\; \psi \circ\varphi_t\circ\psi^{-1}=\varphi_t\;\Leftrightarrow \;\;\psi\circ\varphi_t=\varphi_t\circ\psi
\]
\end{proof}

\textbf{Lemma 3.29}

Sei $X$ ein Vektorfeld mit dem lokalen Fluss $\varphi_t$ und sei $Y$ ein Vektorfeld mit dem lokalen Fluss $\psi_t$. 
Dann gilt 
\[
[X, Y] = 0 \quad \text{genau dann, wenn} \quad \varphi_t \circ \psi_s = \psi_s \circ \varphi_t
\]

\vspace{1em}

Diese rechte Gleichung soll für alle $s$ und $t$ erfüllt sein, für die die entsprechenden Flüsse definiert sind.

\vspace{1em}

Interpretation: Zwei Vektorfelder kommutieren genau dann, wenn die Flüsse kommutieren.

\begin{proof}
\quad
\paragraph{$^"\Leftarrow^"$:}Laut Korollar 3.28: $\varphi_t \circ \psi_s = \psi_s \circ \varphi_t$
\[
\Rightarrow (\varphi_t)_*Y=Y \quad \text{für alle } t, \text{ für die der Fluss definiert ist.}
\]
\[
\overset{\txt{Satz 3.25}}{\Rightarrow} [X,Y]=\mathcal{L}_XY=\lim_{t \to 0} \frac{1}{t} \left((\varphi_t^* Y)_p - Y_p\right)=0
\]
\paragraph{$^"\Rightarrow^"$:} Für alle $q \in M$ gilt $[X, Y]_q = 0$, also 
\[
\lim_{t \to 0} \frac{1}{t} \left( Y_q - (\varphi_t)_* Y_q \right) = 0 \qquad (\star)
\]
Wir definieren die Kurve $c: (-\varepsilon, \varepsilon) \rightarrow T_p M$ durch 
\[
c(t) = ((\varphi_t)_* Y)_p
\]
Der Tangentialvektor von $c$ bei $t$ ist
\[
\dot{c}(t) = \lim_{h \to 0} \frac{1}{h}\underbrace{\left[c(t + h) - c(t)\right]}_{{\txt{Differenz in }  T_pM}} \overset{\txt{Def. }c}{=} \lim_{h \to 0} \frac{1}{h} \left[ (\varphi_{(t+h)})_* Y - (\varphi_t)_* Y \right]_p
\]
\[
\underset{+\txt{Def. Push-forward}}{\overset{(f\circ g)_*X=f_*(g_*Y)}{=}} \lim_{h \to 0} \frac{1}{h} \left[ D \varphi_t ((\varphi_{h})_*Y)_{\varphi_{-t}(p)} - D \varphi_t(Y)_{\varphi_{-t}(p)} \right]
\]
\[
=D \varphi_t \underbrace{\left( \lim_{h \to 0} \frac{1}{h} \left[ (\varphi_h)_* Y \right]_{\varphi_{-t}(p)} - Y_{\varphi_{-t}(p)} \right)}_{=0 \txt{ wegen } (\star) \txt{ in } q=\varphi_{-t}(p)} = 0
\]
\[
\implies c(t) \text{ ist konstant, also } (\varphi_t)_\ast Y|_p = c(t) = c(0) = Y_p \quad \overset{{3.28}}{\Longrightarrow} \quad \text{Beh.}
\]
\end{proof}

Im Satz 3.26 haben wir gesehen, dass es für ein Vektorfeld $X$ mit $X_p \neq 0$ eine Karte $(U, \varphi)$ um $p$ gibt, sodass $X$ ein Koordinatenvektorfeld auf $U$ ist.

\vspace{1em}

\textbf{Was passiert bei zwei oder mehr Vektorfeldern?}

\vspace{1em}

\textbf{Gegeben:} $X, Y$ Vektorfelder, linear unabhängig in $p \in M$ (also auch auf einer kleinen Umgebung von $p$). Gibt es eine Karte um $p$ in der $X$ und $Y$ Koordinatenvektorfelder sind?

\vspace{1em}

\textbf{Notwendige Bedingung} aus dem Satz von Schwarz: In jeder Karte $(U, q)$ gilt 
\[
\left[\frac{\partial}{\partial x_i}, \frac{\partial}{\partial x_j}\right] = 0.
\]
\[
\implies [X, Y] = 0 \quad \text{ist eine notwendige Bed. für die Existenz der gesuchten Karte.}
\]

\vspace{1em}

Der folgende Satz sagt, dass dies auch hinreichend ist.\\


\textbf{Satz 3.30}

Seien $X_1, \dots, X_k$ Vektorfelder auf einer Umgebung $V$ von $p \in M$, die in jedem Punkt von $V$ linear unabhängig sind und es gilt $[X_a, X_b] = 0$ für $1 \leq a, b \leq k$. Dann gibt es eine Karte $(U, \varphi)$ um $p$ mit 
\[
X_a = \frac{\partial}{\partial x_a} \quad \text{auf } U \quad \text{für } a = 1, \dots, k.
\]
\begin{proof}
    Wie im Beweis von Satz 3.26 können wir annehmen, dass $M = \mathbb{R}^n, p = 0$ und $X_a = \frac{\partial}{\partial x_a} \big|_0$ für $a = 1, \dots, k$.

\vspace{1em}

Sei $\varphi^a_t$ der lokale Fluss zum Vektorfeld $X_a$, dann definieren wir:
\[
\psi(a_1, \dots, a_n) = \varphi_{a_1}^1 \left( \varphi_{a_2}^2 \left( ...(\varphi_{a_k}^k (0, 0, a_{k+1}, \dots, a_n))... \right) \right).
\]
Wie in Satz 3.26:
\[
D\psi \left(\frac{\partial}{\partial x_i} \big|_0 \right) = 
\begin{cases} 
X_i(0) = \frac{\partial}{\partial x_i} \big|_0  & \text{für } i = 1, \dots, k \\ 
\frac{\partial}{\partial x_i} \big|_0 & \text{für } i = k+1, \dots, n
\end{cases}
\]

Somit ist $\psi$ auf einer Umgebung von $0$ ein Diffeomorphismus, und durch $\psi^{-1}$ wird die neue Karte definiert.

Analog zu Satz 3.26:
\[
X_1 =  \frac{\partial  }{\partial y_1}
\]
Aus Lemma 3.23 folgt: Die Reihenfolge der $\varphi^i_{a_i}$'s in der Definition von $\psi$ kann vertauscht werden.
\(
\text{Aus } \psi = \varphi_{a_i}^i(\dots) \quad \text{folgt} \quad X_i =  \frac{\partial  }{\partial y_i} 
\)
\end{proof}

\section{{Lie-Gruppen}}
\subsubsection*{{Definition 4.1:}}
Eine Gruppe $G$ heißt \textbf{\underline{Lie-Gruppe}}, falls $G$ eine Mannigfaltigkeit ist und die Abbildung 
\[
G \times G \longrightarrow G, \quad (g, h) \longmapsto g \cdot h^{-1}
\]
differenzierbar ist.\\

\textbf{Bemerkung:}
Die zweite Bedingung ist äquivalent zu der Forderung, dass 
\[
g \longmapsto g^{-1} \quad \text{und} \quad (g, h) \longmapsto g \cdot h
\]
differenzierbar sind.\\

\textbf{Beispiele für Lie-Gruppe:}

\begin{itemize}
    \item[(i)] $(\mathbb{R}^n, +)$
    
    \item[(ii)] $(\mathrm{GL}(n, \mathbb{R}), \cdot)$, $\mathrm{GL}(n, \mathbb{R}) = \det^{-1}(\mathbb{R} \setminus \{0\})$ ist eine offene Teilmenge von $\mathbb{R}^{n^2}$ und trägt die induzierte differenzierbare Struktur.
    
    Also sind $(A, B) \mapsto A \cdot B$ und $A \mapsto A^{-1}$ genau dann differenzierbar, wenn die Einträge von $A \cdot B$ und $A^{-1}$ differenzierbar (im Sinne von Analysis 2) in den Einträgen von $A$ und $B$ sind.
    
    Für das Produkt: Sei $A = (A_{ij})$ und $B = (B_{ij})$, dann gilt: 
    \[
    (A \cdot B)_{ij} = \sum_k A_{ik} B_{kj}
    \]
    Für das Inverse: Laut Cramerscher Regel:
    \[
    (A^{-1})_{ij} = (-1)^{i+j} \frac{1}{\det A} \det \hat{A_{ij}}
    \]
    wobei $\hat{A_{ij}}$ die Teilmatrix von $A$ ist.
    
    \item[(iii)] $O(n)$, $U(n)$ sind Untermannigfaltigkeiten von $\mathrm{GL}(n, \mathbb{R})$ und die Gruppenoperation ist die Einschränkung der Gruppenoperation auf $\mathrm{GL}(n, \mathbb{R})$.


    \item[(iv)] Für $n = p + q$, betrachten wir auf $\mathbb{R}^n$ die symmetrische Bilinearform 
\[
h(x, y) = x_1y_1+x_2y_2+...+x_py_p-x_{p+1}y_{p+1}-...-x_ny_n
\]
Wir definieren 
\[
O(p, q) := \left\{A \in \mathrm{GL}(n, \mathbb{R}) \mid h(Ax, Ay) = h(x, y) \quad \forall x, y \in \mathbb{R}^n \right\}
\]
die \textbf{\underline{Invarianz-Gruppe}}.

\vspace{1em}

Die Gruppe $O(3,1)$ heißt \textbf{Lorentz-Gruppe}, sie ist die Invarianz-Gruppe des Minkowski-Raums $\mathbb{R}^{3,1}$.

\vspace{1em}

$O(3,1)$ hat vier Zusammenhangskomponenten.

\end{itemize}

\subsubsection*{{Definition 4.2:}}
Die \underline{Links- bzw. Rechtstranslationen} auf $G$ sind für jedes fixiertes $g \in G$ definiert durch $g, h:G\rightarrow G$
\[
l_g(h) = gh, \quad r_g(h) = hg.
\]
\textbf{{Bemerkung.:}} 
Die Abbildungen $l_g, r_g: G \rightarrow G$ sind \textbf{Diffeomorphismen}. 

Tatsächlich sind $r_g$ und $l_g$ nach Definition für alle $g \in G$ differenzierbar und es gilt $l_g^{-1} = l_{g^{-1}}$ (analog für $r_g$).

\subsubsection*{{Definition 4.3:}}
Ein Vektorfeld $X \in \Gamma(TG)$ heißt \textbf{\underline{links-invariant}}, falls für alle $g \in G$:
\[
(l_g)_\ast X = X,
\]
d.h. $X$ ist $l_g$-verküpft zu sich selbst.$\forall g\in G$

%%%%%%%%%%%%%%%%%%%%%%%%%%%%%%%%%%%%%%%%%%%%%%%%%%%%%%%%%%%%%%%%%%%%%%%%%%%%%%%%%%% Vorlesung 10 %%%%%%%%%%%%%%%%%%%%%%%%%%%%%

\textbf{Bemerkung}
\begin{itemize}
    \item[(i)] Ein Vektorfeld \( X \) ist links-invariant genau dann, wenn 
    \[ (Dl_g)_h(X_h) = X_{gh} \quad T_hG \rightarrow T_{l_g(h)}G=T_{gh}G \quad l_g(h)=gh\]
    für alle \( g, h \in G \). Denn nach der Definition des push-forwards
    gilt:
    \[
    X_h \overset{\txt{links invar.}}{=}((l_g)_* X)_h \overset{\txt{Def. Push-forward}}{=}\left(Dl_g\right)_{l_{g}^{-1}(h)} \left(X_{l_{g}^{-1}(h)}\right) = \left(Dl_g\right)_{g^{-1}h} \left(X_{g^{-1}h}\right).
    \]
    Setze hier \( \tilde{h} = g \cdot h  \quad (l_g)^{-1}=l_{g^{-1}} \).

    \item[(ii)] Links-invariante Vektorfelder sind bestimmt durch den Wert in einem Punkt. Denn sei \( X \) ein links-invariantes Vektorfeld und \( g \in G \) ein beliebiger Punkt, dann gilt:
    \[
    X_g = X_{g\cdot e} \overset{\txt{i)}}{=} (Dl_g)_e(X_e),
    \]
    wobei \( e \) das neutrale Element in der Gruppe \( G \) ist.

    Die Abbildung
    \[
    T_e G \longrightarrow \Gamma(TG), \quad X_e \longmapsto \tilde{X}, \quad \text{mit } \tilde{X}_g = (Dl_g)_e(X_e) = X_g
    \]
    ist eine lineare Bijektion zwischen \( T_e G \) und dem Vektorraum der links-invarianten Vektorfelder. Die Umkehrabbildung ist gegeben durch:
    \[
    \tilde{X} \longmapsto \tilde{X}_e = X_e
    \]
    Mit anderen Worten: Jedes links-invariante Vektorfeld hat die Form \( \tilde{X} \), wobei \( X_e \in T_e G \) der Wert des Vektorfeldes im neutralen Element \( e \) ist. Insbesondere ist $\tilde{X}$ defineirt durch:
    \[
    g \longmapsto \tilde{X_g}=(Dl_g)_e(X_e)
    \]
    ein differenzierbares Vektorfeld auf \( G \).
    
    \item[(iii)] Lie-Gruppen sind \underline{parallelisierbar}, d.h., es existieren \( \dim(G) \) punktweise linear unabhängige Vektorfelder. 

    \textbf{Idee:} Setze eine Basis in \( T_e G \) zu links-invarianten Vektorfeldern auf \( G \) fort. \\
    Das ist äquivalent dazu, dass \( TG \) trivial ist ($\cong G\times \mathbb{R}^{\txt{dim } G}$).
    
    \item[(iv)] Die Sphären 
    \[
    S^1 \cong SO(2) \cong U(1) \;\; \text{und} \;\; S^3 \cong SU(2) \cong Sp(1) \text{ und } SO(3) \cong SU(2) / \{\pm I\}
    \]
    \[\txt{symplektische Gruppe in Dimension 1 über Quaternion $\mathbb{H}$}\]
    sind Lie-Gruppen. Die gerade-dimensionalen Sphären \( S^{2n} \) sind keine Lie-Gruppen, da auf \( S^{2n} \) kein Vektorfeld ohne Nullstellen existiert. (Satz von Igel)

    \textbf{Satz von Adams:} \( S^1, S^3 \) und \( S^7 \) sind die einzigen parallelisierbaren Sphären. 

    \( S^7 \) ist jedoch keine Lie-Gruppe. (Körper-Multiplikation nicht assoziativ)

    \item[(v)] Analog zu links-invarianten Vektorfeldern kann man auch \textbf{rechts-invariante Vektorfelder} definieren.
    
    \item[(vi)] Sind \( G, H \) Lie-Gruppen, so ist auch \( G \times H \) eine Lie-Gruppe. \\
    Der \( n \)-dimensionale Torus \( \mathbb{T}^n = S^1 \times \cdots \times S^1 \) ist also ebenfalls eine Lie-Gruppe.
\end{itemize}

\underline{Erinnerung: Def. 3.10}
Eine reelle Lie-Algebra ist ein reeller Vektorraum \( V \) mit einer Abbildung
\[
[\cdot, \cdot] : V \times V \longrightarrow V,
\]
die die folgenden Eigenschaften erfüllt:
\begin{itemize}
    \item[(i)] Antisymmetrie: \[ [X, Y] = -[Y, X] \]
    \item[(ii)] Bilinearität: \[ [\alpha X + \beta Y, Z] = \alpha [X, Z] + \beta [Y, Z] \]
    \item[(iii)] Jacobi-Identität: \[ [X, [Y, Z]] + [Y, [Z, X]] + [Z, [X, Y]] = 0 \]
\end{itemize}

Wir haben gesehen, dass \( \Gamma(TM) \), der unendlich-dimensionale Raum der Vektorfelder auf einer Mannigfaltigkeit \( M \), eine Lie-Algebra ist. \([X, Y]\) ist definiert durch:
\[
\mathcal{L}_{[X, Y]} = \mathcal{L}_X \circ \mathcal{L}_Y - \mathcal{L}_Y \circ \mathcal{L}_X.
\]


Es gibt auch endlich-dimensionale Beispiele:
\begin{itemize}
    \item \( \operatorname{End}(V) \) mit der Lie-Klammer:
    \(
    [A, B] = A \circ B - B \circ A
    \)

    \item \( \mathbb{R}^3 \) mit dem Lie-Produkt:
    \(
    [v, w] = v \times w \quad \text{(Kreuzprodukt)}
    \)
\end{itemize}

wollen sehen: Zu jeder Lie-Gruppe gibt es eine \underline{eindeutig} bestimmte Lie-Algebra.

\subsubsection*{Satz 4.4}
Sei \( G \) eine Lie-Gruppe und \( \mathfrak{g} := T_e G \). Dann gilt:
\begin{itemize}
    \item[(i)] Die Menge der links-invarianten Vektorfelder auf \( G \) ist ein reeller Vektorraum, der isomorph zu $\mathfrak{g}=T_eG$ ist. Insbesondere gilt:
    \[
    \txt{dim } \mathfrak{g} =\txt{dim }G.
    \]

    \item[(ii)] Der Kommutator zweier links-invarianter Vektorfelder ist ebenfalls links-invariant. \\
    Daher ist die Menge der links-invarianten Vektorfelder mit der Einschränkung des Kommutators von Vektorfeldern eine Lie-Algebra.

    Auf \( \mathfrak{g} = T_e M \) ist die Lie-Algebra-Struktur definiert durch:
    \[
    [X, Y]^{\mathfrak{g}} := [\tilde{X}, \tilde{Y}]_e^{\txt{VF}}
    \]
    für \( X, Y \in \mathfrak{g} \). (pass auf hier $X, Y$ sind Werte, keine Vektorfelder!). \\
    Insbesondere gilt:
    \(
    \widetilde{[X, Y]}^{\mathfrak{g}} = [\tilde{X}, \tilde{Y}]^{\txt{VF}}
    \)
\end{itemize}
\begin{proof}
\quad
    \begin{itemize}
    \item[(i)] Dies haben wir bereits diskutiert.
    
    \item[(ii)] 
    \(
    X \text{ ist links-invariant} \; \overset{\txt{Def.}}{\Longleftrightarrow} \; X \text{ ist } l_g \text{-verknüpft zu sich selbst für alle } g \in G
    \)

    \(\Longleftrightarrow\; (l_g)_*X=X \quad \txt{für alle } g\in G \)
    
    Nach Lemma 3.15 gilt für links-invariante Vektorfelder \( X, Y \):
    \[
    l_{g*}[X, Y] \overset{3.15}{=} [l_{g*}X, l_{g*}Y] \overset{X, Y\;\txt{links-inv}}{=} [X, Y].
    \]
    D.h. \( [X, Y] \) ebenfalls links-invariant ist.\\
\end{itemize}
Seien \( X, Y \in T_e G \) mit den assoziierten links-invarianten Vektorfeldern \( \tilde{X}, \tilde{Y} \in \Gamma(TG) \). Auf \( T_e G \) definieren wir:
\[
[X, Y] := [\tilde{X}, \tilde{Y}]_e. \quad (*)
\]
Seien \( \tilde{X}, \tilde{Y} \) links-invariante Vektorfelder, dann sind sie nach Definition \( l_g \)-verknüpft mit sich selbst:
\[
\widetilde{[X, Y]}_g^{\mathfrak{g}} \underset{\txt{links-inv. VF}}{\overset{\widetilde{[X, Y]}^{\mathfrak{g}}\;\txt{als}}{=}} Dl_g \left([X, Y]^{\mathfrak{g}}\right) \underset{\txt{Struktur}}{\overset{\txt{Def. Lie-Algebra}}{=}} Dl_g \left([\widetilde{X}, \widetilde{Y}]_e^{\mathsf{VF}}\right) \overset{\txt{Def. links-inv}}{=} [\widetilde{X}, \widetilde{Y}]_g^{\mathsf{VF}}
\]
Daraus folgt, dass \( [\tilde{X}, \tilde{Y}] \) ebenfalls links-invariant ist, und die behauptete Gleichung gilt.\\

Für die Lie-Algebra-Struktur müssen wir die Jacobi-Identität überprüfen. Für \( X, Y, Z \in T_e G \) gilt:
\[
\left[X, [Y, Z]\right] \overset{(*)}{=} [\widetilde{X}, \widetilde{[Y, Z]}]^{\mathsf{VF}}_e 
= \left[\widetilde{X}, [\widetilde{Y}, \widetilde{Z}]_e^{\mathsf{VF}}\right]_e
\]
Damit erfüllt \( [\cdot, \cdot] \) die Jacobi-Identität, da diese für die Lie-Klammer von Vektorfeldern gilt. Insbesondere ist 
\[
T_e G \longrightarrow \{ \text{links-invariante Vektorfelder auf } G \} \subseteq \Gamma(TG), \quad X \longmapsto \tilde{X}
\]
ein Lie-Algebren Isomorphismus von Vektorräumen, der die Lie-Algebra-Struktur überträgt. (Def. 3.11)
\end{proof}

\subsubsection*{Definition 4.5}
Die \underline{Lie-Algebra} \( \mathfrak{g} = \operatorname{Lie}(G) \) einer Lie-Gruppe \( G \) ist definiert als der Raum der links-invarianten Vektorfelder bzw. des Tangentialraums an \( G \) im neutralen Element \( e \). \\

Die Lie-Klammer ist definiert durch den Kommutator von Vektorfeldern (vgl. Satz 4.4).\\

\textbf{Beispiele}
\begin{itemize}
    \item[(i)] Sei \( G = GL(n, \mathbb{R}) \), dann ist:
    \[
    \mathfrak{g} = \mathfrak{g}L(n,\mathbb{R}) = M(n, \mathbb{R}) \quad \text{(da } GL(n, \mathbb{R}) \subseteq M(n, \mathbb{R}) \text{ offen)}
    \]
    Die Lie-Klammer auf \( \mathfrak{g} \) ist der gewöhnliche Kommutator von Matrizen:
    \[
    [A, B] = AB - BA.
    \]
    (Zum Herleiten betrachtet man die Kurve $A(t)=I+tX+o(t^2)$ in $Gl(n,\mathbb{R})$, wobei $X$ die beliebige $n\times n$ Matrix ist. $d/dt|_{t=0}A(t)=X$, verknüpft man mit Links-Abbildung dann für $g\in Gl(n,\mathbb{R})$ ergibt sich die Elemente in $\mathfrak{g}$ $g\cdot X$ was zu einer beliebigen Matrix führt)    
    \item[(ii)] Sei \( G = O(n) \), dann ist:
    \[
    \mathfrak{g} = \mathfrak{o}(n) = \{A \in M(n, \mathbb{R}) \mid A + A^T = 0\}
    \]
    Die Lie-Algebra der orthogonalen Gruppe ist die Menge der schief-symmetrischen Matrizen. 

    Die Lie-Algebra von \( SO(n) \) ist gleich der von \( O(n) \), da \( SO(n) \subset O(n) \) offen ist. (Genauer: \( SO(n) \) ist die Zusammenhangskomponente der Einsheitsmatrix.)\\
    
    Für \( O(3) \) gilt:
    \[
    \mathfrak{o}(3) \cong (\mathbb{R}^3, \times) \quad \times: \txt{Kreuzprodukt}
    \]
\end{itemize}

\subsection{Die Lie-Algebra - Lie-Gruppen Korrespondenz}

\subsubsection*{Definition}
Eine \underline{Lie-Untergruppe} \( H \subseteq G \) einer Lie-Gruppe \( G \) ist eine Untergruppe \( H \) von \( G \) mit Teilraum-Topologie und glatter Struktur, die $H$ zu einer Lie-Gruppe und eine \emph{immersierte Untermannigfaltigkeit} machen.\\

\underline{Bemerkungen}
\begin{itemize}
    \item[(i)] \( H \) ist wieder eine Lie-Gruppe.

    \item[(ii)] Sei \( \iota: H \rightarrow G \) die Immersion, dann gilt:
    \[
    \iota(e_H) = e_G
    \]
    und das Differential definiert einen Homomorphismus:
    \[
    D\iota_e: T_e H \hookrightarrow T_e G.
    \]
    Jedes \( X \in T_e H \) definiert ein links-invariantes Vektorfeld:
    \begin{itemize}
        \item \( X_G \) auf \( G \) (als Fortsetzung von \( D\iota_e(X) \))
        \item \( X_H \) auf \( H \)
    \end{itemize}
    \underline{Behauptung:} Diese Vektorfelder sind $\iota$-verwandt.

Da die Immersion ein Gruppenhomomorphismus ist, gilt:
\[
\iota \circ l_a^H = l_{\iota(a)}^G \circ \iota \quad \text{für alle } a \in H.
\]

Daraus folgt für das Differential:
\[
D\iota X_H(a) = D\iota \left(Dl_a^H(X)\right) \overset{\txt{KR}}{=} D(\iota \circ l_a^H)(X) = D\left(l_{\iota(a)}^G \circ \iota\right)(X) \overset{\txt{KR}}{=} X_G(\iota(a)).
\]

    \item[(iii)] Seien \( X, Y \in T_e H \), dann sind die Kommutatoren \( \iota \)-verknüpft, d.h.,
    \[
    [X_G, Y_G]_e = D\iota \left([X_H, Y_H]_e\right).
    \]

    Daraus folgt:
    \(
    D   \iota(T_e H) \subseteq T_e G
    \)
    ist eine Lie-Unteralgebra, d.h. ein Untervektorraum, der abgeschlossen unter der Lie-Klammer ist.

    Zusätzlich haben wir einen Lie-Algebra-Isomorphismus:
    \[
    (T_e H, [\cdot, \cdot]_H) \cong (D\iota(T_e H), [\cdot, \cdot]_G|_{TeH}).
    \]
    Insbesondere für \( H \subseteq GL(n, \mathbb{R}) \), dann ist der Kommutator auf \( H \), die 
    Einschränkung des Kommutators von Matrizen.
\end{itemize}

\textbf{Beispiel}
\[
G = \mathbb{T}^2 = S^1 \times S^1 \cong \mathbb{R}/\mathbb{Z} \times \mathbb{R}/\mathbb{Z}.
\]
Die Gruppenoperation ist definiert durch:
\[
(\theta, \varphi) + (\theta', \varphi') = (\theta + \theta', \varphi + \varphi').
\]
Dann sind die folgenden Mengen Lie-Untergruppen von \( G \):
\begin{itemize}
    \item \( H_1 = S^1 \times \{1\} \cong \mathbb{R}/\mathbb{Z} \times \{0\} \).
    \item \( H_2 = \{(t, \lambda t) \mid \lambda \in \mathbb{R} \setminus \mathbb{Q} \} \).
\end{itemize}
Lie-Gruppe von $G$, \( H_2 \) ist immersiert, aber nicht eingebettet.

Die Umkehrung von Bemerkung (iii) ist ebenfalls korrekt.\\
  \begin{figure}[H]
    \centering
    \includegraphics[width=8cm]{Image Diffgeo/10.01.png}
	\caption{2-dim Torus, blaue und purpurne Linien $S^1$, orange Linien entsprechen irrationalen Steigungen und liegen dicht in Torus}
 \end{figure}


\textbf{Satz 4.6}

Sei \( G \) eine Lie-Gruppe und \( \mathfrak{h} \subseteq \mathfrak{g} \) eine Lie-Unteralgebra. Dann existiert eine eindeutig bestimmte zusammenhängende Lie-Untergruppe \( H \subseteq G \) mit:
\[
\operatorname{Lie}(H) = \mathfrak{h}.
\]

Das bedeutet, es besteht eine bijektive Beziehung zwischen den zusammenhängenden Lie-Untergruppen von \( G \) und den Lie-Unteralgebren von \( \mathfrak{g} \).\\

\textbf{Bemerkungen}
\begin{itemize}
    \item[(i)] Laut dem \textbf{Satz von Ado} ist jede endlich-dimensionale Lie-Algebra isomorph zu einer Lie-Unteralgebra der Lie-Algebra von \( GL(n, \mathbb{R}) \). \\
    Zusammen mit Satz 4.6 ist jede Lie-Algebra isomorph zu der Lie-Algebra einer Lie-Untergruppe von \( GL(n, \mathbb{R}) \).

    \item[(ii)] Laut dem \textbf{Cartan-Kriterium} ist eine abgeschlossene Untergruppe \( H \subseteq G \) automatisch eine Lie-Untergruppe von \( G \), d.h., \( H \) mit der Teilraum-Topologie besitzt eine differenzierbare Struktur, die \( H \) zu einer Lie-Untergruppe von \( G \) macht.
\end{itemize}

\subsection{Lie-Gruppen Homomorphismen}

\subsubsection*{Definition 4.7}
Ein {\underline{Homomorphismus von Lie-Gruppen}} ist ein Gruppenhomomorphismus, der auch eine differenzierbare Abbildung ist. \\

Ein {\underline{Homomorphismus von Lie-Algebren}} ist eine lineare Abbildung, die die Lie-Klammern erhält (siehe Def. 3.11).\\

\textbf{Lemma 4.8}

Sei \( \varphi: G \rightarrow H \) ein Homomorphismus von Lie-Gruppen, dann ist das Differential am neutralen Element:
\[
D\varphi_e: T_e G \rightarrow T_e H
\]
ein Lie-Algebren Homomorphismus, d.h. für alle \( X, Y \in \mathfrak{g} = T_e G \) gilt:
\[
D\varphi_e([X, Y]^{\mathfrak{g}}) = [D\varphi_e(X), D\varphi_e(Y)]^\mathfrak{h}.
\]
\begin{proof}
\quad

    Per Definition der Lie-Klammer auf den Lie-Algebren von \( G \) bzw. \( H \) müssen wir zeigen, dass für \( X, Y \in T_e G = \mathfrak{g} \) gilt:
\[
D\varphi_e([\tilde{X}, \tilde{Y}]_e^{\mathrm{VF}}) = [\widetilde{D\varphi_e(X)}, \widetilde{D\varphi_e(Y)}]^{\mathrm{VF}}.
\]

Dabei ist \( \tilde{X} \) das links-invariante Vektorfeld zu \( X \in T_e G \), d.h., das Vektorfeld mit:
\[
\tilde{X}_g = Dl_g(X).
\]
Da \( \varphi \) ein Gruppenhomomorphismus ist, gilt:
\(
\varphi \circ l_g = l_{\varphi(g)} \circ \varphi. \quad 
\)

Durch Ableiten erhalten wir:
\[
D\varphi \circ Dl_g = Dl_{\varphi(g)} \circ D\varphi.
\]
Anwenden auf \( X \in T_e G \) ergibt:
\[
D\varphi(\tilde{X}_g) = D\varphi(Dl_g(X)) = D(l_{\varphi(g)})(D\varphi(X))=\widetilde{D\varphi(X)}_{\varphi(g)}.
\]
D.h., die links-invarianten Vektorfelder \( \tilde{X} \) und \( \widetilde{D\varphi(X)} \) sind \( \varphi \)-verknüpft. \\

Seien \( X, Y \in T_e G \), dann sind auch die Kommutatoren \( \varphi \)-verknüpft, d.h.,
\[
D_e \varphi([\tilde{X}, \tilde{Y}]_e) = [\widetilde{D_e \varphi({X})}, \widetilde{D_e \varphi({Y})}]_e.
\]
\end{proof}

\textbf{Beispiel}\\
Sei \( G = H = \mathbb{R} \) mit der Addition als Gruppenoperation. \\
Dann ist jeder differenzierbare/stetige Homomorphismus \( \varphi: \mathbb{R} \rightarrow \mathbb{R} \) von der Form:
\[
\varphi(t) = c \cdot t
\]
für ein geeignetes \( c \in \mathbb{R} \). \\
Denn: Ableiten von 
\(
\varphi(s + t) = \varphi(s) + \varphi(t)
\)
nach \( t \) und in \( s = 0 \) zeigt
\[
\dot{\varphi}(t) = \dot{\varphi}(0) = c
\]
\[
\Rightarrow \quad \varphi(t) = c \cdot t \quad \text{mit} \quad \dot{\varphi}(0) = c.
\]
Jeder differenzierbare Homomorphismus \( \varphi: \mathbb{R} \rightarrow S^1 = \mathbb{R}/\mathbb{Z} \) ist ebenfalls von der Form:
\[
\varphi(t) = e^{ict}
\]
In beiden Fällen ist das Differential \( D\varphi: \mathbb{R} \rightarrow \mathbb{R} \) die Multiplikation mit \( c \).\\

\textbf{Bemerkung}\\
Es gibt keinen nicht-trivialen Lie-Gruppen-Homomorphismus \( \varphi: S^1 \rightarrow \mathbb{R} \). (Trivial z.B. alle Elemente in $S^1$ abgebildet zu $0\in \mathbb{R}$) Denn \( \varphi(S^1) \) wäre eine kompakte Untergruppe in \( \mathbb{R} \), aber \( \{0\} \) ist die einzige kompakte Untergruppe von \( \mathbb{R} \)
\[
\varphi(S^1) = \{0\}.
\]
Sei nun \( G \subseteq \mathbb{R} \) eine kompakte Untergruppe, dann ist \( G \) beschränkt, aber mit \( g \in G \) auch \( ng \in G \) für alle \( n \in \mathbb{N} \) gilt, würde \( G \) unbeschränkt wachsen. Für $g\neq 0$ im Widerspruch zur Kompaktheit von \( G \). Also existieren keine nicht-trivialen Lie-Gruppen-Homomorphismen von \( S^1 \) nach \( \mathbb{R} \).\\

%%%%%%%%%%%%%%%%%%%%%%%%%%%%%%%%%%%%%%%%%%%%%%%%%%%%%%%%%%%%%%%%%%%%%%%%%%%%%%%%%% Vorlesung 11 %%%%%%%%%%%%%%%%%%%%%%%%%%%%%%

\textbf{Folgerung:}

Nicht jeder Lie-Algebren-Homomorphismus ist das Differential eines Lie-Gruppen-Homomorphismus \( G \rightarrow H \). Ein Beispiel dafür ist:
\[
\operatorname{Lie}(S^1) \cong \operatorname{Lie}(\mathbb{R}).
\]
wobei die Lie-Algebren gleich sind aber es existiert kein Lie-Gruppen-Homomorphismus.\\

\textbf{Satz 4.9}

Seien \( G, H \) Lie-Gruppen und sei \( \Phi: \mathfrak{g} \rightarrow \mathfrak{h} \) ein Lie-Algebren-Homomorphismus. Dann existiert eine Umgebung \( U \) von \( e \in G \) und eine glatte Abbildung \( \varphi: U \rightarrow H \) mit:
\begin{itemize}
    \item[(i)] \( \varphi(a \cdot b) = \varphi(a) \cdot \varphi(b) \), falls \( a, b, a \cdot b \in U \),
    \item[(ii)] \( D \varphi(X) = \Phi(X) \) für alle \( X \in \mathfrak{g} = T_e G \),
    \item[(iii)] Sind \( \varphi, \psi: G \rightarrow H \) zwei Lie-Gruppen-Homomorphismen mit \( D\varphi = \Phi = D\psi \) und ist \( G \) zusammenhängend, dann folgt:
    \[
    \varphi = \psi.
    \]
\end{itemize}
Mit Lie-Algebren-Homomorphismus kann man jedoch lokal Lie-Gruppen-Homomorphismus konstruieren!\\

\textbf{Bemerkung}

Der Gruppen-Homomorphismus aus Satz 4.9 ist auf ganz \( G \) definiert, falls \( G \) \textbf{einfach zusammenhängend} ist. Unter dieser Voraussetzung hat man also eine bijektive Beziehung:
\[
\left\{ 
    \text{Lie-Gruppen Homom. von } G \text{ nach } H 
\right\} 
\longleftrightarrow 
\left\{ 
    \text{Lie-Algebren Homom. von } \mathfrak{g} \text{ nach } \mathfrak{h} 
\right\}.
\]
  \begin{figure}[H]
    \centering
    \includegraphics[width=8cm]{Image Diffgeo/11.01.jpg}
	\caption{Ebene mit Loch und die Zylinder Oberfläche sind nicht einfach zusammenhängend}
 \end{figure}

\vspace{0.5em}
\textbf{Hinweis:}

Wir werden noch eine genaue Definition geben. Anschaulich: Jede Schleife in \( G \) lässt sich stetig zu einem Punkt zusammenziehen.\\

\textbf{Korollar 4.10}

Zwei Lie-Gruppen mit isomorphen Lie-Algebren sind \textbf{lokal isomorph}. \\
Die Isomorphie ist \textbf{global}, falls beide Gruppen einfach zusammenhängend sind.
\[\mathfrak{g}\overset{\Phi\cong}{\rightarrow} \mathfrak{h}\quad U\subseteq G\overset{\varphi}{\rightarrow}H \quad V\subseteq H\overset{\psi}{\rightarrow}G\]
\[g\in G\overset{l_g^{-1}}{\rightarrow}e_g\in G\rightarrow e_h\in H \overset{l_h}{\rightarrow}h\in H\]

\textbf{Korollar 4.11}

Eine zusammenhängende Lie-Gruppe mit abelscher Lie-Algebra ist abelsch. \\
(Erinnerung: Eine Lie-Algebra ist abelsch, falls die Lie-Klammer identisch Null ist.)\\
Um zu zeigen braucht man folgenden Satz:\\

\textbf{Satz 4.12}

Sei \( G \) eine zusammenhängende Lie-Gruppe und sei \( U \) eine Umgebung des Einselements in \( G \). Dann gilt:
\[
G = \bigcup_{n=1}^\infty U^n \quad U^n:=\{x_1\cdot...\cdot x_n|x_i\in U\}
\]
wobei \( U^n \) aus dem \( n \)-fachen Produkt von Elementen aus \( U \) besteht. Man sagt: \textbf{Die Umgebung \( U \) erzeugt die Gruppe \( G \)}.\\

\begin{proof}
    (Beweis Korollar 4.11)\\
    Eine abelsche Lie-Algebra \( \mathfrak{g} = \operatorname{Lie}(G) \) ist isomorph zu \( (\mathbb{R}^n, [\cdot, \cdot] = 0) = \operatorname{Lie}(\mathbb{R}^n, +) \), denn sie keine zusätzliche (Gruppen-)Strukturen außer Vektorraum-Struktur besitzt (da Lie-Klammer trivial).\\

Laut Korollar 4.10 ist $G$ dann lokal isomorph zu $\mathbb{R}^n$. Es gilt also:
\[
a\cdot b=b\cdot a \quad \text{für alle } a, b \;\txt{in einer hinreichend kleinen Umgebung von } e_G.
\]

Jede solche Umgebung erzeugt also die ganze Gruppe (Satz 4.12), d.h., jedes Gruppenelement lässt sich als Produkt von Elementen aus der Umgebung schreiben. \\

Damit ist die ganze Gruppe \( G \) abelsch.
\end{proof}

\textbf{Bemerkungen}

\begin{itemize}
    \item[(i)] Alle zusammenhängenden Lie-Gruppen mit der gleichen Lie-Algebra werden von der gleichen einfach-zusammenhängenden Lie-Gruppe überlagert. \\
    (\textcolor{red}{Später: Theorie von Überlagerungen})

    \item[(ii)] Die Lie-Algebra einer Lie-Gruppe \( G \) ist gleich der Lie-Algebra der {zusammenhängenden Komponente des Eins-Elements}\footnote{$e\in G_0$ und $G_0$ zusammenhängend im topologischen Sinne} in \( G \). \\
    Zum Beispiel:
    \[
    \operatorname{Lie}(O(n)) = \operatorname{Lie}(SO(n)).
    \]
    $O(n)$ besitzt 2 Zusammenhangskomponente da det=$\pm 1$
\end{itemize}

\subsection{Die Exponential-Abbildung}

\textbf{Satz 4.13}

Links-invariante Vektorfelder sind \textbf{vollständig}, d.h., für ein links-invariantes Vektorfeld \( X \) auf einer Lie-Gruppe \( G \) ist der Fluss \( \varphi_t \) für alle Zeiten \( t \in \mathbb{R} \) definiert. \\

Weiter gilt:
\[
\varphi_t(g) = g \cdot \varphi_t(e) \quad \text{für alle } g \in G \text{ und } t \in \mathbb{R},
\]
d.h., die Kurve:
\[
l(t) = g \cdot \varphi_t(e)
\]
ist die maximale Integralkurve von \( X \) durch \( g \).
  \begin{figure}[H]
    \centering
    \includegraphics[width=6cm]{Image Diffgeo/11.02.png}
	%\caption{Ebene mit Loch und die Zylinder Oberfläche sind nicht einfach zusammenhängend}
 \end{figure}
\begin{proof}
    Sei \( X \) ein links-invariantes Vektorfeld auf \( G \), dann gilt $X=(l_g)_*X$ und laut Lemma 3.27 gilt für die Flüsse:
\[
\forall g \in G: \quad \varphi_t = l_g \circ \varphi_t \circ l_{g^{-1}}
\]
Angewandt auf \( g \in G \) folgt:
\[
\varphi_t(g) = l_g \circ \varphi_t \circ l_{g^{-1}}(g) = l_g \circ \varphi_t(e) \overset{\txt{Def. Links-Abb}}{=} g \cdot \varphi_t(e) \tag{*}
\]
\(\Rightarrow\) Es reicht zu zeigen: Die Integralkurve von \( X \) durch \( e \) existiert für alle \( t \in \mathbb{R} \).

Sei \( \varepsilon > 0 \) so gewählt, dass \( \varphi_t(e) \) für alle \( t \in (-\varepsilon, \varepsilon) \) definiert ist. Es gilt:
\[
\varphi_{s+t}(e) = \varphi_s(\varphi_t(e)) = \varphi_t(e) \cdot \varphi_s(e) \tag{**}
\]

für alle \( s, t \) klein genug.

Wir schreiben \( s = k \cdot \frac{\varepsilon}{2} + t \), für \( k \in \mathbb{Z} \) und \( |t| < \frac{\varepsilon}{2} \). Für \( s > 0 \) definieren wir:
\[
c(s) := \left( \varphi_{\frac{\varepsilon}{2}}(e) \right)^k \cdot \varphi_t(e)
\]

\underline{Behauptung:} \( c(s) \) ist die Integralkurve von \( X \) durch \( e \).

\[
\left.\frac{d}{ds}\right|_{s=s_0 = k \frac{\varepsilon}{2} + t_0} c(s)
= \left.\frac{d}{dt}\right|_{t = t_0} c\left(k \frac{\varepsilon}{2} + t \right) = \left.\frac{d}{dt}\right|_{t = t_0} \underbrace{\left( \left(\varphi_{\frac{\varepsilon}{2}}(e)\right)^k \cdot \varphi_t(e) \right)}_{=l_{(\varphi_{\varepsilon/2}(e))^k}(\varphi_t(e))}
\]
\[
\overset{\txt{KR}}{=} Dl_{ \left(\varphi_{\frac{\varepsilon}{2}}(e)\right)^k } \left( X_{\varphi_{t_0}(e)} \right) \overset{(Dl_g)_h(X_h)=X_{gh}}{=} X_{\varphi_{\frac{\varepsilon}{2}}(e))^k\varphi_{t_0}(e)} = X_{c(s)}
\]
\end{proof}

\textbf{Korollar 4.14}

Sei \( c_X : \mathbb{R} \rightarrow G \) die Integralkurve durch \( e \) eines links-invarianten Vektorfelds \( X \). \\

Dann ist \( c_X \) ein {Lie-Gruppen-Homomorphismus}, d.h.,
\[
c_X(0) = e \quad \text{und} \quad c_X(s + t) = c_X(s) \cdot c_X(t) \quad \text{für alle } s, t \in \mathbb{R}.
\]

Weiter gilt:
\[
c_{sX}(t) = c_X(s \cdot t).
\]

\begin{proof}
    Es ist \( c_X(t) = \varphi_t(e) \), die {Homomorphismus-Eigenschaft} von \( c_X \) folgt aus \textcolor{red}{(**)}. \\

\underline{bleibt zu zeigen:} Die Abbildung
\(
t \mapsto c(t) := c_X(st)
\)
ist die Integralkurve des links-invarianten Vektorfeldes. \\

Daraus berechnen wir:
\[
\left.\frac{d}{dt}\right|_{t = t_0} c(t)
= \left.\frac{d}{dt}\right|_{t = t_0} c_X(s t)
= s \cdot \dot{c}_X(s t_0)
\overset{c_X\txt{ Integralkurve}}{=} s \cdot X_{c_X(s t_0)}
= s \cdot X_{c(t_0)}
\]
\end{proof}



\textbf{Definition 4.15}

Sei \( G \) eine Lie-Gruppe mit Lie-Algebra \( \mathfrak{g} \). Die Abbildung:
\[
\exp : \mathfrak{g} \rightarrow G, \quad X \mapsto c_X(1)
\]
heißt \textbf{\underline{Exponentialabbildung der Lie-Gruppe}} \( G \). Hierbei ist \( c_X \) die maximale Integralkurve von \( X \) mit:
\[
c_X(0) = e.
\]

\textbf{Bemerkung}

\begin{itemize}
    \item[(i)] Es gilt für alle \( t \in \mathbb{R} \):
    \[
    \exp(tX) =c_{tX}(1)= c_X(t)=\varphi_t(e).
    \]

    \item[(ii)] Homomorphismen \( \mathbb{R} \rightarrow G \) heißen \textbf{1-parametrige Untergruppen} von \( G \).


    \item[(iii)] Für jedes fixierte \( X \in \mathfrak{g} \) ist die Abbildung \( \mathbb{R} \rightarrow G, \; t \mapsto \exp(tX) \) der eindeutig bestimmte glatte Gruppenhomomorphismus \( \gamma : \mathbb{R} \rightarrow G \) mit:
    \[
    \dot{\gamma}(0) = X.
    \]
\end{itemize}
Beweis:\\
Der Homomorphismus \( \gamma : \mathbb{R} \rightarrow G \) definiert eine glatte Kurve in \( G \) mit:
\[
\gamma(0) = e \quad \text{und} \quad \dot{\gamma}(0) = X.
\]

\textbf{Zu zeigen:} 

\( \gamma \) ist die Integralkurve des links-invarianten Vektorfeldes \( \tilde{X} \). \\

Dann folgt aus der Eindeutigkeit der Integralkurven:
\[
\gamma(t) = c_X(t)=c_{tX}(1)=\exp(tX).
\]

Nun berechnen wir den Tangentialvektor an \( \gamma \) zur Zeit \( s \):
\begin{align*}
\left.\frac{d}{dt}\right|_{t=s} \gamma(t) 
&= \left.\frac{d}{dt}\right|_{t=0} \gamma(s+t) 
= \left.\frac{d}{dt}\right|_{t=0} \gamma(s) \cdot \gamma(t) \\
&= \left.\frac{d}{dt}\right|_{t=0} \ell_{\gamma(s)} \gamma(t) 
= D \ell_{\gamma(s)} \, \dot{\gamma}(0) 
= D \ell_{\gamma(s)} X 
= \tilde{X}_{\gamma(s)}.
\end{align*}


\textbf{Beispiel}

Die Exponentialabbildung für \( G = GL(n, \mathbb{R}) \).

Für eine Matrix \( A \in M(n, \mathbb{R}) = \mathfrak{gl}(n, \mathbb{R}) \) definieren wir:
\[
\exp(tA) = e^{tA} = \sum_{k=0}^\infty \frac{(tA)^k}{k!}.
\]

Diese Reihe:
\begin{itemize}
    \item konvergiert absolut (Standardabschätzung mit Norm und absolute Konvergenz von $e^x$ in $\mathbb{R}$).
    \item auf jeder beschränkten Menge in \( \mathfrak{gl}(n, \mathbb{R}) \) gleichmäßig.
\end{itemize}
\[
\Rightarrow \exp(tA) \text{ ist wohldefiniert}.
\]

Dabei benutzen wir, dass für die Operatornorm auf \( GL(n, \mathbb{R}) \) die Abschätzung gilt:
\[
\|A^n\|_{\operatorname{op}} \leq \|A\|^n.
\]

Dies ist tatsächlich die Exponentialabbildung aus Definition 4.15. \\

Laut Bemerkung (iii) müssen wir zeigen:
\[
\exp(0 \cdot A) \underbrace{=}_{\txt{Def.}} I, \quad \exp : \mathbb{R} \underbrace{\rightarrow}_{4.16 i)} GL(n,\mathbb{R)}, \quad t \mapsto \exp(tA)
\]
ist glatt und ein \textbf{Gruppenhomomorphismus} gemäß (4.16.ii) mit $\left.\frac{d}{dt}\right|_{t=0} \exp(tA) \underbrace{=}_{\txt{Def}} A$.

\textbf{Lemma 4.16}

Die Exponentialabbildung für Matrizen hat folgende Eigenschaften:
\begin{enumerate}
    \item Für alle \( A \in M(n, \mathbb{R}) \) gilt:
    \[
    \det(e^A) = e^{\operatorname{Spur}(A)} \quad \text{und somit} \quad e^A \in GL(n, \mathbb{R}).
    \]
    
    \item Für alle \( A, B \in M(n, \mathbb{R}) \) folgt aus \( AB = BA \) die Gleichung:
    \[
    e^{A+B} = e^A \cdot e^B.
    \]
    
    \item Für alle \( B \in GL(n, \mathbb{R}) \) und \( A \in M(n, \mathbb{R}) \) gilt:
    \[
    B e^A B^{-1} = e^{BAB^{-1}}.
    \]
\end{enumerate}
\begin{proof}
    \textbf{Ad (iii):} Multiplizieren der Reihe \( e^A \) von links mit \( B \) und von rechts mit \( B^{-1} \), nutze die Beziehung:
\[
B e^A B^{-1} = \sum_{k=0}^\infty \frac{1}{k!} B A^k B^{-1} = \sum_{k=0}^\infty \frac{1}{k!} (BAB^{-1})^k = e^{BAB^{-1}}.
\]

\textbf{Ad i)} Für Diagonalmatrizen gilt 
\[
    \exp\left(\mathrm{diag}(\lambda_1, \ldots, \lambda_n)\right) = \mathrm{diag}\left(e^{\lambda_1}, \ldots, e^{\lambda_n}\right),
\]
für diese stimmt die Behauptung.
    
\textbf{iii)} $\Rightarrow$ Die Aussage ist richtig für diagonalisierbare Matrizen. ($S^{-1}AS=D$ für ein $S\in \mathrm{GL}(n,\mathbb{R})$).
Die diagonalisierbaren Matrizen sind dicht in den Matrizen (Stetigkeit $\Rightarrow$ Beh.).\\

\textbf{Ad (ii):} Berechne das Cauchy Produkt für die Reihen \( e^A \) und \( e^B \) und forme mit der binomischen Formel um, da \( A \) und \( B \) kommutieren.
\end{proof}

\textbf{Satz 4.17}

Die Exponentialabbildung \( \exp: \mathfrak{g} \rightarrow G \) ist beliebig oft differenzierbar und ein lokaler Diffeomorphismus um \( 0 \in \mathfrak{g} \). Weiter gilt:
\begin{enumerate}
    \item[i)] \(
    \exp(0) = e
    \)
    \item[ii)] \(
    \exp((t+s)X) = \exp(sX) \cdot \exp(tX) \quad \forall s, t \in \mathbb{R}, \, X \in \mathfrak{g}.
    \)
    \item[iii)] \(
    \exp(-X) = \exp(X)^{-1} \quad \forall X \in \mathfrak{g}.
    \)
\end{enumerate}

\begin{proof}
    Die Glattheit der Abbildung \( X \mapsto \exp(X)=c_X(1) \) folgt aus den Sätzen über die glatte Abhängigkeit der Lösungen linearer Differentialgleichungen von den Anfangswerten.

Die Aussagen (i) und (ii) folgen aus Homomorphismus der Exponentialabbildung. Zusammen ergeben sie die Aussage (iii), wenn man \( s = -t \) in (ii) setzt.\\

\underline{bleibt zu zeigen}
Die Exponentialabbildung ist ein lokaler Diffeomorphismus um \( 0 \in \mathfrak{g} \). Dazu zeigen wir \( D\exp_0 = \operatorname{id}_{T_e G} \) und benutzen den Umkehrsatz (Vorlesung 2).\\
Sei \( X \in \mathfrak{g} \), dann folgt:
\[
D\exp_0(X) = \frac{d}{dt} \big|_{t=0} \exp(tX) = X.
\]
\end{proof}

\textbf{Bemerkung}

Da die Exponentialabbildung ein lokaler Diffeomorphismus ist, lassen sich mit Hilfe der Exponentialabbildung speziell an die Lie-Gruppe angepasste Koordinaten definieren.\\
Sei \( V(0) \subseteq \mathfrak{g} \) eine bezüglich 0 sternförmige Umgebung, auf der exp ein Diffeomorphismus ist.\\
Wir definieren:
\[
W(e) := \exp (V(0))\subseteq G, \quad W(g) := l_g(W(e)) \;\; \text{mit} \;\; \varphi_g=\exp^{-1}\circ l_g^{-1}: W(g) \rightarrow V(0) \subseteq \mathfrak{g} \cong \mathbb{R}^n.
\]

\textbf{Bemerkung}

Man kann sogar zeigen: Jeder stetige Homomorphismus zwischen Lie-Gruppen ist differenzierbar.\\

\textbf{Bemerkung}

Im Allgemeinen ist die Exponentialabbildung nicht surjektiv (und injektiv).\\
Man kann zeigen, dass die Exponentialabbildung auf kompakten zusammenhängenden Lie-Gruppen surjektiv ist.

\underline{Beispiel}
\[
G = SL(2, \mathbb{R}) = \{A \in M(2, \mathbb{R}) \mid \det(A) = 1 \}
\]
\[
\mathfrak{g} = \operatorname{Lie}(SL(2, \mathbb{R})) = \{B \in M(2, \mathbb{R}) \mid \operatorname{spur}(B) = 0 \}
\]
Dann ist die Exponentialabbildung nicht surjektiv, denn:
\[
B = \operatorname{diag}\left(-\frac{1}{2}, -2\right) \quad \text{liegt nicht im Bild}.
\]
Wäre \( B = \exp(X) \), so würde folgen:
\[
B = \exp(X) 
= \exp\left( \tfrac{1}{2}X + \tfrac{1}{2}X \right) 
= \exp\left( \tfrac{1}{2}X \right) \cdot \exp\left( \tfrac{1}{2}X \right) 
= \left( \exp\left( \tfrac{1}{2}X \right) \right)^2
\]

Aber es gibt keine Matrix \( A \in SL(2, \mathbb{R}) \) mit \( B = A^2 \).\\
(Cayley-Hamilton-Theorem für \( 2 \times 2 \)-Matrizen):
\[
A^2 - \operatorname{spur}(A) \cdot A + \det(A) \cdot E = 0.
\]
\[
0 = \operatorname{spur} \left(A^2 - \operatorname{spur}(A)A + E \right)=\operatorname{spur}(A^2)-\operatorname{spur}(A)^2+1\leq\operatorname{spur}(A^2)+2
\]
\[
\quad \Longrightarrow-2 \leq \operatorname{spur}(A^2)
\]
Aber:
\[
\operatorname{spur}(B) = -\frac{5}{2}<-2  .
\]
Zu Injektivität einfach $\mathrm{Lie}(S^1)=(\mathbb{R},[\cdot,\cdot]=0)\quad \exp:\mathrm{Lie}(S^1)\rightarrow S^1 \quad t\mapsto e^{it}$\\

\textbf{Satz 4.18}

Sei $\varphi: G \to H$ ein Lie-Gruppen-Homomorphismus. Dann gilt 
\[
\varphi(\exp^G(X))=\exp^H(D\varphi_{e_G}(X))
\]
für alle $X \in \mathfrak{g} = \operatorname{Lie}(G)$, d.h. es gilt das folgende kommutative Diagramm:

\includegraphics[width=5cm]{Image Diffgeo/11.98.png}

\begin{proof}
    Sei $c(t) = \varphi(\exp(tX))$. Dann ist $c(0) = \varphi(e_G) = e_H$ und da $\varphi \circ l_g = l_{\varphi(g)} \circ \varphi$ (*) folgt:
\[
\dot{c}(t) \overset{\txt{KR}}{=} D\varphi\left( \widetilde{X}_{\exp(tX)} \right) 
\overset{\txt{Def }\tilde{X}}{=}D\varphi \circ D l_{\exp(tX)} \left(X_{e_G}\right)
\overset{(*)}{=}D l_{\varphi(\exp(tX))} \circ D\varphi \left(X_{e_G}\right)
= \widetilde{D\varphi(X_{e_G}) _{c(t)}}
\]
d.h. $c(t)$ ist eine Integralkurve von $\widetilde{D\varphi(X_{e_g})}$ durch $e_H$ und somit folgt:
\[
\varphi(\exp(X)) = \exp(D\varphi(X)).
\]
\end{proof}

\textbf{Korollar 4.19}

Jeder injektive glatte Homomorphismus $\varphi: G \to H$ ist eine Immersion und $\varphi(G) \subseteq H$ ist damit eine Lie-Untergruppe von $H$. (immersierte Untermannigfaltigkeit)

\begin{proof}
    Sei $X \in \mathfrak{g} = T_e G$ und sei $\tilde{X}$ das zugehörige Vektorfeld, d.h. $\tilde{X}_g = Dl_g(X)$. 

\medskip

Ist für ein $g \in G$ das Differential von $\varphi$ nicht injektiv, also z.B. $D\varphi(\tilde{X}_g) = 0$, dann folgt:
\[
0 = D\varphi(\tilde{X}_g) = D\varphi \circ D l_g (X) \overset{(*) \txt{Gruppenhomo.}}{=} D l_{\varphi(g)} \, D\varphi (X)
\]
Da $Dl_{\varphi(g)}$ ein Isomorphismus ist: $D\varphi(X)=0\;\;\forall X\in \mathfrak{g}$.
Durch Anwenden der Exponentialabbildung ergibt sich:
\[
e_H  \overset{tX\in \mathfrak{g}}{=} \exp(D\varphi(tX)) = \varphi(\exp(tX))
\]
Dies ist ein Widerspruch, da $\varphi$ injektiv $\Rightarrow D\varphi$ injektiv, d.h. $\varphi$ ist eine Immersion.
\end{proof}

\textbf{Korollar 4.20}

Sei $G$ eine Lie-Gruppe mit der Lie-Algebra $\mathfrak{g}$ und sei $\varphi: \mathbb{R} \to G$ ein stetiger Gruppenhomomorphismus. Dann existiert ein $X \in \mathfrak{g}$ mit 
\[
\varphi(t) = \exp(tX) \quad \text{für alle } t \in \mathbb{R}.
\]
Somit ist jeder stetige Gruppenhomomorphismus $\mathbb{R} \to G$ schon glatt.

\begin{proof}
    Sei $W = \exp(V(0))$ eine Normalumgebung um $e \in G$. Da $\varphi: \mathbb{R} \to G$ stetig ist, mit $\varphi(0) = e$, existiert ein $\varepsilon > 0$ mit $\varphi(t) \in W$ für alle $t$ mit $|t| < 2\varepsilon$. \\

Sei nun $Y \in V(0)$ der Vektor mit 
\[
\exp(Y) = \varphi(\varepsilon),
\]
und setze
\[
X=\frac{1}{\epsilon}Y
\]
\underline{Zu zeigen:} $\varphi(t)=\exp(tx)=:f(t)$ für alle $t\in\mathbb{R}$
\\
Sei $K\subseteq \mathbb{R}$ definiert durch
\[
K = \{t \in \mathbb{R} \mid f(t) = \varphi(t)\}.
\]
Es gilt $0, \varepsilon \in K$. Da $f$ und $\varphi$ stetig sind, ist $K$ eine abgeschlossene Untergruppe von $\mathbb{R}$. Für nicht-triviale abgeschlossene Untergruppen $K \subseteq \mathbb{R}$ gibt es zwei Möglichkeiten: 
\[
K = \mathbb{R} \quad \text{oder} \quad K = K_d := \{nd\mid n\in\mathbb{Z}\}, 
\]
wobei $d$ die kleinste positive Zahl in $K$ ist.\\

\textbf{Annahme: $K = K_d$}

Da $\varepsilon > 0$ und $\varepsilon \in K$, folgt:
\(
\frac{d}{2} < \varepsilon.
\)
Da $\varphi$ und $f$ Homomorphismen sind, gilt:
\[
\left( f\left( \frac{d}{2} \right) \right)^2
= \exp\left( \frac{d}{2} X \right)^2
= \exp(dX)
= f(d)
\overset{d \in K}{=} \varphi(d)
= \left( \varphi\left( \frac{d}{2} \right) \right)^2
\qquad 
\]
Anwenden von $\exp^{-1}$ ergibt:
\[
2 \cdot \exp^{-1}\left( f\left( \frac{d}{2} \right) \right)
= \exp^{-1}\left( f\left( \frac{d}{2} \right)^2 \right)
= \exp^{-1}\left( \varphi\left( \frac{d}{2} \right)^2 \right)
= 2 \cdot \exp^{-1}\left( \varphi\left( \frac{d}{2} \right) \right)
\]
\[
\Rightarrow
f\left(\frac{d}{2}\right) = \varphi\left(\frac{d}{2}\right).
\]
Das bedeutet, $d$ war nicht das kleinste Element in $K$, was ein Widerspruch ist.
\[
\Rightarrow K = \mathbb{R}.
\]
\end{proof}
  \begin{figure}[H]
    \centering
    \includegraphics[width=8cm]{Image Diffgeo/11.03.png}
	%\caption{Ebene mit Loch und die Zylinder Oberfläche sind nicht einfach zusammenhängend}
 \end{figure}
 %%%%%%%%%%%%%%%%%%%%%%%%%%%%%%%%%%%%%%%%%%%%%%%%%%%%%%%%%%%%%%%%%%%%%%%%%%%%%%%% Vorlesung 12 %%%%%%%%%%%%%%%%%%%%%%%%%%%%%%
 \textbf{Satz 4.21}\\
Sei $G$ eine Lie-Gruppe mit Lie-Algebra $\mathfrak{g} = V_1 \oplus \dots \oplus V_r$, \\
dann ist die Abbildung
\[
\begin{aligned}
    \Phi \colon V_1 \oplus \dots \oplus V_r &\longrightarrow G \\
    v_1 + \dots + v_r &\longmapsto \exp(v_1) \dots \exp(v_r)
\end{aligned}
\]
ein lokaler Diffeomorphismus um $0 \in \mathfrak{g}$.
\begin{proof}
\quad\\
    Wir zeigen $D\Phi_0 = \mathrm{id}_{\mathfrak{g}}$, laut dem Umkehrsatz ist $\Phi$ dann ein lokaler Diffeomorphismus um $0 \in \mathfrak{g}$. \\
    Für $i = 1, \dots, r$ sei $X_i \in V_i$, dann gilt
\[
D\Phi_0(X_i) = \left. \frac{d}{dt} \right|_{t=0} \exp(0)\cdot...\cdot\exp(tX_i)\cdot ... \cdot \exp(0) = \left. \frac{d}{dt} \right|_{t=0} \exp(tX_i) = X_i.
\]
\end{proof}

\textbf{Korollar 4.22}\\
Jeder stetige Homomorphismus $\psi \colon G \to H$ ist schon glatt.
\begin{proof}
    Sei $X_1, \dots, X_n$ eine Basis von $T_e G=\mathfrak{g}$. Für $i = 1, \dots, n$ definieren wir stetige Gruppenhomomorphismen
\[
\psi_i \colon \mathbb{R} \to H, \quad
t \mapsto \psi(\exp(t X_i))
\]

Laut Korollar 4.18 gibt es $Y_i \in T_e H$ mit $\psi(\exp(t X_i)) = \exp(D\psi_{e_G}(tX_i))= \exp(t Y_i)$ \quad für $i = 1, \dots, n$.
\[
\Rightarrow \quad
\psi\left( \exp(t_1 X_1) \dots \exp(t_n X_n) \right)
\overset{\txt{Gruppenhomo.}}{=} \exp(t_1 Y_1) \dots \exp(t_n Y_n) \quad (*)
\]

Laut Satz 4.21 ist die Abbildung $\Phi \colon \mathfrak{g} \to G$ definiert durch
\[
\Phi(t_1 X_1 + \dots + t_n X_n) = \exp(t_1 X_1) \dots \exp(t_n X_n)
\]
auf einer Umgebung $U$ von $0 \in \mathfrak{g}$ ein Diffeomorphismus.

Sei $V = \Phi(U)$ die entsprechende Umgebung von $e \in G$. Auf $V$ schreiben wir $\psi$ als
\[
\psi = (\psi \circ \Phi) \circ \Phi^{-1}
\]
Mit Gleichung (*) schreiben wir $\psi \circ \Phi$ als
\[
\psi \circ \Phi \colon (t_1, \dots, t_n) \mapsto \exp(t_1 Y_1) \dots \exp(t_n Y_n)
\]

d.h. $\psi \circ \Phi$ ist glatt auf $U$, und $\Phi$ auch glatt auf $U$ $\Rightarrow \psi$ ist glatt auf $V$.

Der Homomorphismus $\psi$ ist also glatt auf einer Normalumgebung $W(e) \subseteq G$ von $e \in G$.\\

\underline{Behauptung:} $\psi$ ist glatt auf ganz $G$.

Auf $W(g) = l_g(W(e))$ schreiben wir
\[
\psi|_{W(g)}^{(h)} = l_{\psi(g)}^H \circ \psi|_{W(e)} \circ l_{g^{-1}}^G \quad \txt{da\;} \psi(g)\psi(g^{-1}h) = \psi(gg^{-1}h) = \psi(h)
\]
Wir wissen, die Links-Abbildung ist glatt, also ist $\psi$ glatt auf beliebiger $W(g)$.
\end{proof}

\textbf{Korollar 4.23}\\
Eine lokal-euklidische topologische Gruppe mit abzählbarer Topologie besitzt höchstens eine differenzierbare Struktur bzgl. der sie eine Lie-Gruppe wird.

\begin{proof}
    Betrachte die Identität 
\[
\mathrm{id}:(G, \mathscr{A}_1) \to (G, \mathscr{A}_2),
\]
dies ist stetig und damit auch differenzierbar (Kor 4.20) und damit ein Diffeomorphismus zwischen den differenzierbaren Strukturen.
\end{proof}

\textbf{Bemerkung:}\\
Das \emph{fünfte Hilbert-Problem} fragt, ob jede lokal-euklidische Gruppe eine Lie-Gruppe ist. Die Frage wurde 1952 von Montgomery und Zippin positiv beantwortet.\\

\textbf{Lemma 4.24}\\
Sei $\iota \colon H \to G$ eine Lie-Untergruppe und sei $X \in \mathfrak{g}$. Dann gilt:
\begin{itemize}
    \item[i)] $X \in D\iota(\mathfrak{h}) \Rightarrow \exp(tX) \in \iota(H) \subseteq G$ für alle $t \in \mathbb{R}$,
    \item[ii)] Gilt $\exp(tX) \in \iota(H)$ für alle $t$ in einem Intervall $I$, dann folgt $X \in D{\iota}(\mathfrak{h})$.
\end{itemize}

\begin{proof}
    Sei zunächst $X \in D\iota(\mathfrak{h})$, es existiert also ein $X_0 \in \mathfrak{h}$ mit $X = D{\iota}(X_0)$. Dann folgt
\[
\exp(tX) = \exp\left( D{\iota}(tX_0) \right) \overset{\text{4.18}}{=} \iota(\exp(tX_0)) \quad \implies \exp(tX) \in \iota(H) \quad \forall t \in \mathbb{R}.
\]

Für den Beweis der umgekehrten Richtung betrachten wir die glatte Abbildung
\[
I \subseteq \mathbb{R} \to G, \quad t \mapsto \exp(tX).
\]
Nach Voraussetzung liegt das Bild dieser Abbildung in $\iota(H)$. Daher existiert eine glatte Abbildung
\[
\alpha \colon I \to H \quad \text{mit} \quad \exp(tX) = \iota(\alpha(t)).
\]

Sei nun $\tilde{X}$ das links-invariante Vektorfeld auf $H$ zu $\dot{\alpha}(t_0)$. Dann gilt $D\iota(\tilde{X})=X$\\
(vgl. Bemerkungen unter Artikel 4.1 $\iota \circ l_a^H = l_{\iota(a)}^G \circ \iota$ und $D_\iota X_H(a)=X_G(\iota(a))\quad \forall a\in H$ und $X\in T_eH$, $X_H, X_G$ links-invariante Vektorfelder zu $X$)\\
und somit $X\in D\iota(\mathfrak{h})$
\end{proof}


\paragraph{Bemerkung:}
Für eingebettete Lie-Untergruppen $H \subseteq G$ mit Lie-Unteralgebra $\mathfrak{h} \subseteq \mathfrak{g}$ gilt also $X \in \mathfrak{h}$ genau dann, wenn $\exp(tX) \in H$ für alle $t$ aus einem kleinen Intervall um $0$.\\

\textbf{Lemma 4.25}\\
Sei $\varphi \colon G \to K$ ein Homomorphismus von Lie-Gruppen. Dann ist $\ker \varphi$ eine abgeschlossene Untergruppe von $G$ und es gilt
\[
\mathrm{Lie}(\ker \varphi) = \ker(D\varphi).
\]
\begin{proof}
  $\ker \varphi$ ist abgeschlossen: $\mathrm{ker}(\varphi)=\varphi{-1}(\{0\})$ und Urblider von stetiger Abbildung abgeschlosser Menge .\\
    Sei $X \in \mathfrak{g}$, dann ist nach vorigem Lemma
\[
X \in \mathrm{Lie}(\ker \varphi) \; \overset{\exp: \mathfrak{g}\rightarrow G}{\Leftrightarrow }\; \exp(tX) \in \ker(\varphi) \quad \forall t \in \mathbb{R},
\]
d.h.
\[
\varphi(\exp(tX)) = e_k \; \overset{4.18}{\Leftrightarrow} \; \exp(D\varphi(tX)) = e \quad \forall t \in \mathbb{R}.
\]

Da $\exp$ in einer Umgebung der Null ein lokaler Diffeomorphismus ist, folgt $D\varphi(X) = 0$, also $X \in \ker(D\varphi)$.
\end{proof}

\underline{Letzte Woche:}
Für kommutierende Matrizen $A, B$ gilt $e^{A+B} = e^A \cdot e^B$.\\

\textbf{Satz 4.26}\\
Seien $X, Y \in \mathfrak{g}$ mit $[X, Y] = 0$, dann gilt
\[
\exp(X + Y) = \exp(X) \cdot \exp(Y).
\]
\begin{proof}
    Sei $\mathfrak{a} := \mathrm{span} \{X, Y\} \subseteq \mathfrak{g}$, dann ist $\mathfrak{a}$ eine abelsche Unteralgebra von $\mathfrak{g}$. Sei $A \subseteq G$ die zusammenhängende Untergruppe mit Lie-Algebra $\mathfrak{a}$ (Existenz laut Satz 4.6). Laut Folgerung 4.11 ist die Gruppe $A$ abelsch. Wir definieren eine Kurve $\alpha \colon \mathbb{R} \to G$ durch
\[
\alpha(t) := \exp(tX) \cdot \exp(tY).
\]

Da $A$ abelsch ist, ist $\alpha \colon \mathbb{R} \to A \subseteq G$ ein glatter Homomorphismus (Vertauschung $\exp(tX), \exp(tY)$ erlaubt):
\[\alpha(s+t)=\exp(sX)\exp(tX)\exp(sY)\exp(tY)\overset{X,Y\txt{kommu}}{=}\alpha(s)\alpha(t)\]
Daher existiert ein $Z \in \mathfrak{g}$ mit $\alpha(t) = \exp(tZ)$, wobei $Z = \dot{\alpha}(0)\in\mathfrak{g}$.

Berechnet man die Ableitung von $\alpha(t) = \exp(tX) \cdot \exp(tY)$ in $t=0$, so erhält man andererseits
\[
Z=\dot{\alpha}(0) = X + Y.
\]
\[
\Rightarrow \quad \exp(tX) \cdot \exp(tY) = \alpha(t) = \exp(t(X + Y)).
\]
\end{proof}


\subsection{Die adjungierte Darstellung}

\paragraph{Sprechweise:}
Gruppen-Isomorphismen einer Gruppe $G$ auf sich selbst nennen wir \underline{Automorphismen}. Wir schreiben $\mathrm{Aut}(G)$ für die Gruppe aller Automorphismen.

\paragraph{Definition 4.27:}
Die Abbildung $\alpha_g \colon G \to G$ mit
\[
\alpha_g(h) := g h g^{-1}
\]
für ein $g \in G$ heißt \emph{\underline{Konjugation}} mit dem Gruppenelement $g$. ($\alpha_g = l_g \circ r_{g^{-1}} = r_{g^{-1}} \circ l_g $).

\paragraph{Lemma 4.28:}
Die Konjugation $\alpha_g$ hat folgende Eigenschaften:
\begin{itemize}
    \item[i)] Die Konjugation $\alpha_g \colon G \to G$ ist ein Gruppenhomomorphismus.
    \item[ii)] $\alpha_g \colon G \to G$ ist ein Gruppenisomorphismus, es gilt $(\alpha_g)^{-1} = \alpha_{g^{-1}}$.
    \item[iii)] Die Abbildung $\alpha \colon G \to \mathrm{Aut}(G) \quad g \mapsto \alpha_g$ ist ein Gruppenhomomorphismus.
\end{itemize}
\begin{proof}
    Für beliebige $g, h, a, b \in G$ berechnen wir:

\underline{Zu (i):}
\[
    \alpha_g(a \cdot b) = g a b g^{-1} = g a g^{-1} \cdot g b g^{-1} = \alpha_g(a) \cdot \alpha_g(b) \quad \;\alpha_g(e) = g e g^{-1} = e.
\]

\underline{Zu (ii):}
\begin{align*}
    (\alpha_g)^{-1} &= (l_g \circ r_{g^{-1}})^{-1} = r_{g^{-1}}^{-1} \circ l_g^{-1} = r_g \circ l_g^{-1} = l_{g^{-1}} \circ r_g = \alpha_{g^{-1}}.
\end{align*}

\underline{Zu (iii):}
\[
    \alpha_{gh}(a) = gh\cdot a\cdot (gh)^{-1}= g h a h^{-1} g^{-1} = g \cdot \alpha_h(a) \cdot g^{-1} = \alpha_g(\alpha_h(a)) = (\alpha_g \circ \alpha_h)(a)
\]
\[
    \Rightarrow \alpha_{gh} = \alpha_g \circ \alpha_h, \quad \text{und} \quad \alpha_e = \mathrm{id}.
\]
\end{proof}
\paragraph{Bemerkung:}
Ist $G$ eine Lie-Gruppe, so ist die Konjugation eine differenzierbare Abbildung und damit ein Homomorphismus von Lie-Gruppen, da die Gruppenoperationen differenzierbare Abbildungen sind.

Gruppen-Automorphismen von der Form $\alpha_g \colon G \to G,\ g \in G$, heißen \emph{innere Automorphismen}.

Das Differential der Konjugation definiert eine Abbildung $(D\alpha_g)_e \colon \mathfrak{g} \to \mathfrak{g}$. Schreiben:
\[
\mathrm{Ad}(g) := (D\alpha_g)_e, \quad \text{d.h.} \quad \mathrm{Ad}(g) = D\alpha_g \circ Dr_{g^{-1}}
\]

\textbf{Lemma 4.29}\\
Das Differential der Konjugation hat die folgenden Eigenschaften:
\begin{itemize}
    \item[i)] Die Abbildung $\mathrm{Ad}(g) \colon \mathfrak{g} \to \mathfrak{g}$ ist ein Lie-Algebren-Homomorphismus, d.h. es gilt:
    \[
    \mathrm{Ad}(g)([X, Y]) = [\mathrm{Ad}(g)X,\ \mathrm{Ad}(g)Y].
    \]

    \item[ii)] Die Abbildung $\mathrm{Ad}(g)$ ist ein Lie-Algebren-Isomorphismus, d.h. $\mathrm{Ad}(g) \in \mathrm{Aut}(\mathfrak{g}) = \mathrm{GL}(\mathfrak{g})$ und es gilt: 
    \[\mathrm{Ad}(g)^{-1} = \mathrm{Ad}(g^{-1}) \quad\forall g \in G
    \]
    
    \item[iii)] Die Abbildung $\mathrm{Ad} \colon G \to \mathrm{Aut}(\mathfrak{g})\quad g\mapsto \mathrm{Ad}(g)$ ist ein Gruppenhomomorphismus.
\end{itemize}

\begin{proof}
    Zu (i): vgl. Lemma 4.8, dort haben wir gezeigt, dass das Differential eines Lie-Gruppen-Homomorphismus ein Lie-Algebren-Homomorphismus der assoziierten Lie-Algebren ist.

\medskip
Zu (ii):
\(
\mathrm{Ad}(g)^{-1} = (D\alpha_{g})^{-1} = (D\alpha_g^{-1}) = D\alpha_{g^{-1}} = \mathrm{Ad}(g^{-1}).
\)

\medskip
Zu (iii):
\(
\mathrm{Ad}(gh) = D\alpha_{gh} = D(\alpha_g \circ \alpha_h) = D\alpha_g \circ D\alpha_h = \mathrm{Ad}(g) \circ \mathrm{Ad}(h).
\)

\medskip
Es gilt sogar: $\mathrm{Ad} \colon G \to \mathrm{Aut}(\mathfrak{g})$ ist eine differenzierbare Abbildung, also ein \emph{Lie-Gruppen Homomorphismus}.\\

Für kleines $t$ folgt aus
\(
\varphi(\exp(tX)) = \exp(D\varphi(tX)) \quad \text{für } \varphi = \alpha_g \quad \text{(Satz 4.18)}:
\)
\[
\operatorname{Ad}(g)X = \frac{1}{t} \exp^{-1} \left( \alpha_g \left( \exp(tX) \right) \right) 
= \frac{1}{t} \exp^{-1} \left( g \exp(tX) g^{-1} \right)
\]

$\Rightarrow$ \quad $\mathrm{Ad}$ ist stetiger Homomorphismus von Lie-Gruppen $\Rightarrow$ $\mathrm{Ad}$ ist differenzierbar (nach 4.22).
\end{proof}


\paragraph{Bemerkung:}
Das Differential der Abbildung $\mathrm{Ad} \colon G \to \mathrm{GL}(\mathfrak{g})$ ist ein Lie-Algebren-Homomorphismus
\[
(D \mathrm{Ad})_e \colon \mathfrak{g} \to \mathfrak{gl}(\mathfrak{g}).
\]
Schreiben: \quad $\mathrm{ad} := (D \mathrm{Ad})_e$

\subsubsection*{Definition 4.30}
Sei $G$ eine Lie-Gruppe mit Lie-Algebra $\mathfrak{g}$.\\
Eine \underline{Darstellung von $G$} auf einem Vektorraum $V$ ist ein Lie-Gruppen-Homomorphismus
\[
\rho \colon G \to \mathrm{GL}(V) = \mathrm{Aut}(V) =\{\txt{inv. lineare Abb }V\rightarrow V\} \underset{\txt{nach Wahl Basis}}{\cong} \mathrm{GL}(\txt{dim}V)
\]
d.h.
\[
\rho(g_1 g_2) = \rho(g_1) \cdot \rho(g_2) \quad \text{für alle } g_1, g_2 \in G.
\]

\medskip
Eine \underline{Darstellung von $\mathfrak{g}$} auf $V$ ist ein Lie-Algebren-Homomorphismus
\[
\varphi \colon \mathfrak{g} \to \mathfrak{gl}(V) = \mathrm{End}(V) =\{\txt{lineare Abb. }V\rightarrow V\}\cong M(\txt{dim}V, \mathbb{R}) 
\]
d.h. es gilt
\[
\varphi([X, Y]) = \varphi(X) \varphi(Y) - \varphi(Y) \varphi(X).
\]

\medskip
\paragraph{Beispiel:} Darstellung von $S^1 = \mathbb{R}/\mathbb{Z}$ auf $\mathbb{R}^2$
\[
\rho(\theta) = \begin{pmatrix}
\cos(\theta) & \sin(\theta) \\
-\sin(\theta) & \cos(\theta)
\end{pmatrix} \colon \mathbb{R}^2 \to \mathbb{R}^2
\quad \text{(Additions-Theorem)}.
\]

\medskip
(Lie$(S^1) = \mathbb{R}$, $[,] = 0$)

\[
\varphi(t) = \begin{pmatrix}
0 & t \\
-t & 0
\end{pmatrix} \colon \mathbb{R}^2 \to \mathbb{R}^2
\]

\paragraph{Bemerkung:}
Sei $\rho \colon G \to \mathrm{GL}(V)$ eine Darstellung von $G$, dann ist das Differential von $\rho$, also der Homomorphismus
\[
\varphi = D\rho \colon \mathfrak{g} \to \mathrm{End}(V),
\]
eine Darstellung der Lie-Algebra $\mathfrak{g}$.\\

\textbf{Definition 4.31}\\
Der Gruppenhomomorphismus $\mathrm{Ad} \colon G \to \mathrm{GL}(\mathfrak{g})$ heißt \emph{\underline{adjungierte Darstellung}} von $G$.

Der Lie-Algebren-Homomorphismus $\mathrm{ad} \colon \mathfrak{g} \to \mathfrak{gl}(\mathfrak{g})$ heißt \emph{\underline{adjungierte Darstellung}} von $\mathfrak{g}$.\\

\textbf{Satz 4.32}\\
Für die adjungierte Darstellung von $\mathfrak{g}$ gilt:
\[
\mathrm{ad}(X)Y = [X, Y] \quad \forall X, Y \in \mathfrak{g}.
\]

\begin{proof}
    Sei $X$ ein links-invariantes Vektorfeld, dann ist die Integralkurve durch $g \in G$ gegeben durch
\[
c_g(t) = g \cdot \exp(tX).
\]

D.h. der lokale Fluss von $X$ ist gegeben durch
\[
\varphi_t(g) = c_g(t) = r_{\exp(tX)}(g) \qquad \text{(Rechtsmultiplikation)}.
\]
\begin{align*}
\operatorname{ad}(X)Y 
&= D\operatorname{Ad}(X)Y 
= \left[ \frac{d}{dt} \right]_{t=0} \operatorname{Ad}\left(\exp(tX)\right)Y \\
&= \left. \frac{d}{dt} \right|_{t=0} \left( Dr_{{\exp(-tX)}} Dl_{{\exp(tX)}} Y \right) 
\hspace{2em} \textcolor{orange}{\left[\text{Def. von } \operatorname{Ad} = D(r_{g^{-1}}\circ l_g)\right]} \\
&= \left. \frac{d}{dt} \right|_{t=0} \left( Dr_{{\exp(-tX)}} \, \widetilde{Y}_{\exp(tX)} \right) 
\hspace{2em} \textcolor{orange}{\left[\text{Def. linksinv. } \widetilde{Y}\right]} \\
&= \left. \frac{d}{dt} \right|_{t=0} \left( D\varphi_{{-t}} \, \widetilde{Y}_{\exp(tX)} \right) 
\hspace{2em} \textcolor{orange}{\left[\text{Def. von } \varphi_t(g)\right]} \\
&\overset{\txt{Satz 3.25}}{=} [\widetilde{X}, \widetilde{Y}]_e = [X, Y]
\end{align*}
\end{proof}


\paragraph{Bemerkung:}
\begin{itemize}
    \item[i)] Für Matrixgruppen $G \subseteq \mathrm{GL}(n, \mathbb{R})$ gilt:
    \[
    \mathrm{Ad}(g) A = g A g^{-1} \quad \text{für alle } g \in G \text{ und } A \in \mathfrak{g}.
    \]

    \item[ii)] Es gilt:
    \[
    g \exp(X) g^{-1} = \exp(g X g^{-1}) = \exp(\operatorname{Ad}(g)X),
    \]
    d.h. das folgende Diagramm kommutiert (vgl. Satz 4.18):
    \[
    \begin{array}{ccc}
        \mathfrak{g} & \xrightarrow{\mathrm{Ad}(g) = D(\alpha_g)} & \mathfrak{g} \\
        \exp \downarrow &  & \downarrow \exp \\
        G & \xrightarrow{\alpha_g} & G
    \end{array}
    \]
    Speziell für Matrixgruppen:
\[
B e^A B^{-1} = e^{BAB^{-1}}.
\]
  \item[iii)] Es gilt:
\(
\mathrm{Ad}(\exp X) = \exp(\mathrm{ad}(X)),
\)
d.h. das folgende Diagramm kommutiert:
\[
\begin{array}{ccc}
\mathfrak{g} & \xrightarrow{\mathrm{ad} = D\mathrm{Ad}} & \mathrm{End}(\mathfrak{g}) \cong \mathfrak{gl}(n, \mathbb{R}) \\
\exp \downarrow &  & \downarrow \exp \\
G & \xrightarrow{\mathrm{Ad}} & \mathrm{GL}(\mathfrak{g}) \cong \mathrm{GL}(n, \mathbb{R})
\end{array}
\]
\hfill (vgl. Satz 4.18)
\end{itemize}

\subsubsection*{Definition 4.33}
Eine \emph{\underline{Derivation}} einer Lie-Algebra $\mathfrak{g}$ ist eine lineare Abbildung
\(
\psi \colon \mathfrak{g} \to \mathfrak{g}
\)
mit
\[
\psi([X, Y]) = [\psi(X), Y] + [X, \psi(Y)] \quad \forall X, Y \in \mathfrak{g}.
\]

Wir schreiben $\mathrm{Der}(\mathfrak{g})$ für den Vektorraum aller Derivationen von $\mathfrak{g}$.\\

\textbf{Satz 4.34}\\
Die adjungierte Darstellung $\mathrm{ad}(X) \colon \mathfrak{g} \to \mathfrak{g}$ ist eine Derivation von $\mathfrak{g}$.

\begin{proof}
    Die Behauptung ist eine Umformulierung der Jacobi-Identität von $\mathfrak{g}$:
\begin{align*}
\mathrm{ad}(X)([Y, Z]) &\overset{4.32}{=} [X, [Y, Z]] \overset{\text{Jacobi}}{=} -[Y, [Z, X]] - [Z, [X, Y]] \\
&= [ [X, Y], Z ] + [Y, [X, Z]] \overset{4.32}{=} [\mathrm{ad}(X)(Y), Z] + [Y, \mathrm{ad}(X)(Z)].
\end{align*}
\end{proof}

\textbf{Satz 4.35}\\
Es gilt:
\(
\mathrm{Der}(\mathfrak{g}) = \mathrm{Lie}(\mathrm{Aut}(\mathfrak{g})).
\)

Wir benutzen Lemma 4.24: $\iota \colon H \hookrightarrow G$ Lie-Untergruppe. Dann:
\begin{itemize}
    \item[i)] $X \in D{\iota}(H) \Rightarrow \exp(tX) \in \iota(H) \subseteq G$
    \item[ii)] $\exp(tX) \in \iota(H)$ für alle $t$ in einem Intervall $I$ $\Rightarrow X \in D{\iota}(\mathfrak{h})$
\end{itemize}

\underline{Bemerkung:} 
$X \in \mathfrak{h} \quad \Leftrightarrow \quad \exp(tX) \in H$ für alle $t \in (-\varepsilon, \varepsilon)$

\begin{proof}
    \[
\psi \in \mathrm{Lie} \left( \mathrm{Aut}(\mathfrak{g}) \right) 
\subset \mathrm{End}(\mathfrak{g}) 
\;\overset{\txt{Bem.}}{\Longleftrightarrow}\; 
e^{t\psi} \in \mathrm{Aut}(\mathfrak{g}) 
\quad \text{für alle } t \text{ in einem Intervall um } 0
\]
\[
\overset{\exp \;\txt{insbes. Homo}}{\Longleftrightarrow}\; 
e^{t\psi} [X, Y] = [e^{t\psi} X,\, e^{t\psi} Y]
\]
\[
\text{Ableiten nach } t \text{ in } t = 0: 
\psi([X, Y]) = [\psi(X), Y] + [X, \psi(Y)]
\]
\[\Rightarrow \psi \in \mathrm{Der}(\mathfrak{g})
\Rightarrow \mathrm{Lie}(\mathrm{Aut}(\mathfrak{g})) \subseteq \mathrm{Der}(\mathfrak{g})\]
{Für die andere Inklusion:}
Sei \(\psi\) eine Derivation von \(\mathfrak{g}\). Induktiv gilt:
\[
\psi^k([X, Y]) = \sum_{a + b = k} \frac{k!}{a! \cdot b!} \left[ \psi^a(X), \psi^b(Y) \right] \quad (\psi^0(X)=X)
\]
\begin{align*}
e^{t\psi} [X, Y] 
&= \sum_{k} \frac{1}{k!} (t\psi)^k [X, Y] = \sum_{k} \sum_{a + b = k} \frac{1}{a! b!} \left[ (t\psi)^a X, (t\psi)^b Y \right] \\
&= \sum_{a, b} \left[ \frac{1}{a!} (t\psi)^a X, \frac{1}{b!} (t\psi)^b Y \right] = \left[ e^{t\psi}(X),\, e^{t\psi}(Y) \right]
\end{align*}
\[
\Rightarrow \quad e^{t\psi} \text{ ist ein Automorphismus von } \mathfrak{g}
\quad \Rightarrow \quad \psi \in \mathrm{Lie}(\mathrm{Aut}(\mathfrak{g}))
\]
\[
\Rightarrow \quad \mathrm{Der}(\mathfrak{g}) \subseteq \mathrm{Lie}(\mathrm{Aut}(\mathfrak{g}))
\]
\end{proof}
Damit können wir das Diagramm oberhalb von Definition 4.33 präzisieren:
\[
\begin{array}{ccc}
\mathfrak{g} & \xrightarrow{\mathrm{ad}} & \mathrm{Der}(\mathfrak{g}) \subseteq \mathrm{End}(\mathfrak{g}) \\
\exp \downarrow &  & \downarrow \exp \\
G & \xrightarrow{\mathrm{Ad}} & \mathrm{Aut}(\mathfrak{g}) \subseteq \mathrm{GL}(\mathfrak{g})
\end{array}
\]
%%%%%%%%%%%%%%%%%%%%%%%%%%%%%%%%%%%%%%%%%%%%%%%%%%%%%%%%%%%%%%%%%%%%%%%%%%%% Vorlesung 13 %%%%%%%%%%%%%%%%%%%%%%%%%%%%%%%%%%%%%%
Wir betrachten ein paar Anwendungen der adjungierten Darstellung\\

\textbf{Definition 4.36}

Das \underline{Zentrum} \( Z(G) \subseteq G \) von \( G \) bzw. \( Z(\mathfrak{g}) \subseteq \mathfrak{g} \) ist definiert als:
\[
Z(G) = \{ g \in G \mid gh = hg \text{ für alle } h \in G \}
\]
\[
Z(\mathfrak{g}) = \{ X \in \mathfrak{g} \mid [X, Y] = 0 \text{ für alle } Y \in \mathfrak{g} \}
\]

\textbf{Bemerkung:}
\[
Z(G) = \ker(\mathrm{\alpha}) \quad \text{und} \quad Z(\mathfrak{g}) = \ker(\mathrm{ad})
\]

\textbf{Beispiele:}
\[
Z(O(n)) = \{ \pm \mathbbm{1}_n \} \cong \mathbb{Z}_2
\]
\[
Z(SU(n)) = \left\{ \mathrm{diag} \left( e^{\frac{2\pi i k}{n}}, \dots, e^{\frac{2\pi i k}{n}} \right) \;\middle|\; 1 \leq k \leq n \right\} \cong \mathbb{Z}_n
\]
\[
Z(\mathrm{GL}(n, \mathbb{R})) = \{ c \cdot \mathbbm{1}_n \mid c \neq 0 \}
\]

\textbf{Satz 4.37}

Das Zentrum \( Z(G) \subseteq G \) ist eine abgeschlossene Lie-Untergruppe, und es gilt:
\[
Z(G) \subseteq \ker(\mathrm{Ad}).
\]

Für zusammenhängende Gruppen \( G \) gilt: 
\[
Z(G) = \ker(\mathrm{Ad}) \quad \text{und außerdem} \quad \mathrm{Lie}(Z(G)) = Z(\mathfrak{g}).
\]

\begin{proof}
Für \( g \in Z(G) \) gilt:
\[
\alpha_g = r_{g^{-1}} \circ l_g=id_G \; \Rightarrow \; \mathrm{Ad}(g) = D(\alpha_g) = D\mathrm{id}_G=\mathrm{id}_\mathfrak{g}
\; \Rightarrow \; g \in \ker(\mathrm{Ad}) \Rightarrow Z(G) \subseteq \ker(\mathrm{Ad}).
\]

Sei nun \( G \) zusammenhängend und \( g \in \ker(\mathrm{Ad}) \).\\
Da \( G \) von einer Normalenumgebung der Eins erzeugt wird (Satz 4.12), genügt es, die folgende Aussage zu zeigen, dass
\[
\alpha_g(\exp(tX)) = \exp(tX) \quad \text{für alle } X \in \mathfrak{g},\ t \in \mathbb{R}.
\]
also $\alpha_g=\mathrm{id}$ in $\forall g\in W(g)=l_g(W(e))$.\\
\emph{(Siehe Satz 4.12 für die genaue Aussage.)}

Betrachte den Homomorphismus
\[
\mathbb{R} \to G, \quad t \mapsto \alpha_g(\exp(tX)).
\]
Folgerung 4.20:
Es gibt ein \( Y \in \mathfrak{g} \) mit
\[
\exp(tY) = \alpha_g(\exp(tX)) \quad \forall t \in \mathbb{R}.
\]

Ableiten in \( t = 0 \):
\[
Y_e = \left. \frac{d}{dt} \right|_{t=0} \exp(tY) = \left. \frac{d}{dt} \right|_{t=0} \alpha_g(\exp(tX)) \overset{\txt{KR}}{=} D \alpha_g (X_e) = \mathrm{Ad}(g)(X_e) \underset{\mathrm{Ad}(g) = \mathrm{id}}{\overset{g \in \ker(\mathrm{Ad})}{=}} X_e.
\]

Also \( X = Y \) und \( \alpha_g = \mathrm{id} \), d.h. \( g \in Z(G) \).
Für die Lie-Algebra von \( Z(G) \) berechnen wir mit Satz 4.25:
\[
\mathrm{Lie}(Z(G)) =\mathrm{Lie}(\mathrm{ker}(\mathrm{Ad}))\overset{4.25}{=}\ker(D\mathrm{Ad})=\ker(\mathrm{ad})=Z(\mathfrak{g}).
\]
\end{proof}

\textbf{Korollar 4.38}

Eine zusammenhängende Lie-Gruppe ist genau dann abelsch, wenn ihre Lie-Algebra abelsch ist.
\begin{proof}
\[
G \text{ ist abelsch} \; \Leftrightarrow \; Z(G) = G \overset{G \;\txt{zsh} \;(*)}{\Leftrightarrow} \mathfrak{g}=\mathrm{Lie}(G)=\mathrm{Lie}(Z(G))\overset{4.37}{=}Z(\mathfrak{g}) \;\Leftrightarrow\; \mathfrak{g}\text{ ist abelsch} 
\]
Für (*) ist ($\Rightarrow $) aus Satz 4.37, ($\Leftarrow$) folgt aus Definition zusammenhängend, da $Z(G)$ offen und abgeschlossen in G liegt.
\end{proof}

\textbf{Bemerkung:}

Ist \( \mathfrak{g} \) eine Lie-Algebra mit \( Z(\mathfrak{g}) = \{0\} \), dann besitzt \( \mathfrak{g} \) eine injektive Darstellung
\[
\mathrm{ad} \colon \mathfrak{g} \rightarrow \mathrm{End}(\mathfrak{g}),
\]
d.h. \( \mathfrak{g} \) ist isomorph zu einer Lie-Algebra einer Lie-Gruppe. \\

\smallskip
 Das ist das \emph{Theorem von Ado} für \( Z(\mathfrak{g}) = \{0\} \).\\

\textbf{Satz 4.39}

Sei \( H \subseteq G \) eine zusammenhängende Lie-Untergruppe einer zusammenhängenden Lie-Gruppe \( G \). Dann ist die Gruppe \( H \) genau dann Normalteiler in \( G \), wenn \( \mathfrak{h} = \mathrm{Lie}(H) \) ein Ideal in \( \mathfrak{g} = \mathrm{Lie}(G) \) ist.

\underline{Erinnerung:}
\\
Normalteiler:
\[
gHg^{-1} = H \quad \text{für alle } g \in G
\]

Ideal:
\( \mathfrak{g} \) ist eine Algebra wobei \( + \) : Vektorraumaddition und \( \circ=[\cdot,\cdot] \). \( \mathfrak{h} \subseteq \mathfrak{g} \) ist ein \emph{Ideal}, falls \( \mathfrak{h} \) eine Untergruppe von \( (\mathfrak{g}, +) \) ist und für alle \( h \in \mathfrak{h},\ g \in \mathfrak{g} \) gilt: \([h, g] \in \mathfrak{h}\)
(bzw. \([ \mathfrak{h}, \mathfrak{g} ] \subseteq \mathfrak{h}\))

\begin{proof}
Sei \( \mathfrak{h} \subseteq \mathfrak{g} \) ein Ideal von \( \mathfrak{g} \).\\
Sei \( Y \in \mathfrak{h} \), \( X \in \mathfrak{g} \) und \( g = \exp(X) \). Dann gilt:
\[
g \exp(Y) g^{-1}
= \exp(X) \exp(Y) \exp(-X)
\underset{\txt{unter 4.32}}{\overset{\txt{kommut. Diag}}{=}} \exp\left( \mathrm{Ad}(\exp(X)) Y \right)
\]
\[= \exp\left( e^{\mathrm{ad}(X)} Y \right)
= \exp\left( \sum_{n=0}^{\infty} \frac{1}{n!} \mathrm{ad}(X)^n Y \right)\overset{4.32}{=}\exp(Y+[X,Y]+\frac{1}{2!}[X,[X,Y]]+...)\]
\[=\exp(Z)\]

Da alle Kommutatoren in \(\mathfrak{h} \) liegen (da $\mathfrak{h}$ Ideal), konvergiert die Reihe gegen ein \( Z \in \mathfrak{h} \quad (\exp^G(\mathfrak{h})\subseteq H) \).
\[
\Rightarrow \quad g \exp(Y) g^{-1} \in H.
\]

Da \( H \) und \( G \) von Normalumgebung des neutralen Elements erzeugt werden (Satz 4.12), folgt, dass \( H \) ein Normalteiler von \( G \) ist.\\

Für die umgekehrte Richtung: Sei \( H \subseteq G \) ein Normalteiler und \( Y \in \mathfrak{h},\ X \in \mathfrak{g}, g_t = \exp(tX)
\). Wir zeigen $[X,Y]\in\mathfrak{h}$:
\[
g_t \exp(sY) g_t^{-1} =\exp(\mathrm{Ad}(g_t)sY)=\exp(se^{t\;\mathrm{ad}X}Y).
\]

Da \( H \) ein Normalteiler ist, gilt für alle $s\in\mathbb{R}$: \(
g_t \exp(sY) g_t^{-1} \in H \quad 
\)
\[\Rightarrow e^{t\;\mathrm{ad}(X)}Y\in\mathfrak{h }\text{ für alle } t \in \mathbb{R}.\]
\[
\Rightarrow t \mapsto e^{t\;\mathrm{ad}X}Y=Y+t[X,Y]+...
\]
ist eine glatte Kurve in \( \mathfrak{h} \) mit Tangentialvektor
\(
[X, Y] \text{ in } t = 0.
\) Also \( [X, Y] \in \mathfrak{h} \), d.h.
\(\Rightarrow \mathfrak{h}\subseteq\mathfrak{g}\) ist ein Ideal.

\end{proof}


\section{Differentialformen}

\subsection{Der Vektorraum der $k$-Formen}

Sei \( V \) ein \( n \)-dimensionaler Vektorraum und sei
\[
V^* := \mathrm{Hom}(V, \mathbb{R})
\]
der \emph{Dualraum} von \( V \), d.h. der Raum der linearen Abbildungen \( V \to \mathbb{R} \).

Später wollen wir insbesonder \( T_pM \) betrachten.\\

\textbf{Definition 5.1:} 

Eine \( k \)-Linearform (kurz: \( k \)-Form) auf \( V \) ist eine multilineare, alternierende Abbildung
\[
\omega \colon \underbrace{V \times \cdots \times V}_{k\text{-mal}} \longrightarrow \mathbb{R}.
\]
Die Zahl \( k \) heißt der \emph{Grad} der \( k \)-Linearform.\\

Dabei ist \( \omega \) alternierend genau dann, wenn eine der folgenden vier äquivalenten Bedingungen erfüllt ist:
\begin{enumerate}[label=\textit{\roman*)}]
    \item \( \omega(X_{\sigma(1)}, \dots, X_{\sigma(k)}) = \mathrm{sgn}(\sigma)\, \omega(X_1, \dots, X_k) \quad \) für alle \( \sigma \in S_k \),
    
    \item \( \omega(\dots, X_i, \dots, X_j, \dots) = -\omega(\dots, X_j, \dots, X_i, \dots) \quad \) für alle \( i \neq j \),
    
    \item \( \omega(X_1, \dots, X_k) = 0 \quad \) falls es \( i \neq j \) mit \( X_i = X_j \) gibt,
    
    \item \( \omega(X_1, \dots, X_k) = 0 \quad \) falls \( X_1, \dots, X_k \) linear abhängig sind.
\end{enumerate}

\textbf{Bemerkung:}

Die Menge der \( k \)-Linearformen auf \( V \) bildet einen reellen Vektorraum.\\
Diesen bezeichnen wir mit
\[
\Lambda^k V^*.
\]

Konvention: \quad \( \Lambda^0 V^* := \mathbb{R} \)

\medskip
Für \( k > \dim V \) gilt:
\[
\Lambda^k V^* = \{0\} \qquad \text{(Nullform, siehe Charakterisierung iv))}
\]
Addition und skalare Multiplikation sind definiert durch:
\[
(\omega_1 + \omega_2)(X_1, \dots, X_k) := \omega_1(X_1, \dots, X_k) + \omega_2(X_1, \dots, X_k),
\]
\[
(\lambda \cdot \omega)(X_1, \dots, X_k) := \lambda \cdot \omega(X_1, \dots, X_k).
\]

\textbf{Beispiele:}

\begin{itemize}
  \item[(i)] \( 1 \)-Linearformen: \( \Lambda^1 V^* = V^*=\txt{Hom}(V,\mathbb{R)} \). Insbesondere gilt: \( \dim \Lambda^1 V^* = n \), falls \( V \) ein \( n \)-dimensionaler Vektorraum ist.
  
  \item[(ii)] \( 2 \)-Linearformen: \( \Lambda^2 V^*= \) Menge der schief symmetrische bilineare Abblidungen.\\
 \( \omega \colon V \times V \to \mathbb{R} \), d.h. es gilt \( \omega(X, Y) = -\omega(Y, X) \quad \forall X, Y \in V \).
\end{itemize}

\textbf{Lemma 5.2:}
\[
\Lambda^2 V^* \cong \mathfrak{so}(V) := \left\{ A \in  M(n,\mathbb{R}) \mid A^T = -A \right\}.
\]

\begin{proof}
Sei \( \langle \cdot, \cdot \rangle \) ein Skalarprodukt auf \( V \).\\
Definiere die Identifikation durch \( \Lambda^2 V^* \to \mathrm{End}(V) \), 
\[
\omega \mapsto \hat{\omega} \quad \text{mit} \quad \langle \hat{\omega}(X), Y \rangle := \omega(X, Y).
\]
\end{proof}

\textbf{Korollar 5.3:}
\[
\dim \Lambda^2 V^* = \binom{n}{2} =\frac{n(n-1)}{2}.
\]

\begin{itemize}
  \item[(iii)] Sei \( V \) ein \( n \)-dimensionaler Vektorraum und \( \omega \) eine \( n \)-Form auf \( V \). Sei \( e_1, \dots, e_n \) eine Basis von \( V \), und sei \( v_1, \dots, v_n \in V \) als Spaltenvektoren betrachten. Dann gilt:
  \[
  \omega(v_1, \dots, v_n) = \det(v_1,...,v_n)\cdot\omega(e_1,...,e_n),
  \]
   \(\Rightarrow \Lambda^n V^* = \mathbb{R} \cdot \det \cong \mathbb{R}. \)
\end{itemize}

\textbf{Definition 5.4:}

Die \underline{äußere Algebra} von \( V \) ist definiert als
\(
\Lambda(V^*) := \bigoplus_{k=0}^n \Lambda^k V^*,
\)
d.h. als Menge der Formen von beliebigem Grad auf \( V \). \\

Die Vektorraumstruktur kommt von den einzelnen Summanden \( \Lambda^k V^* \),
die Produktstruktur ist definiert durch das Dachprodukt (Wedge-Produkt)
\[
\wedge \colon \Lambda^k V^* \times \Lambda^\ell V^* \longrightarrow \Lambda^{k+\ell} V^*,
\qquad (\alpha, \beta) \mapsto \alpha \wedge \beta.
\]

Dabei ist \( \wedge \) definiert durch:
\[
(\alpha \wedge \beta)(X_1, \dots, X_{k+\ell}) = \frac{1}{k! \, \ell!} \sum_{\sigma \in S_{k+\ell}} \mathrm{sgn}(\sigma)\,
\alpha(X_{\sigma(1)}, \dots, X_{\sigma(k)}) \cdot
\beta(X_{\sigma(k+1)}, \dots, X_{\sigma(k+\ell)}).
\]

\textbf{Bemerkung:}

Der Raum der \((0,k)\)-Tensoren auf \( V \) ist definiert als der Vektorraum der \( k \)-fach multilinearen Abbildungen:
\[
\mathcal{T}^{(0,k)}(V^*) := \left\{ \lambda \colon \underbrace{V \times \cdots \times V}_{k\text{-mal}} \to \mathbb{R} \;\middle|\; \lambda \text{ multilinear} \right\}.
\]

Das \textbf{Tensorprodukt} auf der Tensoralgebra \( \mathcal{T}(V^*) := \bigoplus_k \mathcal{T}^{(0,k)}(V^*) \) ist definiert durch:
\[
(\lambda \otimes \mu)(X_1, \dots, X_{k+\ell}) := \lambda(X_1, \dots, X_k) \cdot \mu(X_{k+1}, \dots, X_{k+\ell}).
\]

Das \textbf{Wedge-Produkt} der äußeren Algebra ergibt sich als Projektion des Tensorprodukts auf den Unterraum
\[
\Lambda^k V^* \subseteq \mathcal{T}^{(0,k)}(V^*).
\]

Man definiert die Abbildung
\[
\mathrm{Alt} \colon \mathcal{T}^{(0,k)}(V^*) \to \Lambda^k V^* \quad \mathrm{Alt}(\lambda)(X_1, ...,X_k) := \sum_{\sigma \in S_{k}} \mathrm{sgn}(\sigma)\lambda(X_{\sigma(1)}, ...,X_{\sigma(k)}) .
\]

Für eine Bilinearform \( \lambda \in \mathcal{T}^{(0,2)}(V^*) \) gilt also:
\[
\mathrm{Alt}(\lambda)(X, Y) = \lambda(X, Y) - \lambda(Y, X).
\]

In dieser Notation kann man für das Wedge-Produkt von \( \alpha \in \Lambda^k V^*, \beta \in \Lambda^\ell V^* \) auch schreiben als:
\[
\alpha \wedge \beta = \frac{1}{k! \, \ell!} \, \mathrm{Alt}(\alpha \otimes \beta).
\]

\textbf{Beispiele:}

\begin{itemize}
  \item[(i)] Seien \( \alpha, \beta \) \( 1 \)-Formen auf \( V \), dann gilt:
  \[
  (\alpha \wedge \beta)(X, Y) = \alpha(X) \beta(Y) - \alpha(Y) \beta(X).
  \]
  
  \item[(ii)] Sei \( \alpha \) eine \( 1 \)-Form und \( \beta \) eine \( 2 \)-Form auf \( V \), dann gilt:
  \begin{align*}
  (\alpha \wedge \beta)(X, Y, Z)
  &= \alpha(X)\beta(Y, Z) - \alpha(Y)\beta(X, Z) + \alpha(Z)\beta(X, Y) \\
  &= \alpha(X)\beta(Y, Z) + \alpha(Y)\beta(Z, X) + \alpha(Z)\beta(X, Y).
  \end{align*}
\end{itemize}
  
Allgemein: Sei \( \alpha \) eine \( 1 \)-Form, \( \beta \) eine \( k \)-Form, dann gilt:
\[
(\alpha \wedge \beta)(X_0, \dots, X_k) = \sum_{i=0}^k (-1)^i \alpha(X_i)\cdot \beta(X_0, \dots, \widehat{X_i}, \dots, X_k),
\]
wobei \( \widehat{X_i} \) bedeutet, dass der entsprechende Eintrag ausgelassen wird.\\

\textbf{Lemma 5.5:}

Das Wedge-Produkt von Formen hat folgende Eigenschaften:

\begin{itemize}
  \item[(i)] \( \alpha \wedge \beta = (-1)^{k \ell} \, \beta \wedge \alpha \quad \)
  für alle \( \alpha \in \Lambda^k V^*, \; \beta \in \Lambda^\ell V^* \)
  
  \item[(ii)] \( \omega \wedge \omega = 0 \quad \)
  für alle \( \omega \) mit ungeradem Grad
  
  \item[(iii)] \( (\lambda \alpha + \mu \beta) \wedge \gamma = \lambda (\alpha \wedge \gamma) + \mu (\beta \wedge \gamma) \quad \)
  für alle \( \lambda, \mu \in \mathbb{R}, \; \alpha, \beta, \gamma \in \Lambda( V^*) \)
  
  \item[(iv)] \( (\alpha \wedge \beta) \wedge \gamma = \alpha \wedge (\beta \wedge \gamma) \quad \)
  für alle \( \alpha, \beta, \gamma \in \Lambda( V^*) \)
\end{itemize}

\subsection{Das Verhalten unter Abbildungen}

Sei \( f \colon V \to W \) eine lineare Abbildung zwischen Vektorräumen \( V \) und \( W \).\\
Wir definieren
\[
f^* := \Lambda^k(f) \colon \Lambda^k W^* \longrightarrow \Lambda^k V^*, \quad
\omega \mapsto f^* \omega \quad (\text{Pullback)},
\]
wobei
\[
f^* \omega(v_1, \dots, v_k) := \omega(f(v_1), \dots, f(v_k)),
\]
für \( v_1, \dots, v_k \in V \) und \( \omega \in \Lambda^k W^* \).\\

Die von \( f \) induzierte Abbildung \( f^* \) heißt das \emph{Zurückziehen} von Formen.\\

Die Zuordnung \( V \mapsto \Lambda^k V^*, \; f \mapsto f^* \) ist ein \emph{kontravarianter Funktor} auf der Kategorie der Vektorräume.\\

\textbf{Lemma 5.6:}

Das Zurückziehen hat folgende Eigenschaften:
\begin{itemize}
  \item[(i)] \( (g \circ f)^* = f^* \circ g^* \)
  \item[(ii)] \( f^*(\alpha \wedge \beta) = f^*(\alpha) \wedge f^*(\beta) \)
  \item[(iii)] \( \det(f) = f^*:\Lambda^nV^* \rightarrow \Lambda^nV^* \), falls \( \dim V = n \)
\end{itemize}

\subsection{Eine Basis im Raum der Formen}

Sei \( e_1, \dots, e_n \) eine Basis in \( V \), und sei \( e^1, \dots, e^n \) die zugehörige duale Basis von \( V^* \), d.h.
\[
e^i(e_j) = \delta^i_j.
\]

Mit anderen Worten: Für \( v = \sum_{j=1}^n v^j e_j \) gilt:
\[
e^i(v) = v^i.
\]

Dann ist
\[
e^{i_1} \wedge \dots \wedge e^{i_k}, \quad 1 \leq i_1 < \dots < i_k \leq n,
\]
eine Basis von \( \Lambda^k V^* \). Das heißt: Für jede \( k \)-Form \( \omega \in \Lambda^k V^* \) existieren eindeutig bestimmte reelle Zahlen \( \omega_{i_1 \dots i_k} \) mit
\[
\omega = \sum \omega_{i_1 \dots i_k} \, e^{i_1} \wedge \dots \wedge e^{i_k}.
\]

Insbesondere gilt:
\[
\dim \Lambda^k V^* = \binom{n}{k}, \quad \text{und} \quad \Lambda^k V^* \cong \Lambda^{n-k}V^*.
\]

\textbf{Lemma 5.7:}

Seien \( \omega_1, \dots, \omega_k \in V^* \) und \( v_1, \dots, v_k \in V \), dann gilt:
\[
(\omega_1 \wedge \dots \wedge \omega_k)(v_1, \dots, v_k)
= \det \begin{pmatrix}
\omega_1(v_1) & \cdots & \omega_k(v_1) \\
\vdots & \ddots & \vdots \\
\omega_1(v_k) & \cdots & \omega_k(v_k)
\end{pmatrix}.
\]

Insbesondere gilt:
\[
\omega_1 \wedge \dots \wedge \omega_k = 0 \quad \text{falls die Linearformen } \omega_1, \dots, \omega_k \in V^* \text{ linear abhängig sind}.
\]

\textbf{Beispiel:}

Seien \( \omega_1, \omega_2 \) zwei \(1\)-Formen auf \( V \) und \( X, Y \in V \). Dann gilt:
\[
(\omega_1 \wedge \omega_2)(X, Y)
= \det \begin{pmatrix}
\omega_1(X) & \omega_2(X) \\
\omega_1(Y) & \omega_2(Y)
\end{pmatrix}
= \omega_1(X)\omega_2(Y) - \omega_1(Y)\omega_2(X).
\]
%%%%%%%%%%%%%%%%%%%%%%%%%%%%%%%%%%%%%%%%%%%%%%%%%%%%%%%%%%%%%%%%%%%%%%%%%%%%%%%%%%%%%%% Vorlesung 14 %%%%%%%%%%%%%%%%%%%%%%%%%

\subsection{Differentialformen auf Mannigfaltigkeiten}

Wir wollen die Objekte aus der letzten Vorlesung punktweise auf Mannigfaltigkeiten übertragen.

\paragraph{Definition 5.8:} Sei \( M \) eine Mannigfaltigkeit. Das \underline{Bündel der \( k \)-Formen} auf \( M \) ist definiert als
\[
\Lambda^k (T^*M) := \bigsqcup_{p \in M} \Lambda^k (T^*_p M)
\]
dabei ist \( T^*_p M := (T_p M)^* \) und \( \bigsqcup \) bezeichnet die disjunkte Vereinigung.

Die kanonische Projektion ist definiert als die Fußpunktabbildung. 
\[
\pi : \Lambda^k (T^*M) \to M, \quad \pi(\omega) = p \quad \text{falls } \omega \in \Lambda^k (T^*_p M)
\]
\paragraph{Bemerkung:} Die Menge \( \Lambda^k(T^*M) \) trägt die Struktur einer differenzierbaren Mannigfaltigkeit (analog zu \( TM \)).\\

Sei \( (U, \varphi) \) ein Karte um \( p \in M \). Dann definieren wir ($\txt{dim}(M)=n$)
\[
\Phi : \pi^{-1}(U) \longrightarrow \varphi(U)\times \Lambda^k\mathbb{R}^n \quad \text{mit} \quad \omega \mapsto \left(\varphi(\pi(U)), (D\varphi^{-1})^*\omega \right)
\]

Mit den Karten \( (\pi^{-1}(U), \Phi) \) wird \( \Lambda^k(T^*M) \) eine differenzierbare Mannigfaltigkeit der Dimension $n+\binom{n}{k}$.

Mit der kanonischen Projektion \( \pi : \Lambda^k(T^*M) \to M \) erhalten wir ein reelles Vektorbündel vom Rang \( \binom{n}{k} \) über \( M \).

\textbf{Spezialfall:} \( k = 1 \): \quad \( \Lambda^1 T^*M = T^*M \) heißt das \underline{Kotangentialbündel} von \( M \).

\paragraph{Definition 5.9:} 
Eine \underline{Differentialform vom Grad \( k \)} auf \( M \) (kurz: k-Form) ist eine differenzierbare Abbildung
\[
\omega : M \longrightarrow \Lambda^k(T^*M) \quad \text{mit} \quad \pi \circ \omega = \mathrm{id}_M,
\]
d.h. \( \omega \) ist ein Schnitt vom Vektorbündel $\Lambda^k(T^*M)$.

Den unendlich-dimensionalen Vektorraum der \( k \)-Formen auf \( M \) bezeichnen wir mit
\[
\Omega^k(M) := \Gamma(\Lambda^k T^*M).
\]
Insbesondere gilt für den Raum der 0-Form $\Omega^0(M)=C^\infty(M)$.\\

\textbf{Beispiel:}
Jede Funktion \( f \in C^\infty(M) \) definiert eine Differentialform vom Grad 1, nämlich das Differential, also die 1-Form
\[
{D}f \in \Omega^1(M) \quad Df_p:T_PM\rightarrow T_{f(p)}\mathbb{R}\cong\mathbb{R}
\]
Der Raum \( \Omega^*(M) \) aller Differentialformen auf \( M \) ist eine reelle Algebra (vgl. äußere Algebra Def. 5.4). Wir definieren für Differentialformen \( \omega_1, \omega_2 \in \Omega^*(M) \) und Skalare \( \lambda \in \mathbb{R} \) in \( p \in M \):
\[
(\omega_1 + \omega_2)(p) := \omega_1(p) + \omega_2(p)
\]
\[
(\lambda \omega)(p) := \lambda \cdot \omega(p)
\]
\[
(\omega_1 \wedge \omega_2)(p) := \omega_1(p) \wedge \omega_2(p)
\]

Der Raum \( \Omega^*(M) \) ist auch ein Modul über dem Ring der Funktionen \( C^\infty(M) \), d.h. für eine glatte Funktion \( f \) und eine Differentialform \( \omega \) definieren wir das Produkt \( f \cdot \omega := f \wedge \omega \), also
\[
(f \cdot \omega)(p) = f(p) \cdot \omega(p)
\]
Jede Differentialform \( \hat{\omega} \in \Omega^k(M) \) (nimmt einen Punkt p, gibt eine k-Form aus, die Tangentialvektoren schnupft.) induziert eine \( C^\infty(M) \)-multilineare, alternierende Abbildung
\[
\omega : \underbrace{\Gamma(TM) \times \cdots \times \Gamma(TM)}_{\text{$k$-mal}} \longrightarrow C^\infty(M),
\]

Für Vektorfelder \( X_1, \dots, X_k \) auf \( M \) definieren wir die Funktion durch
\[
\underbrace{\omega(X_1, \dots, X_k)}_{\in C^{\infty}(M)}(p) := \hat{\omega}(p)(X_1(p), \dots, X_k(p)).
\]

Die Umkehrung ist ebenfalls korrekt, wir erhalten eine äquivalente Definition von Differentialformen.

\paragraph{Lemma 5.10:} 
Jede \( C^\infty(M) \)-multilineare, alternierende Abbildung
\[
\omega : \Gamma(TM) \times \cdots \times \Gamma(TM) \longrightarrow C^\infty(M)
\]
definiert eine \( k \)-Form \( \hat{\omega} \) auf \( M \).

\begin{proof}
    Seien \( X_1, \dots, X_k \in T_p M \) mit Fortsetzungen \( \tilde{X}_1, \dots, \tilde{X}_k \) zu lokal um \( p \) definierten Vektorfeldern. Wir definieren die \( k \)-Form \( \hat{\omega} \) durch
\[
\hat{\omega}(p)(X_1, \dots, X_k) := \omega(\tilde{X}_1, \dots, \tilde{X}_k)(p).
\]

\underline{Zu zeigen:} \( \omega(\tilde{X}_1, \dots, \tilde{X}_k)(p) \) hängt nur von den Werten der Vektorfelder \( \tilde{X}_i \) im Punkt \( p \) ab, also von \( X_1, \dots, X_k \in T_p M \), unabhängig von der Fortsetzung.\\

Wir betrachten nur den Fall \( k = 1 \).\\
Wir wollen also zeigen: \( \omega(X)(p) = 0 \), falls \( X \) ein Vektorfeld mit \( X(p) = 0 \) ist.\\

\textit{Warum reicht das:}  
Seien \( X, X' \) Vektorfelder mit \( X(p) = X'(p) \).  
Dann ist \( Y := X - X' \) ein Vektorfeld mit \( Y(p) = 0 \).  
Nun gilt für eine 1-Form \( \omega \):
\[
0 = \omega(Y)(p) = \left( \omega(X) - \omega(X') \right)(p)
\quad \Rightarrow \quad
\omega(X)(p) = \omega(X')(p)
\]

Sei \( (U, \varphi) \) eine Karte um \( p \), schreiben \( X \) auf \( U \) als
\[
X = \sum_{i=1}^n a_i \frac{\partial}{\partial x^i}
\]
für glatte Funktionen \( a_i \in C^\infty(U) \) mit \( a_i(p)=0 \), \( \forall i = 1, \dots, n \).

Wir wählen nun eine Abschneidefunktion \( \chi \in C^\infty(M) \) mit \(\txt{supp}(\chi)\subset U\) und \(\chi \equiv 1 \) auf einer kleinen Umgebung von \( p \). Insbesondere \( \chi(p) = 1 \). Es folgt
\[
\chi^2 X = \sum_{i=1}^n \chi^2 a_i \frac{\partial}{\partial x^i} = \sum_{i=1}^n \chi a_i \cdot\chi\frac{\partial}{\partial x^i}
\]
$\chi a_i$ ist für \( 1 \leq i \leq n \) eine auf \( M \) definierte glatte Funktion.\\
$\chi \deldel{}{x_i}$ ist für \( 1 \leq i \leq n \) eine auf \( M \) definiertes Vektorfeld auf $M$.\\

Wenden wir \( \omega \) auf die Gleichung an:
\[
\omega(X)(p) \overset{\chi(p)=1}{=} \chi^2(p) \omega(X)(p) = \omega(\chi^2 X)(p) = \sum_{i=1}^n \chi \underbrace{a_i(p)}_{=0} \cdot \omega(\chi\frac{\partial}{\partial x^i})(p) = 0
\]
\end{proof}
\subsection{Differentialformen in lokalen Koordinaten}

Sei \( (U, \varphi) \) eine Karte um \( p \in M \). Dann sind für jeden Punkt \( p \in M \) die Vektoren 
\[
\left. \frac{\partial}{\partial x^i} \right|_p \quad \forall i = 1, \dots, n
\]
eine Basis von \( T_p M \).

Die dazu duale Basis bezeichnen wir wie gewohnt mit \( \mathrm{d}x_i(p)=(dx_i)_p \).

Über \( U \) gilt also
\[
\frac{\partial}{\partial x_i} \in \Gamma(TU), \quad \mathrm{d}x_i \in \Omega^1(U) \quad \text{mit} \quad \mathrm{d}x_i\left( \frac{\partial}{\partial x_j} \right) = \delta_{ij}.
\]

Damit schreibt sich jede Differentialform \( \omega \in \Omega^k(M) \) lokal über \( U \) als
\[
\omega|_U = \sum_{i_1 < \dots < i_k} \omega_{i_1 \dots i_k} \, \mathrm{d}x_{i_1} \wedge \dots \wedge \mathrm{d}x_{i_k},
\]
wobei \( \omega_{i_1 \dots i_k} \in C^\infty(U) \) glatt ist, gegeben durch
\[
\omega_{i_1 \dots i_k} = \omega(\deldel{}{x_{i_1}},\cdots, \deldel{}{x_{i_k}})
\]
\subsection{Das Zurückziehen von Differentialformen}

Sei \( f : M \to N \) eine differenzierbare Abbildung und sei \( \omega \in \Omega^k(N) \). Dann erhalten wir mittels Zurückziehen eine \( k \)-Form \( f^*\omega \) auf \( M \), definiert durch
\[
(f^*\omega)_p(X_1, \dots, X_k) := \omega_{f(p)} \left( \mathrm{D}f_p(X_1), \dots, \mathrm{D}f_p(X_k) \right)
\]
für Tangentialvektoren \( X_i \in T_p M \), d.h. \( (f^*\omega)_p = (Df)^*_p\omega_{f(p)}\) (vgl. 5.2).

Die Eigenschaften für das Zurückziehen mittels linearer Abbildungen von (linear-) Formen auf Vektorräumen übertragen sich auf (Differential-)Formen. Insbesondere:

\paragraph{Lemma 5.11:} 
Seien \( \alpha, \beta \) Differentialformen auf \( N \) und seien \( f : M \to N \), \( g : N \to P \) zwei differenzierbare Abbildungen, dann gilt:
\begin{itemize}
    \item[i)] \( (g \circ f)^* = f^* \circ g^* \)
    \item[ii)] \( f^*(\alpha \wedge \beta) = (f^*\alpha) \wedge (f^*\beta) \)
\end{itemize}

\textbf{Bemerkung:} Sei \( M \subset N \) eine Untermannigfaltigkeit und \( \iota : M \hookrightarrow N \) die Inklusionsabbildung. Dann ist die Einschränkung einer Differentialform \( \omega \in \Omega^*(N) \) auf \( M \) genau das Zurückziehen von \( \omega \) durch \( \iota \),\\
d.h. es gilt 
\[
\omega|_M = \iota^* \omega.
\]
\subsection{Das Differential}

Jede 0-Form \( f \) (sprich: glatte Funktion \( f \in C^\infty(M) \)) definiert eine 1-Form \( \mathrm{d}f \), die sich in lokalen Koordinaten schreibt als
\[
\mathrm{D}f = \sum\deldel{f}{x_i}dx_i
\]

\textbf{Behauptung:} Das Differential von Funktionen lässt sich eindeutig zu einer Folge von linearen Abbildungen fortsetzen:
\[
\Omega^0(M) \xrightarrow{\mathrm{d}} \Omega^1(M) \xrightarrow{\mathrm{d}} \Omega^2(M) \xrightarrow{\mathrm{d}} \dots
\]

\textbf{+ ein paar Bedingungen} (damit $d$ eindeutig ist). Genauer gilt:

\paragraph{Satz 5.12.}
Auf jeder differenzierbaren Mannigfaltigkeit \( M \) gibt es genau eine \( \mathbb{R} \)-lineare Abbildung
\[
\mathrm{d} : \Omega^*(M) \longrightarrow \Omega^*(M)
\]
vom Grad 1 (d.h. \( \mathrm{d} : \Omega^k(M) \to \Omega^{k+1}(M) \)), so dass gilt:

\begin{itemize}
    \item[i)] \( \mathrm{d}|_{C^\infty(M)=\Omega^0M} \), stimmt mit dem Differential von Funktionen überein. (Differentialbedingung)
    \item[ii)] \( \mathrm{d} \circ \mathrm{d} = 0 \). (Komplexeigenschaft)
    \item[iii)] Für \( \omega \in \Omega^k(M) \), \( \sigma \in \Omega^\ell(M) \) gilt: (Leibniz-Regel)
    \[
    \mathrm{d}(\omega \wedge \sigma) = \mathrm{d}\omega \wedge \sigma + (-1)^k \omega \wedge \mathrm{d}\sigma.
    \]
    \( \mathrm{d}\omega \) heißt die \underline{äußere Ableitung} der Form \( \omega \).
\end{itemize}

\textbf{Bemerkung:} Die Leibniz-Regel garantiert, dass \( \mathrm{d} \neq 0 \), denn für \( f \in \Omega^0(M), \omega \in \Omega^2(M) \) gilt:
\[
\mathrm{d}(f\omega) = \mathrm{d}f \wedge \omega + f \, \mathrm{d}\omega = \underbrace{Df\wedge \omega}_{\txt{i.A. }\neq 0} + fd\omega
\]
\begin{proof}
\quad

\textbf{Schritt I:} Definiere \( \mathrm{d} \) lokal.

Sei \( (U, \varphi) \) eine Karte von \( M \). Eine \( k \)-Form \( \omega \) auf \( U \) hat eine eindeutige Darstellung
\[
\omega = \sum_{i_1 < \dots < i_k} \omega_{i_1 \dots i_k} \, \mathrm{d}x_{i_1} \wedge \dots \wedge \mathrm{d}x_{i_k} \quad (5.5)
\]

Gibt es einen Differentialoperator, der den Bedingungen des Satzes genügt, so hat er folgende Darstellung:
\[
\mathrm{d}\omega = \sum_{i_1 < \dots < i_k} \mathrm{d} \omega_{i_1 \dots i_k} \wedge \mathrm{d}x_{i_1} \wedge \dots \wedge \mathrm{d}x_{i_k}
\]

\textit{Denn:} \( \mathrm{d}(\mathrm{d}x^i) = \mathrm{d}(\mathrm{d} \varphi_i) = (\mathrm{d} \circ \mathrm{d})(\varphi_i) = 0 \) (Leibniz-Regel gilt für die 1-Form)

$d\omega$ hat eine eindeutige Darstellung, damit folgt die \emph{Eindeutigkeit}.\\

Um die \emph{Existenz} zu zeigen, legen wir obige Darstellung als Definition für \( \mathrm{d} \) zugrunde. \\
Damit ist \( \mathrm{d} \) \( \mathbb{R} \)-linear und stimmt mit dem Differential von Funktionen überein.\\

\underline{Leibniz / Produktregel:}
o.E. sei 
\[
\omega = f\;\mathrm{d}x_{i_1} \wedge \dots \wedge \mathrm{d}x_{i_k}, \quad 
\sigma = g\;\mathrm{d}x_{j_1} \wedge \dots \wedge \mathrm{d}x_{j_\ell}
\]

Dann gilt:
\[\mathrm{d}(\omega \wedge \sigma) = \mathrm{d}\left(fg\; \mathrm{d}x_{i_1} \wedge \dots \wedge \mathrm{d}x_{i_k} \wedge \mathrm{d}x_{j_1} \wedge \dots \wedge \mathrm{d}x_{j_\ell} \right) \]
\[= \underbrace{\mathrm{d}(fg)}_{(df)g+f(dg)} \wedge \mathrm{d}x_{i_1} \wedge \dots \wedge \mathrm{d}x_{i_k} \wedge \mathrm{d}x_{j_1} \wedge \dots \wedge \mathrm{d}x_{j_\ell}\]
\[=(\mathrm{d}f\wedge \mathrm{d}x_{i_1} \wedge \dots \wedge \mathrm{d}x_{i_k})\wedge (g\;\mathrm{d}x_{j_1} \wedge \dots \wedge \mathrm{d}x_{j_\ell}) +(-1)^k (f\; \mathrm{d}x_{i_1} \wedge \dots \wedge \mathrm{d}x_{i_k})\wedge (\mathrm{d}g\wedge\mathrm{d}x_{j_1} \wedge \dots \wedge \mathrm{d}x_{j_\ell})\]
\[= \mathrm{d}\omega \wedge \sigma + \underbrace{(-1)^k}_{\txt{k-mal Vertauschungen von }dg} \omega \wedge \mathrm{d}\sigma\]

\underline{Komplexeigenschaft:}
Zu zeigen: \( \mathrm{d}^2 = \mathrm{d} \circ \mathrm{d} = 0 \). Sei 
\[
\omega = \sum_{i_1 < \dots < i_k} \omega_{i_1 \dots i_k} \, \mathrm{d}x_{i_1} \wedge \dots \wedge \mathrm{d}x_{i_k}
\]
Dann gilt:
\[
\mathrm{d}(\mathrm{d}\omega) = \mathrm{d}\left( \sum_{i_1 < \dots < i_k} \mathrm{d} \omega_{i_1 \dots i_k} \wedge \mathrm{d}x_{i_1} \wedge \dots \wedge \mathrm{d}x_{i_k} \right)
\]

Wegen der Leibniz-Regel (zweimal Leibniz-Regel anwenden auf $f\wedge \omega$) reicht es also zu zeigen, dass für alle \( f \in C^\infty(M) \)
\[
\mathrm{d}(\mathrm{d}f) = 0
\]

Es gilt:
\[
\mathrm{d}f = Df = \sum_{j=1}^n \frac{\partial f}{\partial x^j} \, \mathrm{d}x^j
\quad \overset{\txt{Def. d auf 1-Form}}{\Rightarrow} \quad
\mathrm{d}(\mathrm{d}f) = \sum_{i,j=1}^n \frac{\partial^2 f}{\partial x^i \partial x^j} \, \mathrm{d}x^i \wedge \mathrm{d}x^j = 0
\]

Da der Hessian von \( f \) symmetrisch ist (Satz von Schwarz) und das äußere Produkt antisymmetrisch, verschwindet die Summe auf der rechten Seite.\\

\textbf{Schritt II: Allgemeiner Fall} (lokal zu global)\\
Sei \( \omega \) eine \( k \)-Form auf \( M \), \( (U, \varphi) \) eine Karte von \( M \), \( \iota: U \hookrightarrow M \) die Inklusionsabbildung. Dann ist die Einschränkung von \( \omega \) auf \( U \)
\[
\omega|_{TU} = \iota^* \omega.
\]

Ist \( p \in U \), so definieren wir:
\[
(\mathrm{d}\omega)_p := (\mathrm{d}_U\omega|_{TU}),
\]
wobei $d_U$ das in Schritt I für \( (U, \varphi) \) definierte äußere Differential ist. Da dies eindeutig ist, folgt für jede weitere Karte \( (V, \varphi') \)

Auf dem Schnitt \( U \cap V \) zweier Karten gilt lokal:
\[
\left. d_U \omega \right|_{T(U \cap V)}
\overset{\txt{lokal Def.}}{=} \sum d\omega_{i_1, \dots, i_k} \wedge dx_{i_1} \wedge \dots \wedge dx_{i_k}
\Big|_{T(U \cap V)} \overset{\txt{lokale Eidneutigkeit}}{=} \left. d_{U \cap V} \omega \right|_{T(U \cap V)}
\]
\[
\overset{\txt{lokal def.}}{=} \sum d\tilde{\omega}_{i_1, \dots, i_k} \wedge dy_{i_1} \wedge \dots \wedge dy_{i_k}
\Big|_{T(U \cap V)}
\overset{\txt{lokal def.}}{=} \left. d_V \omega \right|_{T(U \cap V)}
\]

Dabei sind \( \{x_i\} \) lokale Koordinaten auf \( U \) und \( \{y_i\} \) auf \( V \), und \( \omega = \sum \omega_{i_1, \dots, i_k} dx_{i_1} \wedge \dots \wedge dx_{i_k} \) sowie \( \omega = \sum \tilde{\omega}_{i_1, \dots, i_k} dy_{i_1} \wedge \dots \wedge dy_{i_k} \) jeweils die lokale Darstellung von \( \omega \) in den Karten.\\


Damit stimmen die lokal definierten Ausdrücke überein, und wir erhalten einen wohldefinierten
\[
\text{Differentialoperator} \quad \mathrm{d} : \Omega^k(M) \longrightarrow \Omega^{k+1}(M),
\]
der den Bedingungen des Satzes genügt.\\

\underline{Zur Eindeutigkeit:}
Sei \( \mathrm{d}' \) ein weiterer Operator wie im Satz. Sei \( (U, \varphi) \) eine Karte, \( p \in U \). Es gilt lokal:
\[
\omega|_{TU} = \sum_{i_1 < \dots < i_k} \omega_{i_1 \dots i_k} \, \mathrm{d}x_{i_1} \wedge \dots \wedge \mathrm{d}x_{i_k}
\]

Weiters sei \( \chi \in C^\infty(M, [0,1]) \) eine Abschneidefunktion.
  \begin{figure}[H]
    \centering
    \includegraphics[width=9cm]{Image Diffgeo/14.01.png}
	%\caption{Ebene mit Loch und die Zylinder Oberfläche sind nicht einfach zusammenhängend}
 \end{figure}

Setze durch Null auf ganz \( M \) fort:
\[
\tilde{\omega}_{i_1, \dots, i_k} = \chi \cdot \omega_{i_1, \dots, i_k} \quad \tilde{x}_{i_j} := \chi \cdot x_{i_j} = \chi \cdot \txt{Pr}_{i_j}(\varphi)
\]

Dies definiert eine auf ganz \( M \) erklärte \( k \)-Form:
\[
\widetilde{\omega} := \sum_{i_1 < \dots < i_k} \widetilde{\omega}_{i_1 \dots i_k} \, \mathrm{d} \widetilde{x}_{i_1} \wedge \dots \wedge \mathrm{d} \widetilde{x}_{i_k}
\]

Aus den Eigenschaften eines Differentialoperators wie im Satz beschrieben folgt:
\[
\mathrm{d}' \widetilde{\omega} = \sum \mathrm{d} \widetilde{\omega}_{i_1 \dots i_k} \wedge \mathrm{d} \widetilde{x}_{i_1} \wedge \dots \wedge \mathrm{d} \widetilde{x}_{i_k} 
\]
Da $d'|_{C^0(M)}=d|_{C^0(M)}=D$ und $\tilde{\omega}_{i_1,\dots,i_k}$ glatte Abbildungskoeffiziente sind. Außerdem gilt \( \mathrm{d}(\mathrm{d}x^i) = \mathrm{d}(\mathrm{d} \varphi_i) = (\mathrm{d} \circ \mathrm{d})(\varphi_i) = 0 \).\\

Mit Schritt I und \( \omega = \widetilde{\omega} \) auf einer Umgebung \( V \ni p \) (lokale Eindeutigkeit), folgt:
\[
(\mathrm{d}' \widetilde{\omega})_p = (\mathrm{d}_U \tilde{\omega})_p=(d_U\omega)_p
\]
Wir wollen uns nun 
\[(d'\tilde{\omega})_p=(d'\omega)_p\]
überlegen, dass dies dann
\[
(\mathrm{d}' \omega)_p = (d_U\omega)_p= (\mathrm{d} {\omega})_p
\]
impliziert, also \( \mathrm{d}' = \mathrm{d} \).

\paragraph{Abschneidefunktion:} (Zum Beweis \((d'\tilde{\omega})_p=(d'\omega)_p\))
\[
\widetilde{\chi} \in C^\infty(M, [0,1])
\]
  \begin{figure}[H]
    \centering
    \includegraphics[width=10cm]{Image Diffgeo/14.02.png}
	%\caption{Ebene mit Loch und die Zylinder Oberfläche sind nicht einfach zusammenhängend}
 \end{figure}
Da \( \widetilde{\omega} \) und \( \omega \) auf \( V \) übereinstimmen, folgt:
\(
\tilde{\chi}(\omega-\tilde{\omega})= \omega -\tilde{\omega}
\)
  \begin{figure}[H]
    \centering
    \includegraphics[width=11cm]{Image Diffgeo/14.03.png}
	%\caption{Ebene mit Loch und die Zylinder Oberfläche sind nicht einfach zusammenhängend}
 \end{figure}
\[
\Rightarrow \quad 
(d' \tilde{\omega})_p - (d' \omega)_p
= (D\tilde{\chi})_p \wedge \underbrace{(\tilde{\omega} - \omega)_p}_{=0}
+ \underbrace{\tilde{\chi}(p)}_{=0} \cdot (d'(\tilde{\omega} - \omega))_p
= 0
\]
\end{proof}

%%%%%%%%%%%%%%%%%%%%%%%%%%%%%%%%%%%%%%%%%%%%%%%%%%%%%%%%%%%%%%%%%%%%%%%%%%%%%%%%%% Vorlesung 15 %%%%%%%%%%%%%%%%%%%%%%%%%%%%%%

\textbf{Korollar 5.13 (Natürlichkeit):} \\ 
Sei \( f : M \to N \) eine differenzierbare Abbildung. \\
Dann gilt für alle Differentialformen \( \omega \) auf \( N \):
\[
f^*(\mathrm{d} \omega) = \mathrm{d}(f^* \omega)
\]
\underline{Erinnerung:} Das Pull-Back ist definiert als $(f^*\omega)_p(X_1,\dots X_k)=\omega_{f(p)}(Df_p(X_1),\dots,Df_p(X_k))$\\
wobei $\omega\in \Omega^k(N)\quad f^*\omega\in \Omega^k(M)$
\begin{proof}
Sei \( \omega = g \), wobei \( g : N \to \mathbb{R} \) eine differenzierbare Funktion (0-Form) ist. \\
Dann ist die Behauptung eine Umformulierung der Kettenregel:
\[
\left[ f^*(dg) \right]_p(X)
\overset{\txt{Definiton. }f^*}{=} (dg)_{f(p)} \left( Df_p(X) \right) \overset{\txt{0-Form}}{=} (Dg)_{f(p)} Df_p(X)
\]
\[
\overset{\text{KR}}{=} D(g \circ f)_p(X) = d(g \circ f)_p(X)
\overset{f^*g=g\circ f}{=} d(f^*g)_p(X)
\]

Sei \( \omega \) eine \( k \)-Form und \( (U, \varphi) \) eine Karte. Mit der Inklusionsabbildung \( \iota: U \hookrightarrow N \) \\
erhalten wir mit dem Beweis von Satz 5.12:
\[
\iota^*(d\omega)
= (d\omega)|_{T U}
= d(\omega|_{T U})
= d(\iota^*\omega)
\]


Also können wir \( N = U \) und \( f = \left. f \right|_{f^{-1}(U)} \) annehmen. Damit:
\[
\omega = \sum \omega_{i_1 \dots i_k} \, \mathrm{d}y_{i_1} \wedge \dots \wedge \mathrm{d}y_{i_k}
\]
\[
\mathrm{d}\omega = \sum \mathrm{d}\omega_{i_1 \dots i_k} \wedge \mathrm{d}y_{i_1} \wedge \dots \wedge \mathrm{d}y_{i_k}
\]

Anwenden von \( f^* \) gibt und mit Relation $f^*(\alpha\wedge \beta)=(f^*\alpha)\wedge (f^*\beta)$ gilt:
\[
f^*\omega = \sum f^*(\omega_{i_1 \dots i_k}) \, f^*(\mathrm{d}y_{i_1}) \wedge \dots \wedge f^*(\mathrm{d}y_{i_k})     
\]
\[
f^*d\omega = \sum f^*(d\omega_{i_1 \dots i_k}) \, f^*(\mathrm{d}y_{i_1}) \wedge \dots \wedge f^*(\mathrm{d}y_{i_k})
\]

Für die Koordinatenfunktionen gilt:
\[
f^*(\mathrm{d}y^i) \overset{\txt{Schritt 1}}{=} d(f^*y_j)
\quad \Rightarrow \quad
d(f^*\mathrm{d}y_i) \overset{d\circ d = 0}{=} 0
\]

Aus der Produktregel folgt schließlich:
\[
\mathrm{d}(f^* \omega) = \sum \underbrace{\mathrm{d}(f^*\omega_{i_1 \dots i_k})}_{\overset{\txt{1. Schritt}}{=}f^*d\omega_{i_1,\dots,i_k}} \wedge f^*(\mathrm{d}y_{i_1}) \wedge \dots \wedge f^*(\mathrm{d}y_{i_k})
= f^*(\mathrm{d}\omega)
\]
\end{proof}

\textbf{Satz 5.14}

a) Sei \( \omega \) eine 1-Form und \( X, Y \) beliebige Vektorfelder, dann gilt:
\[
d\omega(X, Y)
= \mathcal{L}_X(\omega(Y))
- \mathcal{L}_Y(\omega(X))
- \omega([X, Y])
\]

b) Sei \( \omega \) eine \( k \)-Form und seien \( X_1, \dots, X_{k+1} \) beliebige Vektorfelder, dann gilt:
\[
\mathrm{d}\omega(X_1, \dots, X_{k+1}) =
\sum_{i=1}^{k+1} (-1)^{i+1} \mathcal{L}_{X_i}\left(\omega(X_1, \dots, \widehat{X_i}, \dots, X_{k+1})\right)
\]
\[+ \sum_{1 \le i < j \le k+1} (-1)^{i+j} \omega([X_i, X_j], X_1, \dots, \widehat{X_i}, \dots, \widehat{X_j}, \dots, X_{k+1})\]

Dabei bedeutet \( \widehat{X_i} \), dass dieser Vektor ausgelassen wird.

\begin{proof}
Übung.
\end{proof}

\subsection{Lie-Ableitung von Differentialformen}

Sei \( X \) ein Vektorfeld mit lokalem Fluss \( \varphi_t \), d.h.\ \( \varphi_t : U \subset M \to M \), definiert durch:
\[
\varphi_t(p) = \gamma_p(t), \quad \varphi_0(p) = p, \quad \dot{\varphi}_t(p) = X_{\varphi_t(p)}
\]
\[
\text{(Integral­kurve von } X \text{ durch } p)
\]

\underline{Erinnerung:} Die Lie-Ableitung einer Funktion \( f \in C^\infty(M) \) ist gegeben durch:
\[
\mathcal{L}_X(f)_p
= Df(X_p)
= \left. \frac{d}{dt} \right|_{t=0} \left( (\varphi_t^* f)(p) \right)
\overset{\varphi_t^*f=f\circ \varphi_t}{=} \left. \frac{d}{dt} \right|_{t=0} \left( f(\varphi_t(p)) \right)
\]

Die Lie-Ableitung eines Vektorfeldes \( Y \in \Gamma(TM) \) ist gegeben durch:
\[
(\mathcal{L}_X Y)_p
= [X, Y]_p
= \left. \frac{d}{dt} \right|_{t=0} \left( (\varphi_t^* Y)_p \right)
= D\varphi_{-t} \, Y_{\varphi_t(p)}
\]

\textbf{Definition 5.15}  
Die \underline{Lie-Ableitung einer Differentialform} \( \omega \in \Omega^k(M) \) nach einem Vektorfeld \( X \in \Gamma(TM) \) ist definiert durch:
\[
\mathcal{L}_X \omega := \left. \frac{d}{dt} \right|_{t=0} (\varphi_t^* \omega)
\]
wobei
\[
(\varphi_t^* \omega)_p(X_1, \dots, X_k) = \omega_{\varphi_t(p)}((D\varphi_t)_p(X_1), \dots, (D\varphi_t)_p(X_k))
\]

\textbf{Lemma 5.16}  
Die Lie-Ableitung von Formen hat folgende Eigenschaften:

i) Sei \( \omega \in \Omega^1(M) \) und \( X, Y \in \Gamma(TM) \), dann gilt:
\[
(\mathcal{L}_X \omega)(Y) = \mathcal{L}_X(\omega(Y)) - \omega(\mathcal{L}_XY) \quad \mathcal{L}_XY = [X,Y]
\]

ii) Sei \( \omega \in \Omega^k(M) \) und \( X, X_1, \dots, X_k \in \Gamma(TM) \), dann gilt:
\[
(\mathcal{L}_X \omega)(X_1, \dots, X_k) = \mathcal{L}_X(\omega(X_1, \dots, X_k)) - \sum_{i=1}^k \omega(X_1, \dots, \mathcal{L}_XX_i, \dots, X_k)
\]

iii) 
\(
\mathcal{L}_X (\alpha \wedge \beta) = (\mathcal{L}_X \alpha) \wedge \beta + \alpha \wedge (\mathcal{L}_X \beta)
\)
d.h. $\mathcal{L}_X$ ist eine Derivation.

iv)
\(
\mathcal{L}_X (\mathrm{d} \omega )= \mathrm{d} (\mathcal{L}_X \omega)
\)

\begin{proof}
\textbf{ad i)}
\[
\mathcal{L}_X(\underbrace{\omega(Y)}_{\txt{eine Fkt}})(p) 
\overset{\txt{Lie-Ableitung Fkt}}{=} \left.\frac{d}{dt}\right|_{t=0} \varphi_t^*(\omega(Y)) (p) 
\underset{\txt{Differentialform}}{\overset{\txt{Pull-Back}}{=}} \left.\frac{d}{dt}\right|_{t=0} \omega_{\varphi_t(p)} \left( (D\varphi_t)_p (Y_p) \right)
\]
\[
\overset{\txt{Integralkurve}}{=} \left.\frac{d}{dt}\right|_{t=0} \omega_{\varphi_t(p)} \left( Y_{\varphi_t(p)} \right)
= \left.\frac{d}{dt}\right|_{t=0} \omega_{\varphi_t(p)} \left( \underbrace{D\varphi_t \circ D\varphi_{-t}}_{id} (Y_{\varphi_t(p)}) \right)
\]
\[
\overset{\txt{Def. Pullback VF}}{=} \left.\frac{d}{dt}\right|_{t=0} \textcolor{blue}{\omega_{\varphi_t(p)} [D\varphi_t}(\varphi_t^* Y)_p] \underset{\txt{Differentialform}}{\overset{\txt{Pullback}}{=}} \left.\frac{d}{dt}\right|_{t=0} \textcolor{blue}{(\varphi_t^*\omega)_p}((\varphi_t^* Y)_p)
\]
\[
\underset{(\varphi_0^*\omega)_p=\omega_p, (\varphi_0^*Y)_p=Y_p}{\overset{\txt{Produktregel}}{=}} \left.\frac{d}{dt}\right|_{t=0} (\varphi_t^* \omega)_p (Y_p)
+ \omega_p \underbrace{\left( \left.\frac{d}{dt}\right|_{t=0} (\varphi_t^* Y)_p \right)}_{=\mathcal{L}_XY_p=[X,Y]_p} = (\mathcal{L}_X \omega)_p (Y_p) + \omega_p([X,Y]_p)
\]

\textbf{ad ii)} \quad Analog zu Teil i).\\

\textbf{ad iii)} 
\[
\mathcal{L}_X(\alpha \wedge \beta) 
= \left. \frac{d}{dt} \right|_{t=0} (\varphi_t^* (\alpha \wedge \beta)) 
= \left. \frac{d}{dt} \right|_{t=0} (\varphi_t^* \alpha) \wedge (\varphi_t^* \beta)
\]
\[\overset{\txt{Produktregel}}{=}\frac{d}{dt} |_{t=0} (\varphi_t^* (\alpha ))\wedge\beta+\alpha\wedge(\frac{d}{dt} |_{t=0} (\varphi_t^* \beta))\]
\[
= (\mathcal{L}_X \alpha) \wedge \beta + \alpha \wedge (\mathcal{L}_X \beta)
\]

\textbf{ad iv)}
\[
\mathcal{L}_X (\mathrm{d} \omega) 
= \left. \frac{d}{dt} \right|_{t=0} \varphi_t^* (\mathrm{d} \omega) 
= \left. \frac{d}{dt} \right|_{t=0} \mathrm{d}(\varphi_t^* \omega)
\overset{\txt{Diff linear}}{=} \mathrm{d} \left( \left. \frac{d}{dt} \right|_{t=0} \varphi_t^* \omega \right) 
= \mathrm{d} (\mathcal{L}_X \omega)
\]
\end{proof}

\textbf{Definiton. 5.17}  
Sei \( \omega \in \Omega^k(M) \), \( X \in \Gamma(TM) \), dann ist die \emph{Kontraktion} (oder das \emph{innere Produkt}) von \( \omega \) mit \( X \) folgendermaßen definiert:
\[
(\iota_X \omega)(X_1, \dots, X_{k-1}) := \omega(X, X_1, \dots, X_{k-1})
\]

Wir schreiben auch \( X \lrcorner \, \omega \) für \( \iota_X \omega \).

D.h., wir haben eine Abbildung
\[
\iota_X : \Omega^k(M) \to \Omega^{k-1}(M)
\]

\textbf{Bemerkung:}  

Sei \( (V, \langle \cdot, \cdot \rangle) \) ein euklidischer Vektorraum. Dann definiert das Skalarprodukt einen Isomorphismus
\(
V \to V^*, \quad v \mapsto v^*, \quad \text{definiert durch } v^*(u) := \langle v, u \rangle.
\)\\

Für jeden Vektor \( v \in V \) ist dann
\[
\iota_v : \Lambda^k V^* \to \Lambda^{k-1} V^*,
\]
die zum Dachprodukt mit \( v \) , also zu $\beta\mapsto v^*\wedge\beta$ adjungierte Abbildung, d.h.:
\[
\langle \iota_v \alpha, \beta \rangle_{\Lambda^{k-1}V^*} = \langle \alpha, v^* \wedge \beta \rangle_{\Lambda^{k}V^*}, \quad \text{für } \alpha \in \Lambda^{k} V^*, \, \beta \in \Lambda^{k-1} V^*.
\]

Dabei ist das Skalarprodukt \( \langle \cdot, \cdot \rangle \) auf \( \Lambda^k V \) durch die Wahl der folgenden Orthonormalbasis festgelegt:
\[
e^{i_1} \wedge \dots \wedge e^{i_k}, \quad 1 \le i_1 < \dots < i_k \le n
\]
wobei \( \{ e^i \} \) eine fixierte Orthonormalbasis von \( V \) ist. \\
Für die Vektoren \( e_i^* \) der dualen Basis schreiben wir auch \( e^i \).\\

\textbf{Lemma 5.18:}  
Das innere Produkt \( \iota_X \) hat folgende Eigenschaften:

\begin{itemize}
  \item[i)] \( \iota_X(f \omega) = f ( \iota_X \omega) \quad \) für alle \( f \in C^\infty(M) \)
  \item[ii)] \( \iota_X(\alpha \wedge \beta) = (\iota_X \alpha) \wedge \beta + (-1)^k \alpha \wedge \iota_X \beta \quad \) für \( \alpha \in \Omega^k(M), \, \beta \in \Omega^*(M) \)
  \item[iii)] \( \iota_X(\mathrm{D}f) = Df(X) \quad \) für alle \( f \in C^\infty(M) \)
  \item[iv)] \( \iota_X^2 = 0 \)
  \item[v)] \( \iota_X f = 0 \quad \forall f \in C^\infty(M) \)
\end{itemize}

\textbf{Satz 5.19 (Cartans magische Formel)}\\
Sei \( \omega \in \Omega^*(M) \) und \( X \in \Gamma(TM) \). Dann gilt:
\[
\mathcal{L}_X \omega = \iota_X \mathrm{d} \omega + \mathrm{d} \iota_X \omega
\]

\begin{proof}
\underline{1. Fall:} Gilt für 0-Formen. Sei \( f \in C^\infty(M) \), dann ist
\[
\mathcal{L}_X f = \mathrm{D}f(X) =\mathrm{d}f(X), \quad \text{und} \quad \iota_X f = 0
\]
Somit gilt:
\[
\iota_X \mathrm{d}f + \mathrm{d}(\underbrace{\iota_X f}_{=0}) = \iota_Xdf \overset{\txt{5.18 iii})}{=} \mathrm{d}f(X) = \mathcal{L}_X f
\]

\underline{2. Fall:} Gilt auch für 1-Formen. Sei \( \omega \in \Omega^1(M) \), dann ist:
\[
(\iota_X \mathrm{d} \omega + \mathrm{d} \iota_X \omega)(Y)
\overset{\txt{Def. }\iota_X}{=} (\mathrm{d} \omega)(X, Y) + d(\omega(X))Y 
\]
\[
\overset{\txt{5.14 i)}}{=}\mathcal{L}_X (\omega(Y))- \mathcal{L}_Y (\omega(X)) - \omega([X,Y]) +\mathcal{L}_Y (\omega(X))\\\]\[
= \mathcal{L}_X (\omega(Y ))- \omega([X,Y])\overset{\txt{5.16 i)}}{=}(\mathcal{L}_X\omega)(Y)
\]

Sei nun \( P_X(\omega) := \iota_X \mathrm{d}\omega + \mathrm{d} \iota_X\omega \). Wir zeigen, dass \( P_X \) ein Derivation ist, d.h.
\[
P_X(\alpha \wedge \beta) = P_X(\alpha) \wedge \beta +  \alpha \wedge P_X(\beta)
\]
für \( \alpha \in \Omega^k(M), \beta \in \Omega^*(M) \).\\

Da \( P_X = \mathcal{L}_X \) auf 0- und 1-Formen, folgt daraus die Behauptung durch vollständige Induktion.

Sei also \( \alpha \in \Omega^k(M), \beta \in \Omega^*(M) \). Dann gilt:
\begin{align*}
P_X(\alpha \wedge \beta) 
&= \iota_X \mathrm{d}(\alpha \wedge \beta) + \mathrm{d} \iota_X(\alpha \wedge \beta) \\\\
&= \iota_X \left( \mathrm{d}\alpha \wedge \beta + (-1)^k \alpha \wedge \mathrm{d}\beta \right)
  + \mathrm{d} \left( \iota_X \alpha \wedge \beta + (-1)^k \alpha \wedge \iota_X \beta \right) \\\\
&= (\iota_X \mathrm{d}\alpha )\wedge \beta + \textcolor{blue}{(-1)^{k+1} \mathrm{d}\alpha \wedge \iota_X \beta} 
  + \textcolor{red}{(-1)^k \iota_X \alpha \wedge \mathrm{d}\beta} +(-1)^k(-1)^k\alpha\wedge\iota_X \mathrm{d}\beta\\\\
&\quad + (\mathrm{d} \iota_X \alpha)  \wedge \beta + \textcolor{red}{(-1)^{k-1}\iota_X\alpha \wedge \beta} + \textcolor{blue}{(-1)^k \mathrm{d}\alpha \wedge \iota_X \beta }
  + (-1)^k(-1)^k \alpha \wedge \mathrm{d} \iota_X \beta \\\\
&= (\iota_X \mathrm{d}\alpha + \mathrm{d} \iota_X \alpha) \wedge \beta 
  + \alpha \wedge (\iota_X \mathrm{d}\beta + \mathrm{d} \iota_X \beta) \\\\
&= P_X(\alpha) \wedge \beta + \alpha \wedge P_X(\beta)
\end{align*}
\end{proof}

Für eine \( k \)-Form \( \omega \) haben wir nach obigem Satz:
\[
\mathrm{d} \omega(X, X_1, \dots, X_k) = d(\iota_X(X_1,\dots,X_k))= (\mathcal{L}_X\omega)(X_1, \dots, X_k)-\iota_X(d\omega(X_1, \dots, X_k))
\]

{Iteration dieses Arguments liefert einen weiteren Beweis für die Formel (5.14 b):}
\[
\mathrm{d}\omega(X_1, \dots, X_{k+1}) =
\sum_{i=1}^{k+1} (-1)^{i+1} \mathcal{L}_{X_i} \left( \omega(X_1, \dots, \widehat{X_i}, \dots, X_{k+1}) \right)
\]
\[+ \sum_{1 \le i < j \le k+1} (-1)^{i+j} \omega([X_i, X_j], X_1, \dots, \widehat{X_i}, \dots, \widehat{X_j}, \dots, X_{k+1})\]

wobei \( X_1, \dots, X_{k+1} \) Vektorfelder auf \( M \) sind und \( \widehat{X_i} \) bedeutet, dass der entsprechende Eintrag ausgelassen wird.\\

\textbf{Beispiele:}
\begin{itemize}
  \item[i)] Sei \( \omega \in \Omega^1(M) \), so ist
  \[
  \mathrm{d}\omega(X, Y) = \mathcal{L}_X(\omega(Y)) - \mathcal{L}_Y(\omega(X)) - \omega([X, Y])
  \]
  
  \item[ii)] Sei \( \omega \in \Omega^2(M) \), so ist
  \[
  \mathrm{d}\omega(X, Y, Z) = \mathcal{L}_X(\omega(Y, Z)) 
 - \mathcal{L}_Y(\omega(X, Z)) 
 + \mathcal{L}_Z(\omega(X, Y)) - \omega([X, Y], Z) 
 \] 
 \[
 + \omega([X, Z], Y) 
 - \omega([Y, Z], X) =\sum_{\sigma_{XYZ}} 
  \left( \mathcal{L}_X(\omega(Y, Z)) - \omega([X, Y], Z) \right)\]
  wobei \( \sum_{\sigma_{XYZ}} \) die zyklische Summe über \( X, Y, Z \) bezeichnet.
\end{itemize}

\subsubsection*{Ein kurzer Ausblick}
\begin{itemize}
    \item Differentialformen haben Informationen über die Topologie der Mannigfaltigkeit. \\
    $\leadsto$ De-Rham-Kohomologie $\quad H_{\text{dR}}^*(M)$
    
    \medskip
    Es gilt\quad $H_{\text{dR}}^*(M) \cong H_{\text{sing}}^*(M)$ \quad \textcolor{orange}{\emph{singuläre Kohomologie}} \quad (für kompakte $M$)
    
    \item Auf orientierbaren (nächste Woche) Mannigfaltigkeiten $M$ lässt sich mit $\dim(M)$-Formen ein Integral definieren: \\
    $\omega \in \Omega^{\dim(M)}(M)$: \quad $\displaystyle \int_M \omega$
    
    \medskip
    Auf Mannigfaltigkeiten mit Rand gilt der Satz von Stokes: \\
    $\displaystyle \int_M d\omega = \int_{\partial M} \omega \qquad \omega \in \Omega^{\dim(M)-1}(M)$
    
    \medskip
    \textit{Insbesondere} \quad $\displaystyle \int_M d\omega = 0 \quad$ für $\partial M = \emptyset$
\end{itemize}
\[
\underbrace{\{ \text{exakte Formen} \}}_{\operatorname{im}(d)} 
\subseteq 
\underbrace{\{ \text{geschlossene Formen} \}}_{\operatorname{ker}(d)} 
\subseteq 
\{ \text{$k$-Formen} \}    \quad d^2=0
\]
\[
\omega \text{ mit } \omega = d\eta 
\quad \Rightarrow \quad
\omega \text{ mit } d\omega = 0
\]
\[
\text{Kohomologiegruppe:} \qquad \ker(d)/\operatorname{im}(d) \quad (\txt{Faktorgruppe})
\]

%%%%%%%%%%%%%%%%%%%%%%%%%%%%%%%%%%%%%%%%%%%%%%%%%%%%%%%%%%%%%%%%%%%%%%%%%%%%%%%%%%%%% Vorlesung 16 %%%%%%%%%%%%%%%%%%%%%%%%%%%

\section{Orientierung}

\underline{Erinnerung (Lineare Algebra):} Eine \underline{Orientierung} auf einem reellen Vektorraum \( V \) ist die Wahl einer Äquivalenzklasse von geordneten Basen
\[
\{v_1, \dots, v_n\} \sim \{w_1, \dots, w_n\}
\quad \Longleftrightarrow \quad
\text{(Übergangsmatrix hat positive Determinante)}
\]
\[\Longleftrightarrow \quad v_1^*\wedge\dots v_n^*=\lambda w_1^*\wedge \dots w_n^*\quad \txt{für ein } \lambda >0\]

\textbf{Sprich:} Diese Basen sind \underline{gleichorientiert}.

Es gibt genau zwei Äquivalenzklassen. Die Basen in einer ausgezeichneten Äquivalenzklasse heißen \underline{positiv orientiert}.\\

Die \underline{Standardorientierung} auf \( \mathbb{R}^n \), also \( or_{\mathrm{std}} \), ist gegeben durch die Äquivalenzklasse der kanonischen Standardbasis.\\

Eine lineare Abbildung \( f : (V, or_V) \to (W, or_W) \) zwischen zwei orientierten Vektorräumen heißt \underline{orientierungserhaltend}, falls \( f \) Basen in \( or_V \) auf Basen in \( or_W \) abbildet, also:
\[
f_* or_V = or_W \quad \Longleftrightarrow \quad \det(f)=\det(\mathcal{M}_{B_w}^{B_v}(f)) > 0.
\]
Es ergibt sich nur Sinn, wenn f ein Isomorphismus ist, also f bijektiv.\\
Sonst: \underline{orientierungsumkehrend}.\\

\textbf{Definition 6.1.}
Eine \underline{Orientierung} auf einer differenzierbaren Mannigfaltigkeit \( M \) ist eine Familie \( \{ or_p \}_{p \in M} \) von Orientierungen auf den Tangentialräumen \( T_p M \), so dass ein Atlas \( \mathcal{A} \) existiert mit
\[
D\varphi_p : (T_p M, or_p) \longrightarrow (\mathbb{R}^n, or_{\mathrm{std}})
\]
orientierungserhaltend für alle \( p \in U \) und alle Karten \( (U, \varphi) \in \mathcal{A} \).\\

Eine Mannigfaltigkeit heißt \underline{orientierbar}, falls eine solche Orientierung auf $M$ existiert, und \underline{orientiert}, falls eine Orientierung fixiert ist.

\vspace{1em}

\textbf{Definition 6.2.}
Eine differenzierbare Abbildung \( f : M \to N \) zwischen zwei orientierten Mannigfaltigkeiten heißt \underline{orientierungserhaltend}, falls
\[
Df_p : (T_p M, or_p) \longrightarrow (T_{f(p)} N, or_{f(p)})
\]
für alle \( p \in M \) orientierungserhaltend ist.\\

\textbf{Beispiel: \( S^1 \)}

\underline{Idee:} Orientiere jeden Tangentialraum \( T_p S^1 \) durch $(-p_2,p_1)$.  
Wir haben bereits die Karten gegeben durch die Projektion auf die \(x\)- bzw. \(y\)-Koordinate betrachtet.

\includegraphics[width=8cm]{Image Diffgeo/16.01.jpg}
\vspace{0.5em}

\underline{Problem:} Diese erfüllen die Bedingung (aus Definition 6.1) nicht. Standard Orientierug von $\mathbb{R}$ ist z.B. das grüne Pfeil in der Abbildung, wenn wir Tangentialvektoren aus oberen rechten auf x-Achse projizieren ergibt sich die umgekehrte Richtung.

\vspace{0.5em}

Betrachten daher die Karten:
\begin{align*}
(U_1, \varphi_1) &= (S^1 \cap \{y > 0\}, \quad (x, y) \mapsto -x) \\
(U_2, \varphi_2) &= (S^1 \cap \{y < 0\}, \quad (x, y) \mapsto x) \\
(U_3, \varphi_3) &= (S^1 \cap \{x > 0\}, \quad (x, y) \mapsto y) \\
(U_4, \varphi_4) &= (S^1 \cap \{x < 0\}, \quad (x, y) \mapsto -y)
\end{align*}

Die Kartenabbildungen \( \varphi_i \) müssen so gewählt werden, dass die Übergangsfunktionen zwischen überlappenden Kartengebieten differenzierbar und orientierungserhaltend sind.\\

\textbf{Satz 6.3.}
Eine Mannigfaltigkeit ist genau dann orientierbar, wenn ein Atlas existiert, sodass die Jacobi-Matrizen aller Kartenwechsel eine positive Determinante haben.

\begin{proof}
„\(\Rightarrow\)“: Sei \( M \) orientiert und seien \( (U, \varphi) \), \( (V, \psi) \) zwei orientierungserhaltende Karten.  
Dann sind die Ableitungen \( D\varphi_p \) und \( D\psi_p \) orientierungserhaltend.\\

Aus den Kettenregeln folgt, dass die Kartenwechselabbildung \( D(\psi \circ \varphi^{-1}) = J(\psi \circ \varphi^{-1}) \) orientierungserhaltend (bzgl. der Standardorientierungvon $\mathbb{R}$) ist und besitzt daher eine positive Determinante hat.\\

„\(\Leftarrow\)“: Gegeben sei ein Atlas \( \mathcal{A} \), sodass für alle Karten \( (U, \varphi), (V, \psi) \in \mathcal{A} \) gilt:
\[
\det(J(\psi \circ \varphi^{-1})) > 0.
\]
Definiere eine Orientierung an einem Punkt \( p \in M \) durch
\[
or_p := [\deldel{}{x_1},\dots, \deldel{}{x_n}].
\]
Dann ist auch jede andere Karte \( (V, \psi) \) orientierungserhaltend bzgl. \( or_p \), weil die Übergangsabbildung eine positive Determinante hat.\\

\underline{Bleibt zu zeigen:} Diese Orientierung ist wohldefiniert.\\
Sei \( (V, \psi) \) eine weitere Karte in \( \mathcal{A} \) mit \( p \in V \). Dann gilt:
\[
dy_1 \wedge \dots \wedge dy_n = \det\left( J(\psi \circ \varphi^{-1}) \right) \cdot dx_1 \wedge \dots \wedge dx_n
\]

{Voraussetzung:} \( \det\left( J(\psi \circ \varphi^{-1}) \right) > 0 \)
\[
\Rightarrow \text{Die Basen } \left\{ \frac{\partial}{\partial x_i} \right\} \text{ und } \left\{ \frac{\partial}{\partial y_i} \right\} \text{ sind gleich-orientiert.}
\]
\end{proof}


\textbf{Beispiel:}  
Mit dem Kriterium aus Satz 7.3 lässt sich zeigen, dass folgende Beispiele orientierbarer Mannigfaltigkeiten sind:

\begin{enumerate}[label=\roman*)]
    \item Die Spähre \( S^n \) ist orientiert.
    
    \item Das Tangentialbündel einer Mannigfaltigkeit \( M \) ist orientierbar.
    
    \item Sind \( M_1, M_2 \) zwei orientierte Mannigfaltigkeiten, so ist \( M_1 \times M_2 \) orientiert.
    
    \item Komplexe Mannigfaltigkeiten sind orientiert.
\end{enumerate}

\(n\)-dimensionale komplexe Mannigfaltigkeiten sind topologische Mannigfaltigkeiten, die Karten in \( \mathbb{C}^n \) haben, für die die Kartenwechsel biholomorphe Abbildungen sind.

Die komplexen projektiven Räume sind Beispiele kompakter komplexer Mannigfaltigkeiten.\\

\textbf{Definition 6.4.}  
Eine \emph{\underline{Volumenform}} auf einer \( n \)-dimensionalen Mannigfaltigkeit \( M \) ist eine \( n \)-Form \( \omega \in \Omega^n(M) \) ohne Nullstellen, d.\,h. \( \omega_p \neq 0 \) für alle \( p \in M \).

\vspace{1em}

\textbf{Lemma 6.5.}  
Auf \( M \) existiert genau dann eine Volumenform, wenn das Geradenbündel \( \Lambda^n T^*M \) trivial ist,  
d.\,h. isomorph zu dem trivialen Geradenbündel \( M \times \mathbb{R} \) ist.

\begin{proof}
„\(\Rightarrow\)“: Sei \( \omega \) eine Volumenform auf \( M \). Definiere den trivialisierenden Homomorphismus
\[
\Phi : \Lambda^n T^*M \to M \times \mathbb{R}
\]
für \( \alpha \in \Lambda^n T^*_p M \) durch
\[
\Phi(\alpha) := (p, f(p)).
\]
wobei \(\alpha = f(p) \cdot \omega_p\), da $\alpha $ als n-dim Form $\{\omega_p\}$ einfach als Basis benutzen kann.\\

„\(\Leftarrow\)“: Das Geradenbündel \( \Lambda^n T^*M \) sei trivial und  
\[
\Phi : \Lambda^n T^*M \to M \times \mathbb{R}
\]
sei ein entsprechender Bündel-Isomorphismus. Definiere eine Volumenform \( \omega \) auf \( M \) durch \( \omega_p := \Phi^{-1}(p, 1) \).

Dann ist \( \omega \) glatt und verschwindet nirgends, also eine Volumenform.
\end{proof}

\textbf{Satz 6.6.}  
Eine Mannigfaltigkeit \( M \) ist genau dann orientierbar, wenn eine Volumenform auf \( M \) existiert.

\vspace{1em}

\textbf{Beispiele:}
\begin{enumerate}[label=\roman*)]
    \item Jede Lie-Gruppe ist orientierbar.  

    Wähle eine Basis \( \{\mu_1, \dots, \mu_n\} \) in \( \mathfrak{g}^*=T_e^*G \) und definiere für \( g \in G \) eine \( n \)-Form durch
    \[
    \omega_g := r_{g^{-1}}^* (\mu_1 \wedge \dots \wedge \mu_n),
    \]
    wobei \( r_g \) die Rechtstranslation ist.  

    \(\Rightarrow\) \( \omega \) ist eine global definierte Form ohne Nullstellen, also eine Volumenform,  
    die eine Orientierung auf \( G \) definiert.

    \item Parallelisierbare Mannigfaltigkeiten mit trivialem Tangentialbündel sind orientierbar.  

    Ist das Tangentialbündel trivial, so ist es isomorph zu \( M \times \mathbb{R}^n \), und jedes \( k \)-Formen-Bündel ist ebenfalls trivial.  

    \(\Rightarrow\) Es existiert eine Volumenform.

    \item Auf der Sphäre lässt sich leicht eine explizite Volumenform hinschreiben.  

Ist \( p \in S^n \) und \( v_1, \dots, v_n \in T_p S^n = p^\perp \subset \mathbb{R}^{n+1} \), dann definieren wir
\[
\Omega_p(v_1, \dots, v_n) := \det(p, v_1, \dots, v_n),
\]

Wählen wir die Vektoren \( v_1, \dots, v_n \) als Ergänzung von \( p \) zu einer orthonormalen Basis von \( \mathbb{R}^{n+1} \) (genauer von $(\mathbb{R}^{n+1}, \mathrm{or}_{\txt{std}})$, 
dann ist der Wert von \( \Omega_p \) auf diesen Vektoren gleich +1.  

\(\Rightarrow\) \( \Omega \) ist damit eine Volumenform auf \( S^n \).
\end{enumerate}
\vspace{1em}

\begin{proof}[Beweis von Satz 6.6, Richtung „\(\Leftarrow\)“]  
Sei \( \omega \) eine Volumenform auf \( M \).

\underline{Behauptung 1:}  
Zu jedem \( p \in M \) gibt es eine Karte \( (U, \varphi) \) mit \( p \in U \) und
\[
\omega|_U = f \, dx_1 \wedge \dots \wedge dx_n, \quad \text{mit } f \in C^\infty(U),\ f > 0.  \quad (*)
\]

\underline{Begründung:}  
Sei \( (U, \varphi) \) eine beliebige Karte um \( p \), ohne Einschränkung sei \( U \) zusammenhängend.  
Dann schreibt sich die Volumenform \( \omega|_U \) als
\[
\omega|_U = f \, dx_1 \wedge \dots \wedge dx_n, \quad \text{für eine glatte Funktion } f.
\]
Da \( \omega \) auf \( U \) keine Nullstellen hat, gilt dies auch für \( f \).  
Also gilt entweder \( f > 0 \) auf ganz \( U \), oder \( f < 0 \) auf ganz \( U \), da $U$ zusammenhängend ist.

Im zweiten Fall definieren wir eine neue Karte \( (U, \tilde{\varphi}) \) mit
\[
\tilde{\varphi}(q) := (\varphi_2(q), \varphi_1(q) ,\varphi_3(q), \dots, \varphi_n(q)),
\]

Bezüglich dieser neuen Karte hat \( \omega \) positive Koeffizientenfunktion auf \( U \) (Kartenwechsel von $\varphi$ nach $\tilde{\varphi}$ ergibt sich ein Minus Zeichen).

Damit erfüllt die Karte die Bedingung aus Satz 7.3 und definiert somit eine wohldefinierte Orientierung.\\

Nun seien \( (U, \varphi) \), \( (V, \psi) \) zwei Karten um \( p \), welche die Bedingung \((*)\) erfüllen.

\underline{Behauptung 2:}  
Die Jacobi-Matrix des Kartenwechsels hat positive Determinante.\\

\underline{Begründung:}  
Auf \( U \cap V \) gilt für die Volumenform \( \omega \):
\[
\omega|_{U \cap V} = f_1 \, dx_1 \wedge \dots \wedge dx_n = f_2 \, dy_1 \wedge \dots \wedge dy_n, \quad \text{mit } f_1, f_2 > 0.
\]
\[
\Rightarrow dx_1 \wedge \dots \wedge dx_n = \frac{f_2}{f_1} \, dy_1 \wedge \dots \wedge dy_n \overset{\txt{Kartenwechsel}}{=} \det(J(\psi \circ \varphi^{-1}))\; dy_1 \wedge \dots \wedge dy_n
\]
Die Jacobi-Matrix des Kartenwechsels hat somit positive Determinante. Nach 6.3 ist $M$ dann orientierbar.

\vspace{0.6cm}

Richtung „\(\Rightarrow\)“ 
Sei \( M \) orientierbar, und sei \( \{(U_i, \varphi_i)\} \) ein Atlas von \( M \) mit 
\[
\det\left(D(\varphi_j \circ \varphi_i^{-1})\right) > 0
\quad \text{für alle } i, j. \quad (6.3)
\]

Auf den offenen Mengen \( \varphi_i(U_i) \subset \mathbb{R}^n \) betrachten wir die Standard-Volumenform
\[
\omega_0 := de_1 \wedge \dots \wedge de_n.
\]

Sei \( \{\chi_i\} \) eine der Überdeckung \( \{U_i\} \) untergeordnete \underline{Zerlegung der Eins},  
d.\,h. die \( \chi_i \) sind glatte Funktionen auf \( M \) mit:

\begin{enumerate}[label=\roman*)]
    \item \( \chi_i(x) \in [0,1] \) für alle \( i \) und \( x \in M \),
    \item \( \operatorname{supp}(\chi_i) \subset U_i \),
    \item Für alle \( x \in M \) existiert eine Umgebung \( V_x \) von \( x \), sodass nur endlich viele \( \chi_i \) auf \( V_x \) nicht verschwinden,
    \item \( \sum_i \chi_i(x) = 1 \) für alle \( x \in M \).
\end{enumerate}

Mithilfe der Funktionen \( \chi_i \), der Pullbacks \( \varphi_i^*\omega_0 \) und der Summation konstruieren wir eine globale glatte \( n \)-Form ohne Nullstellen auf \( M \), also eine Volumenform.\\

Wir definieren die \( n \)-Form \( \omega \) durch
\[
\omega := \sum_i \chi_i \cdot \varphi_i^*(\omega_0),
\]
wobei \( \omega_0 = dx_1 \wedge \dots \wedge dx_n \) die Standard-Volumenform auf \( \mathbb{R}^n \) ist.

\underline{Behauptung 3:} Diese Form ist die gesuchte Volumenform.\\

\underline{Begründung:}
\begin{itemize}
    \item \( \omega \) ist global definiert.
    \item \( \omega \) hat keine Nullstellen: Betrachte \( \omega \) auf der Kartenumgebung \( U_k \). Dort gilt:
    \begin{align*}
        (\varphi_k^{-1})^* \omega
        &= \sum_i (\chi_i \circ \varphi_k^{-1}) \cdot (\varphi_k^{-1})^* \varphi_i^*(\omega_0) \\
        &= \sum_i (\chi_i \circ \varphi_k^{-1}) \cdot (\varphi_i \circ \varphi_k^{-1})^* \omega_0 \\
        &= \sum_i \underbrace{(\chi_i \circ \varphi_k^{-1})}_{\in [0,1]} \cdot \underbrace{\det(J(\varphi_i \circ \varphi_k^{-1}))}_{>0} \cdot \omega_0 \\
        &= f\omega_0
    \end{align*}
    Für eine Funktion \( f \) mit \( f > 0 \) sind die Koeffizienten \(\det(J(\varphi_i \circ \varphi_k^{-1}))\) positiv.  
    Die Funktionen \( \chi_i \circ \varphi_k^{-1} \geq 0 \) sind nicht-negativ, wobei mindestens eine nicht identisch Null ist (da \( \sum_i \chi_i(p) = 1 \) für alle \( p \in M \)).
    \[
    \Rightarrow f:= \sum_i \chi_i \circ \varphi_k^{-1} \cdot \det(\varphi_i \circ \varphi_k^{-1})
    \]
    ist positiv.
\end{itemize}

\(\Rightarrow\) \( \omega \) ist eine global definierte \( n \)-Form ohne Nullstellen, also eine Volumenform.
\end{proof}



\textbf{Bemerkung:}  
Seien \( M, N \) orientierte Mannigfaltigkeiten mit Volumenformen \( \operatorname{vol}_M \) und \( \operatorname{vol}_N \).  

Eine Abbildung \( f : M \to N \) ist genau dann orientierungserhaltend, wenn
\[
f^* \operatorname{vol}_N = \lambda \cdot \operatorname{vol}_M,
\]
für eine positive Funktion \( \lambda \in C^\infty(M) \).

\vspace{1em}

\textbf{Beispiel:}  
Sei \( \Omega \) die Volumenform aus Beispiel (iii) auf \( S^n \), und  
\[
\tau : S^n \to S^n
\]
die Antipodenabbildung, also \( \tau(x) = -x \).

Dann gilt:
\[
\tau^* \Omega = (-1)^{n+1} \Omega,
\]
d.\,h. \( \tau : S^n \to S^n \) ist genau dann orientierungserhaltend, wenn \( n \) ungerade ist.\\

Für welche \( n \) ist \( \mathbb{RP}^n \) orientierbar?

Betrachten wir folgende Situation:  
Sei \( \tau : N \to N \) eine fixpunktfreie Involution,  
d.\,h. \( \tau^2 = \operatorname{id} \) (Involution) und \( \tau(p) \neq p \)  für alle \( p \in N \) (Fixpunktfrei).\\

Dann ist der Quotientenraum \( M = N/\tau \) eine glatte Mannigfaltigkeit (Beweis mit Godement-Kriterium). Die differenzierbare Struktur ist eindeutig durch die Bedingung bestimmt,  
dass die kanonische Projektion
\(
\pi : N \to M
\)
ein lokaler Diffeomorphismus ist.\\

Für \( N = S^n \) und \( \tau : S^n \to S^n \), \( x \mapsto -x \),  
erhalten wir \( \mathbb{RP}^n = S^n / \tau \).\\

Der Beweis für den allgemeinen Fall ist analog zu dem für \( \mathbb{RP}^n \).\\

\textbf{Satz 6.7.}  
Sei \( N \) eine orientierte Mannigfaltigkeit und \( \tau : N \to N \) eine fixpunktfreie Involution.  
Dann ist \( M = N / \tau \) genau dann orientierbar, wenn \( \tau \) orientierungserhaltend ist.

\vspace{1em}

\textbf{Korollar 6.8.}  
Der reell-projektive Raum \( \mathbb{RP}^n \) ist genau dann orientierbar, wenn \( n \) ungerade ist.

\begin{proof}
Es gilt \( \mathbb{RP}^n = S^n / \tau \), wobei \( \tau \) die Antipodenabbildung ist, also eine fixpunktfreie Involution. Wie zuvor gesehen, ist die Antipodenabbildung genau dann orientierungserhaltend, wenn \( n \) ungerade ist. Satz 6.7 liefert dann die behauptete Aussage.
\end{proof}

\textbf{Satz 6.9.}  
Eine \( n \)-dimensionale zusammenhängende Mannigfaltigkeit ist genau dann orientierbar,  
wenn die Menge
\[
\Lambda^n T^*M \setminus \{ \text{Nullschnitt} \}
\]
zwei Zusammenhangskomponenten besitzt. Dabei ist der Nullschnitt die Menge aller Nullvektoren der einzelnen Fasern,  
betrachtet als Teilmenge im Totalraum des Bündels. (Gegenbeispiel: Bei Möbiusband besitzt dieser Menge nur ein Zusammenhangskomponenten)

  \begin{figure}[H]
    \centering
    \includegraphics[width=12cm]{Image Diffgeo/16.02.jpg}
    \caption{Schneiden Möbius-Band}
 \end{figure}

\vspace{1em}

Auf der Menge \( \Lambda^n T^*M \setminus \{ \text{Nullschnitt} \} \)  
führen wir eine Äquivalenzrelation ein:

Zwei \( n \)-Formen \( \alpha \) und \( \beta \) sind äquivalent,  
falls es eine positive relle Zahl \( \lambda \)  gibt, sodass  
\[
\beta = \lambda \cdot \alpha.
\]

In diesem Fall liegen \( \alpha \) und \( \beta \) in derselben Faser über \( M \).

Wir definieren den Bündelraum der (nichtverschwindenden) Volumenformen bis auf positive Skalierung:
\[
\widetilde{M} := (\Lambda^n T^*M \setminus \{ \text{Nullschnitt} \}) / \sim.
\]

\textbf{Satz 6.10.}  
Die Menge \( \widetilde{M} \) (aus Satz 6.9) ist eine differenzierbare Mannigfaltigkeit. Weiter gilt:

\begin{enumerate}[label=\roman*)]
    \item Die Mannigfaltigkeit \( \widetilde{M} \) ist orientierbar.
    
    \item Die kanonische Projektion \( \widetilde{M} \to M \) ist eine 2-fache Überlagerung  
    (die sogenannte \emph{Orientierungsüberlagerung} von \( M \)).

    \item Die Mannigfaltigkeit \( M \) ist genau dann orientierbar, wenn die Orientierungsüberlagerung trivial ist,  
    also \( \widetilde{M}\) diffeomorph \( M \times \mathbb{Z}_2 \).
\end{enumerate}


\textbf{Beispiel:}
\[
Z := ([0,1] \times \mathbb{R}) /\sim_Z, \quad \text{mit } (0,p) \sim_Z (1,p) \quad \cong S^1 \times \mathbb{R} \quad \text{(der Zylinder)}
\]
\[
M := ([0,1] \times \mathbb{R}) / \sim_M, \quad \text{mit } (0,p) \sim_M (1, -p) \quad \text{(das Möbiusband)}
\]

Dann gilt:
\[
\widetilde{Z} = Z \times \{\pm 1\}, \qquad 
\widetilde{M} = ([0,2] \times \mathbb{R}) / \sim_Z
\]
mit geeigneter Äquivalenzrelation.
  \begin{figure}[H]
    \centering
    \includegraphics[width=6cm]{Image Diffgeo/16.03.jpg}
 \end{figure}

Die Projektionen lauten:
\[
\pi_Z : \widetilde{Z} \to Z, \quad (p, \pm 1) \mapsto p
\]
\[
\pi_M : \widetilde{M} \to M, \quad (x,y) \mapsto
\begin{cases}
(x\text{ mod }\mathbb{Z}, y), & \text{falls } x \in [0,1) \\
(x\text{ mod }\mathbb{Z}, -y), & \text{falls } x \in [1,2)
\end{cases}
\]
  \begin{figure}[H]
    \centering
    \includegraphics[width=12cm]{Image Diffgeo/16.99.png}
 \end{figure}
\textbf{Korollar 6.11.}  
Einfach zusammenhängende Mannigfaltigkeiten sind orientierbar.

\begin{proof}
Überlagerungen von einfach zusammenhängenden Mannigfaltigkeiten sind trivial.
\end{proof}

\vspace{1em}
\textbf{Bemerkung:}
\begin{enumerate}[label=\roman*)]
    \item Die Orientierungsüberlagerung \( \widetilde{M} \) ist die Menge aller Orientierungen aller Tangentialräume.  
    Die 2-elementige Menge \( \tilde{\pi}^{-1}(p) \) ($\lambda$ positiv oder nicht positiv) entspricht den beiden Orientierungen auf dem Tangentialraums \( T_p M \).

    \item Eine Orientierung auf \( M \) ist ein Schnitt in dem Bündel \( \widetilde{M} \to M \),  
    also eine glatte Abbildung \( M \to \widetilde{M} \), die in jedem \( p \) eine Orientierung auf \( T_p M \) zuordnet.

    \item Bis auf Übergang zur Orientierungsüberlagerung \( \widetilde{M} \)  
    können wir also immer annehmen, dass eine Mannigfaltigkeit orientierbar ist.
\end{enumerate}

\begin{enumerate}[label=\roman*), resume]
    \item Sei \( M^n \subset \mathbb{R}^{n+1} \).  
    Dann ist \( M \) genau dann orientierbar, wenn es ein glattes Normalenfeld ohne Nullstellen gibt,  
    d.\,h.
    \[
    T_{\iota(p)} \mathbb{R}^{n+1} = \mathbb{R} \cdot \nu_p \oplus D\iota(T_p M),
    \]
    wobei \( \iota \) ein Immersions- oder Inklusionsabbildung ist.

    \item Sei \( c \) ein regulärer Wert einer differenzierbaren Abbildung \( f : M \to N \).  
    Ist \( M \) orientierbar, dann ist auch die Mannigfaltigkeit \( M_c := f^{-1}(c) \) orientierbar.
\end{enumerate}
%%%%%%%%%%%%%%%%%%%%%%%%%%%%%%%%%%%%%%%%%%%%%%%%%%%%%%%%%%%%%%%%%%%%%%%%%%%%%%% Vorlesung 17 %%%%%%%%%%%%%%%%%%%%%%%%%%%%%%%%%

\section{Fundamentalgruppe und Überlagerung}
\underline{Ausgangsfrage:} Wie entscheiden wir, ob zwei topologische Räume (nicht) homöomorph sind?

\begin{itemize}
  \item {Für eine positive Antwort:} Gib einen Homöomorphismus an.
  \item {Für eine negative Antwort:} Zeige es gibt keine solche Abbildung.
\end{itemize}

Dafür nutzen wir häufig topologische Invarianten, also Eigenschaften, die unter Homöomorphismen erhalten bleiben:

\begin{itemize}
  \item Kompaktheit
  \item Zusammenhang (Zusammenhangskomponenten)
  \item Zweitabzählbarkeit
  \item Hausdorff-Eigenschaft (T2)
  \item \textbf{neu:} Fundamentalgruppe
\end{itemize}

\textbf{Beispiel:} Die Intervalle \( (0,1) \) und \( [0,1] \) sind nicht homöomorph.\\

Was ist mit \( \mathbb{R} \) und \( \mathbb{R}^2 \)? (Peano-Kurve)\\

Sei \( f: \mathbb{R}^2 \to \mathbb{R} \) ein Homöomorphismus. Dann ist auch jede Einschränkung
\[
\tilde{f} : \mathbb{R}^2 \setminus \{0\} \to \mathbb{R} \setminus \{f(0)\}
\]
ebenfalls ein Homöomorphismus. Aber: \( \mathbb{R}^2 \setminus \{0\} \) ist zusammenhängend, während \( \mathbb{R} \setminus \{f(0)\} \) nicht. {Analoger für} \( [0,1] \) und der Einheitskreis \( S^1 \) sind ebenfalls nicht homöomorph.\\

Dieser Trick (durch Entfernen eines Punktes den Zusammenhang zu testen) funktioniert nicht für kompliziertere Räume, z.B. $\mathbb{R}^2, \mathbb{R}^3$; $S^2, T^2$ und $T^2, T^2*T^2$ (Zusammensetzung von zwei Tori)\\

{Dafür führen wir den Begriff der \emph{einfachen Zusammenhangseigenschaft} ein.}\\

{Große Idee:} Ein topologischer Raum \( X \) ist \textbf{einfach zusammenhängend}, wenn jede geschlossene Kurve (d.h. stetige Abbildung \( S^1 \to X \)) sich in einem Punkt zusammenziehen lässt — also homotop zu einer konstanten Abbildung ist.

\subsection{Homotopie von Wegen}

\textbf{Definition 7.1} \\
Seien \( f_0, f_1 : X \to Y \) stetige Abbildungen. Man sagt, \( f_0 \) ist \underline{homotop} zu \( f_1 \), falls es eine stetige Abbildung
\[
\mathcal{F} : X \times [0,1] \to Y
\]
gibt mit
\[
\mathcal{F}(x,0) = f_0(x), \quad \mathcal{F}(x,1) = f_1(x)
\quad \text{für alle } x \in X.
\]

Die Abbildung \( \mathcal{F} \) heißt eine \underline{Homotopie} zwischen \( f_0 \) und \( f_1 \). Man schreibt \( f_0 \simeq f_1 \), falls \( f_0 \) homotop zu \( f_1 \) ist. Ist \( f_0 \simeq f_1 \) und \( f_1 \) die konstante Abbildung, dann nennt man \( f_0 \) \underline{nullhomotop}.\\

\textbf{Anschauung}

Eine Homotopie ist eine einparametrige Familie von Abbildungen von \( X \) nach \( Y \).

Dabei stellen wir uns vor, dass \( t \in [0,1] \) die Zeit darstellt. Dann beschreibt die Homotopie \( \mathcal{F} \) eine stetige \emph{Verformung} der Abbildung \( f_0 \) zur Abbildung \( f_1 \), wobei \( t \) von 0 bis 1 läuft.

\vspace{1em}

Wir wollen nun einen wichtigen {Spezialfall} betrachten: Wege (oder: Pfade) in einem topologischen Raum \( X \).\\

Für eine stetige Abbildung
\(
\gamma : [0,1] \to X
\)
mit \( \gamma(0) = x_0 \) und \( \gamma(1) = x_1 \), nennen wir \( \gamma \) einen \underline{Pfad} (oder Weg) von \( x_0 \) nach \( x_1 \).

\begin{itemize}
  \item Der Punkt \( x_0 \) heißt \underline{Startpunkt} (auch: Anfangspunkt) des Pfades \( \gamma \).
  \item Der Punkt \( x_1 \) heißt \underline{Endpunkt} des Pfades \( \gamma \).
\end{itemize}

\vspace{0.5cm}

\textbf{Definition 7.2} \\
Zwei Wege \( f_0, f_1 : [0,1] \to X \) heißen \underline{weghomotop} (also homotop relativ der Endpunkte), wenn sie denselben Startpunkt \( x_0 \) und denselben Endpunkt \( x_1 \) besitzen, und es eine stetige Abbildung
\[
\mathcal{F} : [0,1] \times [0,1] \to X
\]
gibt, sodass
\begin{align*}
\mathcal{F}(s, 0) &= f_0(s), & \mathcal{F}(s, 1) &= f_1(s)  \quad (*)\\
\mathcal{F}(0, t) &= x_0,   & \mathcal{F}(1, t) &= x_1 \quad (**)
\end{align*}
für alle \( s, t \in [0,1] \) gilt.

Die Abbildung \( \mathcal{F} \) heißt eine \underline{Weghomotopie} (Homotopie relativ der Endpunkte) zwischen \( f_0 \) und \( f_1 \). \( f_0 \) ist weghomotop zu \( f_1 \). Man schreibt dann
\[
f_0 \simeq_p f_1 \quad \text{(rel. Endpunkte)}
\]

  \begin{figure}[H]
    \centering
    \includegraphics[width=11cm]{Image Diffgeo/17.01.jpg}
	%\caption{Ebene mit Loch und die Zylinder Oberfläche sind nicht einfach zusammenhängend}
 \end{figure}

\begin{itemize}
  \item[\textcircled{1}] besagt: \( \mathcal{F} \) ist eine Homotopie zwischen \( f_0 \) und \( f_1 \).
  \item[\textcircled{2}] bedeutet: Die Kurve \( f_t(s) := \mathcal{F}(s,t) \) ist für jedes feste \( t \) ein Weg von \( x_0 \) nach \( x_1 \).
\end{itemize}

Das heißt: \( \mathcal{F} \) deformiert \( f_0 \) stetig zu \( f_1 \) und hält dabei die Randpunkte \( x_0 \) und \( x_1 \) fest.

\vspace{1em}

\textbf{Lemma 7.3.} \\
Die Relationen \( \simeq \) (Homotopie) und \( \simeq_p \) (Weghomotopie) sind Äquivalenzrelationen.

Ist \( f \) ein Weg, so bezeichnen wir die Weghomotopie-Äquivalenzklasse von \( f \) mit \( [f] \).

\begin{proof}
Wir zeigen, dass \( \simeq_p \) eine Äquivalenzrelation ist:

\begin{enumerate}
  \item {(Reflexivität):} \( f \simeq_p f \), denn wir können die konstante Homotopie \( \mathcal{F}(x,t) := f(x) \) verwenden.
  
  \item {(Symmetrie):} Angenommen \( f \simeq_p f' \) via \( \mathcal{F}(x,t) \). Dann definiert
  \[
  \mathcal{F}'(x,t) := \mathcal{F}(x,1 - t)
  \]
  eine Weghomotopie von \( f' \) zu \( f \), da Anfangs- und Endpunkte ebenfalls erhalten bleiben.
\end{enumerate}

\begin{enumerate}
  \setcounter{enumi}{2}
  \item (Transitivität): Sei \( f_0 \simeq_p f_1 \) und \( f_1 \simeq_p f_2 \). 

  Sei \( \mathcal{F}_0 \) eine Weghomotopie von \( f_0 \) zu \( f_1 \), und \( \mathcal{F}_1 \) eine Weghomotopie von \( f_1 \) zu \( f_2 \).

  Definiere
  \[
  G : X \times [0,1] \to Y, \quad \text{durch}
  \]
  \[
  G(x,t) := 
  \begin{cases}
    \mathcal{F}_0(x, 2t) & \text{für } t \in \left[0, \tfrac{1}{2} \right] \\
    \mathcal{F}_1(x, 2t - 1) & \text{für } t \in \left[\tfrac{1}{2}, 1 \right]
  \end{cases}
  \]

  \( G \) ist wohldefiniert, denn für \( t = \frac{1}{2} \) gilt:
  \[
  \mathcal{F}_0(x,1) = f_1(x) = \mathcal{F}_1(x,0)
  \]

  \( G \) ist stetig auf den abgeschlossenen Mengen \( X \times \left[0,\tfrac{1}{2}\right] \) und \( X \times \left[\tfrac{1}{2},1\right] \)
  \[\Rightarrow G \;\txt{ist stetig auf }X\times [0,1]\]
\end{enumerate}
\end{proof}
  \begin{figure}[H]
    \centering
    \includegraphics[width=13cm]{Image Diffgeo/17.02.jpg}
	%\caption{Ebene mit Loch und die Zylinder Oberfläche sind nicht einfach zusammenhängend}
 \end{figure}

\textbf{Beispiel 1:}

Seien \( f, g : X \to \mathbb{R}^2 \) stetige Abbildungen. Dann sind \( f \) und \( g \) {homotop}; denn die Abbildung
\[
\mathcal{F}(x,t) := (1 - t) \cdot f(x) + t \cdot g(x)
\]
ist eine Homotopie zwischen ihnen.
Diese nennt man eine \underline{Geraden-Linien-Homotopie}, da sie \( f(x) \) und \( g(x) \) entlang der Geraden zwischen den beiden Punkten verbindet.\\

Sind \( f \) und \( g \) zusätzlich Wege von \( x_0 \) nach \( x_1 \), so ist \( \mathcal{F} \) eine {Weghomotopie}.\\
  \begin{figure}[H]
    \centering
    \includegraphics[width=5cm]{Image Diffgeo/17.03.jpg}
	%\caption{Ebene mit Loch und die Zylinder Oberfläche sind nicht einfach zusammenhängend}
 \end{figure}

\textbf{Bemerkung:} Diese Konstruktion funktioniert allgemein für konvexe Zielräume wie \( \mathbb{R}^n \), da in konvexen Mengen die Verbindungsstrecken zwischen zwei Punkten stets enthalten sind.\\

\textbf{Beispiel 2}

Sei \( X = \mathbb{R}^2 \setminus \{0\} \), also die \emph{gelochte Ebene}. Dann sind zwei Wege \( f \) und \( g : [0,2] \to X \) gegeben mit
\[
f(s) = 
\begin{pmatrix}
\cos(\pi s) \\
\sin(\pi s)
\end{pmatrix},
\quad
g(s) = 
\begin{pmatrix}
\cos(\pi s) \\
-\sin(\pi s)
\end{pmatrix}.
\]

{weghomotop}. Aber: \( f \) ist \emph{nicht} weghomotop zu
\[h(s) = \left( \cos(-\pi s),\, \sin(-\pi s) \right) = \left( \cos(\pi s),\, -\sin(\pi s) \right)\]

über die Geraden-Linien-Homotopie, da \( h \) die $x$-Achse durchläuft und damit das Loch im Ursprung berührt. \( f \) lässt sich nicht stetig in \( h \) deformieren, ohne den Ursprung zu durchqueren.\\

\underline{Fazit:} Der {Bildraum} ist entscheidend für die Frage, ob zwei Abbildungen homotop sind. Die gleiche Formel kann auf verschiedenen Zielräumen unterschiedliche Homotopieeigenschaften haben.\\

\textbf{Definition 7.4} \\
Sei \( f \) ein Weg von \( x_0 \) nach \( x_1 \), und \( g \) ein Weg von \( x_1 \) nach \( x_2 \). Dann definieren wir das \underline{Produkt} (oder die \underline{Verkettung}) \( f * g \) (pass auf Reihenfolge, anders als Abbildung-Verkettung) von \( f \) und \( g \) als den Weg \( h : [0,1] \to X \), gegeben durch
\[
h(s) := 
\begin{cases}
f(2s) & \text{für } s \in [0,\tfrac{1}{2}] \\
g(2s - 1) & \text{für } s \in [\tfrac{1}{2}, 1]
\end{cases}
\]

Die Abbildung \( h \) ist
\begin{itemize}
  \item wohldefiniert (da \( f(1) = x_1 = g(0) \)),
  \item stetig (beide Stücke sind stetig, und sie stimmen bei \( s = \tfrac{1}{2} \) überein),
  \item ein Weg von \( x_0 \) nach \( x_2 \).
\end{itemize}

\underline{Idee:} Die erste Hälfte von \( h \) ist \( f \), die zweite Hälfte ist \( g \).

Diese Produktoperation induziert eine wohldefinierte Operation auf den Weghomotopieklassen durch:
\[
[f] * [g] := [f * g]
\]

Dies bildet die Grundlage für die Definition der \textbf{Fundamentalgruppe} im nächsten Abschnitt.\\

\textbf{Beobachtung:} \\
Für \( f_0 \simeq_p f_1 \) und \( g_0 \simeq_p g_1 \) gilt:
\[
f_0 * g_0 \simeq_p f_1 * g_1
\]

Dazu betrachten wir eine Weghomotopie \( \mathcal{F} \) von \( f_0 \) zu \( f_1 \), und eine Weghomotopie \( \mathcal{G} \) von \( g_0 \) zu \( g_1 \).

Definiere:
\[
H(s,t) :=
\begin{cases}
\mathcal{F}(2s,t) & \text{für } s \in [0, \tfrac{1}{2}] \\
\mathcal{G}(2s - 1, t) & \text{für } s \in [\tfrac{1}{2}, 1]
\end{cases}
\]

Da \( \mathcal{F}(1,t) = x_1 = \mathcal{G}(0,t) \) für alle \( t \in [0,1] \), ist \( H \) wohldefiniert und stetig.

\textbf{Folglich:} \( H \) ist eine Weghomotopie von \( f_0 * g_0 \) nach \( f_1 * g_1 \), also gilt:
\[
[f_0] * [g_0] = [f_1] * [g_1]
\]
  \begin{figure}[H]
    \centering
    \includegraphics[width=11cm]{Image Diffgeo/17.04.jpg}
	%\caption{Ebene mit Loch und die Zylinder Oberfläche sind nicht einfach zusammenhängend}
 \end{figure}

Die Operation \( * \) erfüllt Eigenschaften, die denen einer Gruppe ähnlich sind. 
Wir nennen sie die \emph{Gruppoid-Eigenschaften} der Wegverkettung. Ein wichtiger Unterschied zu Gruppen besteht darin, dass \( [f] * [g] \) 
nicht für alle Klassen \( [f], [g] \) definiert ist, sondern nur dann, 
wenn \( f(1) = g(0) \) gilt.\\


\textbf{Satz 7.5.} \\
Die Operation \( * \) auf den Weghomotopieklassen hat folgende Eigenschaften:

\begin{enumerate}
  \item[(i)] \textbf{(Assoziativität)} \\
  Für drei kompatible Wegeklassen gilt:
  \[
  [f] * ([g] * [h]) = ([f] * [g]) * [h]
  \]
  sofern beide Seiten definiert sind (d.h. Endpunkt von \( f \) = Startpunkt von \( g \), etc.).

  \item[(ii)] \textbf{(Rechts- und Linksneutrales Element)} \\
  Für jeden Punkt \( x \in X \) sei \( e_x : I \to X \) der konstante Weg \( e_x(t) = x \). Dann gilt für jeden Pfad \( f \) mit Anfangs- bzw. Endpunkt \( x \):
  \[
  [f] * [e_{x_1}] = [f], \quad [e_{x_0}] * [f] = [f]
  \]
  wenn \( f : x_0 \to x_1 \) ist.

  \item[(iii)] \textbf{(Inverse)} \\
  Für einen Weg \( f : x_0 \to x_1 \) sei der Umkehrweg \( {f^{-}} \) definiert durch \( f^{-}(s) := f(1 - s) \). Dann gilt:
  \[
  [f] * [f^{-}] = [e_{x_0}], \quad [f^{-}] * [f] = [e_{x_1}]
  \]
\end{enumerate}
  \begin{figure}[H]
    \centering
    \includegraphics[width=7cm]{Image Diffgeo/17.05.jpg}
	%\caption{Ebene mit Loch und die Zylinder Oberfläche sind nicht einfach zusammenhängend}
 \end{figure}

{Der Beweis beruht auf zwei Beobachtungen}
\begin{enumerate}
  \item Sei \( k : X \to Y \) eine stetige Abbildung, und sei \( \mathcal{F} \) eine Weghomotopie in \( X \) zwischen zwei Wegen \( f_0 \) und \( f_1 \). Dann ist
  \[
  k \circ \mathcal{F}(s,t)
  \]
  eine Weghomotopie zwischen \( k \circ f_0 \) und \( k \circ f_1 \) in \( Y \).
  \begin{figure}[H]
    \centering
    \includegraphics[width=13cm]{Image Diffgeo/17.06.jpg}
	\caption{Die Homotopie wird durch die stetige Abbildung weitertransportiert}
 \end{figure}

  \item Sei \( k : X \to Y \) eine stetige Abbildung und seien \( f, g \) Wege in \( X \) mit \( f(1) = g(0) \). Dann gilt:
  \[
  k \circ (f * g) = (k \circ f) * (k \circ g)
  \]
  also erhält die Verkettung von Wegen unter \( k \) ihre Struktur.

\end{enumerate}

\vspace{1em}
Für den Beweis von (ii) und (iii) der Gruppoid-Eigenschaften lassen sich die Homotopien direkt konstruieren. {Der Beweis von (i) ist schwieriger.}\\

\textbf{Satz 7.6.} \\
Sei \( f \) ein Pfad in \( X \), und sei \( 0 = a_0 < a_1 < \dots < a_n = 1 \) eine Zerlegung des Intervalls \([0,1]\). Definiere Teilpfade
\[
f_i := f \circ \lambda_i : [0,1] \to X, \quad \text{für } 1 \leq i \leq n,
\]
wobei \( \lambda_i \) die lineare Abbildung ist, die \([0,1]\) auf das Teilintervall \([a_{i-1}, a_i] \) abbildet.

Dann gilt:
\[
[f] = [f_1] * [f_2] * \cdots * [f_n]
\]

Das ist eine {Verallgemeinerung der Assoziativität} der Wegverkettung.

Kurz gesagt: Für Weghomotopien ist es egal, wie wir das Intervall zerlegen.\\

\textbf{Lemma 7.7.} \\
Seien \( h_0, h_1 : X \to Y \) homotop und \( k_0, k_1 : Y \to Z \) homotop. \\
Dann sind auch die Verkettungen \( k_0 \circ h_0 \) und \( k_1 \circ h_1 : X \to Z \) homotop.

\vspace{1em}

\textbf{Definition 7.8.} \\
Ein topologischer Raum \( X \) heißt \underline{zusammenziehbar}, falls die Identität nullhomotop ist, d.h. \( \text{id}_X : X \to X \) homotop zur konstanten Abbildung \( c : X \to X \), \( c(x) = x_0 \) (für ein festes \( x_0 \in X \)).

\vspace{1em}

\textbf{Beispiele und Bemerkungen:}
\begin{enumerate}
  \item[(i)] Die Intervalle \( [0,1] \) und der Raum \( \mathbb{R}^n \) sind zusammenziehbar. \\
  Eine Homotopie ist gegeben durch:
  \[
  \mathcal{F}(x,t) = (1 - t) \cdot x + t \cdot x_0
  \]
  für ein festes \( x_0 \in X \), z.B. \( 0 \).

  \item[(ii)] Jeder zusammenziehbare Raum ist zusammenhängend.

  \item[(iii)] Ist \( Y \) zusammenziehbar, dann sind für jeden topologischen Raum \( X \) alle zwei stetigen Abbildungen \( f, g : X \to Y \) homotop.

  \item[(iv)] Ist \( X \) zusammenziehbar und \( Y \) weg-zusammenhängend, dann sind alle stetigen Abbildungen \( X \to Y \) homotop.
\end{enumerate}

\subsection{Die Fundamentalgruppe}

Wir möchten die Operation \( * \) als Gruppenoperation zwischen Homotopieklassen von Pfaden auffassen.\\

\underline{Problem:} Die Verkettung \( * \) ist nicht zwischen allen Äquivalenzklassen von Pfaden wohldefiniert, da Anfangs- und Endpunkte übereinstimmen müssen.\\

\underline{Lösung:} Wenn wir einen \textbf{Basispunkt} \( x_0 \in X \) fixieren und nur Wege betrachten, die in \( x_0 \) beginnen und enden (also geschlossene Wege basierend in \( x_0 \)), dann bildet die Menge der Homotopieklassen solcher Wege mit \( * \) eine Gruppe. Diese Gruppe heißt die \textbf{Fundamentalgruppe} von \( X \) zum Basispunkt \( x_0 \), geschrieben \( \pi_1(X, x_0) \).\\

\underline{Ziel:} Die Fundamentalgruppe ist eine topologische Invariante und erlaubt es, wesentliche Informationen über die Struktur von \( X \) zu extrahieren.\\

\textbf{Definition 7.9.} \\
Sei \( X \) ein topologischer Raum und \( x_0 \in X \) ein Punkt.
Ein {Weg} \( f : [0,1] \to X \), der in \( x_0 \) beginnt und endet, also \( f(0) = f(1) = x_0 \), heißt \underline{Schlaufe} mit Basispunkt \( x_0 \).\\

Die Menge der Weghomotopieklassen von Schlaufen mit Basispunkt \( x_0 \), zusammen mit der Verkettung \( * \), heißt die \underline{Fundamentalgruppe} von \( X \) bzgl. Basispunkt \( x_0 \). \\

Wir schreiben:
\(\pi_1(X, x_0)\) \\

\textbf{Bemerkung:}
\begin{enumerate}
  \item[(i)]  Aus Satz 7.5 folgt, dass \( * \) auf dieser Menge (der Weghomotopieklassen von Schlaufen mit Basispunkt \( x_0 \)) die Axiome einer Gruppe erfüllt.

  Weiter ist \( \pi_1(X, x_0) \) abgeschlossen unter \( * \), das heißt: Sind \( f \) und \( g \) Schlaufen mit Basispunkt \( x_0 \), 
  so ist auch \( f * g \) wieder eine Schlaufe mit Basispunkt \( x_0 \).

  \item[(ii)] Man nennt \( \pi_1(X, x_0) \) auch die \textbf{erste Homotopiegruppe}.\\
  \textit{(Es gibt auch höhere Homotopiegruppen, z.B. \( \pi_n(X, x_0) \), für \( n \geq 2 \), wobei man hier nicht $S^1$ auf $X$ abbildet, sondern allgemein $S^n$)}
\end{enumerate}

\vspace{1em}
\textbf{Beispiel:}
\[
\pi_1(\mathbb{R}^n, x_0) = \{e_{x_0}\}
\]
{Analog:} Ist \( X \subseteq \mathbb{R}^n \) konvex, so gilt ebenfalls \( \pi_1(X, x_0) = \{e_{x_0}\} \).\\

{Frage:} Wie hängt die Fundamentalgruppe von der Wahl des Basispunktes ab?

\vspace{1em}
\textbf{Definition 7.10.} \\
Sei \( \alpha \) ein Weg in \( X \) von \( x_0 \) nach \( x_1 \). Wir definieren eine Abbildung (Konjugation)
\[
\hat{\alpha} : \pi_1(X, x_0) \to \pi_1(X, x_1)
\]
durch
\[
\hat{\alpha}([f]) := [\alpha^{-} * f * \alpha] = [\alpha^{-}]*[f]*[\alpha]
\]
also: Weg entlang \( \alpha^{-} \) zurück, dann entlang \( f \), dann entlang \( \alpha \) vorwärts.

  \begin{figure}[H]
    \centering
    \includegraphics[width=7cm]{Image Diffgeo/17.07.jpg}
	%\caption{Ebene mit Loch und die Zylinder Oberfläche sind nicht einfach zusammenhängend}
 \end{figure}

\vspace{1em}
\textbf{Bemerkung:} Die Abbildung \( \hat{\alpha} \) (gesprochen: "$\alpha$-pushforward") ist wohldefiniert, d.h. sie hängt nur von der Homotopieklasse von \( \alpha \) ab.\\

\textbf{Satz 7.11.} \\
Die Abbildung \( \hat{\alpha} : \pi_1(X, x_0) \to \pi_1(X, x_1) \), definiert durch
\[
\hat{\alpha}([f]) := [\alpha^{-} * f * \alpha]
\]
ist ein \textbf{Gruppenisomorphismus}.

\textit{Beweis:} Übungsaufgabe.

\vspace{1em}
\textbf{Korollar 7.12.} \\
Ist \( X \) wegzusammenhängend und \( x_0, x_1 \in X \), dann sind
\[
\pi_1(X, x_0) \cong \pi_1(X, x_1)
\]

\textbf{Bemerkung:}

\begin{itemize}
  \item Sei \( C \subseteq X \) eine wegzusammenhängskomponente von \( X \), die \( x_0 \) enthält. Dann gilt:
\[
\quad \pi_1(C, x_0) \cong \pi_1(X, x_0)
\]
D.h. die Fundamentalgruppe beschreibt nur die Informetionen enthalten im Zusammenhangskomponent. Meistens beschränkt man sich bei der Untersuchung der Fundamentalgruppe auf wegzusammenhängende Räume.

  \item Ist \( X \) wegzusammenhängend, so sind alle Fundamentalgruppen \( \pi_1(X, x_0) \) für verschiedene \( x_0 \in X \) zueinander isomorph. Daher schreibt man oft einfach \( \pi_1(X) \), ohne den Basispunkt zu erwähnen. Aber der Isomorphismus zwischen \( \pi_1(X, x_0) \) und \( \pi_1(X, x_1) \) hängt im Allgemeinen vom gewählten Weg \( \alpha \) zwischen den Basispunkten ab.

  \item Der Isomorphismus ist unabhängig vom gewählten Weg genau dann, wenn \( \pi_1(X, x_0) \) abelsch ist. Man spricht dann von \emph{kanonische Ähnlichkeit}.
\end{itemize}

\textbf{Definition 7.13.} \\
Ein topologischer Raum \( X \) heißt \underline{einfach zusammenhängend}, falls
\begin{itemize}
  \item \( X \) zusammenhängend ist und
  \item die Fundamentalgruppe \( \pi_1(X, x_0) \) für ein (und damit jedes) \( x_0 \in X \) die triviale (eindeutige) Gruppe ist.
\end{itemize}

Man schreibt dann:
\[
\pi_1(X, x_0) = 0
\]
falls $\pi_1(X,x_0)$ die triviale Gruppe ist. Was bedeutet, dass alle Schlaufen (mit Basispunkt \( x_0 \)) homotop zur konstanten Schlaufe sind.

\vspace{1em}
\textbf{Lemma 7.14.} \\
In einem einfach zusammenhängenden Raum \( X \) sind je zwei Wege mit denselben Start- und Endpunkten weghomotop.

\begin{proof}
Seien \( \alpha, \beta \) zwei Wege von \( x_0 \) nach \( x_1 \). Dann ist die in X geschlossene Schlaufe $\alpha * \beta^{-}$
eine Schlaufe mit Basispunkt \( x_0 \). Da \( X \) einfach zusammenhängend ist, ist \( \alpha * \beta^{-} \) weghomotop zur konstanten Weg \( e_{x_0} \).

Also gilt:
\[
[\alpha]=[\alpha]*[e_{x_1}]=[\alpha]*[\beta^{-}*\beta]=[\alpha*\beta^{-}*\beta] = [\alpha*\beta^{-}]*[\beta]=[e_{x_0}]*[\beta]
= [\beta]
\]

Daraus folgt:
\[
[\alpha] = [\beta]
\]
\end{proof}

Sei \( h : X \to Y \) eine stetige Abbildung mit \( h(x_0) = y_0 \). \\
Wir schreiben dann kurz:
\[
h : (X, x_0) \to (Y, y_0)
\]

Ist \( f \) eine Schlaufe in \( X \) mit Basispunkt \( x_0 \), so ist die Verkettung \( h \circ f : I \to Y \) eine Schlaufe in \( Y \) mit Basispunkt \( y_0 \).\\

Durch \( f \mapsto h \circ f \) wird eine Abbildung
\[
\pi_1(X, x_0) \longrightarrow \pi_1(Y, y_0)
\]
induziert.

\vspace{1em}
\textbf{Definition 7.15} \\
Sei \( h : (X, x_0) \to (Y, y_0) \) eine stetige Abbildung. Dann definiert man:
\[
h_* : \pi_1(X, x_0) \to \pi_1(Y, y_0), \quad [f] \mapsto [h \circ f]
\]

Diese Abbildung \( h_* \) heißt der \underline{von \( h \) induzierte Homomorphismus} auf den Fundamentalgruppen bzgl. des Basispunkts \( x_0 \).\\

Die Abbildung \( h_* \) ist wohldefiniert, denn: Ist \( \mathcal{F} \) eine Weghomotopie zwischen \( f_0 \) und \( f_1 \), so ist \( h \circ \mathcal{F} \) eine Weghomotopie zwischen \( h \circ f_0 \) und \( h \circ f_1 \).

Die Tatsache, dass \( h_* \) ein Gruppenhomomorphismus ist, folgt aus:
\[
h_*([f * g]) = [h \circ (f * g)] = [(h \circ f) * (h \circ g)] = h_*([f]) * h_*([g])
\]

\vspace{1em}
\textbf{Bemerkung:} \\
Der Homomorphismus \( h_* \) hängt nicht nur von der Abbildung \( h : X \to Y \) ab, sondern auch von der Wahl des Basispunkts \( x_0 \in X \) durch \( h \) ab (Sobald $x_0$ gewählt ist, ist $y_0$ durch $h$ bestimmt).\\

Daraus ergeben sich Schwierigkeiten für die Notation:
Sind \( x_0 \) und \( x_1 \) verschiedene Basispunkte in \( X \), können wir nicht denselben Ausdruck \( h_* \) für beide induzierten Homomorphismen verwenden, denn:
\[
h_* : \pi_1(X, x_0) \to \pi_1(Y, y_0), \quad \text{versus} \quad \pi_1(X, x_1) \to \pi_1(Y, y_0)
\]

Selbst wenn \( X \) wegzusammenhängend ist und die Gruppen isomorph sind, handelt es sich nicht um dieselbe Gruppe.\\

In solchen Fällen schreibt man explizit:
\[
(h_{x_0})_* : \pi_1(X, x_0) \longrightarrow \pi_1(Y, y_0) \quad (\txt{bzw. } (h_{x_1})_*)
\]

Wenn wir nur einen festen Basispunkt betrachten, können wir zur Vereinfachung wieder einfach \( h_* \) schreiben.\\

\textbf{Satz 7.16 (Funktorielle Eigenschaften):} \\
Seien \( h : (X, x_0) \to (Y, y_0) \) und \( k : (Y, y_0) \to (Z, z_0) \) stetige Abbildungen. Dann gilt:
\[
(k \circ h)_* = k_* \circ h_*
\]

Ist zudem \( \text{id}_{(X,x_0)}:(X,x_0)\to(X,x_0) \) die Identität auf, dann ist
\(
(\text{id}_{(X,x_0)})_* = \text{id}_{\pi_1(X, x_0)}
\) der Indentitätshomomorphismus

\vspace{1em}
\textbf{Korollar 7.17.} \\
Ist \( h : (X, x_0) \to (Y, y_0) \) ein Homöomorphismus, dann ist der induzierte Homomorphismus
\[
h_* : \pi_1(X, x_0) \to \pi_1(Y, y_0)
\]
ein Isomorphismus.

\begin{proof}
Sei \( k : (Y, y_0) \to (X, x_0) \) die Inverse von \( h \). Dann gilt:
\(
k_* \circ h_* =(k\circ h)_*=(id_X)_*
\), wobei $id_X$ die Identität von X ist. Weiter gilt \(
h_* \circ k_* =(h\circ k)_*=(id_Y)_*
\) für die Identität $id_Y$ von Y

Da $(id_X)_*$ und $(id_Y)_*$ die Identität auf $\pi_1(X,x_0)$ bzw. $\pi_1(Y,y_0)$ sind sehen wir, dass $k_*$ die Inverse zu $h_*$ ist.
\end{proof}

%%%%%%%%%%%%%%%%%%%%%%%%%%%%%%%%%%%%%%%%%%%%%%%%%%%%%%%%%%%%%%%%%%%%%%%%%%%%%%%%% Vorlesung 18 %%%%%%%%%%%%%%%%%%%%%%%%%%%%%%%

\subsection{Überlagerungen}

\textbf{Motivation:} \\
Überlagerungen bieten ein Hilfsmittel zur Berechnung von Fundamentalgruppen. \\
Gleichzeitig liefern sie eine geometrische Interpretation von Untergruppen der Fundamentalgruppe.

\vspace{1em}
\textbf{Definition 7.18:} \\
Sei \( B \) ein topologischer Raum. Eine \textbf{\underline{Überlagerung}} von \( B \) ist eine stetige, surjektive Abbildung
\[
p : E \to B,
\]
sodass es für jeden Punkt \( b \in B \) eine Umgebung \( U \subseteq B \) gibt, für die gilt:

\begin{itemize}
  \item \( p^{-1}(U) \) ist eine disjunkte Vereinigung offener Mengen \( V_\alpha \subseteq E \),
  \item für jede solche Menge \( V_\alpha \) ist \( p|_{V_\alpha} : V_\alpha \to U \) ein Homöomorphismus.
\end{itemize}

\textbf{Bezeichnungen:}
\begin{itemize}
  \item \( E \) heißt der \underline{Überlagerungsraum},
  \item \( p \) heißt die \underline{Überlagerungsabbildung},
  \item die Mengen \( V_\alpha \) heißen \underline{Blätter},
  \item die (diskrete) Teilmenge \( p^{-1}(x) \) heißt die \underline{Faser über \( x \)}.
\end{itemize}

  \begin{figure}[H]
    \centering
    \includegraphics[width=9cm]{Image Diffgeo/18.01.jpg}
	%\caption{Ebene mit Loch und die Zylinder Oberfläche sind nicht einfach zusammenhängend}
 \end{figure}

\textbf{Bemerkung:}
\begin{itemize}
  \item Sei \( p : E \to B \) eine Überlagerungsabbildung. Für alle \( b \in B \) trägt der Teilraum \( p^{-1}(b) \subseteq E\) die \textbf{diskrete Topologie}.\\
  Das bedeutet: Jedes Blatt \( V_\alpha \subseteq E \) ist offen und schneidet die Faser \( p^{-1}(b) \) nur in einem Punkt, daher ist diser Punkt offen in $p^{-1}(b)$.

  \item Die Überlagerungsabbildung \( p : E \to B \) ist eine \textbf{offene Abbildung}, d.h. das Urbild offener Mengen (z.B. die Blätter \( V_\alpha \)) sind ebenfalls offen.
\end{itemize}

\textbf{Beispiel: Triviale Überlagerung}

Sei \( X \) ein topologischer Raum.

\begin{itemize}
  \item Die Identität \( \mathrm{id}_X : X \to X \) ist eine Überlagerungsabbildung.

  \item Allgemeiner: Sei
  \[
  E = X \times \{1, \dots, n\}
  \]
  die disjunkte Vereinigung von \( n \) Kopien des Raumes \( X \).

  Dann ist die Abbildung
  \[
  p : E \to X, \quad (x, i) \mapsto x
  \]
  eine Überlagerung. Anschaulich stellen wir uns vor, dass der Raum \( X \) mehrfach „übereinandergelegt“ wird.
\end{itemize}
Um solche trivialen Beispiele zu vermeiden, verlangt man in vielen Anwendungen zusätzlich, dass der Überlagerungsraum \( E \) \textbf{wegzusammenhängend} ist.\\

\textbf{Satz 7.19}

Die Abbildung
\[
p : \mathbb{R} \to S^1, \quad x \mapsto (\cos(2\pi x), \sin(2\pi x))
\]
ist eine {Überlagerungsabbildung}.

\vspace{1em}
\underline{Anschaulich:} 
Wir „wickeln“ \( \mathbb{R} \) um den Einheitskreis. Dabei wird jedes Intervall \( [n, n+1] \) surjektiv auf \( S^1 \) abgebildet. Die Zahl \( n \in \mathbb{Z} \) zählt also die „Umdrehungen“.
\begin{proof}
    Wir benutzen die Eigenschaften von Sinus und Kosinus. \\
Betrachte eine offene Umgebung \( U = S^1 \cap \{x>0\} \). Dann ist das Urbild \( p^{-1}(U) \subseteq \mathbb{R} \) die disjunkte Vereinigung offener Intervalle

  \begin{figure}[H]
    \centering
    \includegraphics[width=11cm]{Image Diffgeo/18.02.jpg}
	%\caption{Ebene mit Loch und die Zylinder Oberfläche sind nicht einfach zusammenhängend}
 \end{figure}
Beschränke \( p \) auf das abgeschlossene Intervall \( \overline{V_n}:=[n-\frac{1}{4},n+\frac{1}{4}] \). \\
Dort ist \( p \) injektiv, da auf diesem Intervall die Funktion \( x \mapsto \sin(2\pi x) \) streng monoton ist. Weiterhin bildet \( p \) das Intervall \( \overline{V_n} \) bzw. $V_n$ surjektiv auf den Abschluss \( \overline{U} \) bzw-. $U$ ab (nach dem Zwischenwertsatz).\\

Da \( \overline{V_n} \) kompakt ist und \( \overline{U} \) Hausdorff, ist
\[
p|_{\overline{V_n}} : \overline{V_n} \to \overline{U}
\]
ein Homöomorphismus. Insbesondere ist die Einschränkung \( p|_{V_n} : V_n \to U \) ein Homöomorphismus zwischen $V_n$ und $U$.\\

Ein ähnliches Argument gilt für die offenen Mengen \( S^1 \cap \{x < 0\} \), \( S^1 \cap \{y > 0\} \), \( S^1 \cap \{y < 0\} \). Diese offenen Mengen überdecken \( S^1 \), und jedes ihrer Urbilder ist eine disjunkte Vereinigung von Intervallen \( V_n \), auf denen \( p \) ein Homöomorphismus ist.

Daher ist \( p : \mathbb{R} \to S^1 \) eine Überlagerungsabbildung.
\end{proof}
\vspace{1em}

Ist \( p : E \to B \) eine Überlagerungsabbildung, dann ist \( p \) ein \textbf{lokaler Homöomorphismus} von \( E \) nach \( B \), d.h. jeder Punkt \( e \in E \) besitzt eine Umgebung \( U \subseteq E \), sodass \( p|_U : U \to p(U) \subseteq B \) ein Homöomorphismus ist.\\

\textbf{Aber:} Die Umkehrung gilt nicht! Ein lokaler Homöomorphismus ist nicht notwendigerweise eine Überlagerung:

\vspace{1em}
\textbf{Beispiel 7.20:} \\
Betrachte die Abbildung
\[
p : \mathbb{R}_{> 0} \to S^1, \quad x \mapsto (\cos(2\pi x), \sin(2\pi x))
\]

Diese Abbildung ist:
\begin{itemize}
  \item surjektiv,
  \item lokal ein Homöomorphismus,
  \item aber \textbf{keine Überlagerung}.
\end{itemize}
  \begin{figure}[H]
    \centering
    \includegraphics[width=9cm]{Image Diffgeo/18.03.jpg}
	%\caption{Ebene mit Loch und die Zylinder Oberfläche sind nicht einfach zusammenhängend}
 \end{figure}
Betrachte den Punkt \( b_0 = (1, 0) \in S^1 \). Jede kleine Umgebung \( U \subseteq S^1 \) von \( b_0 \) sollte durch \( p \) überdeckt werden. \\
Das Urbild \( p^{-1}(U) \subseteq \mathbb{R}_{> 0} \) ist eine Vereinigung von Intervallen \( V_n \), wobei \( n \in \mathbb{N}_{>0} \) und
\[
V_n = (n - \varepsilon, n + \varepsilon)
\]
Diese Intervalle \( V_n \) werden jeweils homöomorph auf \( U \) abgebildet, aber für $V_0=(0,\varepsilon)$ ist die Einschränkung nicht mehr homöomorph, da $\mathbb{R}_{>0}$ keinen Punkt unterhalb von $n=0$ enthält, das Intervall nach links ist einfach abgeschnitten.

\vspace{1em}


\textbf{Beispiel}

Die reellen Zahlen \( \mathbb{R} \) sind nicht die einzige zusammenhängende Überlagerung von \( S^1 \). Man kann \( S^1 \) selbst als Teilmenge von \( \mathbb{C} \) betrachten, dann ist für jede natürliche Zahl \( n \in \mathbb{N} \) die Abbildung
\[
p : S^1 \to S^1, \quad z \mapsto z^n
\]
eine \underline{\( n \)-fache Überlagerungsabbildung}. (Intuitiv wird n-mal umgewickelt)\\

Im Beispiel 7.20 haben wir gesehen, dass die Einschränkung einer Überlagerungsabbildung nicht automatisch wieder eine Überlagerung ist. Aber:\\

\textbf{Satz 7.21}
Sei \( p : E \to B \) eine Überlagerungsabbildung. \\
Ist \( B_0 \subseteq B \) ein Teilraum und \( E_0 = p^{-1}(B_0) \), dann ist die Einschränkung
\[
p_0:=p|_{E_0} : E_0 \to B_0
\]
ebenfalls eine Überlagerungsabbildung.


\begin{proof}
Sei \( b_0 \in B_0 \). \\
Da \( p \) eine Überlagerung ist, gibt es eine offene Menge \( U \subseteq B \), sodass \( b_0 \in U \) und \( p^{-1}(U) = \bigsqcup_\alpha V_\alpha \), wobei jedes \( V_\alpha \subseteq E \) offen ist und \( p|_{V_\alpha} : V_\alpha \to U \) ein Homöomorphismus ist.\\

Setze \( U_0 := U \cap B_0 \), dann ist \( U_0 \subseteq B_0 \) offen bzgl. der Teilraumtopologie.\\

Die Mengen \( V_\alpha' := V_\alpha \cap E_0 \) sind offen in \( E_0 \), disjunkt, und
\[
p|_{V_\alpha'} : V_\alpha' \to U_0
\]
ist ein Homöomorphismus.

Also ist \( p|_{E_0} : E_0 \to B_0 \) eine Überlagerung.
\end{proof}

\textbf{Satz 7.22}

Sind \( p : E \to B \) und \( p' : E' \to B' \) Überlagerungsabbildungen, dann ist auch
\[
p \times p' : E \times E' \to B \times B'
\]
eine \textbf{Überlagerungsabbildung}.

\begin{proof}
    

Sei \( b \in B \), \( b' \in B' \) und \( U \subseteq B \), \( U' \subseteq B' \) offene Umgebungen von \( b \) bzw. \( b' \), die von \( p \) bzw. \( p' \) überlagert werden.

Seien \( \{ V_\alpha \} \) die Zerlegung von \( p^{-1}(U) \) in Blätter von \( p \), und \( \{ V_\beta' \} \) die Zerlegung von \( {p'}^{-1}(U') \) in Blätter von \( p' \).

Dann ist das Urbild der offenen Menge \( U \times U' \subseteq B \times B' \) unter \( p \times p' \) die Vereinigung der Mengen
\[
\{V_\alpha \times V_\beta'\}
\]
Diese sind offen, disjunkt, und bilden eine Überdeckung von \( (p \times p')^{-1}(U \times U') \). \\
Jede Einschränkung
\[
(p \times p)|_{V_\alpha \times V_\beta'} : U_\alpha \times V_\beta' \to U \times U'
\]
ist ein Homöomorphismus, da Produkte von Homöomorphismen Homöomorphismen sind.
\end{proof}

\textbf{Beispiel: Überlagerung des Torus durch die Ebene \(\mathbb{R}^2\)}

Betrachte den Torus \( T = S^1 \times S^1 \). Die Produktabbildung
\[
p \times p : \mathbb{R} \times \mathbb{R} \longrightarrow S^1 \times S^1
\]
ist eine {Überlagerung des Torus} durch die Ebene \( \mathbb{R}^2 \), wobei \( p \) die Abbildung aus Satz 7.19 ist, also
\[
p(x) = (\cos(2\pi x), \sin(2\pi x)).
\]

Jedes Einheitsquadrat $[n,n+1]\times [m.m+1]$ im Gitter \( \mathbb{R}^2 \) wird durch \( p \times p' \) surjektiv und lokal homöomorph auf den Torus abgebildet.
 \begin{figure}[H]
    \centering
    \includegraphics[width=11cm]{Image Diffgeo/18.04.png}
 \end{figure}

\textbf{Beispiel:} Betrachte die Abbildung \( p \times p \) aus dem vorherigen Beispiel. Sei \( b_0 = p(0) \in S^1 \) und
\[
B_0 := \{(x,y) \in S^1 \times S^1 \mid x = b_0 \text{ oder } y = b_0\} \subseteq S^1 \times S^1.
\]
Dann ist \( B_0 \) die Vereinigung von zwei Kreisen mit einem gemeinsamen Punkt – die sogenannte \textit{Figure-Eight}.
 \begin{figure}[H]
    \centering
    \includegraphics[width=4cm]{Image Diffgeo/18.05.png}
 \end{figure}

Die Menge \( E_0 = (p \times p)^{-1}(B_0) \subseteq \mathbb{R}^2 \) ist das sogenannte „unendliche Gitter“:
\[
E_0 = \{(x,y) \in \mathbb{R}^2 \mid x \in \mathbb{Z} \text{ oder } y \in \mathbb{Z} \}=(\mathbb{R}\times \mathbb{Z})\times(\mathbb{Z}\times \mathbb{R}).
\]
(vgl. Bild oben) Die Einschränkung \( p_0 = (p \times p)|_{E_0}:E_0\rightarrow B_0 \) ist eine {Überlagerungsabbildung}.

\vspace{1em}
\textbf{Beispiel:} Die Abbildung
\[
p \times \mathrm{id} : \mathbb{R} \times \mathbb{R}_{>0} \longrightarrow S^1 \times \mathbb{R}_{>0}
\]
ist eine {Überlagerung} nach Satz 7.22.

Verwenden wir nun den Homöomorphismus (Skalierung des Einheitsvektors x mit dem Faktor t)
\[
S^1 \times \mathbb{R}_{>0} \cong \mathbb{R}^2 \setminus \{0\}, \quad (x,t) \mapsto tx,
\]
dann ergibt die Verkettung eine Überlagerung
\[
\mathbb{R} \times \mathbb{R}_{>0} \longrightarrow \mathbb{R}^2 \setminus \{0\} \quad (x,t)\mapsto t(\cos(2\pi x), \sin(2\pi x)).
\]


\subsection{Die Fundamentalgruppe des Kreises}
Wir wollen den Zusammenhang zwischen der Fundamentalgruppe und (zusammenhängende) Überlagerungen untersuchen.\\

\textbf{Definition 7.23} \\
Sei \( p : E \to B \) eine Abbildung. Sei \( f : X \to B \) eine stetige Abbildung. Eine \underline{Anhebung} von \( f \) ist eine Abbildung \( \tilde{f} : X \to E \), sodass das $p\circ\tilde{f}=f$, also das folgende Diagramm kommutiert

\begin{figure}[H]
    \centering
    \includegraphics[width=4cm]{Image Diffgeo/18.97.jpeg}
 \end{figure}

Im Allgemeinen ist nicht klar, ob zu einer gegebenen Abbildung \( f : X \to B \) eine Anhebung existiert oder in wie weit diese eindeutig ist.

Wir wollen diese Fragen untersuchen, falls \( p \) eine Überlagerung ist.\\

\textbf{Beispiel:} Sei \( p : \mathbb{R} \to S^1 \) die Überlagerung definiert durch
\[
p(x) = (\cos(2\pi x), \sin(2\pi x)).
\]

Der Weg \( f : [0,1] \to S^1 \) startet bei $b_0=(1,0)$, gegeben durch
\[
f(s) = (\cos(\pi s), \sin(\pi s)),
\]
Eine Anhebung \( \tilde{f} \) von \( f \), die bei \( 0 \in \mathbb{R} \) startet und endend bei $\frac{1}{2}$, ist
\[
\tilde{f}(s) = \frac{s}{2}.
\]

Im Allgemeinen gilt:
\[
\tilde{f}_n(s) = \frac{s}{2} + n, \quad n \in \mathbb{Z},
\]
sind ebenfalls Anhebungen von \( f \), aber mit anderem Endpunkt.

Weitere Beispiele:
\begin{itemize}
    \item \( g(s) = (\cos(\pi s), -\sin(\pi s)) \) mit Anhebung \( \tilde{g}(s) = -\frac{s}{2} \)
    \item \( h(s) = (\cos(4\pi s), \sin(4\pi s)) \) mit Anhebung \( \tilde{h}(s) = 2s \)
\end{itemize}

\textbf{Bemerkung:} Die Identität \( \mathrm{id}_{S^1} : S^1 \to S^1 \) hebt \emph{nicht} zu einer Abbildung
\[
\tilde{\mathrm{id}} : S^1 \to \mathbb{R}
\]
an – \textcolor{orange}{denn es existiert keine stetige Rückanhebung des gesamten Kreises.}

\begin{figure}[H]
    \centering
    \includegraphics[width=7cm]{Image Diffgeo/18.98.png}
 \end{figure}

\textbf{Lemma 7.24}
Sei \( p : E \to B \) eine Überlagerung, sei \( p(e_0) = b_0 \). Jeder Weg \( f : [0,1] \to B \), der in \( b_0 \) beginnt, hat eine \emph{eindeutige Anhebung} \( \tilde{f} \) zu einem Weg in \( E \), der in \( e_0 \) beginnt. (Basispunkt fixiert)

\begin{proof}
Überdecke \( B \) durch offene Mengen \( U \), die durch \( p \) überdeckt werden. Zerlege \( [0,1] \) in \( 0 = s_0 < s_1 < \dots < s_n = 1 \) so, dass \( f([s_i, s_{i+1}]) \subseteq U \) für jede i in einer der Mengen U liegt. (Die Existenz einer solchen Zerlegung (schafft in endlichen Schritten) garantiert durch das \emph{Lebesgue-Zahl-Lemma} (Da [0,1] kompakt)).\\

Wir definieren die Anhebung \( \tilde{f} \) stückweise.

Zunächst definieren wir \( \tilde{f}(0) = e_0 \). Ist \( \tilde{f}(s) \) bereits für alle \( 0 \le s \le s_i \) definiert, so definieren wir \( \tilde{f} \) auf dem Intervall \( [s_i, s_{i+1}] \) wie folgt:

Die Menge \( f([s_i, s_{i+1}]) \) liegt in einer offenen Menge \( U \), die durch \( p \) überdeckt wird. Sei \( \{ V_\alpha \} \) eine Zerlegung von \( p^{-1}(U) \) in \emph{Blätter}, d.h. jede dieser Mengen \( V_\alpha \) wird durch \( p \) homöomorph auf \( U \) abgebildet. \\

Jetzt liegt \( \tilde{f}(s_i) \text{ in einer der Mengen } V_{\alpha} \) sagen wir $V_0$. Dann definieren wir \( \tilde{f}(s) \) für \( s \in [s_i, s_{i+1}] \) durch
\[
\tilde{f}(s) := (p|_{V_{0}})^{-1}(f(s)).
\]

Da \( p|_{V_0} : V_0 \to U \) ein Homöomorphismus ist, ist \( \tilde{f} \) stetig auf \( [s_i, s_{i+1}] \). \\

Fahren wir so fort, können wir \( \tilde{f} \) auf \( [0,1] \) definieren. Stetigkeit von \( \tilde{f} \) folgt, da \( \tilde{f} \) stetig auf $[s_i,s_{i+1}]$ und $[s_j,s_{j+1}]\cap[s_i,s_{i+1}]$ abgeschlossenen ist. Aus unserer Definition von \( \tilde{f} \) folgt, dass $p\circ\tilde{f}=f$ \\

 \begin{figure}[H]
    \centering
    \includegraphics[width=13cm]{Image Diffgeo/18.06.jpg}
 \end{figure}

Zur \emph{Eindeutigkeit} von \( \tilde{f} \): Annahme: \( \tilde{\tilde{f}} \) ist eine Anhebung von \( f \), die bei \( e_0 \) beginnt.

Also $\tilde{\tilde{f}}(0)=e_0=\tilde{f}(0)$: Wir nehmen an, dass \( \tilde{\tilde{f}}(s)=\tilde{f}(s) \) für alle \( 0 \le s \le s_i \) .

Sei \( V_0 \) wie oben, dann ist \( \tilde{f} \) für \( s \in [s_i, s_{i+1}] \) durch
\[
\tilde{f}(s) := (p|_{V_0})^{-1}(f(s))
\]
gegeben.

Was ist \( \tilde{\tilde{f}}(s) \)? Da \( \tilde{f} \) eine Anhebung von \( f \) ist, muss es das Intervall \( [s_i, s_{i+1}] \) nach \( p^{-1}(U)\cup V_\alpha\) abbilden. 

Die Blätter \( V_\alpha \) sind offen und disjunkt. Da \( \tilde{\tilde{f}}([s_i, s_{i+1}]) \) zusammenhängend ist (Bild von stetiger Abbildung unter zusammenhängendem Urbild), muss es vollständig in $V_0$ liegen.
\[
\Rightarrow \text{Für } s \in [s_i, s_{i+1}] \text{ gilt also: } \tilde{\tilde{f}}(s) =y \text{ für ein }y\in V_0 \cap p^{-1}(f(s))
\]

Aber es gibt nur ein solcher Punkt, nämlich $(p|_{V_0})^{-1}(f(s))$:
\[
\Rightarrow \tilde{\tilde{f}}(s) = \tilde{f}(s) \quad \text{für alle } s \in [s_i, s_{i+1}]
\]

\end{proof}

\textbf{Lemma 7.25}
Sei \( p : E \to B \) eine Überlagerung und \( p(e_0) = b_0 \). Sei 
\[ 
F : [0,1] \times [0,1] \to B 
\]
eine stetige Abbildung mit \( F(0,0) = b_0 \). Dann gibt es eine eindeutige Anhebung von F zueiner stetigen Abbildung
\[ 
\widetilde{F} : [0,1] \times [0,1] \to E 
\]
sodass \( \widetilde{F}(0,0) = e_0 \). Ist \( F \) eine Homotopie von Wegen, dann auch \( \widetilde{F} \).
\begin{proof}
    Analog mit 7.24
\end{proof}

\vspace{1em}

\textbf{Satz 7.26}
Sei \( p : E \to B \) eine Überlagerung, sei \( p(e_0) = b_0 \). 

Seien \( f \) und \( g \) zwei Wege in \( B \) von \( b_0 \) nach \( b_1 \). Weiter seien \( \widetilde{f} \) und \( \widetilde{g} \) ihre Anhebungen zu Wegen in \( E \), die in \( e_0 \) starten. 

Sind \( f \) und \( g \) weghomotop, so enden \( \widetilde{f} \) und \( \widetilde{g} \) am selben Punkt von \( E \) und sind weghomotop.

 \begin{figure}[H]
    \centering
    \includegraphics[width=13cm]{Image Diffgeo/18.07.jpg}
 \end{figure}
 
\begin{proof}
Sei \( \mathcal{F} : [0,1] \times [0,1] \to B \) eine Weghomotopie zwischen \( f \) und \( g \). Dann gilt \( \mathcal{F}(0,0) = b_0 \).

Sei \( \widetilde{\mathcal{F}} : [0,1] \times [0,1] \to E \) die Anhebung von \( \mathcal{F} \) zu \( E \) mit \( \widetilde{\mathcal{F}}(0,0) = e_0 \). 

Aufgrund des vorherigen Lemmas ist \( \widetilde{\mathcal{F}} \) eine Weghomotopie, sodass 
\[
\widetilde{\mathcal{F}}(0 \times [0,1]) = {e_0} \quad \text{und} \quad \widetilde{\mathcal{F}}(1 \times [0,1]) = {e_1}.
\]

Die Einschränkung von \( \widetilde{\mathcal{F}} \) auf \( [0,1] \times \{0\} \) ist ein Weg in \( E \), der in \( e_0 \) beginnt und $F|_{[0,1]\text{ x } {0}}$ anhebt,aufgrund der Eindeutigkeit von Weganhebungen folgt 
\[
\widetilde{\mathcal{F}}(s,0) = \widetilde{f}(s).
\]

Ähnlich folgt
\[
\widetilde{\mathcal{F}}(s,1) = \widetilde{g}(s).
\]

Daher enden \( \widetilde{f} \) und \( \widetilde{g} \) im selben Punkt $e_1$, und \( \widetilde{\mathcal{F}} \) ist eine Weghomotopie.
\end{proof}
\vspace{1em}

\textbf{Definition 7.26 1/2}\\
Sei \( p : E \to B \) eine Überlagerung, \( b_0 \in B \). Wähle \( e_0 \in E \) mit \( p(e_0) = b_0 \). Gegeben \( [f] \in \pi_1(B, b_0) \), sei \( \widetilde{f} \) eine Anhebung von \( f \) zu einem Weg in \( E \), der in \( e_0 \) beginnt. Bezeichne den Endpunkt \( \widetilde{f}(1) \) von \( \widetilde{f} \) mit \( \phi([f]) \). Dann ist \( \phi \) eine wohldefinierte Abbildung zwischen den Mengen
\[
\phi : \pi_1(B, b_0) \longrightarrow p^{-1}(b_0).
\]

Wir nennen \( \phi \) die \underline{Anhebungskorrespondenz} abgebildet von der Überlagerungsabbildung \( p \). Diese hängt natürlich von der Wahl von \( e_0 \) ab.
 \begin{figure}[H]
    \centering
    \includegraphics[width=11cm]{Image Diffgeo/18.08.jpg}
 \end{figure}


\textbf{Theorem 7.27}
Sei \( p : E \to B \) eine Überlagerung, sei \( p(e_0) = b_0 \). Ist \( E \) wegzusammenhängend, dann ist die Anhebungskorrespondenz
\[
\phi : \pi_1(B, b_0) \longrightarrow p^{-1}(b_0)
\]
surjektiv. Ist \( E \) zusätzlich einfach zusammenhängend, dann ist die Abbildung bijektiv.

\begin{proof}
Sei \( E \) wegzusammenhängend. Gegeben \( e_1 \in p^{-1}(b_0) \), dann gibt es einen Weg \( \widetilde{f} \) in \( E \) von \( e_0 \) nach \( e_1 \). Dann ist \( f = p \circ \widetilde{f} \) ein Schlaufe in \( B \) mit Basispunkt \( b_0 \) und \( \phi([f]) = e_1 \) per Definition.\\

Sei \( E \) zusätzlich einfach zusammenhängend. Seien \( [f], [g] \) zwei Elemente von \( \pi_1(B, b_0) \) mit \textcolor{orange}{\( \phi([f]) = \phi([g]) \)} . Dann sind die Anhebungen \( \widetilde{f}, \widetilde{g} \) von \( f \) bzw. \( g \) Wege in \( E \), die in \( e_0 \) beginnen, dann gilt \(\tilde{f}(1)=\tilde{g}(1)\). Da \( E \) einfach zusammenhängend ist, gibt es eine Weg-Homotopie \( \widetilde{F} \) zwischen \( \widetilde{f} \) und \( \widetilde{g} \). Dann ist \( H = p \circ \widetilde{F} \) eine Weg-Homotopie in \( B \) zwischen \( f \) und \( g \), also \( [f] = [g] \).
\end{proof}

\textbf{Satz 7.28}
Die Fundamentalgruppe von \( S^1 \) ist isomorph zu \( (\mathbb{Z}, +) \).


\begin{proof}
Sei \( p : \mathbb{R} \to S^1 \) die Überlagerung \( p(x) = (\cos(2\pi x), \sin(2\pi x)) \), mit \( e_0 = 0 \in \mathbb{R} \) und \( b_0 = p(e_0) \). Dann gilt \( p^{-1}(b_0) = \mathbb{Z} \). Da \( \mathbb{R} \) einfach zusammenhängend ist, ist die Abbildung (Anhebungskorrespondenz)
\[
\phi : \pi_1(S^1, b_0) \longrightarrow \mathbb{Z} = p^{-1}(b_0)
\]
bijektiv nach 7.27.\\

Wir zeigen, dass \( \phi \) ein Homomorphismus ist. Seien \( [f], [g] \in \pi_1(S^1, b_0) \), und seien \( \widetilde{f} \) und \( \widetilde{g} \) die zugehörigen Anhebungen, die in \( 0 \in \mathbb{R} \) beginnen. Sei
\[
n=\tilde{f}(1), \qquad m=\tilde{g}(1),\quad \text{also } \phi([f])=n, \qquad \phi([g])=m
\]

Sei nun \( \tilde{\widetilde{g}} \) definiert durch
\[
\tilde{\widetilde{g}}(s) = n+\tilde{g}(s)
\]

Da \( p(n + x) = p(x) \) für alle \( x \in \mathbb{R} \), ist \( \tilde{\widetilde{g}} \) eine Anhebung von \( g \), die in \( n \) beginnt. Damit ist das Produkt
\(
\phi([\tilde{f} * \tilde{\widetilde{g}}]) 
\) definiert und eine Anhebung von $f*g$ (n+m mal rumgelaufene Kurve), die in 0 beginnt.



Der Endpunkt ist \( \tilde{\widetilde{g}}(1) = n+m \). Aus der Definition folgt also:
\[
\phi([f] \ast [g]) = n+m =\phi([f])+\phi([g])
\]

Also ist \( \phi \) ein Gruppenhomomorphismus. Da \( \phi \) bijektiv ist, folgt daraus, dass
\[
\pi_1(S^1, b_0) \cong \mathbb{Z}.
\]

\end{proof}

\textbf{Satz 7.29}
Sei \( p : E \to B \) eine Überlagerung und \( p(e_0) = b_0 \).

\begin{enumerate}
  \item Der Homomorphismus
  \[
  p_* : \pi_1(E, e_0) \longrightarrow \pi_1(B, b_0)
  \]
  ist injektiv.

  \item Sei \( H = p_*(\pi_1(E, e_0)) \). Die \emph{Anhebungskorrespondenz} \( \phi \) induziert eine injektive Abbildung
  \[
  \Phi : \pi_1(B, b_0)/H \longrightarrow p^{-1}(b_0),
  \]
  wobei \( \pi_1(B, b_0)/H \) die Menge der Rechtsnebenklassen von \( H \) bezeichnet.

  Die Abbildung ist bijektiv, falls \( E \) wegzusammenhängend ist.

  \item Ist \( [f] \in \pi_1(B, b_0) \), so gilt
  \[
  [f] \in H \quad \Leftrightarrow \quad f \text{ hebt sich zu einer Schlaufe mit Basispunkt } e_0 \text{ in } E \text{ an}.
  \]
\end{enumerate}

%%%%%%%%%%%%%%%%%%%%%%%%%%%%%%%%%%%%%%%%%%%%%%%%%%%%%%%%%%%%%%%%%%%%%%%%%%%%%%%%%%%%%%% Vorlesung 19 %%%%%%%%%%%%%%%%%%%%%%%%%

\subsection{Äquivalenz von Überlagerungen}

\underline{Vorbereitung:} Sei \( p : E \to B \) eine Überlagerung. \\

Wir nehmen an, dass \( B \) lokal wegzusammenhängend ist. Zerlegen wir \( B \) in die wegzusammenhängenden Komponenten \( B_\alpha \), so ist die Einschränkung von \( p \) auf \( p^{-1}(B_\alpha) \to B_\alpha \) ebenfalls eine Überlagerung (siehe {Satz 7.21}). Daher beschränken wir uns auf wegzusammenhängend B.\\

Ist \( E_\alpha \) eine Weg-Zusammenhangskomponente von \( E \), dann ist die Einschränkung von \( p \) auf \( E_\alpha \to B \) eine Überlagerung (siehe unten Lemma 7.37).\\

Daher sind alle Überlagerungen des lokal wegzusammenhängenden Raumes \( B \) bestimmt durch die wegzusammenhängenden Überlagerungen der weg-zusammenhangs- komponenten von \( B \).

\vspace{0.5em}
\underline{Konvention:} Für die folgenden Kapitel nehmen wir für eine Überlagerung \( p : E \to B \) an, dass \( E \) und \( B \) 
\begin{itemize}
    \item wegzusammenhängend
    \item lokal wegzusammenhängend
\end{itemize}
sind (wenn nichts anderes gesagt wird).\\
  \begin{figure}[H]
    \centering
    \includegraphics[width=9cm]{Image Diffgeo/19.07.png}
	\caption{Topologischer Kamm: wegzusammenhängend aber nicht lokal-wegzusammenhängend}
 \end{figure}

\underline{Erinnerung:} Ist \( p : E \to B \) eine Überlagerung mit \( p(e_0) = b_0 \), dann ist der induzierte Homomorphismus
\[
p_* : \pi_1(E, e_0) \to \pi_1(B, b_0)
\]

injektiv. Setze \( H_0 := \operatorname{im}(p_*) = p_*\big( \pi_1(E, e_0) \big) \). \( H_0 \) ist eine Untergruppe von \( \pi_1(B, b_0) \), welche isomorph zu \( \pi_1(E, e_0) \) ist.

\vspace{1em}
\underline{Ziel:} \( H_0 \) bestimmt die Überlagerung \( p : E \to B \) bis auf Äquivalenz.

\vspace{0.5em}
Für sinnvoll gewählte Räume \( B \) bestimmt jede Untergruppe von \( \pi_1(B, b_0) \) eine Überlagerung. \emph{(morgen)}

\vspace{1em}
\underline{Idee:} Das ist das Prinzip der algebraischen Topologie: Übersetze ein topologisches Problem in ein algebraisches.

\vspace{1em}
Wir werden \underline{Satz 7.29} benutzen.\\

\textbf{Definition 7.30:} 
Seien \( p : E \to B \) und \( p' : E' \to B \) Überlagerungen. \\
Wir sagen, dass sie \emph{\underline{äquivalent}} sind, falls es einen Homöomorphismus 
\[
h : E \to E'
\]
mit \( p' \circ h = p \) gibt. Der Homöomorphismus heißt eine \underline{Äquivalenz von Überlagerungsabbildungen} oder ein \underline{Äquivalenz von Überlagerungsräumen}.
\[
\begin{array}{ccc}
E & \xrightarrow{h} & E' \\
\searrow p & & p' \swarrow \\
& B &
\end{array}
\]
\vspace{1em}
Seien \( p : E \to B \) und \( p' : E' \to B \) Überlagerungen, so dass
\[
H_0 = p_* \left( \pi_1(E, e_0) \right) = p'_* \left( \pi_1(E', e_0') \right) =: H_0' .
\]

\underline{Ziel:} Zeige: Die Überlagerungen sind äquivalent. Dafür brauchen wir\\

\textbf{Lemma 7.31 (Das allgemeine Anhebungsschema)} \\
Sei \( p : E \to B \) eine Überlagerung und \( p(e_0) = b_0 \). Sei \( f : Y \to B \) eine stetige Abbildung mit \( f(y_0) = b_0 \). Zusätzlich sei \( Y \) wegzusammenhängend und lokal wegzusammenhängend. Dann besitzt die Abbildung \( f \) genau dann eine Anhebung 
\[
\tilde{f} : Y \to E \quad \text{mit} \quad \tilde{f}(y_0) = e_0,
\]
wenn die induzierte Abbildung \( f_* \left( \pi_1(Y, y_0) \right) \subseteq p_* \left( \pi_1(E, e_0) \right) \) gilt.

Diese Anhebung ist dabei eindeutig (falls es existiert).

\begin{proof}
\underline{\( \Rightarrow \)}: Wenn eine Anhebung existiert, dann gilt $f=p\circ \tilde{f}$.\\
\[
f_* \left( \pi_1(Y, y_0) \right) 
= p_* \left( \widetilde{f}_* \left( \pi_1(Y, y_0) \right) \right) 
\subseteq p_* \left( \pi_1(E, e_0) \right)
\]

\underline{Zusatz (Eindeutigkeit):} Sei \( y_1 \in Y \), wähle einen Weg \( \alpha \) in \( Y \) von \( y_0 \) nach \( y_1 \). \\
Bilde den Weg \( f \circ \alpha \) in \( B \) und hebe diesen zu einem Weg \( \gamma \) in \( E \), der in \( e_0 \) beginnt. \\
Falls es eine Anhebung \( \tilde{f} : Y \to E \) von \( f \) gibt, dann muss
\[
\tilde{f}(y_1) = \gamma(1)
\]
gelten, da \( \tilde{f} \circ \alpha \) ein Anhebung von \( f \circ \alpha \) ist, das in \( e_0 \) beginnt. \\
Solche Weganhebungen sind eindeutig.

\vspace{1em}
\underline{\( \Leftarrow \)}: Der Eindeutigkeitsbeweis gibt uns eine Idee, was zu tun ist.\\

Sei \( y_1 \in Y \), wähle einen Weg \( \alpha \) in \( Y \) von \( y_0 \) nach \( y_1 \). Bilde den Weg \( f \circ \alpha \) und hebe diesen zu einem Weg $\gamma$ in \( E \), der in \( e_0 \) beginnt. Definiere dann:
\[
\tilde{f}(y_1) :=\gamma(1),
\]

  \begin{figure}[H]
    \centering
    \includegraphics[width=13cm]{Image Diffgeo/19.01.jpg}
	%\caption{Ebene mit Loch und die Zylinder Oberfläche sind nicht einfach zusammenhängend}
 \end{figure}

Die Wohldefiniertheit von \( \tilde{f} \), also die Unabhängigkeit von der Wahl des Weges \( \alpha \), ist etwas Arbeit. Sobald dies bewiesen ist, können wir die Stetigkeit von \( \tilde{f} \) herleiten.\\

\underline{Stetigkeit:} In \( y_1 \in Y \). Wir zeigen: Gegeben eine Umgebung \( N \) von \( \tilde{f}(y_1) \), dann gibt es eine Umgebung \( W \) von \( y_1 \) derart, dass \( \tilde{f}(W) \subseteq N \).\\

Wir schaffen dies, indem wir eine wegzusammenhängende Umgebung \( U \) von \( f(y_0) \) wählen, die von \( p \) überlagert wird.\\

Zerlege \( p^{-1}(U) \) in Blätter und sei \( V_0 \) das Blatt, das \( \tilde{f}(y_1) \) enthält. Durch Ersetzen von \( U \) durch eine kleinere Umgebung von \( f(y_1) \), können wir $V_0\subseteq N$ annehmen. Sei $p_0: V_0\to U$ die Einschränkung von p, dann ist $p_0$ ein Homöomorphismus. \\

Weil \( f \) stetig in $y_1$ ist und \( Y \) lokal wegzusammenhängend, gibt es eine wegzusammenhängende Umgebung \( W \) von \( y_1 \), so dass \( f(W) \subseteq U \). Wir zeigen jetzt $\tilde{f}(W)\subseteq V_0$, was unsere Behauptung zeigt.\\

Gegeben \( y \in W \), wähle einen Weg \( \beta \) in \( W \) von \( y_1 \) nach \( y \). Da \( \tilde{f} \) wohldefiniert ist, erhalten wir \( \tilde{f}(y) \), indem wir den Weg $\alpha*\beta$ von $y_0$ nach y betrachten, den Weg $f\circ(\alpha*\beta)$ zu einem Weg in E anheben und $\tilde{f}(y)$ als den Endpunkt des angehebenen Weges definieren\\

Sei \( \gamma \) eine Anhebung die in \( e_0 \) beginnt. \\
Da der Weg \( f\circ\gamma \) vollständig in \( U \) liegt ist der Weg $\delta=p_0^{-1}\circ f\circ \beta$ eine Anhebung die in $\tilde{f}(y_1)$ beginnt. Dann ist \( \gamma*\delta \) eine Anhebung von \( f \circ (\alpha*\beta) \), die in \( e_0 \) beginnt, und endet in $\delta(1)\in V_0$ 

$\Rightarrow \quad \tilde{f}(W)\subseteq V_0 $.\\

  \begin{figure}[H]
    \centering
    \includegraphics[width=12cm]{Image Diffgeo/19.08.jpg}
	\caption{verhinderte Situation: nach Anhebung enden zwei Kurven nicht in demselben Punkt}
 \end{figure}

\underline{Wohldefiniertheit:} 
Seien \( \alpha \) und \( \beta \) zwei Wege in \( Y \) von \( y_0 \) nach \( y_1 \). \\
Wir müssen zeigen: Heben wir die Wege \( f \circ \alpha \) und \( f \circ \beta \) zu Wegen in \( E \) an, die in \( e_0 \) beginnen, dann enden die Anhebungen im selben Punkt.\\

Hebe $f\circ\alpha$ zu einem Weg $\gamma$ in E and der in $e_0$ beginnt, dann haben wir $f\circ\beta^-$ zu einem Weg $\delta$ an der in $\gamma(1)$ beginnt.\\
Dann ist \( \gamma*\delta \) eine Anhebung der Schlaufe $f\circ(\alpha*\beta^-)$.

  \begin{figure}[H]
    \centering
    \includegraphics[width=9cm]{Image Diffgeo/19.09.jpg}
	%\caption{verhinderte Situation: nach Anhebung enden zwei Kurven nicht in demselben Punkt}
 \end{figure}

Laut Annahme gilt:
\[
f_*\left(\pi_1(Y, y_0)\right) \subseteq p_*\left(\pi_1(E, e_0)\right).
\]
\[
\Rightarrow[f\circ(\alpha*\beta^-)]\in im(p_*)
\]
Nach {Satz 7.29} ist die Anhebung \( \gamma*\delta \) eine Schleife in \( E \). \\
\( \Rightarrow \tilde{f} \) ist wohldefiniert, denn \( \delta^-\) ist eine Anhebung von $f\circ\beta$ die in \( e_0 \) beginnt, und $\gamma$ ist eine Anhebung von $f\circ\alpha$ die in $e_0$ beginnt und beide Anhebung endet in selbe Punkt in E.
\end{proof}

\textbf{Satz 7.32:} \\
Seien \( p : E \to B \) und \( p' : E' \to B \) Überlagerungen mit $p(e_0)=b_0=p'(e_0')$. Dann gibt es eine Äquivalenz \( h : E \to E' \) mit \(h(e_0)=e_0' \) genau dann, wenn die Gruppen
\[
H_0 := p_* \left( \pi_1(E, e_0) \right), \quad H_0' := p'_* \left( \pi_1(E', e_0') \right).
\] 
gleich sind.

Falls \( h \) existiert, ist es eindeutig.

\begin{proof}
(\( \Rightarrow \)): Sei \( h \) gegeben. Da \( h \) ein Homöomorphismus ist, gilt:
\[
h_*(\pi_1(E,e_0))=\pi_1(E',e_0')  \quad (*)
\]
Da \( p = p' \circ h \), folgt
\[
H_0 = p_* \left( \pi_1(E, e_0) \right) 
= (p' \circ h)_* \left( \pi_1(E, e_0) \right) 
\overset{(*)}{=} {p'}_* \left( \pi_1(E', e_0') \right) 
= H_0'
\]

(\( \Leftarrow \)): Sei \( H_0 = H_0' \), wir wollen die Existenz von \( h \) zeigen. Dafür verwenden wir das vorherige Lemma (viermal).\\

Betrachte die Abbildungen
\[
\begin{array}{ccc}
E &  & E' \\
\searrow p & & p' \swarrow \\
& B &
\end{array}
\]
Da \( p'  \) eine Überlagerung ist und \( E \) wegzusammenhängend und lokal wegzusammenhängend ist, gibt es eine Abbildung
\[
h : E \to E'
\quad \text{mit} \quad h(e_0)=e_0',
\]
welche p anhebt (d.h. \( p = p' \circ h \)).

Durch Vertauschen der Rollen von \( E \) und \( E' \) erhalten wir analog eine Abbildung
\[
k : E' \to E
\quad \text{mit} \quad k(e_0')=e_0 \quad \text{s.d.} \quad p \circ k= p'.
\]

Betrachte nun die Abbildungen
\[
\begin{array}{ccc}
E &  & E \\
\searrow p & & p \swarrow \\
& B &
\end{array}
\]

Dann ist \( k \circ h : E \to E \) eine Anhebung von \( p \) mit Anfangspunkt \(k\circ h(e_0)= e_0 \), denn
\[
p\circ(k\circ h)=p'\circ h = p
\]
Da aber die Identität \( \operatorname{id}_E \) ebenfalls eine Anhebung von \( p \) mit Anfangspunkt \( e_0 \) ist und solche Anhebungen eindeutig sind, folgt
\[
k\circ h= \operatorname{id}_E.
\]

Wiedehole das Argument für $h\circ k$ und $id_{E'}$ um den Beweis abzuschließen (h ist das gesuchte Homöomorphismus).


\end{proof}

Wir haben die Frage nach der Äquivalenz schön gelöst. Aber wir haben einen subtilen Punkt nicht diskutiert.\\

{Satz 7.32} gibt eine notwendige und hinreichende Bedingung für die Existenz einer Äquivalenz \( h : E \to E' \), welche \( e_0 \) auf \( e_0' \) abbildet. Aber wir haben noch keine Bedingung für die generelle Existenz einer Äquivalenz gefunden. Es könnte sein, dass es keine Äquivalenz gibt, die \( e_0 \) auf \( e_0' \) abbildet, wohl aber eine Äquivalenz, die \( e_0 \) auf einen anderen Punkt \( e_1' \in (p')^{-1}(b_0) \) abbildet.\\

Können wir diese Frage vielleicht auf \( H_0 \) und \( H_0' \) zurückführen?

\vspace{1em}
\textbf{Erinnerung:} \\
Sind \( H_1, H_2 \) Untergruppen von \( G \), die Gruppen heißen \emph{konjugiert}, falls es ein \( \alpha \in G \) mit
\[
H_2 = \alpha H_1 \alpha^{-1}
\]
gibt. Mit andere Worten: Sie sind konjugiert, falls der Gruppenautomorphismus
\[
x \mapsto \alpha x\alpha^{-1}
\]
die Gruppe \( H_1 \) auf \( H_2 \) abbildet.

Konjugation ist eine Äquivalenzrelation auf den Untergruppen von \( G \). \\
Die Äquivalenzklasse der Untergruppe \( H \) heißt \textbf{\underline{Konjugationsklasse}} von \( H \).\\

\textbf{Lemma 7.33:} \\
Sei \( p : E \to B \) eine Überlagerung. Seien \( e_0, e_1 \in p^{-1}(b_0) \) und
\(
H_i := p_*\left( \pi_1(E, e_i) \right).
\)

\begin{enumerate}
    \item[(i)] Ist \( \gamma \) ein Weg in \( E \) von \( e_0 \) nach \( e_1 \), und sei \( \alpha := p \circ \gamma \) die Schlaufe in \( B \), dann gilt die Gleichung:
    \[
    H_0 = [\alpha]* H_1* [\alpha]^{-1},
    \]
    also sind \( H_0 \) und \( H_1 \) konjugiert in \( \pi_1(B, b_0) \).
    
    \item[(ii)] Umgekehrt: Seien \( e_0 \in E \) und \( H \subseteq \pi_1(B, b_0) \) eine Untergruppe, die konjugiert zu \( H_0  \) ist, also
    Dann gibt es einen Punkt \( e_1 \in p^{-1}(b_0) \), sodass \( H_1=H \).
\end{enumerate}
  \begin{figure}[H]
    \centering
    \includegraphics[width=12cm]{Image Diffgeo/19.11.jpg}
	%\caption{Ebene mit Loch und die Zylinder Oberfläche sind nicht einfach zusammenhängend}
 \end{figure}

\begin{proof}
\underline{Zu (i):} Zuerst zeigen wir 
\[
[\alpha] * H_1 * [\alpha]^{-1} \subseteq H_0.
\]

Gegeben \([h] \in H_1\), es gilt 
\[
[h] = p_*([\widetilde{h}])
\]
für eine Schlaufe \(\widetilde{h}\) in \(E\) mit Basispunkt \(e_1\).

Sei \(\tilde{k} = (\gamma * \widetilde{h}) * \gamma^-\), dies ist eine Schlaufe mit Basispunkt \(e_0\), und

\begin{align*}
p_*([\tilde{k}]) 
&= p_*\left([(\gamma * \widetilde{h}) * \gamma^-]\right) \\
&= \left[(p \circ \gamma) * (p \circ \widetilde{h}) * (p \circ \gamma^-)\right] \\
&= [\alpha * h * \alpha^{-1}] \\
&= [\alpha] * [h] * [\alpha]^{-1}.
\end{align*}

Zeige nun $H_0\subseteq[\alpha]*H_1*[\alpha]^{-1}$:

Beobachte:  \( \gamma^{-} \) ein Weg in \( E \) von \( e_1 \) nach \( e_0 \), und
\[
\alpha^{-} := p \circ \gamma^{-}
\]

Aus dem vorherigen Argument folgt:
\[
[\alpha^-]*H_0*[\alpha]\subseteq H_1. \implies H_0\subseteq[\alpha]*H_1*[\alpha]^{-1}
\]
$\Rightarrow$ {Behauptung} (i) ist gezeigt. 

\vspace{1em}
\underline{Zu (ii):} Um die Umkehrung zu beweisen, seien \( e_0 \in E \) und \( H \) konjugiert zu \( H_0 \) gegeben. \\
Dann gilt:
\[
H_0 = [\alpha]* H* [\alpha]^{-1}
\]
für eine Schleife \( \alpha \) in \( B \) mit Basispunkt \( b_0 \).

Sei \( \gamma \) eine Anhebung von \( \alpha \) zu einem Weg in \( E \), der in \( e_0 \) beginnt. \\
Setze \( e_1 := \gamma(1) \). \\
a) impliziert:
\[
 \quad H_0=[\alpha]* H_1* [\alpha]^{-1}
\]
$\Rightarrow H=H_1$ 

\end{proof}

\textbf{Satz 7.34:} \\
Seien \( p : E \to B \) und \( p' : E' \to B \) Überlagerungen mit \( p(e_0) = b_0 = p'(e_0') \).

Die Überlagerungen \( p \) und \( p' \) sind genau dann äquivalent, wenn die Untergruppen
\[
H_0 := p_* \left( \pi_1(E, e_0) \right) \quad \text{und} \quad H_0' := p'_* \left( \pi_1(E', e_0') \right)
\]
von \( \pi_1(B, b_0) \) konjugiert sind.

\begin{proof}
Sei \( h : E \to E' \) eine Äquivalenz mit \( e_1'=h(e_0) \quad \text{und} \quad H_1'= p_*'(\pi_1(E',e_1')) \). 
\[
\overset{7.32}{\Rightarrow} H_0 = H_1'
\]
und nach dem vorherigen Lemma ist \( H_1' \) zu \( H_0' \) konjugiert.\\

Umgekehrt: Seien \( H_0 \) und \( H_0' \) konjugiert. Dann besagt {Lemma 7.33}, dass es ein \( e_1' \in E' \) gibt, sodass
\[
H_1' = H_0.
\]

Nach {Satz 7.32} existiert dann eine Äquivalenz \( h : E \to E' \) mit \( h(e_0) = e_1' \).
\end{proof}

\textbf{Beispiel 7.35:} \\
Sei \( B = S^1 \). Dann ist \( \pi_1(S^1, b_0) \cong \mathbb{Z} \) ist abelsch. Deswegen sind zwei Untergruppen von \( \pi_1(S^1, b_0) \) konjugiert genau dann, wenn sie gleich sind. 
\[H'=[\alpha]*H*[\alpha^{-}]=H*[\alpha]*[\alpha^{-}]=H\]
\(\Rightarrow\) Zwei Überlagerungen von \( S^1 \) sind genau dann äquivalent, wenn sie zur selben Untergruppe gehören.\\

Wir haben gesehen: $\pi_1(S^1,b_0)\cong(\mathbb{Z},+)$

Die Untergruppen von \( (\mathbb{Z},+) \) sind die Gruppen $G_n:= \{nz|z\in\mathbb{Z}\},n\in\mathbb{Z}_{\geq0}$

Die Überlagerung \( p : \mathbb{R} \to S^1 \), \( t \mapsto e^{2\pi i t} \), gehört zur trivialen Untergruppe \( \{0\} \), da \( \mathbb{R} \) einfach zusammenhängend ist.

Für die Überlagerung \( p : S^1 \to S^1 \), \( z \mapsto z^n \), wobei \( n \in \mathbb{Z}_{\geq 1} \), bildet \( p_* \) einen Erzeuger von \( \pi_1(S^1, b_0)  \) auf \( n \) mal diesen Erzeuger ab, also:
\[
(p_n)_*\left( \pi_1(S^1, b_0) \right) = G_n.
\]

Aus dem vorherigen Satz folgt: Jede wegzusamenhängende Überlagerung zu einer von diesen äquivalent ist.

\subsection{Die universelle Überlagerung}

\textbf{Definition 7.36:} \\
Sei \( p : E \to B \) eine Überlagerung mit \( p(e_0) = b_0 \). Ist \( E \) einfach zusammenhängend, dann heißt \( E \) eine \emph{\underline{universelle Überlagerung}} von \( B \).\\

Da \( \pi_1(E, e_0) \) trivial ist, korrespondiert diese Überlagerung zur trivialen Untergruppe von \( \pi_1(B, b_0) \).\\

Laut {Satz 7.34} sind zwei universelle Überlagerungen von \( B \) äquivalent. Deshalb sprechen wir von \emph{der} universellen Überlagerung.\\

Nicht jeder topologische Raum \( B \) besitzt eine universelle Überlagerung.\\

Für den Moment nehmen wir an, dass \( B \) eine universelle Überlagerung besitzt, und wollen einige Konsequenzen daraus sehen.\\

\textbf{Lemma 7.37:} \\
Sei \( B \) wegzusammenhängend und lokal wegzusammenhängend. Sei \( p : E \to B \) eine Überlagerung (wobei \( E \) nicht notwendigerweise wegzusammenhängend ist). Ist \( E_0 \) eine Wegzusammenhangskomponente von \( E \), dann ist die Einschränkung
\[
p_0 : E_0 \to B
\]
von p eine Überlagerung.

\begin{proof}
Zuerst zeigen wir, dass \( p_0 \) surjektiv ist.

Da \( E \) lokal Homöomorph zu \( B \) ist und \( E \) lokal wegzusammenhängend ist, folgt:
\[\Rightarrow E_0 \;\txt{offen in } E \quad \Rightarrow p(E_0)\subseteq B \;\txt{ist offen}\]

Bleibt zu zeigen: $ p(E_0)\subseteq B$ ist abgeschlossen. Dann folgt $p(E_0)=B$, da B zusammenhängend.\\

Sei nun \( x \in \overline{p(E_0)} \) und \( U \) eine wegzusammenhängende Umgebung von $x$, die von \( p \) überlagert wird. Da \( U \) einen Punkt aus \( p(E_0) \) enthält, gibt es ein Blatt \( V _\alpha\subseteq p^{-1}(U) \), welche \( E_0 \) schneidet. Da \( V_\alpha \) homöomorph zu U ist, ist \( V_\alpha \) wegzusammenhängend ist, folgt \( V_\alpha \subseteq E_0 \).\\

\(\Rightarrow p(V_\alpha)\overset{\txt{Def. ÜL}}{=}U\subseteq p(E_0)\) insbesonders $x\in p(E_0)$\\

Es bleibt zu zeigen, dass \( p_0 : E_0 \to B \) eine Überlagerung ist.

Sei \( x \in B \), wähle eine Umgebung \( U\) wie oben. Ist $V_\alpha$ ein Blatt von $p^{-1}(U)$, dann ist $V_\alpha$ wegzusammenhängend und falls es $E_0$ schneidet, dann ist es in $E_0$ enthalten. \\

\( \Rightarrow p_0^{-1}(U) \) ist die Vereinigung der Blätter $V_\alpha$, die $E_0$ schneidet, jedes diese Blätter ist offen in $E_0$ und wird durch $p_0$ homöomorph auf U abgebildet.\\

\(\Rightarrow U\) wird von $p_0$ überlagert.

\end{proof}

\textbf{Lemma 7.38:} \\
Seien \( p, q \) und \( r \) stetige Abbildungen mit
\[
p = r \circ q
\]
und folgendem kommutativen Diagramm:
  \begin{figure}[H]
    \centering
    \includegraphics[width=2cm]{Image Diffgeo/19.02.jpg}
	%\caption{Ebene mit Loch und die Zylinder Oberfläche sind nicht einfach zusammenhängend}
 \end{figure}

\emph{(Es werde außerdem angenommen, dass \( X, Y, Z \) lokal wegzusammenhängend und wegzusammenhängend sind.)}

\begin{enumerate}
    \item[(i)] Sind \( p \) und \( r \) Überlagerungen, dann auch \( q \).
    \item[(ii)] Sind \( p \) und \( q \) Überlagerungen, dann auch \( r \).
\end{enumerate}

\begin{proof}[Beweis (nur für (i))] 
Sei \( x_0 \in X \), setze \( y_0 := q(x_0) \) und \( z_0 := p(x_0) \).

Wir nehmen also an, dass \( p \) und \( r \) Überlagerungen sind.\\

Zuerst zeigen wir, dass \( q \) surjektiv ist.

Sei \( y \in Y \), wähle einen Weg \( \tilde{\alpha} \) in \( Y \) von \( y_0 \) nach \( y \). Dann ist \( \alpha=r \circ  \tilde{\alpha} \) ein Weg in \( Z \), der in \( z_0 \) beginnt.\\

Sei \( \tilde{ \tilde{\alpha}} \) eine Anhebung von \( \alpha\) zu einem Weg in \( X \), der in \( x_0 \) beginnt.
Dann ist \( q \circ  \tilde{\tilde{\alpha}} \) eine Anhebung von \( \alpha \) nach \( Y \), die in \( y_0 \) beginnt. Nach der Eindeutigkeit von Weghebungen folgt: \( \widetilde{\alpha}=q\circ\tilde{ \tilde{\alpha}} \) auf den Endpunkt von \( \tilde{\alpha} \) abbildet. Also ist $q$ surjektiv.\\

Gegeben \( y \in Y \), wir suchen eine Umgebung von \( y \), die von \( q \) überdeckt wird.  
Sei \( z := r(y) \). Da \( p \) und \( r \) Überlagerungen sind, finden wir eine wegzusammenhängende Umgebung \( U \) von \( z \), die von \( p \) und \( r \) überdeckt wird. Sei \( V \) das Blatt von \( r^{-1}(U) \), das \( y \) enthält.  
Wir wollen zeigen: \( V \) wird von \( q \) überdeckt.  \\

Sei \( \{ U_\alpha \} \) eine Sammlung der Blätter von \( p^{-1}(U) \). Dann bildet \( q \) jede Menge \( U_\alpha \) auf \( r^{-1}(U) \) ab.  
Da \( U_\alpha \) zusammenhängend ist, wird \( U_\alpha \) auf ein einzelnes Blatt von \( r^{-1}(U) \) abgebildet.  
Daher besteht \( q^{-1}(V) \) aus der Vereinigung der Blätter \( U_\alpha \), die durch \( q \) auf \( V \) abgebildet werden.

  \begin{figure}[H]
    \centering
    \includegraphics[width=12cm]{Image Diffgeo/19.12.jpg}
	%\caption{Ebene mit Loch und die Zylinder Oberfläche sind nicht einfach zusammenhängend}
 \end{figure}

Es ist leicht zu sehen, dass jedes \( U_\alpha \) durch \( q \) homöomorph auf \( V \) abgebildet wird.

Seien \( p_0, q_0 \) und \( r_0 \) die Abbildungen, die wir durch Einschränken von \( p, q \) bzw. \( r \) erhalten:

  \begin{figure}[H]
    \centering
    \includegraphics[width=3cm]{Image Diffgeo/19.10.png}
	%\caption{Ebene mit Loch und die Zylinder Oberfläche sind nicht einfach zusammenhängend}
 \end{figure}

Da \( p_0 \) und \( r_0 \) Homöomorphismen sind, so ist auch
\[
q_0 = r_0^{-1} \circ p_0
\]
ein Homöomorphismus.

\end{proof}

\textbf{Satz 7.39:} \\
Sei \( p : E \to B \) eine Überlagerung, wobei \( E \) einfach zusammenhängend sei. Gegeben eine Überlagerung \( r : Y \to B \), dann gibt es eine Überlagerung \( q : E \to Y \) mit
\(
r \circ q = p.
\)
  \begin{figure}[H]
    \centering
    \includegraphics[width=2cm]{Image Diffgeo/19.03.png}
	%\caption{Ebene mit Loch und die Zylinder Oberfläche sind nicht einfach zusammenhängend}
 \end{figure}
Dieser Satz zeigt, warum wir \( E \) eine \emph{universelle Überlagerung} nennen: \( E \) überlagert jede Überlagerung von \( B \).

\begin{proof}
Sei \( b_0 \in B \), wähle \( e_0 \) und \( y_0 \) mit \( p(e_0) = b_0 = r(y_0) \).  
Wir konstruieren \( q \) durch Lemma 7.31:

Die Abbildung \( r \) ist eine Überlagerung und die Bedingung  
\[
p_* \left( \pi_1(E, e_0) \right) \subseteq r_* \left( \pi_1(Y, y_0) \right)
\]  
ist erfüllt, da \( E \) einfach zusammenhängend ist.

\[
\Rightarrow \text{Es gibt eine Abbildung } q : E \to Y \text{ mit } r \circ q = p \text{ und } q(e_0) = y_0.
\]

Laut dem vorherigem Lemma folgt, dass \( q \) eine Überlagerung ist. 
\end{proof}

\textbf{Lemma 7.40} \\
Sei \( p : E \to B \) eine Überlagerung mit \( p(e_0) = b_0 \). Ist \( E \) einfach zusammenhängend, dann besitzt \( b_0 \) eine Umgebung \( U \), sodass die Inklusion \( \iota : U \hookrightarrow B \) den trivialen Homomorphismus (also Nullabbildung)
\[
\iota_* : \pi_1(U, b_0) \longrightarrow \pi_1(B, b_0)
\]
induziert.

\begin{proof}
Sei \( U \) eine Umgebung von \( b_0 \), die von \( p \) überlagert wird, zerlege $p^{-1}(U)$ in Blätter. Sei \( U_\alpha \) das Blatt, das \( e_0 \) enthält. \\

Sei \( f \) eine Schleife in \( U \) mit Basispunkt \( b_0 \). \\
Da \( p \) ein Homöomorphismus zwischen \( U_\alpha \) und \( U \) ist, hebt die Schlaufe \( f \) zu einer Schlaufe \( \tilde{f} \) in \({U_\alpha} \) mit Basispunkt \( e_0 \) an. \\

Da \( E \) einfach zusammenhängend ist, gibt es eine Homotopie \( \tilde{F} \) in \( E \) zwischen \( \tilde{f} \) und der konstanten Schlaufe $e_0$. \\
\(\Rightarrow p\circ\tilde{F} \) ist eine Weg-Homotopie zwischen \( f \) und der konstanten Schleife $b_0$.
  \begin{figure}[H]
    \centering
    \includegraphics[width=12cm]{Image Diffgeo/19.13.jpg}
	%\caption{Hawaiian earring}
 \end{figure}
\end{proof}

\textbf{Beispiel.}
Sei \( X \) der Hawaiianische Ohrring. Ist \( C_n \) ein Kreis mit Radius \( \frac{1}{n} \) in der Ebene mit Mittelpunkt \( \left(\frac{1}{n}, 0\right) \), dann ist \( X \) die Vereinigung der Kreise \( C_n \).
  \begin{figure}[H]
    \centering
    \includegraphics[width=8cm]{Image Diffgeo/19.05.png}
	\caption{Hawaiian earring}
 \end{figure}

Sei \( b_0 = (0, 0) \). Für jede Umgebung \( U \) von \( b_0 \) gibt es Schleifen in \( U \), die nicht homotop zu einer konstanten Schleife in \( X \) sind, also nicht in \( \ker(\iota_*) \) liegen. (Jede Umgebung enthält unendlich viele infinitesimal kleine Kreise, die nicht nullhomotop sind!)

Das bedeutet: Der Inklusionshomomorphismus
\[
\iota_* : \pi_1(U, b_0) \to \pi_1(X, b_0)
\]
ist nicht injektiv.\\

Sei \( n \) mit \( C_n \subseteq U \), dann gilt für die Inklusionen
  \begin{figure}[H]
    \centering
    \includegraphics[width=8cm]{Image Diffgeo/19.06.png}
	%\caption{Hawaiian earring}
 \end{figure}

\( j_* \) injektiv \( \Rightarrow \iota_* \) ist nicht trivial.

%%%%%%%%%%%%%%%%%%%%%%%%%%%%%%%%%%%%%%%%%%%%%%%%%%%%%%%%%%%%%%%%%%%%%%%%%%%%%%%%%%%%%%% Vorlesung 20 %%%%%%%%%%%%%%%%%%%%%%%%%

\subsection{Existenz von Überlagerungen}

Wir haben gesehen: Jede Überlagerung \( p : E \to B \) definiert eine Konjugationsklasse von Untergruppen von \( \pi_1(B, b_0) \), weiterhin sind zwei solche Überlagerungen genau dann äquivalent, wenn sie zur selben Konjugationsklasse korrespondieren.\\

D.\,h.\ wir haben eine injektive Beziehung:\\
Äquivalenzklassen von Überlagerungen von $B \rightarrow$ Konjugationsklassen von Untergruppen von $\pi_1(B, b_0)$.\\

Wir haben bereits gesehen, dass diese Abbildung im Allgemeinen nicht surjektiv ist.\\
Im Lemma~7.40 haben wir eine notwendige Bedingung für die Existenz einer universellen Überlagerung gesehen.\\
Wir formalisieren diese Bedingung.\\

\textbf{Definition 7.41:} Der topologische Raum \( B \) heißt \emph{semilokal einfach zusammenhängend}, falls es für alle \( b \in B \) eine Umgebung \( U \) von \( b \) gibt, so dass der Homomorphismus
\[
  \iota_* : \pi_1(U, b) \longrightarrow \pi_1(B, b),
\]
der von der Inklusion induziert wird, trivial ist.

\medskip
Anderes gesagt: Jeder Punkt \( b \in B \) hat eine Umgebung \( U \), so dass jede Schleife in \( U \) (mit Basispunkt \( b \)) nullhomotop \underline{in \( B \)} ist (also in ganz \( B \), nicht nur in \( U \)). \( U \) muss nicht einfach zusammenhängend sein, denn es wird erlaubt, dass die Homotopie, welche die stetige Verformung der Schleife zu einem Punkt vornimmt, zeitweise auch außerhalb von $U$ verlaufen darf.\\

\textbf{Bemerkung:}
\begin{itemize}
  \item Wenn \( U \) die Bedingung erfüllt, dann auch jede kleinere Umgebung von \( b \), also hat \( b \) eine „beliebig kleine“ Umgebung, welche die Bedingung erfüllt.
  \item Die Hawaiianische Ohrringe ist weder einfach zusammenhängend, noch lokal einfach zusammenhängend, noch semilokal einfach zusammenhängend. (\emph{pathologisch})
  \item Diese Forderung ist schwächer als \emph{lokal einfach zusammenhängend}, was heißen würde, dass es in jeder Umgebung von \( b \) eine Umgebung von \( b \) gibt, die einfach zusammenhängend ist. (Beispiel: Cone über Hawaiianische Ohrringe)
    \begin{figure}[H]
    \centering
    \includegraphics[width=15cm]{Image Diffgeo/20.06.jpg}
	\caption{Cone (rechts) ist einfach zusammenhängend und damit semilokal einfach zusammenhängend, aber nicht lokal einfach zusammenhängend. Der wesentliche Unterschied zwischen der normalen Hawaiia Ohrringe und dem sogenannten Shrinking Wedge of Circles ist Topologie: hier benutzt man coproduct-topology (ganz abstrakt) während normale Hawaiia Ohrringe mit $\mathbb{R}^2$-Teilraum Topologie versehen wird.}
 \end{figure}
  
  \item \emph{Semilokal einfach zusammenhängend ist eine notwendige und hinreichende Bedingung an \( B \), damit es zu jeder Konjugationsklasse von Untergruppen von \( \pi_1(B, b_0) \) eine geeignete Überlagerung von \( B \) gibt.}\\
  
  Notwendigkeit haben wir in Lemma~7.40 gesehen. Das ist hinreichend wollen wir nun zeigen
\end{itemize}

\medskip


\textbf{Theorem 7.42:} Sei \( B \) wegzusammenhängend, lokal wegzusammenhängend und semilokal einfach zusammenhängend. Sei \( b_0 \in B \). Gegeben eine Untergruppe \( H \) von \( \pi_1(B, b_0) \), dann gibt es eine Überlagerung \( p : E \to B \) und einen Punkt \( e_0 \in p^{-1}(b_0) \), so dass
\[
  p_*(\pi_1(E, e_0)) = H.
\]

\medskip

\textbf{Korollar 7.43:} Der topologische Raum \( B \) besitzt eine universelle Überlagerung genau dann, wenn \( B \) wegzusammenhängend, lokal wegzusammenhängend und semilokal einfach zusammenhängend ist.
\begin{proof}
    Hinrichtung folgt einfach aus Lemma 7.40. Für die Rückrichtung gilt zunächst Theorem 7.42, insbesondere muss auch für $H=\{0\}\subseteq\pi_1(B,b_0)$ gelten, was $\pi_1(E,e_0)=0$ bedeutet da $p_*$ injektiv nach Satz 7.29 ist also muss $\ker(p_*)=0$.
\end{proof}

\medskip

Der Beweis von Theorem 7.42 ist relativ lang und besteht aus sieben Schritten.
\begin{enumerate}
  \item Konstruktion E (als Menge) und p
  \item Gebe \( E \) eine Topologie
  \item Die Abbildung \( p : E \to B \) ist stetig und offen
  \item Jeder Punkt von \( B \) hat eine Umgebung die von p überlagert wird
  \item Anheben eines Weges in B
  \item \( p \) ist eine Überlagerung (d.h. E wegzusammenhängend)
  \item $H=p_*(\pi_1(E,e_0))$
\end{enumerate}
\underline{Anmerkung}: Basis einer Topologie\\

\textbf{Definition 1.2:}
Ein Mengensystem \( \mathcal{B} \subset \mathcal{P}(M) \) heißt \emph{Basis der Topologie} \( \mathcal{O}_M \), falls die offenen Mengen aus \( \mathcal{O}_M \) genau die Vereinigungen der Mengen aus \( \mathcal{B} \) sind,  
Insbesondere \( \mathcal{B} \subseteq\mathcal{O}_M \)

\medskip

In der Topologie benutzt man eine andere:

\medskip 

\textbf{Definition:}  
Ist \( X \) eine Menge, eine \emph{Basis} für eine \emph{Topologie} auf \( X \) ist ein Mengensystem \( \mathcal{B} \) aus Teilmengen von \( X \), so dass:
\begin{enumerate}
  \item[(a)] Für alle \( x \in X \) gibt es (mindestens) ein \( B \in \mathcal{B} \) mit \( x \in B \).
  \item[(b)] Für \( x \in B_1 \cap B_2 \) mit \( B_1, B_2 \in \mathcal{B} \), gibt es ein \( B_3 \in \mathcal{B} \) mit  
  \[
    x \in B_3 \subseteq B_1 \cap B_2.
  \]
\end{enumerate}

{Unter diesen Annahmen definieren wir die von \( \mathcal{B} \) erzeugte Topologie \( \mathcal{T} \) als:}  

Eine Menge \( U \subseteq X \) ist \emph{offen} (bezüglich \( \mathcal{T} \)), falls es für alle \( x \in U \) ein \( B \in \mathcal{B} \) gibt mit  
\[
x \in B \subseteq U.
\]

\textbf{Bemerkung:} \( \mathcal{B} \subseteq \mathcal{T} \).


\begin{proof}
Wir führen nun den Beweis für Theorem 7.42:\\
\underline{Schritt 1: Konstruiere \( E \)}

Sei \(\mathcal{P}\) die Menge aller Wege in \(B\), die in \(b_0\) beginnen. Definiere eine Äquivalenzrelation auf \(\mathcal{P}\) durch
\[
\alpha \sim \beta \;:\Longleftrightarrow\;
\begin{cases}
\alpha \text{ und } \beta \text{ enden im selben Punkt} \\
[\alpha * \beta^-] \in H \subseteq \pi_1(B,b_0)
\end{cases}
\]

Tatsächlich ist \(\sim\) eine Äquivalenzrelation, und wir bezeichnen die Äquivalenzklasse eines Weges \(\alpha\) mit \(\alpha^{\#}\).

Sei nun \(E\) die Menge aller Äquivalenzklassen und definiere
\[
p : E \longrightarrow B \quad \text{durch} \quad p(\alpha^{\#}) = \alpha(1) \quad (\txt{identifizieren mit Endpunkt}).
\]
    \begin{figure}[H]
    \centering
    \includegraphics[width=9cm]{Image Diffgeo/20.07.jpg}
	\caption{rechts zwei Kurven äquivalent aber links nicht}
 \end{figure}
Da \(B\) wegzusammenhängend ist, ist \(p\) surjektiv.\\

Wir wollen \(E\) mit einer Topologie versehen, die \(p\) zu einer Überlagerung macht.

Vorher halten wir zwei Beobachtungen fest.

\begin{enumerate}
\item Ist \([\alpha] = [\beta]\) (d.h. \(\alpha\) weghomotop zu \(\beta\)), dann \(\alpha^{\#} = \beta^{\#}\).
\item Ist \(\alpha^{\#} = \beta^{\#}\), dann \((\alpha * \delta)^{\#} = (\beta * \delta)^{\#}\) für jeden Weg \(\delta\) in \(B\), der in \(\alpha(1)\) beginnt.
\end{enumerate}

\underline{Zu 1}:  
\[
[\alpha] = [\beta] \Rightarrow [\alpha * {\beta}^-] = [e_{b_0}] \in H.
\]

\underline{Zu 2}:  
\[
\alpha * \delta \text{ und } \beta * \delta \text{ enden im selben Punkt, und}
\]
\[
[(\alpha * \delta) * (\beta * \delta)^{-}] = [(\alpha * \delta) * (\delta^{-} * \beta^{-})] = [\alpha * \beta^{-}] \in H.
\]



\bigskip
\underline{Schritt 2: Gib \( E \) eine Topologie.}
\begin{enumerate}
    \item[1.] Möglichkeit: Gib \( \mathcal{P} \) die kompakt-offene Topologie und \( E \) die Quotiententopologie.
    
    \item[2.] Möglichkeit: Direkte Konstruktion.
\end{enumerate}

Sei \(\alpha \in \mathcal{P}\) und \(U\) eine wegzusammenhängende Umgebung von \(\alpha(1)\). Definiere
\[
\mathcal{B}(U, \alpha) := \left\{ (\alpha * \delta)^{\#} \,\middle|\, \delta \text{ ist ein Weg in } U, \text{ der in } \alpha(1) \text{ beginnt} \right\}.
\]

Wir halten fest: \(\alpha^{\#} \in \mathcal{B}(U, \alpha)\), denn für \(b := \alpha(1)\) gilt 
\[
\alpha^{\#} = (\alpha * e_{b})^{\#}
\]
und das ist ein Element in \(\mathcal{B}(U, \alpha)\).\\

Wir behaupten, dass die Mengen \(\mathcal{B}(U, \alpha)\) die Basis einer Topologie auf \(E\) definieren.\\

Zunächst zeigen wir: Falls \(\beta^{\#} \in \mathcal{B}(U, \alpha)\), dann gilt \(\alpha^{\#} \in \mathcal{B}(U, \beta)\) und \(\mathcal{B}(U, \alpha) = \mathcal{B}(U, \beta)\). Ist \(\beta^{\#} \in \mathcal{B}(U, \alpha)\), dann schreiben wir \(\beta^{\#} = (\alpha * \delta)^{\#}\) für einen Weg \(\delta\) in \(U\). Dann:
\[
(\beta * \delta^{-})^{\#} \overset{\txt{laut 2)}}{=} \left( (\alpha * \delta) * \delta^{-} \right)^{\#} \overset{\txt{laut 1)}}{=} \alpha^{\#} \quad \Rightarrow \alpha^{\#} \in \mathcal{B}(U, \beta) \quad \text{per Definition}.
\]

\begin{center}
\includegraphics[width=0.6\textwidth]{Image Diffgeo/20.01.png}
\end{center}

Wir zeigen \(\mathcal{B}(U, \beta) \subseteq \mathcal{B}(U, \alpha)\).  
Ein Element in \(\mathcal{B}(U, \beta)\) hat die Form \((\beta * \gamma)^{\#}\) für einen Weg \(\gamma\) in \(U\). Dann gilt  
\[
(\beta * \gamma)^{\#} = ((\alpha * \delta) * \gamma)^{\#} = (\alpha * (\delta * \gamma))^{\#}
\]
und liegt also (per Definition) in \(\mathcal{B}(U, \alpha)\), da $\delta*\gamma$ eine Kurve in $U$ mit $\alpha(1)$ beginnt. Das symmetrische Argument zeigt \(\mathcal{B}(U, \alpha) \subseteq \mathcal{B}(U, \beta)\). \\

Schließlich zeigen wir, dass die \(\mathcal{B}(U, \alpha)\) eine Basis bilden. Sei  
\[
\beta^{\#} \in \mathcal{B}(U_1, \alpha_1) \cap \mathcal{B}(U_2, \alpha_2).
\]  

Wir wählen eine wegzusammenhängende Umgebung \(V\) von \(\beta(1)\), die in \(U_1 \cap U_2\) enthalten ist.  
Die Inklusion  
\[
\mathcal{B}(V, \beta) \subseteq \mathcal{B}(U_1, \beta) \cap \mathcal{B}(U_2, \beta)
\]  
folgt aus der Definition. Laut der vorherigen Überlegung stimmt die rechte Seite mit  
\(\mathcal{B}(U_1, \alpha_1) \cap \mathcal{B}(U_2, \alpha_2)\) überein.\\


\underline{Schritt 3:} Die Abbildung \( p \) ist stetig und offen.

(i) Offenheit:\\  
Das Bild eines Basiselements \(\mathcal{B}(U, \alpha)\) ist die offene Menge \(U \subseteq B\).  

Denn: Sei \(x \in U\) und wähle einen Weg \(\delta\) in \(U\) von \(\alpha(1)\) nach \(x\).  
Dann gilt \((\alpha * \delta)^{\#} \in \mathcal{B}(U, \alpha)\) und  
\[
p((\alpha * \delta)^{\#}) = (\alpha * \delta)(1) = x.
\]

(ii) Stetigkeit:\\ 
Sei \(\alpha^{\#} \in E\) und \(W\) eine Umgebung von \(p(\alpha^{\#})\). Wähle eine wegzusammenhängende Umgebung \(U\) von \(p(\alpha^{\#}) = \alpha(1)\), die in \(W\) enthalten ist. Dann ist \(\mathcal{B}(U, \alpha)\) eine Umgebung von \(\alpha^{\#}\), die von \(p\) nach \(W\) abgebildet wird:
\[p(\mathcal{B}(U,\alpha))\subseteq W\]

Daher ist \(p\) stetig in \(\alpha^{\#}\).\\

\underline{Schritt 4:} Jeder Punkt von \( B \) hat eine Umgebung, die von \( p \) überlagert wird.

Sei \(b_1 \in B\), wähle eine wegzusammenhängende Umgebung von \(b_1\), für die der Homomorphismus  
\[
\pi_1(U, b_1) \xrightarrow{\iota_*} \pi_1(B, b_1)
\]
trivial ist. \\

Wir zeigen: \(U\) wird von \(p\) überlagert.  \\

Zunächst zeigen wir, dass \(p^{-1}(U)\) die Vereinigung der Mengen \(\mathcal{B}(U, \alpha)\) ist, wobei \(\alpha\) alle Wege \(b_0 \to b_1\) durchläuft. Da \(p\) jede Menge \(\mathcal{B}(U, \alpha)\) surjektiv auf \(U\) abbildet, ist klar, dass \(p^{-1}(U)\) die Vereinigung enthält.  

Andererseits: Sei \(\beta^{\#} \in p^{-1}(U)\), dann gilt \(\beta(1) \in U\).  
Wähle einen Weg \(\delta\) in \(U\) von \(b_1\) nach \(\beta(1)\) und sei \(\alpha\) der Weg \(\beta * \delta^-\) von \(b_0\) nach \(b_1\).  

Dann gilt  
\[
[\beta] = [\alpha * \delta], \quad \text{also} \quad \beta^{\#} = (\alpha * \delta)^{\#} \in \mathcal{B}(U, \alpha)
\]  
\(\Rightarrow\) \(p^{-1}(U)\) ist in der Vereinigung enthalten.
    \begin{figure}[H]
    \centering
    \includegraphics[width=4cm]{Image Diffgeo/20.08.jpg}
	%\caption{rechts zwei Kurven äquivalent aber links nicht}
 \end{figure}

Weiter sind die Mengen \(\mathcal{B}(U, \alpha)\) disjunkt.  

Denn ist \(\beta^\# \in \mathcal{B}(U, \alpha_1) \cap \mathcal{B}(U, \alpha_2)\),  
dann \(\mathcal{B}(U, \alpha_1) = \mathcal{B}(U, \beta) = \mathcal{B}(U, \alpha_2)\) aus Betrachtung in Schritt 2.\\

Zuletzt zeigen wir, dass \(p\) eine bijektive Abbildung zwischen \(\mathcal{B}(U, \alpha)\) und \(U\) ist.  

Daraus folgt, dass \(p|_{\mathcal{B}(U,\alpha)}\) ein Homöomorphismus ist, denn es ist bijektiv, stetig und offen.  
Wir wissen bereits, dass die Abbildung surjektiv ist. Für die Injektivität sei  
\[
p\left((\alpha * \delta_1)^\#\right) = p\left((\alpha * \delta_2)^\#\right),
\]
wobei \(\delta_1\) und \(\delta_2\) Wege in \(U\) sind. Also gilt \(\delta_1(1) = \delta_2(1)\).  

Da der Homomorphismus  
\[
\pi_1(U, b_1) \xrightarrow{\iota_*} \pi_1(B, b_1)
\]
trivial ist, ist \(\delta_1 * \delta_2^-\) in \(B\) homotop zur konstanten Schlaufe.  
    \begin{figure}[H]
    \centering
    \includegraphics[width=6cm]{Image Diffgeo/20.09.jpg}
	%\caption{rechts zwei Kurven äquivalent aber links nicht}
 \end{figure}
Damit gilt  
\[
[\alpha * \delta_1] = [\alpha * \delta_2] \quad \text{somit} \quad (\alpha * \delta_1)^\# = (\alpha * \delta_2)^\#.
\]
Da Abbildung p ist also eine Überlagerung im Sinne von Definition 7.18. Wir müssen noch zeigen, dass E wegzusammenhängend ist.\\

\underline{Schritt 5:} Anheben eines Weges in \( B \)

Sei \( e_0 \) die Äquivalenzklasse des konstanten Weges mit Basispunkt \( b_0 \); dann gilt  
\[
p(e_0) = b_0 \quad \text{per Definition}.
\]

Gegeben ein Weg \( \alpha \) in \( B \), der in \( b_0 \) beginnt.  
Wir berechnen die Anhebung des Weges zu einem Weg in \( E \), der in \( e_0 \) beginnt,  
und zeigen, dass die Anhebung in \( \alpha^\# \) endet.\\

{Zunächst} sei \( c \in [0,1] \) und  
\[
\alpha_c \colon [0,1] \to B \quad \text{der Weg} \quad \alpha_c(t) = \alpha(t \cdot c) \quad \text{für } 0 \leq t \leq 1.
\]

\(\alpha_c\) ist der „Teil“ von \(\alpha\) zwischen \(\alpha(0)\) und \(\alpha(c)\).  
Sei \(\alpha_0\) der konstante Weg mit Basispunkt \( b_0 \), und \(\alpha_1 = \alpha\). Wir definieren \(\widetilde{\alpha}(c) := (\alpha_c)^\#\) und zeigen, \(\widetilde{\alpha}\) ist stetig. Dann ist \(\widetilde{\alpha}\) eine Anhebung von \(\alpha\), denn  
\[
p(\widetilde{\alpha}(c)) = \alpha_c(1) = \alpha(c),
\]
weiter
\[
\widetilde{\alpha}(0) = (\alpha_0)^\# = e_0, \quad \text{und} \quad \widetilde{\alpha}(1) = (\alpha_1)^\# = \alpha^\#.
\]

Um die Stetigkeit zu zeigen, führen wir eine neue Notation ein.

Gegeben \( 0 \le c < d \le 1 \), es bezeichne \( \delta_{c,d} \) den Weg, gegeben durch die positive Gerade von \([0,1]\) auf \([c,d]\), gefolgt von \( \alpha \). Wir stellen fest, dass die Wege \( \alpha_d \) und \( \alpha_c*\delta_{c,d} \) weghomotop sind, denn sie sind Reparametrisierungen voneinander.

\begin{center}
\includegraphics[width=0.6\textwidth]{Image Diffgeo/20.02.png}
\end{center}

Wir zeigen die Stetigkeit von \( \tilde{\alpha} \) in \( c \in [0,1] \). Sei \( W \) ein Basiselement der Topologie von \( E \), welches \( \tilde{\alpha}(c) \) enthält. Dann ist \( W \) gleich der Umgebung \( B(U, \alpha_c) \) für eine wegzusammenhängende Umgebung \( U \) von \( \alpha(c) \).

Wähle \( \varepsilon > 0 \) so, dass für \( |c - t| < \varepsilon \) die Punkte \( \alpha(t) \) in \( U \) liegen.\\

\underline{Behauptung:} Für \( d \in [0,1] \) und \( |c - d| < \varepsilon \) gilt \( \tilde{\alpha}(d) \in W \). Was dann die Stetigkeit von \( \tilde{\alpha} \) in \( c \) zeigt.\\

Sei \( |c - d| < \varepsilon \). Betrachte zunächst den Fall \( d > c \). Setze \( \delta = \delta_{c,d} \), dann
\[
[\alpha_d] = [\alpha_c * \delta] \quad \text{also} \quad \widetilde{\alpha}(d) = \left( \alpha_d \right)^\# = \left( \alpha_c \ast \delta \right)^\#
\]
Da \( \delta \) in \( U \) liegt, gilt \( \tilde{\alpha}(d) \in B(U, \alpha_c) \) wie behauptet.

Für \( d < c \), betrachte \( \delta = \delta_{d,c} \) und verfahre ähnlich.\\

\underline{Schritt 6:} {Die Abbildung \( p \colon E \to B \) ist eine Überlagerung.}

Wir müssen nur noch zeigen, dass \( E \) wegzusammenhängend ist. Sei \( \alpha^\# \in E \), dann ist die Abhebung \( \tilde{\alpha} \) des Weges \( \alpha \) ein Weg in \( E \) von \( e \) nach \( \alpha^\# \) nach Konstruktion in Schritt 5.\\

\underline{Schritt 7:}
\(
H = p_* \left( \pi_1(E, e_0) \right)
\)

Sei \( \alpha \) eine Schlaufe in \( B \) mit Basispunkt \( b_0 \). Sei \( \tilde{\alpha} \) die Anhebung in \( E \), die in \( e_0 \) beginnt.
\[
[\alpha] \in p_*\left( \pi_1(E, e_0) \right) \; \overset{\text{(Satz 7.29)}}\Longleftrightarrow \tilde{\alpha} \text{ eine Schlaufe in } E
\]

Der Endpunkt von \( \tilde{\alpha} \) ist \( \alpha^\# \) und
\[
\alpha^\# = e_0 \; \Leftrightarrow \; \alpha \sim e_0 
\Leftrightarrow \; \left[ \alpha \ast e_{b_0}^- \right] \in H 
\Leftrightarrow \; \left[ \alpha \right] \in H
\]
\end{proof}

\subsection{Höhere Homotopiegruppen}

Die {Fundamentalgruppe} \( \pi_1(X, x_0) \) heißt auch {erste Homotopiegruppe}. Wir haben sie durch Äquivalenzklassen von Wegen \( f: [0,1] \to X \) mit \emph{fixierten Endpunkten} definiert,
\[
f(0) = x_0 = f(1).
\]
Die fixierten Endpunkte wurden von den \emph{Weghomotopien} respektiert.

Da \([0,1]/\{0,1\} \cong S^1\), hätten wir auch \emph{Homotopien von Abbildungen}
\[
(S^1, 1) \longrightarrow (X, x_0)
\]
betrachten können und wären zu demselben Ergebnis gekommen.\\

Der Nachteil ist, dass sich die Gruppenverknüpfung etwas ungewöhnliche definiert. (bei $f:[0,1]\to X$ definiert man die Gruppenwirkung einfach durch Kurvenzusammensetzung $f*g$). Man möchte zwei Schleifen \( \gamma_1, \gamma_2 : S^1 \to X \) so verknüpfen, dass sie „hintereinander“ durchlaufen werden. Dazu definiert man:

\begin{itemize}
  \item Zerlege \( S^1 \) in zwei Halbkreise (z.\,B. obere und untere Hälfte),
  \item auf der ersten Hälfte läuft man \( \gamma_1 \),
  \item auf der zweiten Hälfte läuft man \( \gamma_2 \),
  \item glätte die Übergänge und parametriere zurück auf ganz \( S^1 \).
\end{itemize}

Man kann das formal über eine geeignete Verklebung und Reparametrisierung realisieren – 
aber es ist technisch deutlich komplizierter als bei Wegen.


\begin{center}
\includegraphics[width=0.4\textwidth]{Image Diffgeo/20.03.png}
\end{center}

\noindent
Diese Sichtweise führt zu einer möglichen Verallgemeinerung
\[
[0,1]^n / \partial([0,1]^n) \cong S^n
\]
Die \emph{n-te Homotopiegruppe} ist die Menge der \emph{Homotopieklassen} von Abbildungen
\[
f: S^n\longrightarrow X
\]
die den \emph{Basispunkt} von \( S^n \) auf den Basispunkt \( x_0 \) abbilden.

\begin{center}
\includegraphics[width=0.2\textwidth]{Image Diffgeo/20.04.png}
\end{center}

Wir bezeichnen diese Gruppe mit
\[
\pi_n(X, x_0).
\]

\textbf{{Fakten:}}
\begin{itemize}
    \item $\pi_n(X, x_0)$ ist abelsch für $n \geq 2$.
    \item Eine \emph{Überlagerung} $p : (E, e_0) \to (B, b_0)$ induziert einen Isomorphismus
    \[
    p_* : \pi_n(E, e_0) \longrightarrow \pi_n(X, x_0), \quad n \geq 2.
    \]
\end{itemize}

\begin{center}
\includegraphics[width=0.9\textwidth]{Image Diffgeo/20.05.png}
\end{center}

\textbf{{Beispiel.:}} Die \emph{Hopf-Faserung} (3. Homotopiegruppe von $S^2$)
\[
\mathbb{C}^2 \supset S^3 \xrightarrow{p} S^2 \cong \mathbb{C}P^1, \quad (z, w) \mapsto [z : w]
\]
\[
S^3 = \left\{ (z, w) \in \mathbb{C}^2 \mid |z|^2 + |w|^2 = 1 \right\}
\]

Es gilt: \quad $p^{-1}(q) \cong S^1$ \quad für alle $q \in S^2$.
\[
p_* : \pi_3(S^3, \mathrm{id}) \longrightarrow \pi_3(S^2, \tilde{b})
\]

\textbf{{Theorem (Hopf):}} \quad
\[
p_*\left( [\mathrm{id}_{S^3}] \right) \neq 0, \quad \text{sogar mehr:} \quad \pi_3(S^2) = ([\tilde{b}], p_*[\mathrm{id}_{S^3}]).
\]

%%%%%%%%%%%%%%%%%%%%%%%%%%%%%%%%%%%%%%%%%%%%%%%%%%%%%%%%%%%%%%%%%%%%%%%%%%%%%%%%%%% Vorlesung 21 %%%%%%%%%%%%%%%%%%%%%%%%%%%%%

\section{Homologie}

\subsection{Zell-Komplexe}

Wir haben gesehen, dass der Torus \( T^2 \) durch Verkleben gegenüberliegender Seiten aus dem Quadrat entsteht.

 \begin{figure}[H]
    \centering
    \includegraphics[width=15cm]{Image Diffgeo/21.01.png}

 \end{figure}

Allgemeiner kann eine orientierbare Fläche \( \Sigma_g \) mit Geschlecht \( g \) aus einem Polygon mit $4g$ Seiten (Kanten) durch Identifikationen von Paaren von Seiten konstruiert werden. Aus den \( 4g \) Seiten werden $2g$ Kreise in der Fläche, die sich in einem einzelnen Punkt schneiden. Das Innere des Polygons ist eine offene Scheibe (2-Zelle), die an die \( 2g \)-Kreise angeheftet wird.\\

Man könnte die Vereinigung der Kreise durch das Anbringen von \( 2g \) offenen Bögen (1-Zellen) an einem gemeinsamen Punkt versehen. Die Fläche entsteht also in Schritten: beginne mit einem Punkt, bringe die 1-Zellen an diesem Punkt an, dann bringe die 2-Zelle an.\\

Wir schauen uns das beispielhaft für \( g = 1, 2, 3 \) an.

\begin{figure}[H]
    \centering
    \includegraphics[width=10cm]{Image Diffgeo/21.02.png}
    \caption{Hatcher, Algebraic Topology}
 \end{figure}

Die natürliche Verallgemeinerung dieser Konstruktion ist das folgende Verfahren:
\begin{itemize}
    \item Beginne mit einer diskreten Menge \( X^0 \), deren Punkte wir als 0-Zellen auffassen.
    \item Induktiv bilden wir das \underline{\( n \)-Skelett} \( X^n \) aus \( X^{n-1} \) durch Anheften von \( n \)-Zellen durch Abbildungen $\varphi_\alpha: S^{n-1}\to X^{n-1}$.\\
    
    Das heißt \( X^n \) ist der Quotientenraum der disjunkten Vereinigung 
\[ X^{n-1} \,\dot{\cup}\, \bigsqcup_\alpha D^n_\alpha \quad \text{unter der Identifikation } x \sim \varphi_\alpha(x) \text{ für } x \in \partial D^n_\alpha. \]

Als Menge ist \( X^n :=  X^{n-1} \,\dot{\cup}\, \bigsqcup_\alpha e^n_\alpha \), wobei alle \( e^n_\alpha \) offene Scheiben (n-Zellen) sind.
\item Entweder hört man nach endlich vielen Schritten auf und setzt \( X = X^n \) für ein \( n < \infty \), 
oder man fährt unbegrenzt fort und setzt \( X = \bigcup_n X^n \).\\

Im zweiten Fall geben wir \( X \) die schwache Topologie: 
\( A \subseteq X \) ist offen (bzw. abgeschlossen) genau dann, wenn \( A \cap X^n \) offen (bzw. abg.) für alle \( n \).
\end{itemize}


Ein Raum \( X \), der auf diesem Weg konstruiert wird, heißt \underline{Zell-Komplex} oder \underline{CW-Komplex} (Closure-finite, Weak).\\

Ist \( X = X^n \) für ein \( n \), dann heißt \( X \) endlich dimensional und das kleinste solche \( n \) ist die \underline{Dimension} von \( X \),
also die maximale Dimension der Zellen von \( X \).
\begin{figure}[H]
    \centering
    \includegraphics[width=8cm]{Image Diffgeo/21.99.jpg}
 \end{figure}

\textbf{Beispiel:}
\begin{itemize}
\item Ein 1-dimensionaler Zell-Komplex \( X = X^1 \) wird \underline{Graph} genannt. \\
Er besteht aus den Ecken (0-Zellen), an die Kanten (1-Zellen) angeheftet werden. 
Die zwei Enden einer Kante dürfen an derselben Ecke angeheftet werden.


\item Die Sphäre \( S^n \) hat die Struktur eines Zell-Komplexes mit nur zwei Zelle $e^0$ und $e^n$. Die \( n \)-Zelle wird durch die konstante Abbildung \( S^{n-1} \rightarrow e^0\) angeheftet. 
Diese Konstruktion ist äquivalent zur Definition von \( S^n \) als $D^n/\partial D^{n-1}$
\item Der reell projektive Raum \( \mathbb{RP}^n \) ist gegeben als \( S^n / (v\sim -v) \), 
d.h. die Sphäre bei der gegenüberliegende Punkte identifiziert werden. \\

Das heißt \( \mathbb{RP}^n \) ist ein Quotient einer Hemisphäre $D^n$, bei der gegenüberliegende Punkte von $\partial D^n$ \emph{identifiziert} werden. Aber $\partial D^n \cong S^{n-1}$ mit der Identifikation gegenüberliegender Punkte ist \( \mathbb{RP}^{n-1} \). \\
\(\Rightarrow \mathbb{RP}^n \) entsteht aus \( \mathbb{RP}^{n-1} \) durch das Anbringen einer \( n \)-Zelle 
mit der Quotientenabbildung \( S^{n-1} \rightarrow \mathbb{RP}^{n-1} \) als Anheft-Abbildung.\\

Induktiv zeigt man, dass \( \mathbb{RP}^n \) eine Zell-Komplex Struktur $e^0\cup e^1 \cup...\cup e^n$ mit einer Zelle \( e^i \) in jeder Dimension \( i \leq n \) besitzt.
\end{itemize}



\textbf{Bemerkung:} Auf diese Weise lässt sich \( \mathbb{RP}^\infty = \bigcup_n \mathbb{RP}^n \) definieren.\\

Jede Zelle \( e^n_\alpha \) in einem Zellkomplex \( X \) hat eine \underline{charakteristische Abbildung} $\Phi_\alpha:D_\alpha^n\to X$, 
welche die Anklebe-Abbildung \( \varphi_\alpha \) zu einem Homöomorphismus zwischen \( \mathring{D}_\alpha^n \) und \( e^n_\alpha \) macht. (darf nicht Abschluss nehmen, da zwei Randpunkte zu dem gleichen Punkt abgebildet werden können.)\\

Genau nehmen wir \( \Phi_\alpha \) als die Komposition
\[
D^n_\alpha \hookrightarrow X^{n-1} \dot{\cup} \bigsqcup_\alpha D^n_\alpha \to X^n \hookrightarrow X,
\]
wobei die zweite Abbildung die Quotientenabbildung ist, die \( X^n \) definiert.\\

\textbf{Beispiel}
\begin{itemize}
\item Für das Beispiel \( S^n \) von oben ist die charakteristische Abbildung der \( n \)-Zelle die Quotientenabbildung \( D^n \to S^n \),
welche \( \partial D^n \) zu einem Punkt zusammenschlägt.
\[S^n=D^n/\partial D^n\]

\item Für \( \mathbb{RP}^n \) ist eine charakteristische Abbildung für die Zelle \( e^i \) die Quotientenabbildung 
\( D^i \to \mathbb{RP}^i \subset \mathbb{RP}^\infty \), welche gegenüberliegende Punkte von \( \partial D^n \) identifiziert.
\end{itemize}

\textbf{Definition 8.1} Ein \underline{Unterkomplex} des Zellkomplexes \( X \) ist ein abgeschlossener Unterraum \( A \subset X \), 
der aus einer Vereinigung von Zellen von \( X \) besteht.\\

\textbf{Bemerkung:} Da \( A \) abgeschlossen ist, ist das Bild der charakteristischen Abbildung jeder Zelle in \( A \), in \( A \) enthalten.
Insbesondere ist das Bild der Anklebe-Abbildung jeder Zelle in \( A \) in $A$ enthält, Das heißt \( A \) ist wieder ein Zell-Komplex.\\

\textbf{Definition 8.2} Das Paar \( (X,A) \), bestehend aus dem Zell-Komplex \( X \) und dem Unterkomplex \( A \), heißt 
\underline{CW-Paar}.\\

\textbf{Beispiel} 
\begin{itemize}
\item Jedes Skelett \( X^n \) des Zell-Komplexes \( X \) ist ein Unterkomplex.
\item Für \( \mathbb{RP}^n \) sind alle \( \mathbb{RP}^i\subset \mathbb{RP}^n \) Unterkomplexe. Das sind auch die einzigen Unterkomplexe.
\item Für \( S^n \) gibt es die natürlichen \(S^0\subset S^1\subset... \subset S^n \), aber diese Sphären sind keine Unterkomplexe von \( S^n \) mit der Zellstruktur aus zwei Zellen von oben.
Aber wir können \( S^n \) eine Zellstruktur geben, in der jede der Sphären \( S^k \) ein Teilkomplex ist.
Dabei wird \( S^k \) durch das Anbringen von zwei \( k \)-Zellen konstruiert.

\begin{figure}[H]
    \centering
    \includegraphics[width=10cm]{Image Diffgeo/21.98.jpg}
 \end{figure}
\end{itemize}


\textbf{Bemerkung:} In den bisherigen Beispielen von Zell-Komplexen ist der Abschluss jeder Zelle ein Unterkomplex, 
allgemeiner ist der Abschluss jeder Ansammlung von Zellen ein Unterkomplex. In den meisten natürlichen Fällen stimmt das, 
aber im Allgemeinen ist das nicht der Fall.\\

Nehmen wir beispielsweise \( S^1 \) mit der minimalen Zellstruktur und bringen eine 2-Zelle durch eine Abbildung \( S^1 \to S^1 \) an, 
deren Bild ein nicht-trivialer Bogen von \( S^1 \) ist. Dann der Abschluss der 2-Zelle ist kein Unterkomplex, denn 
dieser enthält nur einen Teil der 1-Zelle.
\begin{figure}[H]
    \centering
    \includegraphics[width=12cm]{Image Diffgeo/21.97.jpg}
 \end{figure}

\textbf{Satz 8.3:} Jede glatte, kompakte Mannigfaltigkeit besitzt eine CW-Struktur.

\subsection{8.2. Die Idee hinter Homologie}

Die Fundamentalgruppe \(\pi_1(X,x_0)\) ist nützlich, wenn wir niedrig dimensionale Räume untersuchen.

In der Sprache von CW-Komplexen: Ist \(X\) ein CW-Komplex, dann hängt \(\pi_1(X,x_0)\) nur vom 2-Skelett von \(X\) ab. Insbesondere \(\pi_n(S^i) = \pi_n(S^j)\) für alle \(i, j \geq 2\). Das kann durch Betrachtung der höheren Homotopie-Gruppen gelöst werden.\\

\underline{Problem:} Das Berechnen von \(\pi_n\) ist schwer.

Es gibt eine Alternative, die sich besser berechnen lässt: Die \textbf{Homologie-Gruppen} \(H_n(X)\).\\

\underline{Vorteil:} \(H_n(X)\) hängt für einen CW-Komplex \(X\) nur vom \((n+1)\)-Skelett \(X^{n+1}\) ab.

Für Sphären: 
\[
H_i(S^n) \cong \pi_i(S^n) \quad \text{für } 1 \leq i \leq n, 
\qquad 
H_i(S^n) = 0 \quad \text{für } i > n.
\]

\underline{Nachteil:} Die Definition von Homologie ist nicht so eingängig wie die Definition von Homotopie-Gruppen.\\

{Unser Plan:}
\begin{itemize}
    \item Betrachte ein Beispiel (nicht stringent)
    \item Der vereinfachte Fall: Simpliziale Homologie
    \item Die allgemeine Theorie: Singuläre Homologie
\end{itemize}

\textbf{Anmerkung:} Häufig kennt man nur die Eigenschaften und nicht die Definition.

\hspace{1em} \(\leadsto\) Es soll einen axiomatischen Zugang zur Homologie geben. (Vorlesung Algebraische Topologie.)\\

Wir betrachten ein Beispiel, um die Idee zu veranschaulichen. Dazu betrachten wir den Raum \(X\),

\begin{itemize}
    \item 2 Ecken: x,y
    \item 4 Kanten: $a,b,c,d$ von \(x\) nach \(y\).
\end{itemize}

\begin{figure}[H]
    \centering
    \includegraphics[width=3cm]{Image Diffgeo/21.03.png}

 \end{figure}
 
Wir untersuchen die Fundamentalgruppe von \(X_1\), bzgl. des Basispunktes \(x_0\). Zum Beispiel die Schleife \(ab^{-1}\), wobei der Exponent \(-1\) heißt, dass wir \(b\) rückwärts durchlaufen. Oder komplizierter: \(ac^{-1}d^{-1}ca^{-1}\). Die Kanten in anderer Reihenfolge durchlaufen gibt die Schleife \(b^{-1}a\) mit Basispunkt \(y\). Aber geometrisch beschreiben sie im wesentlichen denselben Kreis.\\

Wenn wir die Knotenstruktur abelsch machen, also \(ab^{-1}=b^{-1}a\) verlang betrachten, betrachten wir Schleifen ohne Basispunkt, sogenannte \underline{Zyklen}. Dabei schreiben wir die Verknüpfung additiv, z.\,B.\ \(a - b + c - d\). Solche linearen Kombinationen nennen wir \underline{Ketten} von Kanten.\\

Manche Ketten können auf verschiedene Weisen in \emph{Zyklen} zerlegt werden, z.\,B.\
\[
(a - c) + (b - d) = (a - d) + (b - c)
\]

und aus algebraischer Sicht unterscheiden wir nicht zwischen diesen Zerlegungen.

Wir erweitern die Bedeutung des Begriffs \underline{Zyklus}, nämlich jede Linearkombination von Kanten, für die ein Zerlegung in Zykel im geometrischen Sinne existiert.\\

Gibt es einfache Kriterien zu entscheiden, wann eine Kette ein Zyklus im algebraischen Sinne ist?

Ein \emph{geometrischer Zykel}, gedacht als Weg, kommt an jeder Ecke genauso oft an wie er diese verlässt. Für eine Kette \(ka + lb + mc + nd\), die Kette kommt \(k+l+m+n\) mal bei \(y\) an, denn \(a, b, c\) und \(d\) kommen genau einmal bei \(y\) an.
Ähnlich verlässt jede der Kanten \(a, b, c\) und \(d\) die Ecke \(x\) einmal. Die Gesamtzahl, die die Kette bei \(x\) ankommt, ist also
\(
-k - l - m - n
\) mal

\(\Rightarrow ka + lb + mc + nd\) ist ein Zyklus, falls \(k + l + m + n = 0\).\\

Wir beschränken dies auf eine Weise, das sich auf alle Graphen verallgemeinern lässt:

Sei \(C_1\) die freie, abelsche Gruppe mit der Basis aus den Kanten \(a, b, c, d\) und \(C_0\) die freie, abelsche Gruppe mit der Basis aus den Ecken \(x, y\).
Elemente von \(C_1\) sind Ketten von Kanten, oder \(1\)-dimensionalen Ketten.  
Elemente von \(C_0\) sind linearkombinationen von Ecken, oder \(0\)-dimensionalen Ketten.\\

Wir definieren einen Kettenhomomorphismus \(\partial: C_1 \to C_0\), indem wir jedes Basiselement \(a, b, c, d\) auf \(y - x\) abbilden, die Spitze der Kante minus der Ausgangspunkt der Kante.
D.h.
\[
\partial(ka + lb + mc + nd) = (k + l + m + n)y - (k + l + m + n)x
\]
und die Zykel sind der Kern von \(\partial\). \\

Eine einfache Rechnung zeigt \(\ker(\partial) = \langle a-b,b-c,c-d\rangle_\mathbb{Z}\). Jeder Zykel in \(X_1\) ist eine eindeutige Linearkombination von den drei offensichtlichen Zykeln. Damit vermitteln wir die Information, dass \(X_1\) drei sichtbare Löcher hat, den leeren Raum zwischen den vier Kanten.\\

Wir vergrößern den Graphen \(X_1\), durch Anbringen einer \(2\)-Zelle \(A\) entlang des Zyklus \(a - b\), wodurch der \(2\)-dim Zellkomplex \(X_2\) entsteht:

\begin{figure}[H]
    \centering
    \includegraphics[width=3cm]{Image Diffgeo/21.04.png}

 \end{figure}

Wir orientieren die \(2\)-Zelle \(A\) im Uhrzeigersinn, dann fassen wir den Rand von \(A\) als den Zyklus \(a - b\) auf. Dieser Zyklus kann nun jetzt zu einem Punkt zusammengezogen werden, ist also trivial in Homotopie. Mit andere Worten \(a - b\) umschließt also Loch in \(X_2\) mehr ein.\\

Wir bilden die Quotienten \(\ker(\partial)/\langle a - b \rangle\). In diesem Quotienten sind die Zykel \(a - c\) und \(b - c\) äquivalent, was damit zusammenpasst, dass sie in \(X_2\) homotop sind.\\

Algebraisch definieren wir ein Paar von Homomorphismen \(C_2 \xrightarrow{\partial_2} C_1 \xrightarrow{\partial_1} C_0
\), wobei \(C_2\) die unendlich zyklische Gruppe erzeugt von \(A\) ist und \(\partial_2(A) = a - b\) (Rand-Abbildung). Der Homomorphismus \(\partial_1\) ist die Randabbildung aus dem vorherigen Beispiel.\\

Die für uns interessante Quotientengruppe ist \( \ker \partial_1 / \operatorname{im} \partial_2 \), also die \(1\)-dim. Zykel modulo denen, die Bilder von \(\partial_2\) sind, also Vielfache von \(a - b\). Diese Quotientengruppe ist die \underline{Homologiegruppe} \(H_1(X_2)\).\\

Das vorige Beispiel passt in dieses Schema, indem wir \(C_2 = 0\) setzen, denn es gibt keine \(2\)-Zellen in \(X_1\).  
In diesem Fall \(H_1(X_1) = \ker \partial_1 / \operatorname{im} \partial_2  = \ker\partial_1\), also die freie abelsche Gruppe in drei Erzeugern.\\

Im aktuellen Beispiel ist \(H_1(X_2)\) eine freie abelsche Gruppe in zwei Erzeugern, \(b - c\) und \(c - d\). Also die Tatsache widerspiegelt, dass wir durch das Einfüllen der \(2\)-Zelle \(A\) die Anzahl der Löcher von drei auf zwei reduziert haben.\\

Wir vergrößern \(X_2\) zu einem Raum \(X_3\), durch das Anfügen einer zweiten \(2\)-Zelle \(B\) entlang des selben Zyklus \(a - b\).

\begin{figure}[H]
    \centering
    \includegraphics[width=3cm]{Image Diffgeo/21.05.png}

 \end{figure}

Damit besteht die \(2\)-dimensionale Kettengruppe \(C_2\) aus den Linearkombinationen von \(A\) und \(B\), und die Randabbildung bildet \(A\) und \(B\) auf \(a - b\) ab.\\

Die Homologiegruppe \(H_1(X_3)=\ker \partial_1 / \operatorname{im} \partial_2  \) ist dieselbe wie für \(X_2\), aber \(\partial_2\) hat nicht-trivialen Kern, die unendliche zyklische Gruppe erzeugt von \(A - B\). Wir erhalten \(A - B\) als den \(2\)-dim. Zyklus, der die Homologiegruppe \(H_2(X_3) = \ker(\partial_2) \cong \mathbb{Z}\) erzeugt.\\

Topologisch ist der Zyklus \(A - B\) die Sphäre bestehend aus \(A\), \(B\) und ihrem gemeinsamen Randkreis.
Dieser sphärische Zyklus entdeckt ein „Loch“ in \(X_3\), das fehlende Innere der Sphäre. Aber dieses Loch wird von einer Sphäre gebildet und nicht von einem Kreis, es ist also von anderer Gestalt als die Löcher, die von \(H_1(X_3) =\langle b-c,c-d\rangle \cong \mathbb{Z}\text{ x }\mathbb{Z}\) erkannt werden.\\

Wir gehen wieder einen Schritt weiter und konstruieren \(X_4\) aus \(X_3\), durch Anfügen einer \(3\)-Zelle \(C\), entlang der \(2\)-Sphäre, die durch \(A\) und \(B\) gebildet wird. Dadurch entsteht die Kettengruppe \(C_3\), erzeugt durch die \(3\)-Zelle \(C\), und wir definieren den Randhomomorphismus
\[
\partial_3: C_3 \to C_2 \quad \text{durch} \quad C \mapsto A - B.
\]
Wir stellen uns vor, dass \(A - B\) der Rand von \(C\) ist, in ähnlicher Weise wie der Zyklus \(a - b\) der Rand von \(A\) ist.\\

Wir haben eine Sequenz aus drei Randhomomorphismen
\[
\cdots \to C_3 \xrightarrow{\partial_3} C_2 \xrightarrow{\partial_2} C_1 \xrightarrow{\partial_1} C_0 
\]
und der Quotient \(H_2(X_4) = \ker(\partial_2)/\operatorname{im}(\partial_3)\) ist trivial geworden.

Außerdem: 
\begin{itemize}
    \item \(H_3(X_4) = \ker(\partial_3) = 0\)
    \item \( H_1(X_4) \cong H_1(X_3) \cong \mathbb{Z}\text{ x }\mathbb{Z},\) das ist die einzige nicht-triviale Homologiegruppe von \(X_4\).
\end{itemize}

Der Muster wird erkennbar.

Für einen Zellkomplex \(X\) ist die Kettengruppe \(C_n(X)\) die freie abelsche Gruppe deren Basis die \(n\)-Zellen von \(X\) sind. Weiter gibt es einen Rand-Homomorphismus 
\[
\partial_n: C_n(X) \longrightarrow C_{n-1}(X).
\]
Damit definieren wir die \underline{Homologiegruppen} 
\[
H_n(X) = \ker \partial_n / \operatorname{im} \partial_{n+1}.
\]

Das verbleibende Problem: Wie definieren wir \(\partial_n\)?

\begin{itemize}
  \item Für \(n = 1\): Einfach. Der Rand einer orientierten Kante ist die Ecke wo sie endet minus die Ecke wo sie startet.
  
  \item Für \(n = 2\): Machbar für Zellen, die entlang von Schlaufen von Kanten angebracht werden. Dann ist der Rand dieser Zelle der Zyklus, der aus diesen Kanten besteht.

  \item Für größere \(n\): ?

  Selbst für Zellkomplexe, die nur aus Polyhedra-Zellen mit guten Anklebeabbildung bestehen, bleibt die Frage der Orientierung.
\end{itemize}

\subsubsection*{Lösung 1: (Simpliziale Homologie)}
\underline{Beobachtung:} Beliebige Polyhedra können immer in besondere Polyhedra, sogenannte Simplizes, zerlegt werden. (z.B. 2-dim. Dreiecke; 3-dim. Tetraeder). Daher können wir ohne Einschränkung. mit Simplizes statt mit allgemeinen Polyhedra arbeiten.\\

\underline{Vorteile:} Können Orientierungen und RandAbbildung leicht definieren.

\underline{Nachteile:} 
\begin{itemize}
  \item Verlieren Effizienz bei der Zerlegung
  \item Nur für wenige Räume anwendbar
  \item Zu starr, um damit zu arbeiten.
\end{itemize}

\subsubsection*{Lösung 2: (Singuläre Homologie)}

\underline{Idee:} Ignoriere die Darstellung von \( X \) als Zellkomplex.
Betrachte alle stetigen Abbildungen definiert auf Simplizes mit Bild in \( X \).
Das sorgt für unglaublich große Kettengruppen \( C_n(X) \), aber die Quotienten \( H_n(X) = \ker(\partial_n)/\mathrm{im}(\partial_{n+1}) \) sind für hinreichend gute Räume \( X \).\\

Für Räume wie in den obigen Beispielen stimmen die singulären Homologiegruppen mit den Gruppen überein, die wir aus den zellulären Ketten berechnet haben.\\

Wollen sehen: Singuläre Homologie erlaubt uns die Definition der zellulären Homologiegruppen für alle Zellkomplexe. Insbesonders löst sie die Frage nach der Randabbildung.

%%%%%%%%%%%%%%%%%%%%%%%%%%%%%%%%%%%%%%%%%%%%%%%%%%%%%%%%%%%%%%%%%%%%%%%%%%%%%%%%%%%%%%%%%%% Vorlesung 22 %%%%%%%%%%%%%%%%%%%%%

\subsection{$\Delta$-Komplexe}
\textbf{Beispiel 8.4:} 
\begin{itemize}
    \item Der Torus, die reell projektive Ebene und die Kleinsche Flasche können durch Identifizieren von gegenüberliegenden Kanten eines Quadrats definiert werden.

\begin{figure}[H]
    \centering
    \includegraphics[width=13cm]{Image Diffgeo/22.01.png}
 \end{figure}

Schneiden entlang der Diagonalen zerlegt die Quadrate in Dreiecke. D.h. die Flächen können durch Zusammenkleben von Dreiecken gebaut werden.


  \item Ähnlich lässt sich jedes Polygon entlang von Diagonalen in Dreiecke zerlegen, d.\,h. jede Fläche entsteht durch das Zusammenkleben von Dreiecken.
  \begin{figure}[H]
    \centering
    \includegraphics[width=4cm]{Image Diffgeo/22.02.png}
 \end{figure}
\end{itemize}

$\Rightarrow$ Allgemein lassen sich noch viele andere 2-dimensionalen Räumen in Dreiecke zerlegen:

\begin{figure}[H]
    \centering
    \includegraphics[width=5cm]{Image Diffgeo/22.03.jpg}
 \end{figure}

\textbf{Definition 8.5:}\\
Ein \underline{$n$-Simplex} $C$ in $\mathbb{R}^m$ ist die konvexe Hülle von $n+1$ Punkten $u_0, \dots, u_n$, so dass
\(
u_1 - u_0,\ u_2 - u_0,\ \dots,\ u_n - u_0\) { linear unabhängig sind} {(affin unabhängig)}.
D.\,h.
\[
C = \left\{ \lambda_0 u_0 + \dots + \lambda_n u_n \,\middle|\, \sum_{i=0}^n \lambda_i = 1,\ \lambda_i \geq 0 \right\}.
\]

Die $u_i$ heißen \underline{Ecken} oder \underline{Vertices}.

\vspace{1em}
\textbf{Beispiel:} \ Der Standard-$n$-Simplex:
\[
\Delta^n = \left\{ (\lambda_1, \dots, \lambda_n) \in \mathbb{R}^{n+1} \,\middle|\, \sum_i \lambda_i = 1,\ \lambda_i \geq 0 \ \forall i \right\}
\]

Also $v_0 = e_0,\ v_1 = e_1,\ \dots,\ v_n = e_n$.
\begin{figure}[H]
    \centering
    \includegraphics[width=12cm]{Image Diffgeo/22.04.png}
 \end{figure}

\textbf{Bemerkung:}
Für Homologie ist die Reihenfolge der Ecken wichtig, d.\,h. ein $n$-Simplex heißt eigentlich \emph{„$n$-Simplex mit einer Ordnung der Ecken“}.

\begin{itemize}
  \item Durch die Ordnung erhalten wir eine Orientierung auf den Kanten $\left[ v_i, v_j \right]$.
  
  \item Wir orientieren von kleineren zum größeren Index.
  
  \item Durch die Ordnung erhalten wir einen Homöomorphismus des Standard-Simplex $\Delta^n$ auf jeden $n$-Simplex $\left[ v_0, \dots, v_n \right]$, welcher die Orientierung erhält, nämlich die \emph{Baryzentrischen Koordinaten}
  \[
  (t_0, \dots, t_n) \longmapsto \sum_{i=0}^n t_i v_i
  \]
  
  \item Nach Entfernen einer der $n+1$ Ecken aus dem $n$-Simplex $\left[ v_0, \dots, v_n \right]$ bilden die verbleibenden $n$ Ecken ein $(n-1)$-Simplex, eine sogenannte \underline{Seite} (\emph{face}) von $\left[ v_0, \dots, v_n \right]$.
\end{itemize}

\textbf{Konvention:} 
Die Ecken einer Seite (oder allgemein eines Unterkomplexes aufgespannt von Ecken) ordnen wir wie im größeren Simplex. (z.B. für Dreieck $[v_0, v_1, v_2]$, nach Entfernen $v_1$ ergibt $[v_0,v_2]$)\\

\textbf{Definition 8.6:}
\begin{itemize}
  \item Die Vereinigung aller Seiten von $\Delta^n$ heißt der \emph{Rand} von $\Delta^n$: $\partial \Delta^n$.

  \item Der \underline{offene Simplex} $\mathring{\Delta}^n$ ist das Innere von $\Delta^n$, $\mathring{\Delta}^n = \Delta^n \setminus \partial \Delta^n$.

  \item Eine \underline{$\Delta$-Komplex Struktur} auf einem Raum $X$ ist eine Sammlung von Abbildungen $\sigma_\alpha \colon \Delta^n \to X$, wobei $n$ von $\alpha$ abhängt, so dass:
  \begin{enumerate}[i)]
    \item Die Einschränkung $\sigma_\alpha|_{\mathring{\Delta}^n}$ ist \emph{injektiv} und jeder Punkt von $X$ liegt im Bild von genau einer Einschränkung $\sigma_\alpha|_{\mathring{\Delta}^n}$.
    
    \item Die Einschränkung von $\sigma_\alpha$ auf eine Seite von $\Delta^n$ ist eine der Abbildungen $\sigma_\beta \colon \Delta^{n-1} \to X$. Dabei identifizieren wir die Seite von $\Delta^n$ durch den kanonisch linearen Homöomorphismus mit $\Delta^{n-1}$, so dass die Orientierung der Kanten erhalten bleibt.
    
    \item $A \subset X$ ist offen $\iff \sigma_\alpha^{-1}(A)$ ist offen für alle $\sigma_\alpha$.
  \end{enumerate}
\end{itemize}
\begin{figure}[H]
    \centering
    \includegraphics[width=10cm]{Image Diffgeo/22.99.jpg}
    \caption{beide dreidimensionaler Simplexialkomplex, da 3 Ecken enthalten}
 \end{figure}

\textbf{Bemerkung:}
\begin{itemize}
\item Einschränkung auf jedem einzelnen Dreieck $\Delta ^n$ aus Bedingung i) verhindert Situation z.B. aus 2. Beispiel oben (oranger Punkt, wo die Abbildungen dort nicht mehr injektiv sind).
  \item Bedingung (iii) verhindert, dass wir jeden Punkt in \( X \) zu einer Ecke machen.

  \item Die Beispiele aus 8.4 sind \(\Delta\)-Komplexe für den Torus, projektiven Raum und die Kleinsche Fläche mit:
  \begin{itemize}
    \item 2 Dreiecken, 3 Kanten und 1 oder 2 Ecken,
    \item 6 \(\sigma_\alpha\)'s für Torus und Kleinsche Fläche (2+3+1),
    \item 7 \(\sigma_\alpha\)'s für projektive Ebene (2+3+2).
  \end{itemize}

  \item (iii) impliziert: \(X\) lässt sich als Quotientenraum von disjunkten Simplizes \(\Delta^n_\alpha\) auffassen. Einer für jedes \(\sigma_\alpha \colon \Delta^n \to X\).

  \medskip
  Der Quotientenraum entsteht durch Identifikation jeder Seite von \(\Delta^n_\alpha\) mit \(\Delta^{n-1}_\beta\), da der Einschränkung \(\sigma_\beta\) von \(\sigma_\alpha\) gehört (Bedingung ii)).

  \medskip
  Vorstellung: Wir bauen den Raum induktiv. Beginn mit einer diskreten Menge von Ecken, füge Kanten an um einen Graphen zu konstruieren, dann 2-Simplexe usw.
  D.h. ein \(\Delta\)-Komplex kann kombinatorisch aus einer Sammlung von \(n\)-Simplizes \(\Delta^n_\alpha\) und Abbildungen auf den Seiten der \(n\)-Simplexe \(\Delta^n_\alpha\), je einem \((n-1)\)-Simplex \(\Delta^{n-1}_\beta\) zunehm.
\end{itemize}

\textbf{Beispiel}

\begin{center}
\setlength{\extrarowheight}{5.0ex}
\begin{tabular}{>{\centering\arraybackslash}m{0.25\linewidth} >{\raggedright\arraybackslash}m{0.65\linewidth}}
  \includegraphics[width=0.4\linewidth]{Image Diffgeo/22.05.png} & 
  Identifikation der 3 Kanten in der angegebenen Richtung ergibt den sog. Eselsrücken. \\
  
  \includegraphics[width=0.4\linewidth, trim=0 0 0 0, clip]{Image Diffgeo/22.06.png} & 
  Identifikation der 3 Kanten in der angegebenen Richtung gibt \textit{keine} $\Delta$-Komplex-Struktur. \\
  
  \includegraphics[width=0.43\linewidth, trim=0 0 0 0, clip]{Image Diffgeo/22.07.png} & 
  Mit Triangulierung ergibt aber eine $\Delta$-Komplex-Struktur auf dem Quotientenraum. \\
\end{tabular}
\end{center}

\textbf{Bemerkung}
Aus der Auffassung als Quotientenraum folgt: $X$ ist hausdorffsch. \\

Die Abbildung $\sigma_\alpha \vert_{\mathring{\Delta}^n}$ sind Homöomorphismen {aufs Bild}.\\

\(
\Rightarrow\sigma_\alpha(\mathring{\Delta}^n) \subseteq X \)  ist offen und Zellen einer CW-Komplex-Struktur auf X mit Charakteristik Abbildung $\sigma_\alpha$.



\subsection{Simplexiale Homologie}

\underline{Ziel:} Definieren \emph{Simplexialer Homologiegruppen} für $\Delta$-Komplexe.\\

\textbf{Definition 8.7}
Sei $X$ ein $\Delta$-Komplex. Sei $\Delta_n(X)$ die freie abelsche Gruppe mit der Basis aus den offenen $n$-Simplexes $e^n_\alpha$ von $X$. \\
Die Elemente von $\Delta_n(X)$ heißen \underline{$n$-Ketten} und können als endliche Summen 
\[
\sum_\alpha n_\alpha e^n_\alpha \quad \text{mit Koeffizienten } n_\alpha \in \mathbb{Z}
\]
geschrieben werden.\\

\textbf{Bemerkung}
Wir könnten auch $\sum_\alpha n_\alpha \sigma_\alpha$ mit $\sigma_\alpha \colon \Delta^n \to X$ schreiben, 
die $\sigma_\alpha$ sind die charakteristischen Abbildungen auf $e^n_\alpha$ mit Bild im Abschluss von $e^n_\alpha$. \\

Der Rand eines $n$-Simplex $[v_0, \dots, v_n]$ besteht aus $(n{-}1)$-Simplexes \([v_0, \dots, \hat{v}_i, \dots, v_n]\),
wobei $\hat{\ }$ bedeutet, dass der entsprechende Eintrag ausgelassen wird. Wollen wir dies als Grundlage für den Randabbildung verwenden, so führen wir aus technischen Gründen Vorzeichen ein:
\[
\partial([v_0, \dots, v_n]) = \sum_i (-1)^i [v_0, \dots, \hat{v}_i, \dots, v_n]
\]

\emph{Heuristisch:} Füge Vorzeichen ein, damit die Orientierung mit der Orientierung der UnterSimplexes übereinstimmt.

\begin{center}
\setlength{\extrarowheight}{5.0ex}
\begin{tabular}{>{\centering\arraybackslash}m{0.25\linewidth} >{\raggedright\arraybackslash}m{0.65\linewidth}}
  \includegraphics[width=0.8\linewidth]{Image Diffgeo/22.08.png} & 
  \(\partial [v_0, v_1] = [v_1] - [v_0]\) \\
  
  \includegraphics[width=0.8\linewidth, trim=0 0 0 0, clip]{Image Diffgeo/22.09.png} & 
  \(\partial [v_0, v_1, v_2] = [v_1, v_2] - [v_0, v_2] + [v_0, v_1]\) \\
  
  \includegraphics[width=0.8\linewidth, trim=0 0 0 0, clip]{Image Diffgeo/22.10.png} & 
  \(\partial [v_0, v_1, v_2, v_3] = [v_1, v_2, v_3] - [v_0, v_2, v_3] + [v_0, v_1, v_3] - [v_0, v_1, v_2]\) \\
\end{tabular}
\end{center}


\textbf{Definition 8.8:} \\
Sei $X$ ein $\Delta$-Komplex. Der \underline{Randhomomorphismus} ist die lineare Fortsetzung
\[
\partial_n : \Delta_n(X) \to \Delta_{n-1}(X)
\]
von
\[
\partial_n(\sigma_\alpha) = \underbrace{\sum_i (-1)^i \, \sigma_\alpha [v_0, \dots, \hat{v}_i, \dots, v_n]}_{\in \Delta_{n-1}(X), \text{ laut Bedingung (ii) ein $\Delta$-Komplex}}
\]

\textbf{Lemma 8.9} \\
Die Komposition
\[
\Delta_n(X) \xrightarrow{\partial_n} \Delta_{n-1}(X) \xrightarrow{\partial_{n-1}} \Delta_{n-2}(X)
\]
ist \emph{null}.

\begin{proof}
\[
\partial_n(\sigma) = \sum_i (-1)^i \, \sigma [v_0, \dots, \hat{v}_i, \dots, v_n]
\]
\[
\Rightarrow  \partial_{n-1} \big(\partial_n(\sigma)\big)
= \sum_{i < j} (-1)^i (-1)^{j-1} \, \sigma [v_0, \dots, \hat{v}_i, \dots, \hat{v}_j, \dots, v_n] \]
\[\qquad \qquad\qquad\qquad+ \sum_{j < i} (-1)^i (-1)^j \, \sigma [v_0, \dots, \hat{v}_j, \dots, \hat{v}_i, \dots, v_n]
\]

Die beiden Summen heben sich auf, da nach Vertauschen von $i$ und $j$ die erste Summe mit anderem Vorzeichen erscheint.    
\end{proof}


Aus algebraischer Sicht: Wir haben eine Sequenz von Homomorphismen von abelschen Gruppen
\[
\cdots C_{n+1}\xrightarrow{\partial_{n+1}} C_n \xrightarrow{\partial_n} C_{n-1} \xrightarrow{\partial_{n-1}} \cdots \xrightarrow{\partial_1} C_0 \xrightarrow{\partial_0} 0,
\]
mit $\partial_n \circ \partial_{n+1} = 0$ für alle $n$.\\

\textbf{Definition 8.10:}
Diese Sequenz heißt \underline{Kettenkomplex}.\\

\textbf{Bemerkung:}
\begin{itemize}
  \item Wir haben die Sequenz nach rechts um $0$ und $\partial_0 := 0$ ergänzt.
  \item $\partial_n \circ \partial_{n+1} = 0$ bedeutet: $\mathrm{Im}(\partial_{n+1}) \subseteq \ker(\partial_n)$.
\end{itemize}

\textbf{Definition 8.11:} Die $n$-te \underline{Homologiegruppe} des Kettenkomplexes ist
\[
H_n = \frac{\ker(\partial_n)}{\mathrm{Im}(\partial_{n+1})}.
\]
\begin{itemize}
  \item Elemente von $\ker(\partial_n)$ heißen \underline{Zykel}.
  \item Elemente von $\mathrm{Im}(\partial_{n+1})$ heißen \underline{Ränder}.
  \item Elemente von $H_n$ heißen \underline{Homologieklassen}.
  \item Zwei Zykel, die die gleiche Homologieklasse repräsentieren, heißen \underline{homolog}.
\end{itemize}

\textbf{Definition 8.12:}
Sei $C_n = \Delta_n(X)$, dann schreiben wir
\[
H_n^{\Delta}(X) := \frac{\ker(\partial_n)}{\mathrm{Im}(\partial_{n+1})}
\]
und nennen sie die $n$-te \underline{Simplexiale Homologiegruppe}.\\

\textbf{Beispiel 1:} $X = S^1$, mit einer Ecke $v$ und einer Kante $e$.

Dann gilt: 
  \begin{figure}[H]
    \centering
    \includegraphics[width=2cm]{Image Diffgeo/22.11.png}
 \end{figure}

\begin{itemize}
  \item \(
\Delta_1(S^1) \cong \mathbb{Z} \cong\Delta_0(S^1)
\) 
  \item $\partial_1 = 0$, denn $\partial_1(e) = v - v = 0$,  $\ker \partial_1 =\Delta_1(S^1)$
  \item $\partial_0 =0$ nach Definition, $\ker \partial_0 = \Delta_0(S^1)$
  \item $\Delta_n(S^1) = 0$, für $n \ge 2$
  \item
  \(
   H_0^{\Delta}(S^1)=\frac{\ker(\partial_0)}{\txt{im}(\partial_1)}\cong\frac{\mathbb{Z}    }{0}
  \) \quad
    \(
   H_1^{\Delta}(S^1)=\frac{\ker(\partial_1)}{\txt{im}(\partial_2)}\cong\frac{\mathbb{Z}    }{0}
  \)
  \[
 \Rightarrow H_n^{\Delta}(S^1) \cong 
  \begin{cases}
    \mathbb{Z}, & \text{für } n = 0,1, \\
    0, & \text{für } n \ge 2
  \end{cases}
  \]
\end{itemize}

\textbf{Beobachtung:} Sind alle Randabbildungen gleich $0$, so gilt
\[
H_n^{\Delta}(X) \cong \Delta_n(X).
\]

\vspace{1cm}

\textbf{Beispiel 2:} $X = T^2$ mit der Struktur von oben: 1 Ecke $v$; 3 Kanten $a, b, c$; 2 2-Simplexes $U$ und $L$
    \begin{figure}[H]
    \centering
    \includegraphics[width=3cm]{Image Diffgeo/22.12.png}
 \end{figure}

   Es gilt:
   \begin{itemize}
       \item \(
    \partial_1 = 0 \;\txt{(jede Kante bilden immer von v nach v)}\; \Rightarrow \; H_0^\Delta(T^2) \cong \mathbb{Z}
  \)
  
        \item  \(
    \partial_2 U = a + b - c = \partial_2 L
  \)
  
        \item \(
    \{a, b, a + b - c\} \text{ ist eine Basis von } \Delta_1(T^2)
  \; \Rightarrow \; 
    H_1^\Delta(T^2) \cong \mathbb{Z} \oplus \mathbb{Z} \quad \text{mit Basis } [a], [b]
  \)\\
  ($\frac{\ker \partial_1}{\txt{im} \partial_2}=\frac{\Delta_1(T^2)}{[a+b-c]}\underset{=x_1,x_2}{\overset{(x_1,x_2,x_3\;\txt{mod\;}x_3)}{=}}[a]\times [b]$)

  \item Keine 3-Simplexes $\Rightarrow H_2^\Delta(T^2) = \ker \partial_2\;(\txt{im}\partial_3=0) =$ \\
  \(
  \text{unendliche zyklische Gruppe erzeugt von } U - L \cong \mathbb{Z}
  \)
  \[
  \txt{denn:}\;\partial_2(pU + qL) = (p + q)(a + b - c) = 0 
  \quad \Leftrightarrow \quad p = -q
  \]
\[
\Rightarrow H_n^\Delta(T^2) \cong
\begin{cases}
  \mathbb{Z} & n = 0,2 \\
  \mathbb{Z} \oplus \mathbb{Z} & n = 1 \\
  0 & n \geq 3
\end{cases}
\]
   \end{itemize}
\textbf{Beispiel 3:} $X = \mathbb{R}P^2$ mit der Zerlegung von oben:2 Ecken $v, w$; 3 Kanten $a, b, c$; 2 2-Simplexes $U, L$.
    \begin{figure}[H]
    \centering
    \includegraphics[width=3cm]{Image Diffgeo/22.13.png}
 \end{figure}

\begin{itemize}
  \item 
  \(
    \operatorname{im}(\partial_1) = \langle w - v \rangle 
    \; \Rightarrow \; H_0^\Delta(X) \cong \mathbb{Z}
    \quad \text{mit } v \text{ oder } w \text{ als Erzeuger}
  \)\\
  ($\frac{\ker \partial_0}{\txt{im}\partial_1}=\frac{\Delta_0=\txt{all points}}{<w-v>}=\frac{<v>=<w>}{<w-v>}\approx <v>+<w-v>=<w>\cong \mathbb{Z}$)

  \item \(
    \partial_2 U = -a + b + c,\; 
    \partial_2 L = a - b + c \;\Rightarrow \;\partial_2 \text{ ist injektiv} 
    \; \Rightarrow \; H_2^\Delta(X) = 0
  \)

\item 
  \(
    \ker \partial_1 \cong \mathbb{Z} \oplus \mathbb{Z}
    \quad \text{mit Basis } a - b \text{ und } c
  \) (da \(\partial_1(a-b)=0, \partial_1(c)=0\))
 
  \item 
  \(
    \operatorname{im}(\partial_2) \text{ ist eine Index-2-Untergruppe von } \partial_1
  \)
. Wähle $c$, $a - b + c$ als Basis von $\ker \partial_1$:
  \[
    a-b + c=\partial _2L, \; 2c = (a - b + c) + (-a + b + c) 
    = \partial_2 L + \partial_2 U \quad \text{als Basis von } \operatorname{im} \partial_2
  \]
\[
  \Rightarrow H_1^\Delta(X)=\frac{<a-b+c,c>}{<a-b+c,2c>} \cong \mathbb{Z}_2
\]
\end{itemize}

$\Delta$-Komplex-Struktur auf $S^1$: Zwei Kopien von $\Delta^n$, deren Ränder mit der Identität identifiziert werden. Wir bezeichnen die $n$-Simplexes mit $U$ und $L$:
  \[
    \Rightarrow \ker \partial_n = \langle U - L \rangle_{\mathbb{Z}} \cong \mathbb{Z}.
  \]

\vspace{1em}

\noindent\textbf{Offene Frage:} Ist $H_n^\Delta(X)$ unabhängig von der gewählten $\Delta$-Komplex-Struktur?

\medskip

\noindent
M.a.W.: Sei $X$ homöomorph zu $Y$. Ist dann $H_n^\Delta(X)$ isomorph zu $H_n^\Delta(Y)$?

\subsection{Singuläre Homologie}

\textbf{Definition 8.13:} Ein \underline{singulärer $n$-Simplex} in einem topologischen Raum $X$ ist eine stetige Abbildung $\sigma \colon \Delta^n \to X$.\\

\textbf{Bemerkung:} „Singulär“ soll bedeuten, dass $\sigma$ keine Einbettung sein muss, sondern Singularitäten haben darf (nicht injektive Abbilung erlaubt).\\

\textbf{Definition 8.14:} Sei $C_n(X)$ die freie abelsche Gruppe erzeugt von den singulären $n$-Simplexes in $X$. Elemente von $C_n(X)$, genannt \emph{(singuläre) $n$-Ketten}, sind endliche formale Summen
\[
\sum_i n_i \, \sigma_i, \quad n_i \in \mathbb{Z}, \quad \sigma_i \colon \Delta^n \to X.
\]

Die Randabbildung $\partial_n \colon C_n(X) \to C_{n-1}(X)$ ist definiert durch
\[
\partial_n(\sigma) = \sum_i (-1)^i \, \sigma|_{[v_0, \ldots, \hat{v}_i, \ldots, v_n]}.
\]


Die \underline{singuläre Homologiegruppe} ist
\[
H_n(X) = \ker \partial_n / \operatorname{im} \partial_{n+1}.
\]

\textbf{Bemerkung:} Aus der Definition folgt sofort, dass homöomorphe Räume isomorphe Homologiegruppen $H_n$ besitzen. (Im Gegensatz zu $H_n^\Delta(X)$.)

\begin{itemize}
  \item Sei $X$ ein $\Delta$-Komplex mit endlich vielen Simplexes. Es ist nicht klar, ob $H_n(X)$ endlich erzeugt für alle $n$.
  \item Ebenso ist nicht klar, ob $H_k(X) = 0$ für $k > \dim(X)$.
\end{itemize}

\textbf{Bemerkung:} Singuläre Homologie kann als Spezialfall von Simplexialer Homologie aufgefasst werden.\\

Sei $X$ ein beliebiger topologischer Raum, definieren den \underline{singulären Komplex} $S(X)$ als den $\Delta$-Komplex mit einem $n$-Simplex $\Delta^n_\sigma$ 
für jeden singulären $n$-Simplex $\sigma \colon \Delta^n \to X$, wobei $\Delta^n_\sigma$ auf dem offensichtlichen Weg an den $(n-1)$-Simplizes von $S(X)$ angeklebt wird, die den Einschränkungen von $\sigma$ auf die $(n-1)$-Simplizes von $\partial \Delta^n$ entsprechen.
\[
\Rightarrow \quad H_n^\Delta(S(X)) \cong H_n(X).
\]

\textbf{Bemerkung} (Geometrische Interpretation von Zyklen):\\
\underline{Beobachtung:} Eine singuläre $n$-Kette $\xi$ kann immer in der Form 
\[
\xi = \sum_i \varepsilon_i \sigma_i
\]
mit $\varepsilon_i = \pm 1$ geschrieben werden, wobei Abbildungen $\sigma_i$ die $n$-Simplexe darstellen.\\

Gegeben $n$-Kette $\xi = \sum_i \varepsilon_i \sigma_i$, betrachten wir $\partial \xi$ als Summe von singulären $(n-1)$-Simplizes mit Vorzeichen $\pm 1$. Dabei können Paare auftreten, die sich aufheben, d.h. zweimal der gleiche $(n-1)$-Simplex mit entgegengesetzten Vorzeichen. \\

Wählen wir eine maximale Sammlung sich aufhebender Paare, können wir einen $n$-dimensionalen $\Delta$-Komplex $K_\xi$ aus der disjunkten Vereinigung von $n$-Simplizes $\Delta^n_i$ konstruieren, einen für jedes $\sigma_i$. Dabei identifizieren wir die Paare von $(n-1)$-Simplizes aus der Sammlung.\\

Die $\sigma_i$ induzieren eine Abbildung $K_\xi \to X$. Ist $\xi$ ein \emph{Zyklus}, d.h. alle $(n-1)$-dim Seiten der $\Delta^n_i$'s sind in identifizierten Paaren, dann ist $K_\xi$ in einer Umgebung jedes Punktes, der nicht im $(n-2)$-Skelett $K_\xi^{n-2}$ von $K_\xi$ liegt, homöomorph zu $\mathbb{R}^n$, also eine Mannigfaltigkeit.\\

Alle $n$-Simplexe von $K_\xi$ können zusammenpassend orientiert werden, sodass 
\[
K_\xi^n \big/ K_\xi^{n-2}
\]
eine orientierte Mannigfaltigkeit ist.

  \begin{figure}[H]
    \centering
    \includegraphics[width=2cm]{Image Diffgeo/22.16.jpg}
 \end{figure}

\textbf{Bemerkung.} Elemente von $H_1(X)$ sind durch Sammlungen von orientierten Schleifen in $X$ repräsentiert, und Elemente von $H_2(X)$
durch Abbildungen von geschlossenen, orientierten Flächen nach $X$.\\

Man kann zeigen: Ein orientierter $1$-Zyklus $\coprod_\alpha S_\alpha^1 \to X$ ist null in $H_1(X)$, genau dann, wenn es eine Fortsetzung zu einer Abbildung
von einer kompakten, orientierten Fläche mit Rand $\coprod_\alpha S_\alpha^1$ gibt.\\

Die Aussage bleibt wahr für $2$-Zyklen, aber nicht für höhere Dimensionen (In der Nähe von 2-Zykeln trifft keine Aussage über Mannigfaltigkeit). Der Zusammenhang zur Mannigfaltigkeit geht also verloren.
  \begin{figure}[H]
    \centering
    \includegraphics[width=5cm]{Image Diffgeo/22.15.jpg}
 \end{figure}

\textbf{Proposition 8.15.} Zerlegen wir $X$ in seine Wegzusammenhangskomponenten $X_\alpha$, dann ist 
\[
H_n(X) \cong \bigoplus_\alpha H_n(X_\alpha).
\]

\begin{proof}
    Jeder singuläre Simplex hat wegzusammenhängendes Bild, d.h. $C_n(X)$ ist eine direkte Summe der Untergruppen $C_n(X_\alpha)$. Die Randabbildung $\partial_n$ erhält die direkte Summe, d.h. $\partial_n(C_n(X_\alpha)) \subseteq C_{n-1}(X_\alpha)$, 
d.h. auch $\ker \partial_n$ und $\operatorname{im} \partial_{n+1}$ sind direkte Summen, also auch die Homologiegruppen:
\[
H_n(X) \cong \bigoplus_\alpha H_n(X_\alpha)
\]
\end{proof}

\textbf{Proposition 8.16:} Ist $X$ nicht-leer und wegzusammenhängend, dann gilt $H_0(X) \cong \mathbb{Z}$.
Für einen beliebigen Raum $X$ ist $H_0(X)$ eine direkte Summe von $\mathbb{Z}$'s, ein $\mathbb{Z}$ für jede Wegzusammenhangskomponente.

\begin{proof}
Per Definition:
\[
H_0(X) = \frac{\ker(\partial_0)}{\mathrm{im}(\partial_1)} = \frac{C_0(X)}{\mathrm{im}(\partial_1)}
\]
denn $\partial_0 = 0$.

Wir definieren einen Homomorphismus 
\[
\varepsilon : C_0(X) \to \mathbb{Z}, \quad \varepsilon\left( \sum_i n_i \sigma_i \right) := \sum_i n_i.
\]

Für $X \neq \varnothing$ ist die Abbildung surjektiv.

\underline{Ziel:} $\ker(\varepsilon) = \mathrm{im}(\partial_1)$, falls $X$ wegzusammenhängend.

\begin{enumerate}
    \item $\mathrm{im}(\partial_1) \subseteq \ker(\varepsilon)$, denn für jeden singulären 1-Simplexis $\sigma: \Delta^1 \to X$ gilt
    \[
    \partial_1(\sigma) = \sigma|_{[v_1]} - \sigma|_{[v_0]} \Rightarrow \varepsilon(\partial_1(\sigma)) = 1 - 1 = 0.
    \]

    \item $\ker(\varepsilon) \subseteq \mathrm{im}(\partial_1)$: Sei $\sum_i n_i \sigma_i \in \ker(\varepsilon)$, also $\sum_i n_i = 0$.

    Die $\sigma_i$ sind singuläre 0-Simplexes, also Punkte von $X$.

    Wähle einen Weg $\alpha_i: [0,1] \to X$ vom Basispunkt $x_0$ nach $\sigma_i(v_0)$, und sei $\sigma_i$ der singuläre 0-Simplex mit Bild $x_0$.

    Wir betrachten $\tau_i$ als singulären 1-Simplex, es sei $\tau_i : [v_0, v_1] \to X$ und 
\[
\partial_1 \tau_i = \sigma_i - \sigma_0.
\]
Also
\[
\partial_1 \left( \sum_i n_i \tau_i \right)
= \sum_i n_i \partial_1 \tau_i 
= \sum_i n_i \sigma_i - \sum_i n_i \sigma_0 
= \sum_i n_i \sigma_i - \underbrace{\left( \sum_i n_i \right) }_{=0}\sigma_0 
= \sum_i n_i \sigma_i, 
\]
\[
\Rightarrow \sum_i n_i \sigma_i \text{ ist ein Rand, also } \ker(\varepsilon) \subseteq \mathrm{im}(\partial_1).
\]
\end{enumerate}
\end{proof}
  \begin{figure}[H]
    \centering
    \includegraphics[width=5cm]{Image Diffgeo/22.14.jpg}
 \end{figure}

\textbf{Proposition 8.17.} Ist $X$ ein Punkt, dann gilt
\[
H_n(X) =
\begin{cases}
\mathbb{Z} & n = 0, \\
0 & n > 1.
\end{cases}
\]

\begin{proof}
In diesem Fall gibt es genau einen $n$-Simplex $\sigma_n$ für jedes $n$, und 
\[
\partial(\sigma_n) = \sum_i (-1)^i \sigma_{n-1},
\]
eine Summe mit $n+1$ Summanden, also 0 für $n$ ungerade und $\sigma_{n-1}$ für $n$ gerade, $n \ne 0$. Das gibt den Kettenkomplex
\[
\cdots \longrightarrow \mathbb{Z} \xrightarrow{\cong} \mathbb{Z} \xrightarrow{0} \mathbb{Z} \xrightarrow{\cong} \mathbb{Z} \xrightarrow{0}\mathbb{Z}\to 0.
\]

$\Rightarrow$ Die Homologiegruppen sind trivial, außer für $H_0 \cong \mathbb{Z}$.
\end{proof}

\textbf{Bemerkung} (Reduzierte Homologie) Leichte Varianten der singulären Homologie für den Fall $H_n(\{*\}) = 0$ für alle $n$ (insbesondere $n = 0$). Dafür definieren wir die reduzierte Homologie $\tilde{H}_n(X)$ als die Homologiegruppen des erweiterten Kettenkomplexes
\[
\cdots \longrightarrow C_2(X) \xrightarrow{\partial_2} C_1(X) \xrightarrow{\partial_1} C_0(X) \xrightarrow{\varepsilon} \mathbb{Z} \longrightarrow 0,
\]
wobei $\varepsilon\left( \sum_i n_i \sigma_i \right) = \sum_i n_i$ (wie im Beweis von Prop. 8.16). Wir sollten $X \neq \emptyset$ fordern, damit wir keine nicht-triviale Homologie in Dimension $-1$ erzeugen.\\

Es gilt $\varepsilon \circ \partial_1 = 0$, da $\varepsilon$ verschwindet auf $\operatorname{im} \partial_1 \Rightarrow \varepsilon$ induziert eine Abbildung
\[
H_0(X) = \frac{\ker \varepsilon}{\operatorname{im} \partial_1} \longrightarrow \mathbb{Z},
\]
mit Kern $\tilde{H}_0(X)$ 
\[\Rightarrow H_0 \cong \tilde{H}_0(X) \oplus \mathbb{Z}\]

Offensichtlich: $H_n(X) \cong \tilde{H}_n(X)$ für $n > 0$.\\

\underline{Formal:} Das zusätzliche $\mathbb{Z}$ an den erweiterten Ketten-Komplex wird gedacht als erzeugt von der eindeutigen Abbildung
\[
[\emptyset] \longrightarrow X,
\]
wobei $[\emptyset]$ der leere Simplex ohne Ecken ist (mit Dimension -1). Die Abbildung $\varepsilon$ ist dann die gewöhnliche Randabbildung:
\[
\partial[\sigma_0] = [\sigma_1] - [\emptyset].
\]

%%%%%%%%%%%%%%%%%%%%%%%%%%%%%%%%%%%%%%%%%%%%%%%%%%%%%%%%%%%%%%%%%%%%%%%%%%%%%%%%%%%%%%%%%%%%% Vorlesung 23 %%%%%%%%%%%%%%%%%%

\subsection{Eigenschaften der Homologie}

Sei $f \colon X \to Y$ stetig, $C_n(X)$ die singulären $n$-Ketten von $X$. Dann definiert $f$ eine Abbildung
\[
f_\# \colon C_n(X) \to C_n(Y)
\]
als lineare Fortsetzung der Abbildung $\sigma \mapsto f \circ \sigma$. D.h.
\[
f_\#\left( \sum_i n_i \sigma_i \right) = \sum_i n_i f_\# \sigma_i = \sum_i n_i \, f_\#\circ\sigma_i.
\]

Diese Abbildung erfüllt
\[
f_\# \circ \partial = \partial \circ f_\#.
\]

Daraus gilt:
\[
f_\#(\ker \partial) \subseteq \ker \partial \quad \text{und} \quad f_\#(\operatorname{im} \partial) \subseteq \operatorname{im} \partial.
\]

$\Rightarrow f$ induziert einen {Homomorphismus}
\[
f_* \colon H_n(X) \to H_n(Y).
\]

Diese Abbildung hat zwei wichtige Eigenschaften:

\begin{enumerate}
    \item[(i)] $(g \circ f)_* = g_* \circ f_*$ für Abbildungen $X \xrightarrow{f} Y \xrightarrow{g} Z$.
    \item[(ii)] $(\operatorname{id}_X)_* = \operatorname{id}_{H_n(X)}$.
\end{enumerate}

\textbf{Satz 8.18:}
Sind $f, g \colon X \to Y$ stetig und homotop, dann induzieren sie den gleichen Homomorphismus:
\[
f_* = g_* \colon H_n(X) \to H_n(Y).
\]

\textbf{Definition 8.19:}
Eine stetige Abbildung $f \colon X \to Y$ heißt \underline{Homotopieäquivalenz}, wenn es eine stetige Abbildung $g \colon Y \to X$ gibt, sodass $g\circ f$ homotop zur Identität auf $X$ und $f\circ g$ homotop zur Identität auf $Y$ ist.  
Die Abbildung $g$ heißt \underline{Homotopieinverse} von $f$ (diese ist i.A. nicht eindeutig).\\

\textbf{Korollar 8.20:}
Sei $f \colon X \to Y$ eine Homotopieäquivalenz, dann sind die Abbildungen
\[
f_* \colon H_n(X) \to H_n(Y)
\]
{Isomorphismen} (für alle $n$).\\

\textbf{Beispiel:}
$\{0\}$ ist {homotopieäquivalent} zu $\mathbb{R}^n$:
\[
\begin{array}{ccc}
\{0\} & \longrightarrow & \mathbb{R}^n \\
0 & \overset{f}{\longmapsto} & 0 \\
0 & \overset{g}{\longleftarrow} & x
\end{array}
\]
Bleibt zu zeigen: $f\circ g\simeq id_{\mathbb{R}^n}$ (da $g\circ f$ offensichtlich ist)
\begin{proof}
Finde die Homotopie H:
\[
H:(t, x) \mapsto (1 - t)x \colon [0,1] \times \mathbb{R}^n \to \mathbb{R}^n.
\]
\end{proof} 

Sei $X$ ein $\Delta$-Komplex, wir definieren eine Abbildung 
\[
\Delta_n(X) \to C_n(X)
\]
indem wir jeden $n$-Simplex auf die charakteristische $\sigma \colon \Delta^n \to X$ schicken.\\

\textbf{Satz 8.21:} Der induzierte Homomorphismus 
\[
H_n^\Delta(X) \to H_n(X)
\]
ist ein {Isomorphismus}.\\

\textbf{Axiomatischer Blickpunkt}

Eine \underline{Homologie-Theorie} ordnet jedem (nicht-leeren) CW-Komplex $X$ eine Folge von abelschen Gruppen $\widetilde{h}_n(X)$
und jeder stetigen Abbildung $f \colon X \to Y$ zwischen CW-Komplexen eine Folge von Homomorphismen 
\[
f_* \colon \widetilde{h}_n(X) \to \widetilde{h}_n(Y)
\]
zu, sodass $(g \circ f)_* = g_* \circ f_*$ und $\mathrm{id}_X = \mathrm{id}$, und sodass die folgenden drei Axiome erfüllt sind:

\begin{enumerate}
    \item Ist $f$ homotop zu $g$, dann gilt $f_* = g_* \colon \widetilde{h}_n(X) \to \widetilde{h}_n(Y)$.
    
    \item Sei $(X, A)$ ein CW-Paar, dann gibt es einen \emph{Randhomomorphismus}
    \[
    \partial \colon \widetilde{h}_n(X/A) \to \widetilde{h}_{n-1}(A),
    \]
    welcher in die lange exakte Sequenz
    \[
    \cdots \overset{\partial}{\to} \widetilde{h}_n(A) \xrightarrow{\iota_*} \widetilde{h}_n(X) \xrightarrow{q_*} \widetilde{h}_n(X/A) \xrightarrow{\partial} \widetilde{h}_{n-1}(A) \xrightarrow{\iota_*} \cdots
    \]
    passt, wobei $\iota$ die Inklusion $A \to X$ und $q$ die Quotientenabbildung ist.
    \begin{figure}[H]
    \centering
    \includegraphics[width=8cm]{Image Diffgeo/23.02.jpg}
 \end{figure}

    Desweiteren ist die Randabbildung auch natürlich: Sei $f \colon (X, A) \to (Y, B)$, diese Abbildung definiert eine Abbildung
\[
\overline{f} \colon X/A \to Y/B
\]
auf den Quotienten. Dann ist das folgende Diagramm kommutativ:
\begin{figure}[H]
    \centering
    \includegraphics[width=4cm]{Image Diffgeo/23.01.png}
 \end{figure}
 \item Für eine Wedge-Summe $X = \bigvee_\alpha X_\alpha$ mit Inklusionen $\iota_\alpha \colon X_\alpha \hookrightarrow X$
ist der direkte Summenabbildung
\[
\bigoplus_\alpha{\iota_\alpha}_*:\bigoplus_\alpha \widetilde{h}_n(X_\alpha) \rightarrow \widetilde{h}_n(X)
\]
ein Isomorphismus, für alle $n$.
\[
\left[ \text{Eilenberg--Steenrod--Axiome} \right]
\]
\end{enumerate}

\textbf{Bemerkung:}
\begin{itemize}
    \item Es gibt verschiedene Homologien, die obige Definition. erfüllen (d.h. nicht-isomorphe).
    \item Simpliziale/Singuläre Homologien erfüllen
    \[
    H_n(\{\text{pt}\}) =
    \begin{cases}
        \mathbb{Z} & \text{für } n = 0 \\
        0 & \text{sonst}
    \end{cases}
    \]
    Für alle Homologien ist dies nicht der Fall.
\end{itemize}

\vspace{1cm}

\textbf{\underline{Zusammenhang zur Fundamentalgruppe:}}\\

$\pi_1(X,x_0)$ und $H_1(X)$ sind durch Abbildungen $f : [0,1] \to X$ definiert.

Ist $f : [0,1] \to X$ ein Schleife, also $f(0) = f(1)$, dann ist der zugehörige singuläre $1$-Simplex gegeben durch $f$, ein Zyklus, denn
\[
\partial f = f(1) - f(0) .
\]

\textbf{Satz 8.22:} 

Fassen wir Schleifen als singuläre $1$-Zykel auf, erhalten wir einen Homomorphismus
\[
h : \pi_1(X,x_0) \to H_1(X).
\]

Wenn $X$ wegzusammenhängend ist, dann ist $h$ surjektiv und der Kern ist die Kommutator-Untergruppe von $\pi_1(X,x_0)$:
\[
\left\{ a b a^{-1} b^{-1} \;\middle|\; a,b \in \pi_1(X,x_0) \right\} = \mathcal{K}(\pi_1(X)).
\]

D.h. $h$ induziert einen Isomorphismus zwischen der Abelisierung
\[
\pi_1(X,x_0)/\mathcal{K}(\pi_1(X))
\]
und $H_1(X)$.

\begin{proof}
Wir schreiben $f \simeq g$ für homotope Wege mit festgehaltenem Endpunkt  
und $f \sim g$ für homologe Ketten, d.h. $f - g = \partial \sigma$ für ein $2$-Kette $\sigma$.

Diese Relation erfüllt:

\begin{itemize}
  \item[i)] Ist $f$ der konstante Weg, dann gilt $f \sim 0$. \\
  $f$ ist ein Zyklus, denn es ist eine Schleife, und da $H_1(\mathrm{pt}) = 0$, muss $f$ ein Rand sein ($f\in \ker(\partial)$). 
  Explizit: $f$ ist der Rand des konstanten singulären $2$-Simplex $\sigma$ mit dem gleichen Bild wie $f$:
  \[
  \partial \sigma = \sigma |_{ [v_1, v_2]} - \sigma |_{ [v_0, v_2]} + \sigma |_{ [v_0, v_1]} = f - f + f = f.
  \]

  \item[ii)] Ist $f \simeq g$, dann $f \sim g$. \\
  Sei $F : [0,1] \times [0,1] \to X$ eine Homotopie von $f$ nach $g$. \\
  Dann gibt es ein Paar von singulären $2$-Simplexen $\sigma_1$ und $\sigma_2$ in $X$, welche das Quadrat $[0,1] \times [0,1]$ in zwei Dreiecke $[v_0,v_1,v_3]$ und $[v_0,v_2,v_3]$ zerlegen.
      \begin{figure}[H]
    \centering
    \includegraphics[width=4cm]{Image Diffgeo/23.03.png}
 \end{figure}

Wir berechnen
\begin{align*}
  \partial(\sigma_1 - \sigma_2)
  &= \partial \sigma_1 - \partial \sigma_2 \\
  &= \sigma_1|_{[v_3,v_1]} - \sigma_1|_{[v_0,v_1]} + \sigma_1|_{[v_0,v_3]}
     - \left( \sigma_2|_{[v_3,v_2]} - \sigma_2|_{[v_0,v_2]} + \sigma_2|_{[v_0,v_3]} \right) \\
  &= \sigma_1|_{[v_3,v_1]} - \sigma_1|_{[v_0,v_1]} - \sigma_2|_{[v_3,v_2]} + \sigma_2|_{[v_0,v_2]}  \\
  &= \text{const}(f(1)) - f - (-g) + \text{const}(f(0)) \\
  &\sim g - f
\end{align*}
\[
\Rightarrow\quad g - f \text{ ist ein Rand} \quad \Rightarrow\quad f \sim g
\]

\medskip

\item[iii)] $f * g \sim f + g$, wobei $f * g$ die Verkettung von Wegen ist.  
Sei $\sigma : \Delta^2 \to X$ die orthogonale Projektion von $f * g$ auf  
$\sigma([v_0,v_2])$, dann
\[
\partial \sigma = g - f * g + f \;\implies\; f+g\sim f*g
\]
\begin{figure}[H]
    \centering
    \includegraphics[width=4cm]{Image Diffgeo/23.04.png}
 \end{figure}

\item[iv)] \quad $f^- \sim -f$, wobei $f^-$ die Umkehr des Weges $f$.

Das folgt aus der vorigen Beobachtung:
\[
f + f^- \sim f * f^- \sim 0.
\]

Wenden wir (ii) und (iii) auf Schleifen an, erhalten wir einen wohldefinierten Homomorphismus
\[
h : \pi_1(X,x_0) \to H_1(X),
\]
welcher die Homotopieklasse von $f$ auf die Homologieklasse des 1-Zyklus $f$ abbildet.

\medskip
\end{itemize}
\underline{1. Behauptung:} $X$ wegzusammenhängend $\Rightarrow h$ ist surjektiv.\\

Sei $\sum_i n_i \sigma_i$ ein Zyklus, der ein Element von $H_1(X)$ repräsentiert.  
Nach Umbenennung der $\sigma_i$ können wir annehmen, dass $n_i = \pm 1$.  
Laut {iv)} können wir $n_i = +1$ annehmen, sodass wir den 1-Zyklus $\sum_i \sigma_i$ betrachten.

Ist ein $\sigma_i$ keine Schleife, dann folgt aus $\partial\left( \sum_i \sigma_i \right) = 0$, 
dass es ein anderes $\sigma_j$ gibt, sodass die Verkettung $\sigma_i * \sigma_j$ definiert ist.
Laut (iii) können wir die Terme $\sigma_i$ und $\sigma_j$ zu einem einzigen Term $\sigma_i * \sigma_j$ 
zusammenfassen. Führen wir dies fort, enden wir mit dem Fall, in dem jedes $\sigma_i$ eine Schleife ist.\\

Da $X$ wegzusammenhängend ist, können wir einen Weg $\gamma_i$ von $x_0$ zum Basispunkt von $\sigma_i$ wählen. 
Dann gilt:
\[
\gamma_i * \sigma_i * \gamma^-_i \sim \sigma_i \quad \text{(laut (iii) und (iv))}.
\]

Wir können also annehmen, dass $\sigma_i$ eine Schleife mit Basispunkt $x_0$ ist.  
Wir können alle $\sigma_i$'s zu einem einzigen $\sigma$ zusammenfassen (siehe (ii)).
\[
\Rightarrow \text{d.h. das gegebene Element liegt im Bild von } h.
\]

   %%%%%%%%%%%%%%%%%%%%%%%%%%%%%%%%%%%%%%%%%%%%%%%%%%
\underline{2. Behauptung:} $\mathcal{K}(\pi_1(X,x_0)) = \ker(h)$

\medskip

$\mathcal{K}(\pi_1(X,x_0)) \subseteq \ker(h)$, da $H_1(X)$ abelsch ist.
\[
\left[ h(aba^{-1}b^{-1}) = a + b - a - b = 0 \right]
\]

Für die umgekehrte Inklusion zeigen wir, dass alle Klassen $[f] \in \ker(h)$ in der Abelisierung $\pi_1(X,x_0)_{\text{ab}}$ trivial sind.\\

Sei $[f] \in \ker(h)$. Als 1-Zyklus ist $f$ der Rand einer 2-Kette $\sum n_i \sigma_i$. Wieder können wir $n_i = \pm 1$ annehmen. 
Wir können die Kette $\sum \sigma_i$ einem 2-dimensionalen $\Delta$-Komplex $K$ zuordnen, indem wir für jedes $\sigma_i$ einen simplizialen 2-Simplex wählen und bestimmte Paare von Kanten identifizieren.

\begin{figure}[H]
    \centering
    \includegraphics[width=5cm]{Image Diffgeo/23.05.png}
 \end{figure}

Mit der üblichen Randformel
\[
\partial \sigma_i = \tau_{i0} - \tau_{i1} + \tau_{i2}
\quad \Rightarrow \quad
f = \sum_i \partial \sigma_i = \sum_i (\tau_{i0} - \tau_{i1} + \tau_{i2})
\]

Wir können alle außer einem $\tau_{ij}$ in Paare zusammenfassen, für die die Koeffizienten $(-1)^{j} n_i$ in jedem Paar $+1$ und $-1$ sind. Das verbleibende ist gleich $f$. Wir identifizieren Kanten der $\Delta^2_j$'s, die zu den gepaarten $\tau_{ij}$ gehören, wobei wir die Orientierung der Kanten beibehalten $\rightsquigarrow$ Erhalten eine orientierte Fläche $\Sigma$ mit Rand $f$.\\

Zusammenfassung der Abbildung $\sigma_i$ gibt eine Abbildung $\Sigma \to X$.\\

\underline{Blackbox:} Wir können $\sigma$ deformieren, sodass der Rand, der zu $f$ gehört, unverändert bleibt und jede Ecke auf den Basispunkt $x_0$ abgebildet wird.

\medskip

Wir benutzen additive Notation auf der abelschen Gruppe $\pi_1(X)_{\text{ab}}$ und schreiben
\begin{align*}
  [f] &= 0 && \text{(da $h([f]) = 0$)} \\
      &= \left[ \partial \sum \sigma_i \right] && \text{(da $f$ Rand einer 2-Kette ist)} \\
      &= \sum_i [\partial \sigma_i] = 0.
\end{align*}

$\Rightarrow f$ ist ein Produkt von Kommutatoren $\Rightarrow f \in \mathcal{K}(\pi_1(X,x_0))$

\[
\Rightarrow \ker(h) \subseteq \mathcal{K}(\pi_1(X,x_0))
\quad \Rightarrow \quad \ker(h) = \mathcal{K}(\pi_1(X,x_0))
\]
\end{proof}

%%%%%%%%%%%%%%%%%%%%%%%%%%%%%%%%%%%%%%%%%%%%%%%%%%%%%%%%%%%%%%%%%%%%%%%%%%%%%%%%% Vorlesung 24 %%%%%%%%%%%%%%%%%%%%%%%%%%%%%

\section{Ergänzende Themen}
\subsection{Quotientenmannigfaltigkeiten}

\textbf{Definition 9.1.} Sei $G$ eine Gruppe und $\mathcal{M}$ eine Mannigfaltigkeit. Eine \underline{(Links-)Wirkung} von $G$ auf $\mathcal{M}$
ist eine Abbildung 
\[
\theta: G \times \mathcal{M} \to \mathcal{M} \quad (g, p) \mapsto g \cdot p
\]
die die folgenden Eigenschaften erfüllt:
\[
g_1 \cdot (g_2 \cdot p) = (g_1 g_2) \cdot p \quad \forall\, g_1, g_2 \in G,\, p \in \mathcal{M},
\]
\[
e \cdot p = p \quad \forall\, p \in \mathcal{M},
\]
wobei $e$ das neutrale Element von $G$ ist.

Ist $G$ eine Lie-Gruppe, so sprechen wir von einer \emph{glatten Wirkung}, falls $\theta$ eine glatte Abbildung ist.\\

\textbf{Bemerkung.}
\begin{itemize}
    \item Häufig stellen wir uns vor, dass jedes $g \in G$ eine Abbildung $\theta_g = \theta(g, \cdot): \mathcal{M} \to \mathcal{M}$ definiert.\\
    Mit dieser Schreibweise sind die Bedingungen:
    \[
        \theta_{g_1} \circ \theta_{g_2} = \theta_{g_1 g_2}, \qquad \theta_e = \mathrm{id}_{\mathcal{M}}.
    \]
    
    \item Analog kann man Rechtswirkungen als Abbildungen $\mathcal{M} \times G \to \mathcal{M}$ definieren. Die erste Forderung wird dann zu
    \[
        \theta_{g_2} \circ \theta_{g_1} = \theta_{g_1 g_2}.
    \]

    \item Für eine glatte Wirkung ist jedes $\theta_g$ ein Diffeomorphismus von $\mathcal{M}$, denn $\theta_{g^{-1}}$ ist eine glatte Inverse (bijektiv).

    \item Wir schreiben auch $G \curvearrowright \mathcal{M}$.
\end{itemize}

\textbf{Beispiel:}
\begin{enumerate}
    \item $G$ beliebige Lie-Gruppe, $\mathcal{M}$ beliebige glatte Mannigfaltigkeit. Die \underline{triviale Wirkung} von $G$ auf $\mathcal{M}$ ist gegeben durch
    \[
        g \cdot p = p \qquad \forall g \in G,\, p \in \mathcal{M}.
    \]
    Die Wirkung ist glatt.

    \item Die \underline{natürliche Wirkung} von $GL(n, \mathbb{R})$ auf $\mathbb{R}^n$ ist eine Linkswirkung gegeben durch Matrixmultiplikation
    \[
        (A, x) \mapsto A x.
    \]
    Dies ist eine Wirkung, denn
    \[
        \mathbb{I} x = x \quad \text{und} \quad (AB)x = A(Bx).
    \]
    Die Wirkung ist glatt, denn $Ax$ hängt polynomial von den Einträgen von $A$ und $x$ ab.

    \item Jede Lie-Gruppe wirkt durch \underline{Linkstranslation} glatt auf sich selbst.\\
    Allgemein wirkt jede Lie-Untergruppe $H \subseteq G$ durch Linkstranslation auf $G$.

    \item Jede Lie-Gruppe wirkt durch \underline{Konjugation}
    \[
        \theta(g, h) = g h g^{-1}
    \]
    glatt auf sich selbst.

    \item Die Wirkung einer \underline{diskreten Gruppe} $\Gamma$ auf einer Mannigfaltigkeit $\mathcal{M}$ ist glatt genau dann, wenn die Abbildung
    \[
        p \mapsto g \cdot p
    \]
    für jedes $g \in \Gamma$ eine glatte Abbildung auf $\mathcal{M}$ ist. Z.\,B. ist die Wirkung von $\mathbb{Z}^n$ auf $\mathbb{R}^n$ glatt:
    \[
        (m_1, \dots, m_n) \cdot (x_1, \dots, x_n) = (x_1 + m_1, \dots, x_n + m_n).
    \]
\end{enumerate}

\textbf{Definition 9.2}

Sei $\theta: G \times \mathcal{M} \to \mathcal{M}$ eine Links\-wirkung der Gruppe $G$ auf der Menge $\mathcal{M}$. (Für diese Definition sind weder Stetigkeit noch Differenzierbarkeit notwendig)

\begin{itemize}
    \item Für alle $p \in \mathcal{M}$ ist die \underline{Orbit von $p$} oder \underline{Bahn} $G \cdot p$ die Menge
    \[
        G(p) = G \cdot p = \{ g \cdot p \mid g \in G \}.
    \]
    (Also die Menge aller Bilder von $p$ unter den Elementen von $G$.)
    
    \item Für jedes $p \in \mathcal{M}$ ist der \underline{Stabilisator} oder die \underline{Isotropiegruppe} von $p$, bezeichnet mit $G_p$, die Menge der Elemente von $G$, die $p$ fixieren:
    \[
        G_p = \{ g \in G \mid g \cdot p = p \}.
    \]
    $G_p$ ist eine Untergruppe von $G$.
    
    \item Die Wirkung heißt \underline{transitiv}, falls es für jedes Paar $p, q \in \mathcal{M}$ ein $g \in G$ mit $g \cdot p = q$ gibt.\\
    (\emph{Äquivalent:} Der einzige Orbit ist ganz $\mathcal{M}$.)
    
    \item Die Wirkung heißt \underline{frei}, falls das einzige Element von $G$, das ein Element von $\mathcal{M}$ festhält, die Identität ist, d.\,h.
    \[
        g \cdot p = p \quad \text{für ein } p \in \mathcal{M} \ \Rightarrow\ g = e.
    \]
    (\emph{Äquivalent:} Alle Stabilisatoren sind trivial.)
\end{itemize}

\textbf{Beispiele} (Nummerierung wie oben):
\begin{enumerate}
    \item $G ,\ M$ beliebig,\quad $g \cdot p = p$. Dann gilt $G \cdot p = \{p\},\quad G_p = G$.

    \item $\mathrm{GL}(n) \curvearrowright \mathbb{R}^n$:\quad Für je zwei $v, w \in \mathbb{R}^n \setminus \{0\}$ gibt es eine invertierbare lineare Abbildung $A$ mit $Av = w$. Für $p=0$ oder $p\neq 0$ in $\mathbb{R}^n$:
    \[
        \Rightarrow\quad G \cdot p = \{0\} \quad \text{oder} \quad \mathbb{R}^n \setminus \{0\}.
    \]

    \item $G \curvearrowright G$, \quad $(g, h) \mapsto g h$.\quad Gegeben $g_1, g_2$, dann gibt es genau eine Linkstranslation auf $G$, die $g_1$ auf $g_2$ abbildet (nämlich $g_2 g_1^{-1}$).\\
    $\Rightarrow$ Die Wirkung ist \emph{frei} und \emph{transitiv}.

    \item $\mathbb{Z}^n \curvearrowright \mathbb{R}^n$ ist \emph{frei}, aber nicht \emph{transitiv}.
\end{enumerate}

Seien \(E, M\) topologische Räume, \(\pi: E \to M\) eine Überlagerung und \(\theta: E \to E\) ein Homöomorphismus mit
\[
\pi \circ \theta = \pi.
\]
\begin{center}
\begin{tikzpicture}[baseline=(current bounding box.center)]
\node (E1) at (0,1.2) {\(E\)};
\node (E2) at (2,1.2) {\(E\)};
\node (M)  at (1,0) {\(M\)};
\draw[->] (E1) to node[above]{\(\theta\)} (E2);
\draw[->] (E1) to node[left]{\(\pi\)} (M);
\draw[->] (E2) to node[right]{\(\pi\)} (M);
\end{tikzpicture}
\end{center}

Wir bezeichnen mit \(\operatorname{Aut}_\pi(E)\) die Menge aller solcher Homöomorphismen, die \emph{Automorphismengruppe von \(\pi\)} mit der Komposition als Verknüpfung.

\(\operatorname{Aut}_\pi(E)\) wirkt auf \(E\).

\paragraph{Bemerkung:} \(\operatorname{Aut}_\pi(E)\) wirkt transitiv auf \emph{jedes Faser} von \(\pi\) genau dann, wenn \(\pi_* \left( \pi_1(E, q) \right)\) ist eine \emph{normale Untergruppe} von \(\pi_1(M, \pi(q))\) für alle \(q \in E\),
\[
\left( \forall g \in G: \quad g \mathcal{N} g^{-1} = \mathcal{N} \right).
\]

\textbf{Proposition 9.3.} \\
Seien \(E, M\) Mannigfaltigkeiten und \(\pi: E \to M\) eine Überlagerung. Versehen mit der diskreten Topologie ist die Automorphismengruppe \(\operatorname{Aut}_\pi(E)\) eine \emph{nulldimensionale} Lie-Gruppe, die \emph{glatt} und \emph{frei} auf \(E\) wirkt.
\vspace{0.8cm}

Sei \( G \) eine Gruppe und \( M \) ein topologischer Raum, \( G \) wirke auf \( M \), \( (g, p) \mapsto g \cdot p \). Wir definieren eine Relation auf \( M \) durch \( p \sim q \), falls es ein \( g \in G \) gibt, sodass \( g \cdot p = q \). Das ist eine Äquivalenzrelation auf \( M \) und die Äquivalenzklassen sind die \emph{Orbits von \( G \) in \( M \)}. Die Menge der Orbits \( M / G \) versehen wir mit der Quotiententopologie und nennen sie den \underline{Orbitraum}. \\[1em]

\textbf{Lemma 9.4.} 

Für jede stetige Wirkung einer topologischen Gruppe \( G \) auf einem topologischen Raum \( M \) ist die Quotientenabbildung \( \pi : M \to M / G \) eine \emph{offene} Abbildung.\\


Es gibt viele Lie-Gruppen und Wirkungen, sodass \( M/G \) wieder eine \emph{Mannigfaltigkeit} ist. \\

\textbf{Beispiele:} (Orbiträume für glatte Lie-Gruppen-Wirkungen)
\begin{enumerate}
  \item \( G \) beliebige Gruppe, \( M \) Mannigfaltigkeit, \( g \cdot p = p \) für alle \( g \in G, p \in M \). Jeder Orbit besteht aus einem Punkt \( \Rightarrow M/G = M \) ist Mannigfaltigkeit (altriviales Beispiel).

  \item Einfaches nicht-triviales Beispiel: \( G = \mathbb{R}^k \), \( M = \mathbb{R}^k \times \mathbb{R}^n \), wirkt durch Translation in dem \( \mathbb{R}^k \)-Faktor:
  \[
    v \cdot (x, y) = (v + x, y)
  \]
  Die Orbits sind Unterräume parallel zu \( \mathbb{R}^k \) und der Orbitraum
  \[
    \mathbb{R}^k \times \mathbb{R}^n / \mathbb{R}^k
  \]
  ist homöomorph zu \( \mathbb{R}^n \). \\
  Die Quotientenabbildung \( \pi: \mathbb{R}^k \times \mathbb{R}^n \to \mathbb{R}^n \) ist eine glatte Submersion.


  \item Die Gruppe \( S^1 \) wirkt auf \( \mathbb{C} \) durch komplexe Multiplikation: \( z \cdot w = zw \). \\
  Die Orbits sind Kreise um den Ursprung oder \( \{0\} \). \\
  Der Orbitraum ist \( [0, \infty) \), aber \emph{keine} Mannigfaltigkeit. \\
  Die Quotientenabbildung \( \pi: \mathbb{C} \to \mathbb{C}/S^1 \), \( z \mapsto |z| \).

  \item \( \mathrm{GL}(n, \mathbb{R}) \curvearrowright \mathbb{R}^n \): Zwei Orbits: \( \{0\} \) und \( \mathbb{R}^n \setminus \{0\} \). \\
  Offene Mengen im Quotienten: \( \varnothing, \quad \mathbb{R}^n / \mathrm{GL}(n, \mathbb{R}), \quad \left\{ [\mathbb{R}^n \setminus \{0\}] \right\} \) \\
  \(\Rightarrow\) nicht hausdorffsch, also keine Mannigfaltigkeit.

  \item \( \mathrm{O}(n) \curvearrowright \mathbb{R}^n \) durch Matrixmultiplikation: Die Orbits sind Sphären um den Ursprung und \( \{0\} \). Der Orbitraum ist \( [0, \infty) \).

  \item Entfernt man \( 0 \) aus den vorigen Beispielen, werden die Quotienten zu Mannigfaltigkeiten.
\end{enumerate}

\textbf{Beispiel 9.5}

Sei \( \alpha \in \mathbb{R} \setminus \mathbb{Q} \). \(\mathbb{R}\) wirke auf \( T^2 = S^1 \times S^1 \) durch
\[
t \cdot (w, z) = \left( e^{2\pi i t} w,\, e^{2\pi i \alpha t} z \right).
\]

Die Wirkung ist glatt, frei und hat dichte Orbits.

\(\Rightarrow\) Die einzigen offenen Mengen in \( T^2 / \mathbb{R} \) sind \( \varnothing \) und \( T^2 / \mathbb{R} \). \\
\textit{Nicht hausdorffsch} \( \Rightarrow \) Keine Mannigfaltigkeit.
  \begin{figure}[H]
    \centering
    \includegraphics[width=14cm]{Image Diffgeo/24.01.png}
    \caption{Alle Linien mit irrationaler Neigung können unendlich oft um den Torus gewickelt werden, ohne sich jemals zu schließen}
 \end{figure}

Um solche Beispiele zu verhindern, müssen wir unsere Wirkungen beschränken.\\

\textbf{Definition 9.6}

Eine stetige Links­wirkung einer Lie-Gruppe \( G \) auf einer Mannigfaltigkeit \( M \) heißt 
\underline{eigentliche Wirkung}, falls die Abbildung
\[
\begin{aligned}
G \times M &\longrightarrow M \times M \\
(g, p) &\longmapsto (g \cdot p,\, p)
\end{aligned}
\]
\emph{eigentlich} ist (d.h. Urbilder kompakter Mengen sind kompakt).\\

\textbf{Proposition 9.7}

Wirkt die Lie-Gruppe \( G \) stetig und eigentlich auf der Mannigfaltigkeit \( M \), 
dann ist der Orbitraum \( M / G \) hausdorffsch.\\


\textbf{Proposition 9.8} (Charakterisierung eigentlicher Wirkungen:)

Sei \( M \) eine Mannigfaltigkeit und \( G \) eine Lie-Gruppe, die stetig auf \( M \) wirkt. \\
Die folgenden Aussagen sind äquivalent:
\begin{enumerate}
  \item Die Wirkung ist eigentlich.
  \item Ist \( (p_i) \) eine Folge in \( M \) und \( (g_i) \) eine Folge in \( G \), sodass \( (p_i) \) und \( (g_i\cdot p_i) \) konvergieren, dann hat \( (g_i) \) eine konvergente Teilfolge.
  \item Für alle kompakten Teilmengen \( K \subseteq M \) ist die Menge
  \[
    G_K = \{ g \in G \mid g K \cap K \neq \emptyset \}
  \]
  kompakt.
\end{enumerate}


\textbf{Korollar 9.9}

Jede stetige Wirkung einer kompakten Lie-Gruppe auf einer Mannigfaltigkeit ist eigentlich.


\begin{proof}
Seien \( (p_i) \) und \( (g_i) \) wie in Bedingung (2) von Proposition 9.8, dann hat \( (g_i) \) eine konvergente Teilfolge, da jede Folge in \( G \) eine konvergente Teilfolge besitzt.

\bigskip

(\emph{Lie-Gruppe} \( \Rightarrow \) \emph{Mannigfaltigkeit} \( \Rightarrow \) \emph{metrischer Raum} \\
Daher ist \emph{Überdeckungskompaktheit} und \emph{Folgenkompaktheit} äquivalent.)
\end{proof}

\textbf{Proposition 9.10} (Orbitsen eigentlicher Wirkungen)

Sei \( \theta \) eine eigentliche glatte Wirkung der Lie-Gruppe \( G \) auf der Mannigfaltigkeit \( M \).

Für jeden Punkt \( p \in M \) ist die Orbitabbildung
\[
\theta^{(p)} : G \to M
\]
eine \textit{eigentliche} Abbildung, und der Orbit \( G \cdot p = \theta^{(p)}(G) \) ist \textit{abgeschlossen} in \( M \).

Ist zusätzlich \( G_p = \{e\} \), dann ist \( \theta^{(p)} \) eine glatte \textit{Einbettung} und der Orbit ist eine \textit{eigentlich eingebettete Untermannigfaltigkeit}.\\


\textbf{Korollar 9.11}

Wirkt eine Lie-Gruppe \( G \) {eigentlich} auf der Mannigfaltigkeit \( M \), dann ist jeder Orbit 
eine {abgeschlossene} Teilmenge von \( M \) und jede {Isotropiegruppe} ist {kompakt}.

\begin{proof}
Die erste Aussage folgt aus Proposition 9.10.

Die zweite aus Proposition 9.8, da die Isotropiegruppe eines Punktes \( p \) die Menge 
\[
G_K \text{ für } K = \{p\}
\]
ist.
\end{proof}

\textbf{Beispiel:} Die Wirkung \( \mathbb{R}^+ \) auf \( \mathbb{R}^n \) gegeben durch
\[
t \cdot (x_1, \dots, x_n) = (t x_1, \dots, t x_n)
\]
ist {nicht eigentlich}.

\begin{itemize}
    \item 1. Möglichkeit: Die Isotropiegruppe von \( 0 \in \mathbb{R}^n \) ist \( \mathbb{R}^+ \), also {nicht kompakt}.
    \item 2. Möglichkeit: Die anderen Orbits sind {offene Strahlen}, also nicht {abgeschlossen} in \( \mathbb{R}^n \).
\end{itemize}

\textbf{Theorem 9.12} (Quotientenmannigfaltigkeitstheorem)

Sei \( G \) eine Lie-Gruppe, die {glatt}, {frei} und {eigentlich} auf der Mannigfaltigkeit \( M \) wirkt.

Dann ist der Orbirtraum \( M / G \) eine topologische Mannigfaltigkeit der Dimension 
\[
\dim M - \dim G,
\]
und hat eine eindeutige glatte Struktur mit der Eigenschaft, dass die Quotientenabbildung
\[
\pi : M \to M/G
\]
eine {glatte Submersion} ist.
  \begin{figure}[H]
    \centering
    \includegraphics[width=6cm]{Image Diffgeo/24.02.png}
 \end{figure}

\textbf{Beispiele:}
\begin{enumerate}
  \item \(\mathbb{C}P^n = \left( \mathbb{C}^{n+1} \setminus \{0\} \right) / \mathbb{C}^* 
        = S^{2n+1} / S^1\)
        
        Dabei gilt: \( \mathbb{C}^* \cong (0, \infty) \times S^1 \)

  \item \( S^2 = \mathrm{SO}(3) / S^1 \)
  
  \item {Triviale Wirkung:} \( G \curvearrowright M,\quad g \cdot p = p \)

        \hspace{1em} $\Rightarrow$ hat „falsche“ Dimension (erfüllt Theorem 9.12 nicht)

  \item \( S^2/\mathbb{Z}_3,\quad \mathbb{C}/\mathbb{Z}^3 \quad \leadsto \) {Orbifolds}
\end{enumerate}

%%%%%%%%%%%%%%%%%%%%%%%%%%%%%%%%%%%%%%%%%%%%%%%%%%%%%%%%%%%%%%%%%%%%%%%%%%%%%%%%%%% Vorlesung 25 %%%%%%%%%%%%%%%%%%%%%%%%%%%

\subsection{Zerlegung der Eins}
\textbf{Lemma 3.4.} \\
Sei $R > 0$, dann existiert ein glattes Feld $\chi_R : \mathbb{R}^n \to [0,1]$ mit
\[
\chi_R \big|_{\overline{B_{\frac{1}{3}R}(0)}} \equiv 1
\quad \text{und} \quad
\chi_R \big|_{\mathbb{R}^n \setminus B_{\frac{2}{3}R}(0)} \equiv 0.
\]

{Insbesondere:} Diese Funktion ist
\begin{itemize}
    \item glatt auf $\mathbb{R}^n$,
    \item identisch $1$ für $|x| \leq \frac{1}{3}R$,
    \item identisch $0$ für $|x| \geq \frac{2}{3}R$.
\end{itemize}
      \begin{figure}[H]
    \centering
    \includegraphics[width=4cm]{Image Diffgeo/25.01.png}
 \end{figure}

{Es gibt eine alternative Konstruktion zu diesen Funktionen.}

Dazu sei
\[
e_1 \colon \mathbb{R} \to \mathbb{R}, \quad
e_1(t) =
\begin{cases}
0, & t \leq 0 \\
e^{-\frac{1}{t}}, & t > 0
\end{cases}
\]
      \begin{figure}[H]
    \centering
    \includegraphics[width=6cm]{Image Diffgeo/25.02.png}
 \end{figure}
\[
e_2 \colon \mathbb{R} \to \mathbb{R}, \quad
e_2(t) = e_1(t) \cdot e_1(1 - t)
\]
      \begin{figure}[H]
    \centering
    \includegraphics[width=6cm]{Image Diffgeo/25.03.png}
 \end{figure}
\[
e_3 \colon \mathbb{R} \to \mathbb{R}, \quad
e_3(t) = \frac{1}{C} \int_{-\infty}^{t} e_2(s)\, ds
\quad \text{mit} \quad
C = \int_{-\infty}^{\infty} e_2(s)\, ds
\]
      \begin{figure}[H]
    \centering
    \includegraphics[width=6cm]{Image Diffgeo/25.04.png}
 \end{figure}
{Denn $e_3$ ist glatt und es gilt:}
\[
e_3(t) =
\begin{cases}
0, & t \leq 0 \\
\in (0,1), & t \in (0,1) \\
1, & t \geq 1
\end{cases}
\quad \Rightarrow \quad
e_3(1 - t) = 1 - e_3(t)
\]
{Für $0 < a < b$ und $q \in \mathbb{R}^n$ definieren wir:}
\[
S_{a,b,q} \colon \mathbb{R}^n \to \mathbb{R}, \quad
S_{a,b,q}(x) = 1 - e_3 \left( \frac{|x - q| - a}{b - a} \right)
\]
{Dann gilt:}
\[
S_{a,b,q}(x) =
\begin{cases}
1, & x \in \overline{B_a(q)} \\
\in (0,1), & x \in B_b(q) \setminus \overline{B_a(q)} \\
0, & x \notin B_b(q)
\end{cases}
\]
{Daraus können wir eine allgemeine Aussage ableiten:}

\vspace{1em}

\textbf{Lemma 3.13.} \\
Sei $M$ eine Mannigfaltigkeit, $K \subseteq U \subseteq M$ mit $K$ kompakt und $U$ offen. Dann gibt es für alle $f \colon U \to \mathbb{R}$ glatt eine Funktion $\widetilde{f} \colon M \to \mathbb{R}$ mit
\[
\widetilde{f}\big|_K = f
\quad \text{und} \quad
\operatorname{supp}(\widetilde{f}) = \overline{\left\{ p \in M \,\middle|\, \widetilde{f}(p) \neq 0 \right\} }\subseteq U.
\]

\underline{Ziel:} \\
Gegeben sei eine offene Überdeckung $\{U_i\}_{i \in I}$ und eine Funktion $f \colon M \to \mathbb{R}$. \\[0.5em]
Finde Funktionen $\tilde{f}_i \colon M \to \mathbb{R}$ mit:
\begin{itemize}
    \item $\operatorname{supp}(\tilde{f}_i) \subseteq U_i$
    \item $\displaystyle \sum_{i \in I} \tilde{f}_i = f$ \hfill (potenziell unendliche Indexmenge)
\end{itemize}

\vspace{1em}

\underline{Wohldefiniert:} \\
Verlange, dass die Träger $\operatorname{supp}(\tilde{f}_i)$ {lokal endlich} sind. \\[0.5em]
„Lokal endlich“ heißt: Jeder Punkt hat eine Umgebung, die nur mit endlich vielen der Mengen (nicht-leer) schneidet.

\vspace{1em}

\textbf{Bemerkung:} Es reicht, diese Funktionen für $f \equiv 1$ zu finden. Für andere $f$ setze dann $\tilde{f}_i := f \cdot \chi_i$.\\

\textbf{Definition 3.14.} \\
Eine \underline{Zerlegung der Eins} ist eine Familie
\[
\mathcal{S} = \{ s_i \colon M \to \mathbb{R} \}_{i \in I}
\]
von glatten Funktionen, so dass:
\begin{enumerate}
    \item die Familie der Tr\"ager $\{ \operatorname{supp}(s_i) \}_{i \in I}$ eine lokal endliche \"Uberdeckung von $M$ ist,
    \item $0 \leq s_i \leq 1$ f\"ur alle $i \in I$ und $\displaystyle 1 = \sum_{i \in I} s_i$.
\end{enumerate}
\textbf{Bemerkung.} Eigenschaft 2 $\Rightarrow$ $\{ \operatorname{supp}(s_i) \}$ ist eine \"Uberdeckung von $M$.\\

\textbf{Beispiel:}
\[
\mathcal{M} = (0, \infty), \qquad \mathcal{U} = \left\{ \{k, k+2\} \right\}_{k \in \mathbb{N}_0}
\]
\[
\mathbb{R} \setminus \left\{ S_{\frac{1}{4}, \frac{3}{4}, k+1} \right\}_{k \in \mathbb{N}_0}
\cup \left\{ S_{\frac{5}{4}, \frac{7}{4}, 0} \right\}
\]
mit
\[
S_{a,b,q}(x) =
\begin{cases}
1 & \text{für } x \in B_a(q), \\
\in (0,1) & \text{für } x \in B_b(q) \setminus \overline{B_a(q)}, \\
0 & \text{für } x \notin B_b(q)
\end{cases}
\]
      \begin{figure}[H]
    \centering
    \includegraphics[width=6cm]{Image Diffgeo/25.05.png}
 \end{figure}

ist eine Zerlegung der Eins bezüglich $\mathcal{U}$.
      \begin{figure}[H]
    \centering
    \includegraphics[width=6cm]{Image Diffgeo/25.06.png}
 \end{figure}
\textbf{Bemerkung:}
\begin{itemize}
    \item Dieses Beispiel zeigt, dass Träger nicht immer kompakt sind:
    \[
    \operatorname{supp} \left( S_{\frac{5}{4}, \frac{7}{4}, 0} \right) = \left(0, \frac{7}{4}\right]
    \]
    
    \item Aber man kann zeigen: maximal abzählbar viele der Träger sind nicht leer.
\end{itemize}

\textbf{Satz 9.15}\\
Sei $M$ hausdorffsch mit präkompakter, abzählbarer Basis der Topologie $\mathcal{B}$.\\

\textnormal{(Präkompakt: Alle Elemente in $\mathcal{B}$ haben kompakten Abschluss.)}

Dann gilt: Für alle offenen Überdeckungen $\mathcal{U}$ von $M$ existiert eine Teilüberdeckung 
\[
\mathcal{U}' \text{ von } \mathcal{B} \text{ sodass } \overline{\mathcal{U}'} := \left\{ \overline{B} \mid B \in \mathcal{U}' \right\}
\]
eine lokal endliche Verfeinerung von $\mathcal{U}$ ist.\\

\textnormal{(Verfeinerung: Für alle $B \in \mathcal{U}'$ existiert ein $U \in \mathcal{U}$ mit $\overline{B} \subseteq U$.)}
\vspace{1em}

\textbf{Satz 9.16}\\
Für alle offenen Überdeckungen $\mathcal{U} = \{U_i\}_{i \in I}$ einer glatten Mannigfaltigkeit existieren
Zerlegungen der Eins
\[
\mathcal{R}^0_{\mathcal{U}} = \{\rho_j^0\}_{j \in J'} \quad \text{und} \quad \mathcal{R}^1_{\mathcal{U}} = \{\rho^1_i\}_{i \in I}
\]
mit
\begin{itemize}
    \item für alle $j \in J'$ ist $\operatorname{supp}(\varphi_j)$ kompakt und es gilt: $\exists \, i(j) \in I$ mit $\operatorname{supp}(\varphi_j) \subseteq U_{i(j)}$.
    
    \item $\mathcal{R}^1_{\mathcal{U}}$ ist eine Zerlegung der Eins bezüglich $\mathcal{U}$.
\end{itemize}

\textit{\small{(Keine Zerlegung bzgl. $\mathcal{U}$, da $i$ nicht injektiv sein muss.)}}

\begin{proof}
Man kann zeigen: Es gibt eine abzählbare Basis 
\[
\mathcal{B} = \left\{ B_j := \varphi_j^{-1}(B_{2a_j}(q_j)) \right\}_{j \in J}
\]
von $M$, mit $V \subseteq B \in \mathcal{B}$ existiert eine Karte, die $B$ auf eine offene Kugel in $\mathbb{R}^n$ abbildet, und $\mathcal{B}$ ist präkompakt.

\medskip

\textit{(benutzt Zweitzählbarkeit von $M$)}

\medskip

Nach Satz 9.15: Es gibt eine Teilüberdeckung 
\[
\mathcal{B}' = \left\{ B_{j} \right\}_{j \in J'}
\quad \text{von } \mathcal{B}, \text{ sodass } 
\quad \overline{\mathcal{B}'} = \left\{ \overline{B_{j}} \right\}_{j \in J'}
\]
eine lokal endliche Verfeinerung von $\mathcal{U}$ ist.

\medskip

Definiere $\rho_j \colon M \to \mathbb{R}$ durch
\[
\rho_j(x) =
\begin{cases}
\rho_{a_j, 2a_j, q_j} \circ \varphi_j(x) & \text{für } x \in B_j, \\
0 & \text{sonst}.
\end{cases}
\]

Dann gilt:
\[
\operatorname{supp}(\rho_j) = \overline{B_j}.
\]

Definiere eine Funktion $\sigma \colon M \to \mathbb{R}$ durch
\[
\sigma := \sum_{j \in J'} \rho_j
\]

Für alle $p \in M$ gilt: Es gibt eine Karte $(U_p, \varphi_p)$ und eine Teilmenge $J_p' \subseteq J'$, sodass
\begin{itemize}
    \item $J_p'$ endlich ist und
    \item $\rho_j|_{U_p} = 0$ für alle $j \notin J_p'$ \quad (denn $\overline{\mathcal{B}}$ ist lokal endlich).
\end{itemize}

Dann ist $\sigma|_{U_p} = \sum_{j \in J_p'} \rho_j$ eine (endliche) Summe glatter Funktionen.
\[
\Rightarrow \sigma \text{ ist glatt}.
\]

\underline{Behauptung:} $\sigma(p) > 0$ für alle $p \in M$.

Sei $p \in M$. Da $\rho_j \geq 0$, genügt es zu zeigen, dass es ein $\rho_j$ mit $\rho_j(p) > 0$ gibt.\\

$\mathcal{B}'$ Überdeckung: Für $p \in M$ gilt:
\[
p \in \varphi_{j_p}^{-1}(B_{2a_{j_p}}(q_{j_p}))
\]

Dann ist
\[
\rho_{a_{j_p}, 2a_{j_p}, q_{j_p}} \circ \varphi_{j_p} > 0 \quad \text{(in Umgebung von $p$)}
\quad \Rightarrow \text{Beh.}
\]

\medskip

\textcolor{orange}{\small $\sigma$ ist überall positiv, müssen noch normalisieren.}

\medskip

Wir setzen für $\rho_j^0 \colon M \to \mathbb{R}$
\[
\rho_j^0 := \frac{\rho_j}{\sigma}
\]

\underline{Behauptung:} $\mathcal{R}_{\mathcal{U}}^0 := \{\rho_j^0\}_{j \in J'}$ ist eine Zerlegung der Eins.

\begin{itemize}
    \item $\overline{\mathcal{B}'} = \left\{ \operatorname{supp}(\rho_j) = \operatorname{supp}(\rho_j') \right\}_{j \in J'}$ ist lokal endliche Überdeckung, kompakt.
    
    \item 
    \(
    \sum_{j \in J_p'} \rho_j^0 = \sum_{j \in J_p'} \frac{\rho_j}{\sigma} = \frac{1}{\sigma} \sum_{j \in J_p'} \rho_j = \frac{\sigma}{\sigma} = 1
    \)
    
    \item $\overline{\mathcal{B}'}$ ist Verfeinerung von $\mathcal{U}$: Für jedes $j \in J'$ gibt es ein $i(j) \in I$, sodass
    \[
    \overline{B_j} \subseteq U_{i(j)}
    \]
\end{itemize}

Um aus $\mathcal{R}_{\mathcal{U}}^1$ eine Zerlegung der Eins \emph{bezüglich} $\mathcal{U}$ zu erhalten, modifizieren wir $\mathcal{R}_{\mathcal{U}}^0$.\\
(→ Ziel: Eine Zerlegung der Eins bezüglich $\mathcal{U}$ zu erhalten.)

\medskip

Die Familie der Träger von $\mathcal{R}_{\mathcal{U}}^0$ ist lokal endlich.\\

Setze
\[
\rho_i^1 \colon M \to \mathbb{R}, \qquad \rho_i^1 := \sum_{\substack{j \in J' \\ i(j) = i}} \rho_j^0
\]

\underline{Behauptung:} $\mathcal{R}_{\mathcal{U}}^1 := \{ \rho_i^1 \}_{i \in I}$ ist eine Zerlegung der Eins bezüglich $\mathcal{U}$.

\medskip

Nach Konstruktion gilt:
\[
\sum_{i \in I} \rho_i^1 
= \sum_{i \in I} \sum_{\substack{j \in J' \\ i(j) = i}} \rho_j^0 
= \sum_{j \in J'} \rho_j^0 
= 1
\]
\underline{Behauptung:} 
\[
\operatorname{supp}(\rho_i^1) = \bigcup_{\substack{j \in J' \\ i(j) = i}} \operatorname{supp}(\rho_j^0)
\]
Daraus folgt:
\[
\left\{ p \in M \mid \rho_i^1(p) \neq 0 \right\} = \bigcup_{\substack{j \in J' \\ i(j) = i}} \left\{ p \in M \mid \rho_j^0(p) \neq 0 \right\}
\]
{Bei lokal endlichen Familien vertauschen Abschluss und Vereinigung.}
\[
\operatorname{supp}(\rho_j^0) \subseteq U_{i(j)} = U_i \quad \Rightarrow \quad \operatorname{supp}(s_i^1) \subseteq U_i
\]

\underline{Behauptung:} Die Familie $\{\operatorname{supp}(\rho_i^1)\}_{i \in I}$ ist eine lokal endliche Familie.

\medskip

Für alle $p \in M$ existieren eine Umgebung $V_p$ von $p$ und eine endliche Menge $J_p' \subseteq J'$ mit
\[
\operatorname{supp}(\rho_j^0) \cap V_p = \varnothing \quad \text{für alle } j \notin J_p'
\]

Definiere die endliche Teilmenge:
\[
I_p := \{ i(j) \mid j \in J_p' \}
\]

Wenn $i \notin I_p$, dann gilt:
\[
\rho_i^1|_{V_p} = \sum_{\substack{j \in J' \\ i(j) = i}} \rho_j^0|_{V_p} = 0  \;\Rightarrow \;\operatorname{supp}(\rho_i^1) \cap V_p = \varnothing 
\]
\end{proof}

\textit{Topologische Anmerkung:} Sei $M$ topologischer Raum, $U \subseteq M$ offen, $S \subseteq M$. Dann gilt:
\[
S \cap U \neq \varnothing \quad \Longleftrightarrow \quad \overline{S} \cap U \neq \varnothing
\]
und
\[
S \cap U = \varnothing \quad \Longleftrightarrow \quad \overline{S} \cap U = \varnothing
\]

\textbf{Korollar 9.17}\\
Sei $U \subseteq M$ offen, $f \colon U \to \mathbb{R}$ glatt. Dann:
\begin{enumerate}
    \item Für alle $C \subseteq M$ abgeschlossen mit $C \subseteq U$ existiert eine glatte Funktion $\tilde{f}_C \colon M \to \mathbb{R}$ mit
    \[
    \operatorname{supp}(\tilde{f}_C) \subseteq U \quad \text{und} \quad \tilde{f}_C|_C = f|_C
    \]

    \item Es gibt eine Familie glatter Funktionen $\{f_i \colon M \to \mathbb{R}\}_{i \in I}$, sodass die Familie $\{ \operatorname{supp}(f_i) \}_{i \in I}$ eine lokal endliche Familie von kompakten Mengen ist, und
    \[
    f = \sum_{i \in I} f_i
    \]
\end{enumerate}

\subsection{Vergleichsgeometrie}

\underline{Euklidische Geometrie (von $\mathbb{R}^2$)}

\begin{itemize}
    \item Was ist die kürzeste Verbindung zwischen zwei Punkten $x, y \in \mathbb{R}^2$?
    
    \item Gegeben: Gerade $L \subseteq \mathbb{R}^2$ und $x \notin L$ ein Punkt, der nicht auf $L$ liegt. \\
    Wie viele Geraden durch $x$ gibt es, die $L$ nicht schneiden?
    
    \item Was ist die Innenwinkelsumme eines Dreiecks?
\end{itemize}

\underline{Sphärische Geometrie (von $S^2$)}

\begin{itemize}
    \item Dieselben Fragen für $x, y \in S^2$.
\end{itemize}

%%%%%%%%%%%%%%%%%%%%%%%%%%%%%%%%%%%%%%%%%%%%%%%%%%%%%%%%%%%%%%%%%%%%%%%%%%%%%%%%%%%%%% Vorlesung 26 %%%%%%%%%%%%%%%%%%%%%%%%

Wir betrachten
\[
S^2 = \left\{ (x, y, z) \in \mathbb{R}^3 \,\middle|\, x^2 + y^2 + z^2 = 1 \right\} \subseteq \mathbb{R}^3
\]
die Einheitskugel (Radius 1) um den Ursprung \( O \).\\

\underline{Beobachtung:}
Der Schnitt einer Ebene in \( \mathbb{R}^3 \) mit einer Sphäre ist entweder leer, ein Punkt oder ein Kreis. (Kein Oval.)\\


\textbf{Definition 9.18:}
Alle Kreise auf der Sphäre \( S^2 \), deren Mittelpunkt \( O \) ist, heißen \emph{Großkreise}.\\

\textbf{Bemerkung:}
Großkreise sind die Kreise, die durch Schnitte mit Ebenen durch \( O \) mit \( S^2 \) entstehen.\\

\textbf{Definition 9.19}\\
Zwei Punkte \( P, Q \) auf \( S^2 \) heißen \underline{diametral}, wenn gilt:
\[
\{P, Q\} = \ell \cap S^2
\]
für eine Gerade \( \ell \) durch \( O \). (Man sagt auch: „P und Q liegen sich diametral gegenüber.“)\\

\textbf{Satz 9.20}\\
Jeder Kreis auf der Sphäre, der ein diametrales Punktpaar enthält, ist ein Großkreis.\\

\underline{Beobachtung:}
Der Abstand von zwei diametralen Punkten auf der Sphäre, gemessen in \( \mathbb{R}^3 \), ist 2 \quad \text{(Zweimal der Radius)}.

\medskip

{Frage:} Was ist der Abstand gemessen in \( S^2 \)?

{Idee:} Betrachte alle Kurven in \( S^2 \), welche zwei Punkte verbinden, berechne deren Länge und verwende die minimale Länge.

\medskip

Sei \( M \subseteq \mathbb{R}^n \) eine eingebettete Untermannigfaltigkeit. Für \( x, y \in M \) definieren wir:
\[
d_M(x, y) := \inf \left\{ L(\gamma) \,\middle|\, \gamma \colon [0,1] \to M \text{ stetig},\, \gamma(0) = x,\, \gamma(1) = y \right\}
\]

Dabei bezeichnet \( L(\gamma) \) die (euklidische) Länge der Kurve \( \gamma \).\\

\underline{Fakten:}
\begin{itemize}
    \item \( d_M \) ist eine Metrik auf \( M \).
    
    \item Die metrische Topologie von \( d_M \) ist die Teilraumtopologie von \( M \).
    
    \item Ist \( (M, d_M) \) vollständig (als metrischer Raum), dann ist das Infimum in der Definition von \( d_M(x,y) \) ein Minimum.
\end{itemize}

\medskip

\textbf{Beispiel:} Sei \( M = \mathbb{R}^2 \setminus \{0\} \). Dann ist \( d_M = d_{\txt{eukl.}} \), aber nicht vollständig.
      \begin{figure}[H]
    \centering
    \includegraphics[width=6cm]{Image Diffgeo/26.01.png}
 \end{figure}
 
\textbf{Definition 9.21:}  
Eine \underline{Geodätische} \( \gamma \) in einem metrischen Raum \( (X, d) \) ist eine \emph{stetige, abstandserhaltende} Abbildung
\[
\gamma \colon I \to X,
\]
wobei \( I \) eines der folgenden Teilintervalle in \( \mathbb{R} \) ist:
\[
\mathbb{R},\quad [a, b],\quad [a, \infty),\quad (-\infty, b]
\]

\medskip

\textbf{Satz 9.22:}  
Die Geodätische zwischen zwei nicht diametralen Punkten auf einer Sphäre ist der \emph{kürzeste Bogen} des (eindeutigen) Großkreises durch die beiden Punkte.

Für zwei diametrale Punkte ist jeder verbindende Halbkreis­bogen eine Geodätische.
\begin{proof}
    Riemannsche Geometrie
\end{proof}
      \begin{figure}[H]
    \centering
    \includegraphics[width=3cm]{Image Diffgeo/26.02.png}
 \end{figure}
Der Satz liefert: Eine (sphärische) Geodätische ist ein Stück eines Großkreises, das nicht mehr als zwei diametrale Punkte enthält. Sphärische Geodätische sind also isometrische Einbettungen von Intervallen der Länge höchstens
\[
\frac{1}{2} \cdot \text{(Umfang der Sphäre)}.
\]

Insbesondere sind Geodätische zwischen diametralen Punkten \emph{nicht eindeutig}.

\medskip

\underline{Fakt:} Großkreise sind genau die Bilder von lokalen Geodätischen auf der Sphäre.\\

\textbf{Satz 9.23:}  
Seien \( a, b \) zwei nicht-diametrale Punkte auf der Sphäre. Wir bezeichnen die Bogenlänge des kürzesten verbindenden Großkreisbogens als \emph{sphärischen Abstand} \( |ab|_{S} \) von \( a \) und \( b \). Der sphärische Abstand zweier diametraler Punkte ist die halbe Länge des Umfangs eines Großkreises. Genauer:
\[
|ab|_{S} = \angle aOb
\]
       \begin{figure}[H]
    \centering
    \includegraphics[width=4cm]{Image Diffgeo/26.03.png}
 \end{figure}
 
\underline{Vergleich mit euklidischer Geometrie}\\

\begin{tabular}{>{\bfseries}m{4cm}|m{5cm}|m{5cm}}
 & Euklidische Geometrie \( \mathbb{R}^2 \) & Sphärische Geometrie \( S^2 \) \\
\hline
{Schnitt von Geraden} &
0, 1 oder unendlich viele Schnittpunkte &
Immer 2 Schnittpunkte (außer identisch) \newline (es gibt keine Parallelen) \\
\hline
{Dreiecke} &
Innenwinkelsumme = \( 180^\circ \) &
Innenwinkelsumme \( > 180^\circ \) \newline (Großkreis-Dreiecke) \\
\end{tabular}

\vspace{1cm}

\textbf{Definition 9.24:}  
Das Maß des Winkels zwischen zwei Kurven \( c_1, c_2 \colon [0,1] \to S^2 \) auf \( S^2 \) mit \( c_1(0) = c_2(0) = p \) ist das euklidische Winkelmaß zwischen den Tangentialvektoren \( v_1, v_2 \) an \( c_1, c_2 \) in \( p \). Wir schreiben dafür:
\[
\angle_p(c_1, c_2)
\]

\medskip

\textbf{Definition 9.25:}  
Sei \( U \subseteq S^2 \) offen. Eine Abbildung \( f \colon U \to \mathbb{R}^2 \) heißt \underline{winkeltreu}, falls für alle Paare \( c_1, c_2 \) von Geodäten auf \( S^2 \) mit gemeinsamem Startpunkt \( p \in U \) gilt:
\[
\angle_p(c_1, c_2) = \angle_{f(p)}(f(c_1), f(c_2))
\]

\medskip

\textbf{Satz 9.26:}  
Die stereographische Projektion ist winkeltreu.

\subsubsection*{Hyperbolische Geometrie}
Gibt es auch eine 2-dimensionale Geometrie, bei der es zu viele „Parallelen“ gibt?

\medskip

\textbf{Definition 9.27 (Das Poincaré-Halbebenenmodell):}  
Die Punktmenge der hyperbolischen Geometrie in diesem Modell ist gegeben durch
\[
\mathbb{H}^2 := \left\{ z \in \mathbb{C} \,\middle|\, \operatorname{Im}(z) > 0 \right\}
\]

{Hyperbolische Geraden} in \( \mathbb{H}^2 \) sind:
\begin{itemize}
    \item die in \( \mathbb{H} \) liegenden Teile von Kreisen mit Mittelpunkt \( p \in \mathbb{C} \) mit \( \operatorname{Im}(p) = 0 \),
    
    \item sowie Mengen der Form \( \left\{ a + i b \mid b \in \mathbb{R}_{>0} \right\} \), d.\,h. euklidische Geraden, die senkrecht auf der Achse \( \operatorname{Im}(z) = 0 \) stehen.
\end{itemize}
       \begin{figure}[H]
    \centering
    \includegraphics[width=10cm]{Image Diffgeo/26.04.png}
 \end{figure}

 Die Metrik auf \( \mathbb{H}^2 \) verhält sich so, dass Abstände von Punkten in der Ebene davon abhängen, wie nah diese an der reellen Achse sind. Je näher man an der reellen Achse (im euklidischen Sinne) kommt, desto mehr verändert sich die Metrik.

\medskip

\textbf{Satz 9.28:}  
Zu je zwei verschiedenen Punkten \( p, q \) in der hyperbolischen Ebene existiert genau eine hyperbolische Gerade, die \( p \) und \( q \) enthält.\\

\textbf{Bemerkung (Schnittpunkt im Unendlichen):}

Sind zwei hyperbolische Geraden senkrecht zur reellen Achse, so schneiden sie sich in \( \mathbb{C} \) und auch in \( \mathbb{H}^2 \) nicht.  
Wir fügen daher einen abstrakten Schnittpunkt, genannt \( \infty \), im Rand \emph{hinzu}, der auf all diesen Geraden liegt.\\

Es schneiden sich also zwei solche \emph{hyperbolische Geraden im Punkt \( \infty \)}.\\

Ein zweiter \emph{Ausnahmefall} sind Geraden, die die reelle Achse im selben Punkt schneiden.  
Solche Geraden schneiden sich ebenfalls im Grenzwert.  
Wir sagen in beiden Fällen, dass sich die hyperbolischen Geraden \emph{im Unendlichen schneiden}.\\

Wir definieren einen tatsächlichen Schnittpunkt durch die erweiterte Menge:
\[
\widehat{\mathbb{H}}^2 := \mathbb{H} \cup \{ \infty \} \cup \left\{ z \in \mathbb{C} \mid \mathrm{Im}(z) = 0 \right\}
\]

\textbf{Definition 9.29:} Unter \emph{hyperbolischen Dreiecken} verstehen wir die Figuren,  
die von drei hyperbolischen, sich paarweise schneidenden Geradenstücken begrenzt werden.  
Die Geradenschnittpunkte heißen \emph{Ecken} des Dreiecks. Dabei sind Schnittpunkte im Unendlichen explizit erlaubt.\\

\textbf{Definition 9.30:} Der Winkel zwischen zwei sich in \( \mathbb{H}^2 \) schneidenden Geraden  
ist der Winkel zwischen den (euklidischen) Tangentialvektoren im Schnittpunkt.  
Der Winkel zwischen hyperbolischen Geraden der Form  
\[
\{ a + ib \mid b \in \mathbb{R}_{>0} \} \quad \text{und} \quad \{ c + id \mid d \in \mathbb{R}_{>0} \}
\]  
ist \( 0 \) (per Definition).\\

\textbf{Definition 9.31:} Der Abstand zwischen Punkten \( p, q \in \mathbb{H}^2 \) ist gegeben durch
\[
d(p, q) = \log \left( \frac{ |p - \overline{q}| + |p - q| }{ |p - \overline{q}| - |p - q| } \right)
\]
\( |\cdot| \): Betrag der komplexen Zahl, \quad \( \overline{q} \): komplex konjugiert von \( q \).

\vspace{1em}

\textbf{Satz 9.32:} Der obige Abstand ist eine Metrik auf \( \mathbb{H}^2 \).  
Die Abbildungen
\[
f(z) = \frac{az + b}{cz + d} \in \mathrm{PSL}(2, \mathbb{R})
\]
mit \( a, b, c, d \in \mathbb{R} \) erhalten \( \mathbb{H}^2 \) als Menge und bilden hyperbolische Geraden auf hyperbolische Geraden ab.  
Diese Abbildungen heißen \emph{Möbiustransformationen}, falls \( ad - bc = 1 \).

%%%%%%%%%%%%%%%%%%%%%%%%%%%%%%%%%%%%%%%%%%%%%%%%%%%%%%%%%%%%%%%%%%%%%%%%%%%%%%%%%%%%% the end %%%%%%%%%%%%%%%%%%%%%%%%%%%%%%

\end{document}
